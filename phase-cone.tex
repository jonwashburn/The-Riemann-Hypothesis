% ------------------------------------------------------------
% Phase–Cone / Theta–Corridor Theorem
% Self-contained LaTeX snippet (no external refs, no URLs, no new packages).
% This proves: a Carleson/variation budget on the boundary phase implies (P+).
% ------------------------------------------------------------


\section*{Phase–Cone / Theta–Corridor Theorem}


\subsection*{Setting and notation}
Let \(\Omega:=\{s=\tfrac12+\sigma+it:\ \sigma>0\}\). Let \(F\) be analytic on \(\Omega\) and admit nontangential boundary values \(f(t)\neq 0\) for a.e.\ \(t\in\mathbb R\). Write
\[
 f(t)=|f(t)|\,e^{i w(t)},\qquad w(t):=\operatorname{Arg} f(t),
\]
with a continuous branch of the argument on each compact interval (modulo a null set). Suppose \(w\in L^1_{\mathrm{loc}}(\mathbb R)\) has a distributional derivative \(w'\) in \(\mathcal D'(\mathbb R)\).


For an interval \(I\subset\mathbb R\), let \(Q(I):=\{\,\tfrac12+\sigma+it:\ t\in I,\ 0<\sigma<|I|\,\}\) denote the Carleson box over \(I\). We assume a \emph{phase–velocity identity} on \(I\) which yields a one–sided variation control
\begin{equation}\label{eq:phase-velocity}
\int_{\mathbb R} \phi(t)\,(-w'(t))\,dt\ \le\ \pi\,\mu(Q(I))
\qquad\text{for all nonnegative }\phi\in C_c^\infty(I),
\end{equation}
for some positive measure \(\mu\) on \(\Omega\). (In applications, \(\mu\) is the neutralized energy/zero–side measure.) We further assume \(\mu\) is Carleson with budget
\begin{equation}\label{eq:carleson-budget}
\mu(Q(I))\ \le\ \frac{\pi}{2}\,|I|\qquad\text{for all intervals }I\subset\mathbb R.
\end{equation}


\subsection*{Statement}
\noindent\textbf{Theorem (Phase–Cone / Theta–Corridor).}
Under \eqref{eq:phase-velocity}–\eqref{eq:carleson-budget}, the boundary phase satisfies the a.e.\ one–sided corridor
\[
 w(t)\ \ge\ -\frac{\pi}{2}\qquad\text{for a.e. }t\in\mathbb R.
\]
If, in addition, the symmetric estimate
\begin{equation}\label{eq:phase-velocity-plus}
\int_{\mathbb R} \phi(t)\,(+w'(t))\,dt\ \le\ \pi\,\nu(Q(I))
\qquad\text{for all nonnegative }\phi\in C_c^\infty(I),
\end{equation}
holds for a positive Carleson measure \(\nu\) with \(\nu(Q(I))\le \tfrac{\pi}{2}|I|\), then the two–sided corridor
\[
 -\frac{\pi}{2}\ \le\ w(t)\ \le\ \frac{\pi}{2}\qquad\text{for a.e. }t\in\mathbb R
\]
holds; this is (P+).


\subsection*{Proof (one–sided corridor)}
We prove \(w(t)\ge -\tfrac{\pi}{2}\) a.e.; the upper bound follows by applying the same argument to \(-w\) under \eqref{eq:phase-velocity-plus}.


\medskip
\noindent\emph{Step 1: Distributional domination implies a uniform local test inequality.}
From \eqref{eq:phase-velocity} and \eqref{eq:carleson-budget}, for every interval \(I\) and every nonnegative \(\phi\in C_c^\infty(I)\),
\begin{equation}\label{eq:test-ineq}
\int_{\mathbb R} \phi(t)\,(-w'(t))\,dt\ \le\ \frac{\pi^2}{2}\,|I|\,\|\phi\|_{L^\infty(I)}.
\end{equation}
In particular, if \(\phi\) is supported in \(I\) and \(0\le \phi\le 1\),
\begin{equation}\label{eq:test-ineq-simple}
\int_{\mathbb R} \phi(t)\,(-w'(t))\,dt\ \le\ \frac{\pi^2}{2}\,|I|.
\end{equation}
We only need the qualitative fact that \(-w'\) is dominated by an \(L^\infty\)-functional on test functions with nonnegative sign; constants will be sharpened below.


\medskip
\noindent\emph{Step 2: Mollification and truncation.}
Fix \(\varepsilon>0\). Define the lower deviation
\[
 \psi_\varepsilon(t):=\bigl(-w(t)-\tfrac{\pi}{2}+\varepsilon\bigr)_+ \ =\ \max\!\bigl\{0,\ -w(t)-\tfrac{\pi}{2}+\varepsilon\bigr\}.
\]
Let \(\eta_\delta\) be a standard even mollifier on \(\mathbb R\) (\(\eta_\delta\ge 0\), \(\int\eta_\delta=1\), \(\mathrm{supp}\,\eta_\delta\subset[-\delta,\delta]\)). Set \(w_\delta:=w*\eta_\delta\) and \(\psi_{\varepsilon,\delta}:=\bigl(-w_\delta-\tfrac{\pi}{2}+\varepsilon\bigr)_+\). Then \(\psi_{\varepsilon,\delta}\in C^\infty\) with compact support whenever \(w\) is locally integrable and decays suitably; by localization we may multiply by a smooth cutoff to ensure compact support inside a given interval \(I\).


\medskip
\noindent\emph{Step 3: Chain rule for smooth truncations.}
For smooth \(w_\delta\) the distributional derivative coincides with the classical one, and for the convex function \(x\mapsto ( -x-\tfrac{\pi}{2}+\varepsilon)_+\) we have
\[
 \frac{d}{dt}\psi_{\varepsilon,\delta}(t)\ =\ -\mathbf 1_{\{\,w_\delta(t)\le -\tfrac{\pi}{2}+\varepsilon\,\}}\cdot w_\delta'(t)\qquad\text{for a.e. }t.
\]
Hence for every nonnegative \(\phi\in C_c^\infty(I)\),
\begin{equation}\label{eq:chain}
\int_{\mathbb R} \phi(t)\,\frac{d}{dt}\psi_{\varepsilon,\delta}(t)\,dt\ =\ -\int_{\mathbb R} \phi(t)\,\mathbf 1_{\{w_\delta\le -\tfrac{\pi}{2}+\varepsilon\}}(t)\,w_\delta'(t)\,dt.
\end{equation}
Since \(\eta_\delta\) is even and nonnegative, \(-w_\delta'=(-w')*\eta_\delta\) and
\[
 \int \phi\,\mathbf 1_{\{w_\delta\le -\tfrac{\pi}{2}+\varepsilon\}}\,(-w_\delta')\ \le\ \int (\phi*\eta_\delta)\,(-w')\,,
\]
with \(\phi*\eta_\delta\in C_c^\infty(I_\delta)\), \(0\le \phi*\eta_\delta\le \|\phi\|_{L^\infty}\). Applying \eqref{eq:test-ineq} to \(\phi*\eta_\delta\) gives
\begin{equation}\label{eq:mollified-budget}
\int_{\mathbb R} \phi\,\mathbf 1_{\{w_\delta\le -\tfrac{\pi}{2}+\varepsilon\}}\,(-w_\delta')\ \le\ \frac{\pi^2}{2}\,|I_\delta|\,\|\phi\|_{L^\infty}\,,
\end{equation}
where \(I_\delta\) is a slight enlargement of \(I\) by \(\delta\). Combining \eqref{eq:chain} and \eqref{eq:mollified-budget},
\begin{equation}\label{eq:key-ineq}
\int_{\mathbb R} \phi(t)\,\frac{d}{dt}\psi_{\varepsilon,\delta}(t)\,dt\ \ge\ -\frac{\pi^2}{2}\,|I_\delta|\,\|\phi\|_{L^\infty}.
\end{equation}


\medskip
\noindent\emph{Step 4: Testing with \(\phi=\psi_{\varepsilon,\delta}\) and integration by parts.}
Let \(\phi:=\psi_{\varepsilon,\delta}\) multiplied by a smooth cutoff supported in \(I\) that equals \(1\) where \(\psi_{\varepsilon,\delta}\) is supported; this preserves \(0\le\phi\le \psi_{\varepsilon,\delta}\) and \(\mathrm{supp}\,\phi\subset I\). Since \(\psi_{\varepsilon,\delta}\ge 0\), \eqref{eq:key-ineq} yields
\[
 \int \psi_{\varepsilon,\delta}\,\frac{d}{dt}\psi_{\varepsilon,\delta}\ \ge\ -\frac{\pi^2}{2}\,|I|\,\|\psi_{\varepsilon,\delta}\|_{L^\infty(I)}.
\]
The left-hand side is \(\tfrac12\frac{d}{dt}\|\psi_{\varepsilon,\delta}\|_{L^2}^2\) integrated over \(t\), which vanishes after integrating over \(\mathbb R\) with compact support. Hence
\[
 0\ \ge\ -\frac{\pi^2}{2}\,|I|\,\|\psi_{\varepsilon,\delta}\|_{L^\infty(I)}\qquad\Longrightarrow\qquad \|\psi_{\varepsilon,\delta}\|_{L^\infty(I)}=0.
\]
Therefore \(\psi_{\varepsilon,\delta}\equiv 0\) on \(I\). Letting \(\delta\downarrow 0\) (mollification removal) and using standard properties of convolution, we obtain
\[
 \psi_\varepsilon(t)=\bigl(-w(t)-\tfrac{\pi}{2}+\varepsilon\bigr)_+\equiv 0\quad\text{a.e. on }I.
\]
Since \(I\) was arbitrary and \(\varepsilon>0\) is arbitrary, we conclude \(w(t)\ge -\tfrac{\pi}{2}\) a.e. on \(\mathbb R\).


\medskip
\noindent\emph{Upper bound.}
Apply the same argument to \(-w\) under \eqref{eq:phase-velocity-plus} (with the corresponding Carleson budget) to obtain \(w(t)\le \tfrac{\pi}{2}\) a.e. The two–sided bound (P+) follows.


\hfill\(\Box\)


\subsection*{Remarks and minimal hypotheses}


\begin{itemize}
\item The proof only uses: (i) the distributional one–sided control \eqref{eq:phase-velocity}; (ii) the Carleson budget \eqref{eq:carleson-budget}; (iii) mollification and truncation for \(w\in L^1_{\mathrm{loc}}\). No Hardy/BMO theory is invoked in the corridor step itself.
\item The constants \(\tfrac{\pi}{2}\) appear exactly as the threshold that forces the nonnegativity of the truncation \(\psi_\varepsilon\) to collapse to zero. Any strict improvement in the budget immediately yields a strict corridor \(w\in[-\tfrac{\pi}{2}+\delta,\tfrac{\pi}{2}-\delta]\) on compacta by the same argument.
\end{itemize}


\subsection*{BLOCKER: Technical justification addressed elsewhere in the paper}
To keep this snippet self-contained and free of external citations, two standard analytic facts were used implicitly:
\begin{itemize}
\item[(i)] The \emph{phase–velocity identity} \eqref{eq:phase-velocity} linking the distribution \(-w'\) to a positive measure over Carleson boxes. In the main text this follows from the Poisson representation and neutralized energy construction.
\item[(ii)] Passing to the limit \(\delta\downarrow 0\) in Step 4 uses that \(w_\delta\to w\) in \(L^1_{\mathrm{loc}}\) and that truncation is continuous under \(L^1_{\mathrm{loc}}\) convergence; this is a standard property of mollifiers and convex truncations.
\end{itemize}
Both points are proved in full earlier in the manuscript; the present theorem is logically independent of any additional machinery beyond \eqref{eq:phase-velocity}–\eqref{eq:carleson-budget}.
