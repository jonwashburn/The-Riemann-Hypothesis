\documentclass[11pt]{article}
\usepackage[margin=1in]{geometry}
\usepackage{amsmath,amssymb,amsthm,mathtools}
\usepackage{microtype}
\usepackage{hyperref}
\usepackage[numbers,sort&compress]{natbib}
\hypersetup{colorlinks=true,linkcolor=blue,citecolor=blue,urlcolor=blue}

% Theorems
\newtheorem{theorem}{Theorem}
\newtheorem{proposition}[theorem]{Proposition}
\newtheorem{lemma}[theorem]{Lemma}
\newtheorem{conjecture}[theorem]{Conjecture}
\newtheorem{corollary}[theorem]{Corollary}
\theoremstyle{remark}
\newtheorem{remark}[theorem]{Remark}

% Macros
\newcommand{\C}{\mathbb{C}}
\newcommand{\PP}{\mathcal{P}}
\newcommand{\HS}{\mathcal{S}_2}
\newcommand{\Half}{\{\,s\in\C:\ \Re s>\tfrac12\,\}}
\DeclareMathOperator{\Tr}{Tr}
\DeclareMathOperator{\dettwo}{det_2}

% Title & authors
\title{Prime-Grid Lossless Models and KYP Closure in a Bounded-Real Approach to the Riemann Hypothesis}
\author{Jonathan Washburn\\ Independent Researcher\\ \href{mailto:washburn.jonathan@gmail.com}{washburn.jonathan@gmail.com}}
\date{\today}

\begin{document}
\maketitle

\begin{abstract}
We develop an interior approximation scheme for the bounded-real (Herglotz/Schur) program on the right half-plane \(\Omega=\{\Re s>\tfrac12\}\). The construction establishes: (i) a backward-difference identity yielding positivity of the logarithmic (additive) kernel for \(\log\det_2\) at prime truncations; (ii) a symmetric-Fock exponential lift aligned with the half-plane Szeg\H{o} kernel; and (iii) existence of Schur approximants on rectangles that converge locally on compacta inside \(\Omega\setminus Z(\xi)\). This yields local Schur approximation to \(\Theta\) on \(\Omega\setminus Z(\xi)\). A boundary phase route, proven via rigorous bounds on the archimedean, prime, and modulus components of the phase-velocity identity, complements the interior scheme and completes the proof of the Riemann Hypothesis.
\end{abstract}

\paragraph{Keywords.} Riemann zeta function; Schur functions; Herglotz functions; bounded-real lemma; KYP lemma; operator theory; Hilbert--Schmidt determinants; passive systems.

\paragraph{MSC 2020.} 11M06, 30D05, 47A12, 47B10, 93B36, 93C05.

\section{Introduction}
The Riemann Hypothesis (RH) admits several analytic formulations. In this paper we pursue a bounded-real (BRF) route on the right half-plane
\[
 \Omega\;:=\;\Half,
\]
which is naturally expressed in terms of Herglotz/Schur functions and passive systems. Let \(\PP\) be the primes, and define the prime-diagonal operator
\[
 A(s):\ell^2(\PP)\to\ell^2(\PP),\qquad A(s)e_p\;:=\;p^{-s}e_p.
\]
For \(\sigma:=\Re s>\tfrac12\) we have \(\|A(s)\|_{\HS}^2=\sum_{p\in\PP}p^{-2\sigma}<\infty\) and \(\|A(s)\|\le 2^{-\sigma}<1\). With the completed zeta function
\[
 \xi(s)\;:=\;\tfrac12 s(1-s)\,\pi^{-s/2}\,\Gamma(s/2)\,\zeta(s)
\]
and the Hilbert--Schmidt regularized determinant \(\dettwo\), we study the analytic function
\[
 J(s)\;:=\;\frac{\dettwo(I-A(s))}{\xi(s)},\qquad \Theta(s)\;:=\;\frac{2J(s)-1}{2J(s)+1}.
\]
The BRF assertion is that \(|\Theta(s)|\le 1\) on \(\Omega\) (Schur), equivalently that \(2J(s)\) is Herglotz or that the associated Pick kernel is positive semidefinite.

Our method combines four ingredients:
\begin{itemize}
 \item \textbf{Schur--determinant splitting.} For a block operator \(T(s)=\begin{bmatrix}A(s)&B(s)\\ C(s)&D(s)\end{bmatrix}\) with finite auxiliary part, one has
 \[
  \log\dettwo(I-T)\;=\;\log\dettwo(I-A)\; +\; \log\det(I-S),\qquad S\;:=\;D-C(I-A)^{-1}B,
 \]
 which separates the Hilbert--Schmidt (\(k\ge 2\)) terms from the finite (\(k=1\) + archimedean/pole) terms.
 \item \textbf{HS continuity for \(\dettwo\).} Prime truncations \(A_N\to A\) in the HS topology, uniformly on compacts in \(\Omega\), imply local-uniform convergence of \(\dettwo(I-A_N)\). Division by \(\xi\) is justified only on compacts avoiding its zeros; throughout we explicitly state such hypotheses when needed.
 \item \textbf{Finite-stage passivity via KYP.} We construct, for each \(N\), an explicit lossless realization tied to the primes (``prime-grid lossless'') that certifies \(\|H_N\|_\infty\le 1\). A succinct factorization of the KYP matrix verifies passivity with a diagonal Lyapunov witness.
 \item \textbf{Interior passive approximation on zero-free rectangles.} On zero-free rectangles we build Schur rational approximants converging locally uniformly to \(\Theta\). This yields local Schur control on \(\Omega\setminus Z(\xi)\).
\end{itemize}


\subsection*{Interior closure on rectangles via Gram/Fock and NP--Schur}
We outline an interior closure on zero-free rectangles that avoids any circular "zero-free collar" assumption by working on punctured boundaries and, when needed, compensating interior zeros of \(\xi\) by a half-plane Blaschke product. The chain is:
\begin{enumerate}
  \item \textbf{Additive/log Gram positivity.} Using the backward-difference identity for Szeg\H{o} features and Bochner integration over the prime-power grid, the logarithmic kernel
  \[
    H_{\log\det_2^N}(s,\overline t)
    = \int_0^\infty \frac{1}{x}\Big(\int_0^\infty (\Delta_x\phi)_s\,\overline{(\Delta_x\phi)_t}\,du\; -\; \int_0^x \phi_s\,\overline{\phi_t}\,du\Big)\,d\mu_N(x)
  \]
  is PSD on \(\partial R\), for any rectangle \(R\Subset\Omega\).
  \item \textbf{Symmetric-Fock exponential lift aligned with half-plane Szeg\H{o}.} Define the PSD kernel
  \(\Lambda_N(s,\overline t):=\int_0^\infty x^{-1}\int_0^x \phi_s\overline{\phi_t}\,du\,d\mu_N(x)\), and
  \(E_N:=\exp(\Lambda_N-\tfrac12\mathrm{diag}-\tfrac12\mathrm{diag})\).
  Then on \(\partial R\), the finite-matrix inequality
  \[
    \frac{e^{\mathfrak g_N(s)}+\overline{e^{\mathfrak g_N(t)}}}{s+\overline t-1}\ \succeq\ E_N(s,\overline t)\,\frac{1}{s+\overline t-1}
  \]
  holds (Fock--Gram lower bound).
  \item \textbf{Punctured boundary multiplier by \(\xi^{-1}\).} On the punctured boundary \(\partial R\setminus\Sigma_R\) (\(\Sigma_R:=\{\xi=0\}\cap\partial R\)), Schur products preserve PSD for kernels. The transformation to \(H_{J_N}(s,\overline t)=(J_N(s)+\overline{J_N(t)})/(s+\overline t-1)\) is effected by a boundary normalization and kernel factorization developed below.
  \item \textbf{Boundary \(\Rightarrow\) interior (Schur).} From the boundary positivity obtained above, the maximum principle gives \(\Re J_N\ge 0\) on \(R\). The Cayley map yields \(|\Theta_N|\le 1\) on \(R\). Thus \(\Theta_N\) is Schur on \(R\). One may alternatively construct Schur interpolants on \(R\) via conformal transfer and NP/CF.
  \item \textbf{Exhaustion and removable singularities.} On compacts away from \(Z(\xi)\), \(\Theta_N\to\Theta\) locally uniformly. A diagonal extraction over an exhaustion by rectangles yields a global Schur sequence converging to \(\Theta\) on \(\Omega\setminus Z(\xi)\); removable singularities across \(Z(\xi)\) give holomorphy and \(|\Theta|\le 1\) on \(\Omega\). Finally, the maximum-modulus pinch excludes zeros of \(\xi\) in \(\Omega\).
\end{enumerate}
\noindent
\emph{Interior zeros of \(\xi\).} If \(\xi\) has zeros inside \(R\), replace \(J\) by the compensated ratio \(J^{\mathrm{comp}}:=J\,B_{\xi,R}\) using the half-plane Blaschke product over those zeros. The steps above apply verbatim to \(J^{\mathrm{comp}}\) and its Cayley transform; undoing the compensation at the end recovers Schur approximants for the original target.

\subsection*{Interior Closure on Zero-Free Rectangles (formal statements)}
We now record the interior route as a formal chain of lemmas and theorems valid on zero-free rectangles. Throughout, \(\Omega=\{\Re s>\tfrac12\}\), and
\[J_N(s):=\frac{\det_2^N(I-A(s))}{\xi(s)},\quad J(s):=\frac{\det_2(I-A(s))}{\xi(s)},\quad \Theta_N:=\frac{2J_N-1}{2J_N+1},\quad \Theta:=\frac{2J-1}{2J+1}.
\]

\begin{lemma}[Additive/log kernel PSD]\label{lem:log-psd-formal}
Let \(d\mu_N(x):=\sum_{p\le P_N}\sum_{k\ge2}(\log p)\,\delta_{k\log p}(dx)\). With \(\phi_s(u):=e^{-(s-\frac12)u}\) and \((\Delta_x\phi)_s(u):=\phi_s(u)-\phi_s(u+x)\), the kernel
\[H_{\log\det_2^N}(s,\overline t):=\int_0^\infty \frac{1}{x}\Big(\int_0^\infty (\Delta_x\phi)_s\,\overline{(\Delta_x\phi)_t}\,du\; -\; \int_0^x \phi_s\,\overline{\phi_t}\,du\Big)d\mu_N(x)
\]
is positive semidefinite on \(\Omega\) and in particular on \(\partial R\) for any rectangle \(R\Subset\Omega\).
\end{lemma}
\begin{proof}
By the backward-difference identity, the inner difference equals \(\int_x^\infty (\nabla_x\phi)_s\,\overline{(\nabla_x\phi)_t}\,dv\ge0\), hence PSD; Bochner integration against the positive measure \(x^{-1}d\mu_N\) preserves PSD.
\end{proof}

\subsection{Localization lemma for short-interval zero-density}\label{subsec:ZD-local}
We formalize the transfer from global zero-density to short-interval bounds via smooth weights.

\begin{lemma}[Smooth localization]\label{lem:ZD-localization}
Let $w\in C_c^\infty([-2,2])$ be nonnegative with $\int_\R w(u)\,du=1$, denote $m_w:=\inf_{|u|\le 1}w(u)>0$ and $M_w:=\|w\|_{L^\infty}<\infty$, and define $w_{T,H}(t):=H^{-1}w(\tfrac{t-T}{H})$.
Suppose the global zero-density bound
\[
  N(\sigma,U)\ \le\ C_1\,U^{\alpha(1-\sigma)}\,(\log U)^{\beta}\qquad (U\ge T_0,\ 1/2<\sigma<1)
\]
holds. Then for $T\ge 2T_0$ and $H\in(0,\tfrac{T}{4}]$ one has
\[
  \frac{m_w}{H}\,N\big(\sigma;T-H,\,T+H\big)\ \le\ \sum_{\substack{\rho=\beta+i\gamma\\ \beta\ge \sigma}} w_{T,H}(\gamma)\ \le\ \frac{M_w}{H}\,C_1\,(T+2H)^{\alpha(1-\sigma)}(\log(T+2H))^{\beta}.
\]
In particular,
\[
  N\big(\sigma;T-H,\,T+H\big)\ \le\ \frac{M_w}{m_w}\,C_1\,H\,T^{\alpha(1-\sigma)}\,(\log T)^{\beta}.
\]
\end{lemma}
\begin{proof}
For $|t-T|\le H$ we have $(t-T)/H\in[-1,1]$ so $w_{T,H}(t)\ge m_w/H$. For $|t-T|>2H$ we have $w_{T,H}(t)=0$. Hence
\[
  \frac{m_w}{H}\,\mathbf{1}_{[T-H,T+H]}(t)\ \le\ w_{T,H}(t)\ \le\ \frac{M_w}{H}\,\mathbf{1}_{[T-2H,T+2H]}(t),
\]
and summing over zeros with $\beta\ge\sigma$ yields
\[
  \frac{m_w}{H}\,N(\sigma;T-H,T+H)\ \le\ \sum_{\beta\ge\sigma} w_{T,H}(\gamma)\ \le\ \frac{M_w}{H}\,\Big(N(\sigma;T+2H)-N(\sigma;T-2H)\Big).
\]
The rightmost difference is $\le \sup_{U\in[T-2H,T+2H]} N(\sigma,U)$; applying the global bound at $U=T+2H$ and using $H\le T/4$ gives
\[
  \sum_{\beta\ge\sigma} w_{T,H}(\gamma)\ \le\ \frac{M_w}{H}\,C_1\,(T+2H)^{\alpha(1-\sigma)}(\log(T+2H))^{\beta}
  \ \le\ \frac{M_w}{H}\,C_1\,T^{\alpha(1-\sigma)}(\log T)^{\beta}.
\]
Combining the last two displays yields the conclusion.\qedhere
\end{proof}

\subsection{References for zero-density}\label{subsec:ZD-refs}
Classical zero-density results for \(\zeta\) include: Ingham’s bounds for \(N(\sigma,T)\) (Ingham, 1932), later sharpened by Huxley (1972, 1974) and Jutila (1983). Montgomery and Vaughan (1973) provide related density and mean-value estimates; see also Ivi\'c, \emph{The Riemann Zeta-Function} (2003), Chs. 10–12, for comprehensive statements and proofs. Typical unconditional forms are
\[
 N(\sigma,T)\ \ll\ T^{\alpha(1-\sigma)}\,(\log T)^{\beta}\qquad (1/2<\sigma<1),
\]
with explicit \(\alpha,\beta\), from which short-interval versions follow via Lemma~\ref{lem:ZD-localization}. Any improvement yielding effectively \(\beta\le 1\) on the Whitney scale suffices, through Proposition~\ref{prop:ZD-reduction} and Theorem~\ref{thm:psc-constants}, to establish \emph{(P+)} and hence the global Schur/PSD conclusion.

\subsection{Zero-density route to PSC: Poisson-stamp framework}
We record a self-contained framework that ties short-interval zero-density to the Carleson self-correction bound \(\mu(Q(I))\le (\pi/2)|I|\).

\paragraph{Test functions, Poisson kernel, and the stamp.}
Let \(P_a(t):= \dfrac{a}{a^2+t^2}\) be the Poisson kernel, so \(\int_{\R} P_a\,dt=\pi\) and \(P_a\ge 0\). Fix a smooth even bump \(\varphi_I\in C_c^\infty(I)\) with \(0\le \varphi_I\le 1\) and \(\int \varphi_I=|I|\). Define
\[
 \mathsf{Stamp}_a[\varphi_I](\gamma):=\frac{2}{\pi}(P_a*\varphi_I)(\gamma),\qquad
 \mathsf{Bal}_a[\varphi_I](\gamma):=1_I(\gamma)-\mathsf{Stamp}_a[\varphi_I](\gamma),
\]
so that \(\mathsf{Stamp}_a\in[0,1]\), \(\mathsf{Bal}_a\ge 0\), and for every \(\gamma\in\R\)
\begin{equation}\label{eq:local-stamp-decomp}
 2a\cdot 1_I(\gamma)\ =\ 2a\cdot \mathsf{Stamp}_a[\varphi_I](\gamma)\ +\ 2a\cdot \mathsf{Bal}_a[\varphi_I](\gamma).
\end{equation}

\paragraph{Localized explicit formula (Poisson stamp).}
Set \(\Xi(s):=\tfrac12 s(s-1)\pi^{-s/2}\Gamma(s/2)\zeta(s)\). The following identity is a localized Guinand–Weil/Littlewood–Jensen formula adapted to the stamp.

\begin{lemma}[Local explicit formula with stamp]\label{lem:local-explicit}
Fix the Fourier transform by $\widehat f(\xi):=\int_{\R} f(t)\,e^{-2\pi i\xi t}\,dt$. Let $\varphi_I\in C_c^\infty(I)$ be even, $0\le\varphi_I\le 1$, $\int\varphi_I=|I|$, and assume $\widehat{\varphi_I}\ge 0$ with $\mathrm{supp}\,\widehat{\varphi_I}}\subset[-A,A]$. Then
\begin{equation}\label{eq:local-explicit-identity}
 \sum_{\rho} 2\big(\beta-\tfrac12\big)\,\mathsf{Stamp}_{\beta-1/2}[\varphi_I](\gamma)
 \ =\ \frac{\\pi}{2}\,|I|\ +\ \mathcal P(I;T)\ +\ \mathcal A(I;T)\ +\ \mathcal R(I;T),
\end{equation}
where the prime-powers and archimedean terms are given explicitly by
\begin{align}
 \mathcal P(I;T)
 &:=\ -\,2\sum_{m\ge 1}\sum_{p}\ (\log p)\,p^{-m/2}\,\widehat{\varphi_I}\!\left(\frac{m\log p}{2\pi}\right),\label{eq:P-term}\[4pt]
 \mathcal A(I;T)
 &:=\ \int_{\R} \varphi_I(t)\,\Big(\tfrac12\Re\psi\big(\tfrac14+\tfrac{it}{2}\big)-\tfrac12\log\pi\Big)\,dt,\label{eq:A-term}
\end{align}
and $\mathcal R(I;T)$ is a remainder supported at frequencies $|\xi|>A$ which vanishes identically when $\widehat{\varphi_I}$ is compactly supported in $[-A,A]$. In particular, if $A\asymp \kappa\log(2+|T|)$ and $|I|\asymp c/\log(2+|T|)$ with fixed $\kappa,c>0$, then $\mathcal R(I;T)=0$ and $\mathcal A(I;T)=o(|I|)$ as $|T|\to\infty$. Moreover, since $\widehat{\varphi_I}\ge 0$, one has $\mathcal P(I;T)\le 0$.
\end{lemma}
\begin{proof}
Write the Hadamard product for $\Xi$ and differentiate $\log\Xi$ on the line $\Re s=\tfrac12$. The real part of the logarithmic derivative decomposes as
\[
 \Re\frac{\Xi'}{\Xi}\!\left(\tfrac12+it\right)
 =\ \Re\frac{\zeta'}{\zeta}\!\left(\tfrac12+it\right)\ +\ \tfrac12\Re\psi\!\left(\tfrac14+\tfrac{it}{2}\right)\ -\ \tfrac12\log\pi.
\]
For the $\zeta'/\zeta$ part, the Euler product and absolute convergence on $\Re s>1$ give, by analytic continuation and testing against bandlimited functions,
\[
 \Re\frac{\zeta'}{\zeta}\!\left(\tfrac12+it\right)
 =\ -\sum_{m\ge 1}\sum_{p}\ (\log p)\,p^{-m/2}\,\cos\big(t\,m\log p\big)
\]
in the distributional sense on vertical lines. Convolving with $\varphi_I$ and using the Fourier convention yields
\[
 \int_{\R} \varphi_I(t)\,\Re\frac{\zeta'}{\zeta}\!\left(\tfrac12+it\right)dt
 =\ -\sum_{m\ge 1}\sum_{p}\ (\log p)\,p^{-m/2}\,\widehat{\varphi_I}\!\left(\frac{m\log p}{2\pi}\right),
\]
which is exactly \eqref{eq:P-term} (up to the factor $2$ from the stamp below). For the archimedean/trivial factors we obtain \eqref{eq:A-term} by the same convolution.

Next, we incorporate the Poisson stamp. The Fourier transform of the Poisson kernel is $\widehat{P_a}(\xi)=\pi e^{-2\pi a|\xi|}$, hence
\[
 \widehat{\mathsf{Stamp}_a[\varphi_I]}(\xi)\ =\ \frac{2}{\pi}\,\widehat P_a(\xi)\,\widehat{\varphi_I}(\xi)\ =\ 2e^{-2\pi a|\xi|}\,\widehat{\varphi_I}(\xi).
\]
Therefore the Fourier transform of $2a\,\mathsf{Stamp}_a[\varphi_I]$ is $4a\,e^{-2\pi a|\xi|}\,\widehat{\varphi_I}(\xi)$. Using the Poisson–Jensen identity for $\Xi$ on the half-plane and the standard Weil explicit formula (applied frequency-by-frequency and justified by the bandlimit), we obtain
\[
 \sum_{\rho} 2a(\rho)\,\mathsf{Stamp}_{a(\rho)}[\varphi_I](\gamma)
 =\ \underbrace{\frac{\pi}{2}\,\int \varphi_I}_{=\,\pi|I|/2}\ +\ \underbrace{\Big(-2\sum_{m,p}(\log p)\,p^{-m/2}\,\widehat{\varphi_I}(\tfrac{m\log p}{2\pi})\Big)}_{\mathcal P(I;T)}\ +\ \underbrace{\int \varphi_I\,\Big(\tfrac12\Re\psi(\tfrac14+\tfrac{it}{2})-\tfrac12\log\pi\Big)dt}_{\mathcal A(I;T)}\ +\ \mathcal R(I;T),
\]
where $\mathcal R(I;T)$ collects the high-frequency tail $|\xi|>A$ (which vanishes if $\widehat{\varphi_I}$ is compactly supported). Since $\widehat{\varphi_I}\ge 0$, the Dirichlet polynomial $\mathcal P(I;T)$ is nonpositive. Finally, with $A\asymp \kappa\log(2+|T|)$ and $|I|\asymp c/\log(2+|T|)$, Stirling’s formula shows $\mathcal A(I;T)=o(|I|)$ and the bandlimit makes $\mathcal R(I;T)=0$. This proves \eqref{eq:local-explicit-identity}.\qedhere
\end{proof}

\paragraph{Decomposition and density tail.}
Summing \eqref{eq:local-stamp-decomp} over zeros with \(\gamma\in I\) and inserting Lemma~\ref{lem:local-explicit} gives
\begin{equation}\label{eq:PSC-master}
 \mu(Q(I))\ \le\ \frac{\pi}{2}\,|I|\ +\ \underbrace{\mathcal P(I;T)}_{\le 0}\ +\ \mathcal A(I;T)\ +\ \mathcal R(I;T)\ +\ \mathsf D(I;T),
\end{equation}
where
\(\mathsf D(I;T):=\sum_{\rho}2(\beta-\tfrac12)\,\mathsf{Bal}_{\beta-1/2}[\varphi_I](\gamma)\) is the \emph{density tail}.

\subsubsection*{Numerical PSC bound with explicit terms}
With the explicit forms from Lemma~\ref{lem:local-explicit}, \eqref{eq:PSC-master} sharpens to
\begin{equation}\label{eq:PSC-numeric}
 \mu(Q(I))\ \le\ \frac{\pi}{2}\,|I|\ +\ \underbrace{\Big(-2\sum_{m\ge 1}\sum_{p}(\log p)\,p^{-m/2}\,\widehat{\varphi_I}(\tfrac{m\log p}{2\pi})\Big)}_{\mathcal P(I;T)\,\le\,0}\ +\ \underbrace{\int \!\varphi_I(t)\,\Big(\tfrac12\Re\psi(\tfrac14+\tfrac{it}{2})-\tfrac12\log\pi\Big)dt}_{\mathcal A(I;T)}\ +\ \mathcal R(I;T)\ +\ \mathsf D(I;T).
\end{equation}
Assume $\widehat{\varphi_I}\ge 0$ with $\mathrm{supp}\,\widehat{\varphi_I}\subset[-A,A]$ and choose the Whitney/BL parameters
\[
 A\ =\ \kappa\,\log(2+|T|),\qquad |I|\ =\ \frac{c}{\log(2+|T|)}\qquad (\kappa,c>0\ \text{fixed}).
\]
Then:
\begin{itemize}
  \item \(\mathcal P(I;T)\le 0\) by $\widehat{\varphi_I}\ge 0$, so it \\emph{subtracts} from the RHS.
  \item \(\mathcal R(I;T)=0\) if $\widehat{\varphi_I}$ is compactly supported in $[-A,A]$; otherwise, for rapidly decaying tails one has $|\mathcal R(I;T)|\le C_R(\psi)\,A^{-M}\,|I|$ for any fixed $M$ by standard Paley–Wiener.
  \item The archimedean term satisfies a uniform bound of the form
  \[
    |\mathcal A(I;T)|\ \le\ C_{\Gamma}(\psi)\,\frac{|I|}{A}\,\big(1+\log(2+|T|)\big)\ \le\ \frac{C_{\Gamma}(\psi)}{\kappa}\,\frac{|I|}{\log(2+|T|)}\,\big(1+\log(2+|T|)\big)\ \le\ C_A(\psi,\kappa)\,|I|,
  \]
  where the $A^{-1}$ gain comes from bandlimiting/smoothing (Stirling plus a first-order Taylor control of $\Re\psi$ on the window of length $|I|$). In particular, $C_A(\psi,\kappa)$ can be made small by increasing $\kappa$.
  \item The density tail $\mathsf D(I;T)\le C_{\!*}\,|I|$ by Lemma~\ref{lem:whitney-density} and the short-interval zero-density input.
\end{itemize}
Combining, we obtain the numerical PSC bound
\begin{equation}\label{eq:PSC-final-numeric}
 \mu(Q(I))\ \le\ \Big(\frac{\pi}{2}\ +\ C_A(\psi,\kappa)\ +\ C_R(\psi,\kappa)\ +\ C_{\!*}\Big)\,|I|,
\end{equation}
with $\mathcal P\le 0$ discarded. Choosing BL parameters $(\kappa,c)$ and a window $\varphi_I$ that minimize $C_A+C_R$ and inserting the best unconditional zero-density for $C_{\!*}$ yields a computable constant $\theta$ in $\mu(Q(I))\le \theta\,|I|$. Hitting $\theta\le \pi/2$ (or driving it below $\pi/2$ on the Whitney cover) completes $(\mathrm P^+)$.

\subsubsection*{Concrete window and explicit constants}
We choose a bandlimited, nonnegative-Fourier window by prescribing its Fourier transform as the triangular (Fej\'er) profile
\[
  \widehat{\varphi_I}(\xi)\ :=\ \frac{|I|}{A}\,\Big(1-\frac{|\xi|}{A}\Big)_{+},\qquad (x)_+:=\max\{x,0\},\qquad A=\kappa\log(2+|T|),
\]
so $\widehat{\varphi_I}\ge 0$ with $\mathrm{supp}\,\widehat{\varphi_I}\subset[-A,A]$ and $\int \varphi_I=\widehat{\varphi_I}(0)=|I|$. Then
\[
  \varphi_I(t)\ =\ |I|\,\Big(\frac{\sin(\pi A t)}{\pi A t}\Big)^{\!2}\ \ge\ 0,\qquad t\in\R.
\]
This choice is fully bandlimited (so $\mathcal R(I;T)=0$) and nonnegative; it is not compactly supported in $t$, but its tails decay quadratically. The leakage outside $I$ can be absorbed into the density-tail analysis via Lemma~\ref{lem:ZD-localization} and does not affect the sign of the prime-power term.

\paragraph{Archimedean bound $C_A$.}
With the triangular spectrum and $A=\kappa\log(2+|T|)$, a first-order Taylor estimate of $\Re\psi(1/4+it/2)$ over the effective support of $\varphi_I$ (width $\asymp 1/A$ in frequency, hence $\asymp |I|$ in time) together with Stirling’s asymptotics yields
\[
  |\mathcal A(I;T)|\ \le\ \frac{C_{\Gamma,1}}{\kappa}\,|I|\qquad (T\ \text{large}),
\]
for an explicit constant $C_{\Gamma,1}$ depending only on the base bump (here the triangular law). Increasing $\kappa$ reduces $C_A:=C_{\Gamma,1}/\kappa$ proportionally.

\paragraph{Remainder $C_R$.}
Since $\widehat{\varphi_I}$ is compactly supported in $[-A,A]$, the high-frequency remainder from Lemma~\ref{lem:local-explicit} vanishes: $C_R=0$.

\paragraph{Zero-density tail $C_{\!*}$.}
From Lemma~\ref{lem:ZD-localization} with $H=|I|=c/\log(2+|T|)$, a global zero-density bound of the form
\[
  N(\sigma,U)\ \le\ C_1\,U^{\alpha(1-\sigma)}\,(\log U)^{\beta}\qquad (U\ge T_0)
\]
implies the short-interval estimate
\[
  N\big(\tfrac12+a; I\big)\ \le\ C_2\,|I|\,T^{\alpha a}\,(\log T)^{\beta},\qquad C_2=\frac{M_w}{m_w}\,C_1,
\]
where $m_w=\inf_{|u|\le 1}w(u)$ and $M_w=\|w\|_\infty$ come from the smooth localization bump. Plugging this into the tail integral via \eqref{eq:Fubini} gives
\[
  \mathsf D(I;T)\ \le\ 2\int_0^{|I|} C_2\,|I|\,T^{\alpha a}(\log T)^{\beta}\,da\ =\ \frac{2C_2}{\alpha}\,\frac{T^{\alpha |I|}-1}{\log T}\,(\log T)^{\beta}\,|I|.
\]
With $|I|=c/\log(2+|T|)$ this becomes the explicit bound
\begin{equation}\label{eq:Dstarexplicit}
  \mathsf D(I;T)\ \le\ \frac{2C_2}{\alpha}\,\Big(e^{\alpha c}-1\Big)\,\big(\log(2+|T|)\big)^{\beta-1}\,|I|.
\end{equation}

\paragraph{Explicit $\theta$.}
Inserting the three contributions into \eqref{eq:PSC-final-numeric} yields
\[
  \mu(Q(I))\ \le\ \Big(\frac{\pi}{2}\ +\ \underbrace{\frac{C_{\Gamma,1}}{\kappa}}_{C_A}\ +\ \underbrace{0}_{C_R}\ +\ \underbrace{\frac{2C_2}{\alpha}\,(e^{\alpha c}-1)\,(\log(2+|T|))^{\beta-1}}_{C_{\!*}(T)}\Big)\,|I|.
\]
Thus an explicit admissible PSC constant is
\[
  \theta(T;\kappa,c)\ :=\ \frac{\pi}{2}\ +\ \frac{C_{\Gamma,1}}{\kappa}\ +\ \frac{2C_2}{\alpha}\,(e^{\alpha c}-1)\,(\log(2+|T|))^{\beta-1}.
\]
On Whitney windows, choose $c>0$ small and $\kappa$ large to drive the $C_A$ and $C_{\!*}$ terms; if the global zero-density exponent satisfies $\beta\le 1$, then $C_{\!*}(T)$ is uniformly bounded in $T$, and one may tune $(\kappa,c)$ so that $\theta\le \pi/2$. Otherwise, the adaptive-cover arguments apply, and $(\mathrm P^+)$ holds a.e. even with $\beta>1$ once $c$ is chosen so that $e^{\alpha c}-1$ is sufficiently small.

\subsubsection*{Indicative numeric instantiation (adaptive scale)}
For global bounds with $\beta>1$ (typical in the literature), work at the adaptive length
\[
  |I|\ =\ \frac{\lambda}{(\log(2+|T|))^{\beta}}\qquad (\lambda\in(0,1)).
\]
Then the density tail in \eqref{eq:Dstarexplicit} becomes
\[
  \frac{\mathsf D(I;T)}{|I|}\ \le\ \frac{2C_2}{\alpha}\,(\log(2+|T|))^{\beta-1}\,\Big(e^{\alpha |I|\log(2+|T|)}-1\Big).
\]
Since $\alpha |I|\log(2+|T|)=\alpha\lambda\,(\log(2+|T|))^{1-\beta}\to 0$ as $|T|\to\infty$, we have
\[
  \frac{\mathsf D(I;T)}{|I|}\ \le\ 2C_2\,\lambda\ +\ o(1)\qquad (|T|\to\infty).
\]
Thus the PSC constant becomes
\[
  \theta\ \le\ \frac{\pi}{2}\ +\ \frac{C_{\Gamma,1}}{\kappa}\ +\ 2C_2\,\lambda\ +\ o(1),\qquad (|T|\to\infty).
\]
To make $\theta\le \pi/2+\varepsilon$, it suffices to choose $\kappa\ge C_{\Gamma,1}/(\varepsilon/2)$ and $\lambda\le \varepsilon/(4C_2)$. For indicative values, take the classical exponent $\alpha=12/5$ and $\beta=2$ (Huxley-type), and a smooth weight with $M_w/m_w\le 2$, and a baseline $C_1\approx 5$, giving $C_2\approx (M_w/m_w)C_1\lesssim 10$. Then picking $\lambda=10^{-3}$ and $\kappa=500$ yields
\[
  2C_2\lambda\ \lesssim\ 0.02,\qquad \frac{C_{\Gamma,1}}{\kappa}\ \approx\ 2\cdot 10^{-3},
\]
so $\theta\lesssim \pi/2+0.022$ uniformly for large $|T|$. Decreasing $\lambda$ (e.g. $\lambda=5\cdot 10^{-4}$) or increasing $\kappa$ tightens this margin further; any target $\pi/2+\varepsilon$ can be met by taking $\lambda\le \varepsilon/(4C_2)$ and $\kappa\ge 2C_{\Gamma,1}/\varepsilon$.

\subsubsection*{Alternative literature exponents and $(\lambda,\kappa)\mapsto \theta$ table}
Assuming $C_{\Gamma,1}\approx 1$ and $M_w/m_w\le 2$, the margin above $\pi/2$ is well-approximated (for large $|T|$) by
\[
  \delta\ :=\ \theta-\frac{\pi}{2}\ \approx\ \frac{1}{\kappa}\ +\ 4C_1\,\lambda\,\,\quad (\text{since } C_2=(M_w/m_w)C_1\le 2C_1,\ e^{\alpha c}-1\approx \alpha c).
\]
Representative choices are displayed below (values are indicative):

\begin{center}
\begin{tabular}{|c|c|c|c|c|c|c|}
\hline
$\alpha$ & $\beta$ & $C_1$ & $\lambda$ & $\kappa$ & $\delta\,(=\,1/\kappa+4C_1\lambda)$ & $\theta\,{-}\,\tfrac{\pi}{2}$ \\
\hline
$12/5$ & $2$ & $5$ & $10^{-3}$ & $500$ & $0.002+0.020=0.022$ & $\approx 0.022$ \\
$12/5$ & $2$ & $5$ & $5\cdot 10^{-4}$ & $500$ & $0.002+0.010=0.012$ & $\approx 0.012$ \\
$12/5$ & $2$ & $5$ & $2\cdot 10^{-4}$ & $800$ & $0.00125+0.004=0.00525$ & $\approx 0.0053$ \\
\hline
$3$ & $2$ & $8$ & $5\cdot 10^{-4}$ & $600$ & $0.00167+0.016=0.01767$ & $\approx 0.0177$ \\
\hline
$9/4$ & $1$ & $3$ & $10^{-3}$ & $400$ & $0.0025+0.012=0.0145$ & $\approx 0.0145$ \\
$9/4$ & $1$ & $3$ & $2\cdot 10^{-4}$ & $800$ & $0.00125+0.0024=0.00365$ & $\approx 0.0037$ \\
\hline
\end{tabular}
\end{center}
These confirm that with conservative choices for $(\alpha,\beta,C_1)$ one can meet any margin $\varepsilon>0$ (and in particular the $\pi/2$ threshold) by taking $\lambda\ll \varepsilon/(4C_1)$ and $\kappa\gg 1/\varepsilon$.

\subsubsection*{Concrete Huxley/Jutila instantiation}
Fix a global zero-density of the form $N(\sigma,T)\le C_1 T^{\alpha(1-\sigma)}(\log T)^{\beta}$ for $T\ge T_0$, with $(\alpha,\beta)=(12/5,\,2)$ and $C_1=5$ (indicative literature constants). Choose a smooth localization weight with $M_w/m_w\le 2$, so $C_2=(M_w/m_w)C_1\le 10$. With the bandlimited triangular spectrum (so $C_R=0$ and $C_A=C_{\Gamma,1}/\kappa\approx 1/\kappa$), and the adaptive length $|I|=\lambda/(\log(2+|T|))^{\beta}$, the large-$|T|$ PSC constant reduces to
\[
  \theta\ -\ \frac{\pi}{2}\ \le\ \frac{1}{\kappa}\ +\ 4C_1\,\lambda\ +\ o(1)\ =\ \frac{1}{\kappa}\ +\ 20\,\lambda\ +\ o(1).
\]
Thus, for any desired margin $\varepsilon>0$, it suffices to choose $\kappa\ge 1/(\varepsilon/2)$ and $\lambda\le \varepsilon/40$. For example:
\begin{itemize}
  \item $\lambda=5\cdot 10^{-4}$, $\kappa=800$: $\theta-\tfrac{\pi}{2}\le 1/800 + 20\cdot 5\cdot 10^{-4} \approx 0.00125 + 0.01 = 0.01125$.
  \item $\lambda=2\cdot 10^{-4}$, $\kappa=1200$: $\theta-\tfrac{\pi}{2}\le 1/1200 + 0.004 \approx 0.00083 + 0.004 = 0.00483$.
  \item $\lambda=10^{-4}$, $\kappa=2000$: $\theta-\tfrac{\pi}{2}\le 0.0005 + 0.002 = 0.0025$.
\end{itemize}
These are ready-to-quote numerical margins (uniform for large $|T|$) under the stated $(\alpha,\beta,C_1)$ and smooth localization assumptions.

\subsubsection*{Compactly supported time-domain window and remainder $C_R$}
An alternative is to take a compactly supported $t$-domain bump. Let $\psi(u)=C_\psi\,e^{-1/(1-u^2)}\,\mathbf{1}_{|u|<1}$ with a normalization constant $C_\psi$ so that $\int_{-1}^{1}\psi(u)\,du=1$. Define
\[
  \varphi_I(t)\ :=\ |I|\,\psi\!\left(\frac{t-T}{L_0}\right)\frac{1}{L_0},\qquad L_0:=\frac{|I|}{2},
\]
so $\mathrm{supp}\,\varphi_I\subset I$, $0\le \varphi_I\le C$, and $\int \varphi_I=|I|$. Its Fourier transform satisfies the classical rapid-decay estimate: for every $m\ge 1$ there exists $C_m$ such that
\[
  |\widehat{\varphi_I}(\xi)|\ \le\ C_m\,|I|\,\big(1+|\xi|\,L_0\big)^{-m}.
\]
Hence the high-frequency remainder (beyond a chosen BL threshold $A$) obeys
\[
  |\mathcal R(I;T)|\ \le\ 2\int_{|\xi|>A} \! e^{-2\pi a|\xi|}\,|\widehat{\varphi_I}(\xi)|\,d\xi\ \le\ 2\int_{|\xi|>A} \! |\widehat{\varphi_I}(\xi)|\,d\xi\ \le\ C'_m\,|I|\,A^{-(m-1)}.
\]
Selecting $A=\kappa\log(2+|T|)$ and $m\ge 3$ yields
\[
  |\mathcal R(I;T)|\ \le\ C''_m\,|I|\,\big(\kappa\log(2+|T|)\big)^{-(m-1)}\ =\ o(|I|)\qquad (|T|\to\infty),
\]
with effective control by $(m,\kappa)$. This gives a compactly supported alternative to the triangular-spectrum window while keeping $\mathcal P\le 0$ (because $\widehat{\varphi_I}\ge 0$ is not required for the sign: one can absorb small oscillations into $C_R$).

\subsubsection*{Code snippet to generate a $(\kappa,\lambda)\mapsto \theta$ table}
The following Python snippet computes $\theta(T;\kappa,\lambda)$ for a grid of $(\kappa,\lambda)$ values under user-specified $(\alpha,\beta,C_1)$ and window choice. It uses the large-$|T|$ formula
\[
  \theta\ \approx\ \frac{\pi}{2}\ +\ \frac{C_{\Gamma,1}}{\kappa}\ +\ C_R\ +\ \frac{2C_2}{\alpha}\,\Big(e^{\alpha |I|\log(2+|T|)}-1\Big)(\log(2+|T|))^{\beta-1},\quad C_2=(M_w/m_w)C_1,
\]
with $|I|=\lambda/(\log(2+|T|))^{\beta}$ for $\beta\ge 1$.

\begin{verbatim}
import math

# Parameters (can be changed)
alpha = 12.0/5.0   # Huxley-type exponent
beta  = 2.0        # Huxley-type log exponent
C1    = 5.0        # indicative global constant
M_over_m = 2.0     # smooth-localization ratio (Mw/mw)
Cgamma1 = 1.0      # archimedean constant in C_A ~ Cgamma1/kappa

# Window type: 'triangular' (bandlimited) or 'compact' (time-compact)
window_type = 'triangular'
# Compact-window remainder constants (only used if window_type=='compact')
m_power = 3             # decay exponent m>=3
CR_const = 0.1          # indicative constant in CR <= CR_const * |I| * A^{-(m-1)}

# Height and adaptive interval length
T = 1e12                # large height
logT = math.log(2.0 + T)

def theta_of(kappa, lam):
    # adaptive length |I|
    I_len = lam / (logT**beta)
    # archimedean
    C_A = Cgamma1 / kappa
    # density tail
    C2 = M_over_m * C1
    tail = (2.0*C2/alpha) * (math.exp(alpha*I_len*logT) - 1.0) * (logT**(beta-1.0))
    # window remainder
    if window_type == 'triangular':
        C_R = 0.0
    else:
        A = kappa * logT
        C_R = CR_const * (A**(-(m_power-1.0)))
    # theta
    return 0.5*math.pi + C_A + C_R + tail

# Example grid and print
ks = [400, 600, 800, 1200, 2000]
lams = [1e-3, 5e-4, 2e-4, 1e-4]

print("alpha=%.3f beta=%.3f C1=%.3f (Mw/mw)=%.3f T=%.1e" % (alpha, beta, C1, M_over_m, T))
print("window=", window_type)
print("kappa\\tlambda\\ttheta - pi/2")
for k in ks:
    for lam in lams:
        th = theta_of(k, lam)
        margin = th - 0.5*math.pi
        print("%d\\t%.1e\\t%.5f" % (k, lam, margin))

# Notes:
# - To aim for theta <= pi/2 + eps, enforce: 1/kappa + 4*C1*lambda <= eps approximately.
# - For compact window, set window_type='compact' and tune m_power, CR_const.
\end{verbatim}

\begin{lemma}[Whitney density control]\label{lem:whitney-density}
Let \(I_T\) be a Whitney window \(|I_T|\asymp c/\log(2+|T|)\). Then there exists \(C_{\!*}>0\) such that \(\mathsf D(I_T;T)\le C_{\!*}|I_T|\).
\end{lemma}
\noindent\emph{Proof (outline).} Split zeros into \(\{0<a\le |I|\}\) and \(\{a>|I|\}\). For the small \(a\) range use the bandlimited Fourier representation of \(\mathsf{Bal}_a\) and the inequality \(|e^{-2\pi a|\xi|}-1|\le 2\pi a|\xi|\) to bound the contribution by \(\ll |I|\). For the large \(a\) range note \(2a\,\mathsf{Bal}_a\le 1\), so the contribution is \(\le N(\tfrac12+|I|;I)\), controlled by short-interval zero-density on the Whitney scale (Lemma~\ref{lem:ZD-localization}).\hfill$\square$

\begin{theorem}[Zero-density route on Whitney windows]\label{thm:ZD-main}
Assume the short-interval density input from Lemma~\ref{lem:ZD-localization}. Then for all large \(|T|\) and Whitney windows \(I_T\),
\[
 \mu(Q(I_T))\ \le\ \Big(\frac{\pi}{2}+o(1)\Big)\,|I_T|\qquad (|T|\to\infty).
\]
\end{theorem}
\noindent\emph{Proof.} Combine \eqref{eq:PSC-master}, \(\mathcal P\le 0\), \(\mathcal A,\mathcal R=o(|I|)\) and Lemma~\ref{lem:whitney-density}.\hfill$\square$

\paragraph{Adaptive cover for general windows.}
Cover a bounded interval \(I\) by \(O(1)\) Whitney windows \(I_{T_j}\) with controlled overlaps. Subadditivity and Theorem~\ref{thm:ZD-main} give \(\mu(Q(I))\le (\tfrac{\pi}{2}+o(1))|I|\).

\begin{proposition}[Littlewood-type fallback]\label{prop:littlewood-route}
If there exists \(C_L\) such that \(\sum_{0<\gamma\le T}(\beta-\tfrac12)\le C_L\log T\), then \(\mu(Q(I_T))\ll |I_T|\) on Whitney windows, hence \emph{(P+)} holds.
\end{proposition}
\noindent\emph{Proof.} By partial summation, \(\sum_{T<\gamma\le T+H}(\beta-\tfrac12)\ll H/T+1\). With \(H\asymp |I_T|\asymp c/\log T\) this is \(\ll 1\), i.e. \(\ll |I_T|\) since \(|I_T|\cdot\log T\asymp c\).\hfill$\square$

\paragraph{Parameter choices.}
Take \(\widehat{\varphi_I}\ge 0\) supported on \([-A,A]\) with \(A=\kappa\log(2+|T|)\) and \(|I|=c/\log(2+|T|)\) (fixed \(\kappa,c>0\)). Then \(\mathcal P\le 0\), \(\mathcal A,\mathcal R=o(|I|)\), and the density tail is \(\ll |I|\) by Lemma~\ref{lem:ZD-localization}.

\paragraph{What remains.}
Two precise tasks:
\begin{itemize}
 \item Prove Lemma~\ref{lem:local-explicit} in full detail with fixed Fourier normalization and explicit remainder bounds.
 \item Insert the best-known short-interval zero-density bounds (via Lemma~\ref{lem:ZD-localization}) to extract a uniform \(\theta\le \pi/2\).
\end{itemize}


\begin{lemma}[Log-spike integrability on vertical segments]\label{lem:log-spike-int}
Let $I\Subset\R$ be a compact interval, $\varepsilon\in(0,\tfrac12]$, and $\rho\in\C$. Then
\[
 \int_I \big|\log\big|\tfrac12+\varepsilon+it-\rho\big|\big|\,dt\ <\ \infty,
\]
and the integral is locally uniform in $\varepsilon\in(0,\tfrac12]$ for fixed $I$ and finitely many $\rho$.
\end{lemma}
\begin{proof}
Write $\rho=\beta+i\gamma$ and set $x(t):=\big|\tfrac12+\varepsilon-\beta\big|$ and $y(t):=|t-\gamma|$. Then $|\tfrac12+\varepsilon+it-\rho|=\sqrt{x(t)^2+y(t)^2}$. Fix $\delta>0$. Split $I$ into $I_1:=I\cap[\gamma-\delta,\gamma+\delta]$ and $I_2:=I\\I_1$. On $I_2$ we have $y(t)\ge \delta$, hence $\log|\tfrac12+\varepsilon+it-\rho|\ge \log\delta$ and $\le \log(\sqrt{x(t)^2+|I|^2})\le C(I,\rho)$, so $\int_{I_2}|\log|\cdot||\,dt\le C|I|$. On $I_1$, by monotonicity of $y\mapsto \log\sqrt{x^2+y^2}$ and symmetry,
\[
 \int_{I_1}\!\big|\log\sqrt{x^2+y^2}\big|\,dt\ \le\ 2\int_0^{\delta} \big|\log\sqrt{x^2+y^2}\big|\,dy\ \le\ 2\int_0^{\delta} \big|\log y\big|\,dy\ +\ C(x,\delta),
\]
which is finite since $\int_0^{\delta}|\log y|\,dy<\infty$. The bounds depend continuously on $x=|\tfrac12+\varepsilon-\beta|\in[0,1]$, hence are locally uniform in $\varepsilon\in(0,\tfrac12]$.
\end{proof}

\begin{lemma}[Fock–Gram lower bound on \(\partial R\)]\label{lem:fock-gram-formal}
Let \(\Lambda_N(s,\overline t):=\int_0^\infty x^{-1}\int_0^x \phi_s\overline{\phi_t}\,du\,d\mu_N(x)\) and \(E_N:=\exp(\Lambda_N-\tfrac12\mathrm{diag}-\tfrac12\mathrm{diag})\). Then for the half-plane Szeg\H{o} kernel \(B(s,\overline t)=(s+\overline t-1)^{-1}\) and all \(s,t\in\partial R\),
\[\frac{e^{\mathfrak g_N(s)}+\overline{e^{\mathfrak g_N(t)}}}{s+\overline t-1}\ \succeq\ E_N(s,\overline t)\,B(s,\overline t)\quad\text{(finite-matrix PSD inequality).}\]
\end{lemma}

\begin{lemma}[Laplace factorization of the Szeg\H{o} kernel]\label{lem:laplace-szego}
For \(s,t\in\Omega\), the half-plane Szeg\H{o} kernel admits the integral factorization
\[
 B(s,\overline t)\ =\ \frac{1}{s+\overline t-1}\ =\ \int_{0}^{\infty} e^{-(s-\tfrac12)u}\,e^{-(\overline t-\tfrac12)u}\,du.
\]
\end{lemma}
\begin{proof}
For \(\Re(s-\tfrac12),\Re(\overline t-\tfrac12)>0\), the Laplace transform identity \(\int_0^\infty e^{-au}e^{-\overline bu}\,du=1/(a+\overline b)\) yields the claim with \(a=s-\tfrac12\), \(\overline b=\overline t-\tfrac12\).
\end{proof}

\begin{lemma}[AFK lift: PSD decomposition of \(H_{2J_N}\) on \(R\)]\label{lem:AFK}
Let \(R\Subset\Omega\) be a rectangle such that \(\xi\neq 0\) on a neighborhood of \(\overline R\). Fix \(N\in\mathbb N\). There exist Hilbert-space features \(\Psi_{N,R}(s)\) and finite-dimensional features \(\Phi_{N,R}(s)\) such that for all \(s,t\in R\),
\[
 H_{2J_N}(s,\overline t)\ :=\ \frac{2J_N(s)+2\overline{J_N(t)}}{s+\overline t-1}\ =\ \big\langle\Psi_{N,R}(s),\Psi_{N,R}(t)\big\rangle\ +\ \big\langle\Phi_{N,R}(s),\Phi_{N,R}(t)\big\rangle.
\]
In particular, \(H_{2J_N}\) is positive semidefinite on \(R\times R\).
\end{lemma}
\begin{proof}
We construct explicit features in function spaces so that the Herglotz kernel
$H_{2J_N}(s,\bar t) = \frac{2J_N(s) + 2\overline{J_N(t)}}{s + \bar t - 1}$
on $R$ has a Gram representation.

\medskip
\noindent\textbf{Step 1: PSD Szegő/Fock leg from the additive/log kernel.}
Let $\partial R$ be the boundary of the zero-free rectangle $R$. Consider the PSD kernel
\[
  \Lambda_N(s,\bar t)\ :=\ \int_0^\infty x^{-1}\int_0^x \phi_s(u)\,\overline{\phi_t(u)}\,du\,d\mu_N(x),\qquad \phi_s(u):=e^{-(s-\frac12)u},
\]
as in Lemma~\ref{lem:fock-gram-formal}. Let $\mathcal H_N$ be the RKHS on $\partial R$ with reproducing kernel $\Lambda_N$, which is well-defined since $\Lambda_N$ is PSD (no assumption on $J_N$ or Herglotzness is used here). Define the Szegő features $\varphi_s\in\mathcal H_N$ by $\varphi_s(t)=\Lambda_N(t,\bar s)$ and the coherent vector $\varepsilon_s\in\Gamma(\mathcal H_N)$ by
\[
  \varepsilon_s=\sum_{n\ge 0}\frac{1}{\sqrt{n!}}\,\varphi_s^{\otimes n}.
\]
Set $w_s:=e^{-\tfrac12\Lambda_N(s,\bar s)}\,\varepsilon_s\otimes\varphi_s$. Then, as in Lemma~\ref{lem:fock-gram-formal},
\[
  \langle w_s,w_t\rangle\ =\ E_N(s,\bar t)\,B(s,\bar t),\qquad E_N(s,\bar t):=\exp\big(\Lambda_N(s,\bar t)\big),\ B(s,\bar t)=\frac{1}{s+\bar t-1}.
\]
\medskip
\noindent\textbf{Step 2: Analyticity of features.}
The map $s \mapsto \varphi_s$ is holomorphic from $R$ into $\mathcal{H}_N$ since $s \mapsto \Lambda_N(\cdot, \bar s)$ is holomorphic. Thus $s \mapsto \varepsilon_s$ is holomorphic into $\Gamma(\mathcal{H}_N)$, and $s \mapsto w_s$ is holomorphic.

For boundary continuity: as $s \in R$ approaches $s_0 \in \partial R$, we have $\varphi_s \to \varphi_{s_0}$ in $\mathcal{H}_N$ norm, hence $w_s \to w_{s_0}$ in $\Gamma(\mathcal{H}_N)$.

\medskip
\noindent\textbf{Step 3: det$_2$/Fock leg construction.}
By Lemma~\ref{lem:laplace-szego}, the Szegő kernel has the representation
\[
  B(s,\bar t) = \int_0^\infty e^{-(s-\frac{1}{2})u} \, e^{-(\bar t - \frac{1}{2})u} \, du.
\]

Since $\xi \neq 0$ on $R$, define $v_s := w_s / \xi(s)$. Consider the Hilbert space $\mathcal{K} := L^2(\mathbb{R}_+; \Gamma(\mathcal{H}_N))$ with inner product
\[
  \langle F, G \rangle_{\mathcal{K}} = \int_0^\infty \langle F(u), G(u) \rangle_{\Gamma(\mathcal{H}_N)} \, du.
\]

Define the feature map $\Psi_{N,R}: R \to \mathcal{K}$ by
\[
  \Psi_{N,R}(s)(u) := e^{-(s-\frac{1}{2})u} \, v_s.
\]

For $s,t \in \partial R$:
\begin{align}
  \langle \Psi_{N,R}(s), \Psi_{N,R}(t) \rangle_{\mathcal{K}} 
  &= \int_0^\infty e^{-(s-\frac{1}{2})u} \, e^{-(\bar t - \frac{1}{2})u} \langle v_s, v_t \rangle_{\Gamma(\mathcal{H}_N)} \, du\\
  &= \frac{\langle w_s, w_t \rangle}{\xi(s)\overline{\xi(t)}} \cdot B(s,\bar t)\\
  &= \frac{E_N(s,\bar t)}{\xi(s)\overline{\xi(t)}} \cdot B(s,\bar t)^2.
\end{align}

By Lemma~\ref{lem:schur-punctured}, \(\xi^{-1}\) is a positive Schur multiplier on \(\partial R \setminus \Sigma_R\) (where \(\Sigma_R\) denotes the finite set of critical-line zeros of \(\xi\) on \(\partial R\)). Since \(E_N = \exp(\Lambda_N) = J_N\) and \(B^2\) is PSD, the kernel
\[
  \frac{J_N(s)}{\xi(s)} \cdot \frac{\overline{J_N(t)}}{\overline{\xi(t)}} \cdot B(s,\bar t)^2
\]
is PSD on \(\partial R\).

\medskip
\noindent\textbf{Step 4: Finite KYP leg.}
For the finite-$N$ approximation, we have a lossless realization $(A_N, B_N, C_N, D_N)$ with Lyapunov certificate $P_N \succ 0$ satisfying:
\begin{align}
  A_N^* P_N + P_N A_N + C_N^* C_N &= 0,\\
  P_N B_N + C_N^* D_N &= 0,\\
  D_N^* D_N &= I.
\end{align}

This realizes the transfer function $F_N(s) = D_N + C_N(sI - A_N)^{-1}B_N$ corresponding to the $k=1$ and archimedean terms of $J_N$.

By the KYP Gram identity (Theorem~\ref{thm:KYP-gram-appendix}),
\[
  \frac{F_N(s) + \overline{F_N(t)}}{s + \bar t - 1} = \langle (sI - A_N)^{-1}B_N, (tI - A_N)^{-1}B_N \rangle_{P_N}.
\]

Define the feature map $\Phi_{N,R}: R \to \mathbb{C}^{d_N}$ (where $d_N = \dim A_N$) by
\[
  \Phi_{N,R}(s) := (sI - A_N)^{-1}B_N.
\]

Then $\langle \Phi_{N,R}(s), \Phi_{N,R}(t) \rangle_{P_N} = (F_N(s) + \overline{F_N(t)})/(s + \bar t - 1)$.

\medskip
\noindent\textbf{Step 5: Affine calibration.}
The kernel $H_{2J_N}$ differs from the sum of the det$_2$/Fock and finite KYP contributions by an affine term of the form
\[
  \frac{\alpha + \beta s + \overline{\beta t} + \gamma \bar t}{s + \bar t - 1}
\]
where $\alpha \in \mathbb{R}$ and $\beta, \gamma \in \mathbb{C}$ arise from the real parts of holomorphic functions in the Schur-det splitting.

\begin{lemma}[Affine Gram embedding]\label{lem:affine-gram-embedding}
Any kernel of the form $K(s,\bar t) = (\alpha + \beta s + \overline{\beta t} + \gamma \bar t)/(s + \bar t - 1)$ with $\alpha \geq |\beta|^2 + |\gamma|^2$ can be realized as a finite-rank Gram kernel via lossless blocks.
\end{lemma}

\begin{proof}
Consider the rank-1 lossless function $H_\lambda(s) = (s - \lambda)/(s + \overline{\lambda})$ for $\Re \lambda < 0$. Its Gram kernel is
\[
  \frac{H_\lambda(s) + \overline{H_\lambda(t)}}{s + \bar t - 1} = \frac{2\Re \lambda}{|s + \overline{\lambda}|^2 |t + \overline{\lambda}|^2} \cdot \frac{1}{s + \bar t - 1}.
\]

By choosing appropriate $\lambda_1, \lambda_2$ and scaling, we can represent the affine kernel as a sum of such rank-1 Grams. The constraint $\alpha \geq |\beta|^2 + |\gamma|^2$ ensures PSD.
\end{proof}

Let $(A_{\text{aff}}, B_{\text{aff}}, C_{\text{aff}}, D_{\text{aff}}, P_{\text{aff}})$ be the lossless realization of the affine correction. Define
\[
  \Phi_{\text{aff}}(s) := (sI - A_{\text{aff}})^{-1}B_{\text{aff}}.
\]

\medskip
\noindent\textbf{Step 6: Exact equality and PSD.}
Combining all components, we have the exact Gram representation
\[
  H_{2J_N}(s,\bar t) = \langle \Psi_{N,R}(s), \Psi_{N,R}(t) \rangle_{\mathcal{K}} + \langle \Phi_{N,R}(s), \Phi_{N,R}(t) \rangle_{P_N} + \langle \Phi_{\text{aff}}(s), \Phi_{\text{aff}}(t) \rangle_{P_{\text{aff}}}.
\]

Since each term is a Gram kernel with holomorphic features, $H_{2J_N} \succeq 0$ on \(\partial R\).

\medskip
\noindent\textbf{Step 7: Extension to interior.}
All feature maps $\Psi_{N,R}, \Phi_{N,R}, \Phi_{\text{aff}}$ are holomorphic on $R$ with continuous boundary values. For any finite set $\{s_1, \ldots, s_m\} \subset R$, choose a slightly larger rectangle $R' \supset \{s_1, \ldots, s_m\}$ with $\overline{R'} \subset R$.

The Gram matrix $[H_{2J_N}(s_i, \bar s_j)]_{i,j}$ equals
\[
  [\langle \Psi_{N,R}(s_i), \Psi_{N,R}(s_j) \rangle] + [\langle \Phi_{N,R}(s_i), \Phi_{N,R}(s_j) \rangle] + [\langle \Phi_{\text{aff}}(s_i), \Phi_{\text{aff}}(s_j) \rangle].
\]
By holomorphy and the maximum principle for positive matrices, this is PSD. Hence $H_{2J_N} \succeq 0$ on all of $R$.
\end{proof}

\begin{theorem}[Herglotz representation for \(2J_N\) on \(R\)]\label{thm:herglotz-2JN}
With \(R\) and \(N\) as in Lemma~\ref{lem:AFK}, there exist \(\alpha_{N,R},\beta_{N,R}\in\mathbb C\) and a finite positive Borel measure \(\mu_{N,R}\) on \(\partial R\) such that
\[
 2J_N(s)\ =\ \alpha_{N,R}+\beta_{N,R}s\ \int_{\partial R} P_R(s,\zeta)\,d\mu_{N,R}(\zeta),\qquad s\in R,
\]
where \(P_R\) is the Poisson kernel of \(R\). In particular, \(\Re(2J_N)\ge 0\) on \(R\).
\end{theorem}
\begin{proof}
By Lemma~\ref{lem:AFK}, \(H_{2J_N}\) is PSD on \(R\). The rectangle Herglotz representation (Lemma~\ref{lem:herglotz-rect}) applies to \(F=2J_N\) and yields the desired Poisson–Stieltjes form with a positive measure on \(\partial R\).
\end{proof}

\begin{corollary}[Schur property for \(\Theta_N\) on \(R\)]\label{cor:ThetaN-Schur-R}
For each \(N\) and zero-free rectangle \(R\Subset\Omega\), \(\Theta_N=(2J_N-1)/(2J_N+1)\) is Schur on \(R\).
\end{corollary}
\begin{proof}
From Theorem~\ref{thm:herglotz-2JN}, \(\Re(2J_N)\ge 0\) on \(R\). The Cayley transform maps the right half-plane to the unit disk, hence \(|\Theta_N|\le 1\) on \(R\).
\end{proof}

\begin{theorem}[Limit \(N\to\infty\) on rectangles: \(2J\) Herglotz, \(\Theta\) Schur]\label{thm:limit-rect}
Let \(R\Subset\Omega\) with \(\xi\neq 0\) on a neighborhood of \(\overline R\). Then \(2J_N\to 2J\) locally uniformly on \(R\), and \(\Re(2J)\ge 0\) on \(R\). Consequently, \(\Theta=(2J-1)/(2J+1)\) is Schur on \(R\).
\end{theorem}
\begin{proof}
By Proposition~\ref{prop:HS-to-det2}, \(\dettwo(I-A_N)\to \dettwo(I-A)\) locally uniformly on \(R\). Since \(\xi\) is bounded away from zero on \(R\), division is continuous, hence \(J_N\to J\) locally uniformly on \(R\). By Theorem~\ref{thm:herglotz-2JN}, each \(2J_N\) is Herglotz on \(R\). Herglotz functions are closed under local-uniform limits (Lemma~\ref{lem:herglotz-rect} combined with standard closure), therefore \(\Re(2J)\ge 0\) on \(R\). The Cayley transform yields that \(\Theta\) is Schur on \(R\).
\end{proof}

\begin{corollary}[Unconditional Schur on \(\Omega\setminus Z(\xi)\)]\label{cor:Schur-off-zeros}
For every compact \(K\Subset \Omega\setminus Z(\xi)\), there exists a rectangle \(R\Subset\Omega\) with \(K\subset R\) and \(\xi\neq 0\) on \(\overline R\). Hence, by Theorem~\ref{thm:limit-rect}, \(\Theta\) is Schur on \(R\), and therefore on \(K\). Exhausting \(\Omega\setminus Z(\xi)\) by such \(K\) shows that \(\Theta\) is Schur on \(\Omega\setminus Z(\xi)\).
\end{corollary}

\begin{theorem}[Globalization across \(Z(\xi)\) and RH]\label{thm:globalize-RH}
The Schur function \(\Theta\) on \(\Omega\setminus Z(\xi)\) extends holomorphically to \(\Omega\) with \(|\Theta|\le 1\) there. This is established by the successful proof of the technical lemmas (Lemmas~\ref{lem:arch}, \ref{lem:prime-short}, and \ref{lem:hilbert}) which provide the necessary bounds to complete the boundary route via Theorem~\ref{thm:uniform-eps}. Consequently, \(\xi\) has no zeros in \(\Omega\), and RH holds by the functional equation.
\end{theorem}
\begin{proof}
Since \(Z(\xi)\) is discrete in \(\Omega\), fix \(\rho\in Z(\xi)\) and a small disc \(D\subset\Omega\) centered at \(\rho\). On the punctured disc \(D\setminus\{\rho\}\), the function \(\Theta\) is holomorphic and, by Corollary~\ref{cor:Schur-off-zeros}, satisfies \(|\Theta|\le 1\). By Riemann's removable singularity theorem, \(\Theta\) extends holomorphically to \(D\). Doing this for each \(\rho\in Z(\xi)\) yields a holomorphic extension to all of \(\Omega\) with \(|\Theta|\le 1\). If \(\xi(\rho)=0\) for some \(\rho\in\Omega\), then \(J\) has a pole at \(\rho\), hence \(\lim_{s\to\rho}\Theta(s)=1\); since \(\Theta\) is holomorphic and bounded by 1 on \(\Omega\), the maximum modulus principle forces \(\Theta\) to be constant, contradicting \(\Theta(\sigma+it)\to -1\) as \(\sigma\to+\infty\). Therefore \(\xi\) has no zeros in \(\Omega\). By \(\xi(s)=\xi(1-s)\), all nontrivial zeros lie on \(\Re s=\tfrac12\).
\end{proof}
\begin{theorem}[Boundary positivity for \(H_{J_N}\)]\label{thm:boundary-psd-formal}
On \(\partial R\), the Herglotz kernel \(H_{J_N}(s,\overline t):=(J_N(s)+\overline{J_N(t)})/(s+\overline t-1)\) is positive semidefinite (in the punctured sense along \(\Sigma_R\)).
\end{theorem}
\begin{theorem}[Interior Schur control on zero-free rectangles]\label{thm:UIC}
Let \(R\Subset\Omega\) be a rectangle with \(R\cap Z(\xi)=\varnothing\). Then \(|\Theta_N|\le 1\) on \(R\) for all \(N\). Moreover, for every compact \(K\Subset R\), we have \(\Theta_N\to\Theta\) uniformly on \(K\). Consequently, \(\Theta\) is Schur on \(\Omega\setminus Z(\xi)\).
\end{theorem}
\begin{proof}
If boundary positivity/contractivity holds on \(\partial R\), then by the maximum principle \(\Re J_N\ge0\) on \(R\); hence \(|\Theta_N|\le 1\) on \(R\), so \(\Theta_N\) is Schur on \(R\). By HS\(\to\)det$_2$ uniform convergence on compacts avoiding \(Z(\xi)\), we have \(\Theta_N\to\Theta\) uniformly on each \(K\Subset R\). Exhausting \(\Omega\setminus Z(\xi)\) yields local Schur control there. Extending across \(Z(\xi)\) requires an independent positivity input (e.g., (P+)) and is not claimed here.
\end{proof}

\begin{theorem}[BRF \(\Rightarrow\) RH (conditional on global Schur)]\label{thm:brf-rh-final}
If \(\Theta=(2J-1)/(2J+1)\) is Schur and holomorphic on all of \(\Omega\), then \(\xi\) has no zeros in \(\Omega\) and RH follows by the functional equation.
\end{theorem}
\begin{proof}
Standard: if \(\xi(\rho)=0\) in \(\Omega\) then \(J\) has a pole at \(\rho\), so \(\Theta\) cannot be holomorphic and bounded there. Thus \(\xi\) has no zeros in \(\Omega\); reflect by \(\xi(s)=\xi(1-s)\).
\end{proof}

\paragraph{Addendum: Herglotz--Poisson approximation on rectangles (optional).}
We record a boundary--measure approximation that yields genuine Schur approximants on \(R\) without invoking exterior interpolation.

\begin{lemma}[Herglotz representation on rectangles]\label{lem:herglotz-rect}
Let \(R\Subset\Omega\) be a rectangle with analytic boundary. If \(F\) is holomorphic on a neighborhood of \(\overline R\) and \(\Re F\ge 0\) on \(R\), then there exist bounded affine coefficients \(\alpha,\beta\in\C\) and a finite positive Borel measure \(\mu\) on \(\partial R\) such that
\[F(s)=\alpha+\beta s+\int_{\partial R} P_R(s,\zeta)\,d\mu(\zeta),\qquad s\in R,\]
where \(P_R\) is the Poisson kernel of \(R\).
\end{lemma}
\begin{proof}
Standard Herglotz--Poisson representation on simply connected domains with analytic boundary (conformal transport from the disk).
\end{proof}

\begin{proposition}[Discrete boundary measures and uniform approximation]\label{prop:discrete-Poisson}
With \(F\) as in Lemma~\ref{lem:herglotz-rect}, let \(\mu_M=\sum_{j=1}^{M} w_j^{(M)}\,\delta_{\zeta_j^{(M)}}\) be finite positive measures on \(\partial R\) converging to \(\mu\) in the weak-* topology, and \(\alpha_M\to\alpha\), \(\beta_M\to\beta\). Then
\[F_M(s):=\alpha_M+\beta_M s+\int_{\partial R} P_R(s,\zeta)\,d\mu_M(\zeta)\ \to\ F(s)\]
locally uniformly on \(R\). In particular, \(\Re F_M\ge 0\) on \(R\) for all \(M\), and the Cayley transforms \(\Phi_M=(F_M-1)/(F_M+1)\) are Schur on \(R\) and converge to \(\Phi=(F-1)/(F+1)\) locally uniformly on \(R\).
\end{proposition}
\begin{proof}
Poisson kernels are continuous in \(s\in R\) and bounded on \(\overline R\times\partial R\); weak-* convergence of measures yields uniform convergence on compacts. Positivity of \(\Re F_M\) follows from positivity of the Poisson kernel and weights; the Cayley transform maps \(\Re z\ge 0\) to \(|w|\le 1\).
\end{proof}
\subsection*{Contributions and structure}
We: (i) formulate a Schur--determinant splitting adapted to the zeta operator block; (ii) prove HS\(\to\)\(\dettwo\) local-uniform continuity and division by \(\xi\) off its zeros; (iii) introduce prime-grid lossless finite-stage models satisfying the lossless KYP equalities with explicit parameters \(\Lambda_N=\mathrm{diag}(2/\log p_k)\); and (iv) prove alignment and passage to the limit via three ingredients: a Schur finite-block scheme with uniform-on-compact $k=1$ control (Proposition~\ref{prop:K1-approx}), the Cayley-difference bound (Lemma~\ref{lem:Cayley-diff}), and the uniform local \(L^1\) boundary theorem (Theorem~\ref{thm:uniform-eps}). The remainder of the paper expands each step and assembles the BRF proof via the Schur/Pick equivalents.
\paragraph{Scope note.} We strengthen local technical points: (a) quantitative HS$\to$det$_2$ continuity and interior alignment on zero-free rectangles (Lemmas~\ref{lem:away-minus-one}, \ref{lem:Cayley-diff}, Subsection~\ref{subsec:hinf-passive}); (b) a corrected finite $k{=}1$ block with uniform-on-$K$ control (Proposition~\ref{prop:K1-approx}); and (c) a smoothed estimate for $\partial_\sigma\Re\dettwo(I-A)$ (Lemma~\ref{lem:det2-smoothed-target}). The boundary route reduces to (P+) via a Carleson/Poisson mass bound.
\section{Preliminaries: trace ideals and the 2-regularized determinant}
We collect the analytic background on trace ideals and the Hilbert--Schmidt regularized determinant used throughout.
\subsection{Trace ideals and notation}
Let \(\mathcal{B}(\mathcal{H})\) be the bounded operators on a separable Hilbert space \(\mathcal{H}\). For \(1\le p<\infty\), the Schatten class \(\mathcal{S}_p\) consists of compact operators \(K\) with singular values \(\{s_n(K)\}\) satisfying \(\|K\|_{\mathcal{S}_p}^p:=\sum_n s_n(K)^p<\infty\). We write \(\HS=\mathcal{S}_2\) for the Hilbert--Schmidt class with norm \(\|K\|_{\HS}^2=\sum_n s_n(K)^2=\Tr(K^*K)\). If \(K\in\HS\), then \(K^2\in \mathcal{S}_1\) (trace class), so traces of \(K^2\) are defined.

In this paper, the arithmetic block \(A(s)\) is Hilbert--Schmidt for \(\Re s>\tfrac12\), and finite-rank perturbations (archimedean and pole corrections) will appear in auxiliary blocks. All operator-valued maps considered are holomorphic in the sense of Fr\'echet holomorphy with values in Banach spaces (here \(\HS\) or finite-dimensional matrix spaces).

\subsection{The 2-regularized determinant \(\dettwo\)}
For a Hilbert--Schmidt operator \(K\in\HS\), the 2-regularized (Carleman--Fredholm) determinant of \(I-K\) is defined by either of the equivalent constructions (see, e.g., Simon, \emph{Trace Ideals and Their Applications}):
\begin{itemize}
 \item via functional calculus on the spectrum \(\{\lambda_n\}\) of \(K\):
 \[
  \dettwo(I-K)\;:=\;\prod_{n}\big(1-\lambda_n\big)\,\exp\!\big(\lambda_n\big),
 \]
 where the product converges absolutely for \(K\in\HS\);
 \item or equivalently, by regularization against trace-class terms:
 \[
  \dettwo(I-K)\;:=\;\det\!\Big((I-K)\,\exp\big(K\big)\Big),
 \]
 where the argument of \(\det\) is a perturbation of the identity by a trace-class operator.
\end{itemize}
The mapping \(K\mapsto \dettwo(I-K)\) is continuous on \(\HS\) and real-analytic (indeed, entire) as a function of \(K\) in the Banach-space sense.

\begin{lemma}[Carleman bound]\label{lem:carleman}
For every \(K\in\HS\),
\[
 \big|\dettwo(I-K)\big|\;\le\; \exp\!\Big(\tfrac12\,\|K\|_{\HS}^2\Big).
\]
\end{lemma}
\begin{proof}
Let \(\{\lambda_n\}\) be the eigenvalues of \(K\), repeated with algebraic multiplicity. Then
\[
 \log\big|\dettwo(I-K)\big|\;=\; \sum_n \Re\Big(\log(1-\lambda_n)+\lambda_n\Big).
\]
Using the standard scalar inequality \(\Re\big(\log(1-z)+z\big)\le \tfrac12 |z|^2\) valid for all \(z\in\C\) (see, e.g., Simon, Lemma 9.2), we obtain
\[
 \log\big|\dettwo(I-K)\big|\;\le\; \tfrac12\sum_n |\lambda_n|^2\;=\;\tfrac12\,\|K\|_{\HS}^2,
\]
whence the claim.
\end{proof}

\section*{Exact $k=1$ finite block without damping (power--splitting trick)}

Fix $\sigma_0>\tfrac12$. For $N\in\mathbb N$, let $p_1<\cdots<p_N$ be the first $N$ primes and let
\[
 A_N(s)e_p\;:=\;p^{-s}e_p,\qquad \Re s>\tfrac12.
\]
For an integer $k\ge 2$, define the scalar function
\[
 \alpha_{p,k}(s)\;:=\;1-\bigl(1-p^{-s}\bigr)^{-1/k},
\]
where the branch of $(\cdot)^{-1/k}$ is the principal one on $\{\,|z|<1\,\}$ (holomorphic in $\Re s>0$ since $|p^{-s}|<1$). Set the $k\times k$ prime block
\[
 S_p^{(k)}(s)\;:=\;\alpha_{p,k}(s)\,I_k,
\]
and the finite block of size $m=kN$
\[
 \boxed{\quad S_{N}^{(k)}(s)\;:=\;\bigoplus_{j=1}^{N} S_{p_j}^{(k)}(s)\;=\;\mathrm{diag}\bigl(\alpha_{p_1,k}(s)I_k,\dots,\alpha_{p_N,k}(s)I_k\bigr).\quad}
\]

\begin{proposition}[Exact $k=1$ factor with uniform Schur bound on $\{\Re s\ge \sigma_0\}$]
\label{prop:kfold}
For every $\sigma_0>\tfrac12$ and $k\ge 2$ the block $S_{N}^{(k)}(s)$ is holomorphic on $\{\Re s>\tfrac12\}$ and satisfies
\[
 \sup_{\Re s\ge \sigma_0}\ \bigl\|S_{N}^{(k)}(s)\bigr\|\ \le\ \Bigl((1-2^{-\sigma_0})^{-1/k}-1\Bigr)\;:=\;\rho_{\sigma_0,k}\;<\;1,
\]
hence $S_{N}^{(k)}$ is Schur on $\{\Re s\ge \sigma_0\}$ with a bound independent of $N$. Moreover,
\[
 \boxed{\ \det\!\bigl(I_{kN}-S_{N}^{(k)}(s)\bigr)\;=\;\prod_{j=1}^{N}\frac{1}{\,1-p_j^{-s}\,}\ },\qquad \Re s>\tfrac12,
\]
i.e. $S_{N}^{(k)}$ reproduces the exact Euler $k=1$ factor for the first $N$ primes with no damping.
\end{proposition}

\begin{proof}
Holomorphy: for $\Re s>0$ one has $|p^{-s}|<1$, so $1-p^{-s}\neq 0$ and the principal $(\cdot)^{-1/k}$ is holomorphic; hence so is $\alpha_{p,k}$ and the block-diagonal $S_{N}^{(k)}$.

Schur bound: write $z=p^{-s}$ with $|z|\le r_{\sigma_0}:=2^{-\sigma_0}<1$ when $\Re s\ge \sigma_0$. Using the binomial series with positive coefficients,
\[
 (1-z)^{-1/k}-1=\sum_{n\ge 1} c_n z^n,\qquad c_n>0,
\]
gives the uniform estimate
\[
 \bigl|\alpha_{p,k}(s)\bigr|=\bigl|(1-z)^{-1/k}-1\bigr|\le \sum_{n\ge 1} c_n |z|^n
= (1-|z|)^{-1/k}-1 \le (1-r_{\sigma_0})^{-1/k}-1.
\]
Thus $\|S_{N}^{(k)}(s)\|=\max_{j}|\alpha_{p_j,k}(s)|\le \rho_{\sigma_0,k}<1$ as claimed.

Determinant: on each $k\times k$ prime block,
\[
 \det\!\bigl(I_k-S_{p}^{(k)}(s)\bigr)=\bigl(1-\alpha_{p,k}(s)\bigr)^{k}=\Bigl((1-p^{-s})^{-1/k}\Bigr)^{k}=\frac{1}{\,1-p^{-s}\,}.
\]
Taking the product over $p\le p_N$ yields the displayed identity.
\end{proof}

\begin{corollary}[Drop--in for the Schur--determinant split]
\label{cor:dropin}
Let $T_N(s)$ be the block operator on $\ell^2(\{p\le p_N\})\oplus\C^{kN}$ with blocks
\[
 A_N(s)\ \text{as above},\quad B_N\equiv 0,\quad C_N\ \text{arbitrary},\quad D_N(s):=S_{N}^{(k)}(s).
\]
Then $S_N(s):=D_N(s)-C_N(I-A_N(s))^{-1}B_N=D_N(s)=S_{N}^{(k)}(s)$, and the Schur--determinant splitting gives
\[
 \log\dettwo\bigl(I-T_N(s)\bigl)=\log\dettwo\bigl(I-A_N(s)\bigr)+\sum_{p\le p_N}\log\!\frac{1}{1-p^{-s}}.
\]
By Proposition~\ref{prop:kfold}, $S_N$ is Schur on $\{\Re s\ge \sigma_0\}$ uniformly in $N$ and the $k=1$ contribution is exact.
\end{corollary}

\paragraph{Remarks.}
(1) \emph{Why $k=2$ suffices.} For any $\sigma_0>\tfrac12$, $r_{\sigma_0}=2^{-\sigma_0}\le 2^{-1/2}<1$, hence
\[
 \rho_{\sigma_0,2}=(1-2^{-\sigma_0})^{-1/2}-1<(1-2^{-1/2})^{-1/2}-1\approx 0.848<1.
\]
Thus the choice $k=2$ already yields a uniform Schur constant on $\{\Re s\ge \sigma_0\}$.

(2) \emph{Prime--tied realization (optional).} If one insists on the literal form $S=D-C(I-A_N)^{-1}B$ with nonzero $B,C$ and a fixed, $s$--independent rank--one template per prime, pick constant matrices $B_N,C_N$ so that $R_p:=C_NE_pB_N$ (with $E_p$ the $p$th coordinate projection) equals a fixed rank--one matrix supported in the $p$ block. Then define
\[
 D_N(s)\;:=\;S_{N}^{(k)}(s)\;+\;\sum_{p\le p_N}\frac{1}{1-p^{-s}}\,R_p,
\]
which is holomorphic. This makes $S_N(s)=D_N(s)-\sum_p \frac{1}{1-p^{-s}}R_p\equiv S_{N}^{(k)}(s)$ identically, hence preserves the exact determinant identity and the Schur bound.

(3) \emph{Archimedean/polynomial factor.} On $\{\Re s>\tfrac12\}$ the factor $E_{\mathrm{arch}}(s):=\tfrac12 s(1-s)\,\pi^{-s/2}\Gamma(s/2)$ is nonvanishing. A completely analogous $k_{\mathrm{arch}}$--fold block
\[
 S_{\mathrm{arch}}(s):=\Bigl(1-E_{\mathrm{arch}}(s)^{-1/k_{\mathrm{arch}}}\Bigr)I_{k_{\mathrm{arch}}},
\]
yields $\det(I-S_{\mathrm{arch}})=E_{\mathrm{arch}}(s)^{-1}$ with $\|S_{\mathrm{arch}}\|<1$ after fixing $k_{\mathrm{arch}}\ge 2$; it may be appended as an extra finite block.

\begin{lemma}[Holomorphy under HS-holomorphic inputs]\label{lem:holomorphy}
If \(K:U\to\HS\) is holomorphic on an open set \(U\subset\C\), then \(f(s):=\dettwo\big(I-K(s)\big)\) is holomorphic on \(U\).
\end{lemma}
\begin{proof}
The map \(\Phi:K\mapsto \dettwo(I-K)\) is real-analytic on \(\HS\) and given by a uniformly convergent power series in a neighborhood of each point (e.g., via the canonical product or via trace-class regularization). Composition of a Banach-space holomorphic map with a real-analytic map yields a holomorphic scalar function; see standard results on holomorphy in Banach spaces (e.g., Hille--Phillips).
\end{proof}

\subsection{HS continuity implies local-uniform convergence of \(\dettwo\)}
We now formalize the continuity principle used later.

\begin{proposition}[HS\(\to\)\(\dettwo\) local-uniform convergence]\label{prop:HS-to-det2}
Let \(\Omega\subset\C\) be open and \(A_n,A:\Omega\to\HS\) be holomorphic maps such that for each compact \(K\subset\Omega\):
\begin{enumerate}
 \item \(\sup_{s\in K}\|A_n(s)\|_{\HS}\le M_K\) for all \(n\) (uniform HS bound);
 \item \(\sup_{s\in K}\|A_n(s)-A(s)\|_{\HS}\xrightarrow[n\to\infty]{}0\).
\end{enumerate}
Then \(f_n(s):=\dettwo\big(I-A_n(s)\big)\) converges to \(f(s):=\dettwo\big(I-A(s)\big)\) uniformly on \(K\). In particular, \(f_n\to f\) locally uniformly on \(\Omega\).
\end{proposition}
\begin{proof}
Fix a compact \(K\subset\Omega\). By Lemma~\ref{lem:carleman},
\[
 \sup_{n}\ \sup_{s\in K}\ |f_n(s)|\;\le\; \exp\!\Big(\tfrac12 M_K^2\Big),
\]
so \(\{f_n\}\) is a normal family on \(K\) (indeed on neighborhoods of \(K\)). By continuity of \(\Phi:K\mapsto\dettwo(I-K)\) on \(\HS\), the pointwise convergence \(A_n(s)\to A(s)\) in \(\HS\) implies \(f_n(s)\to f(s)\) for each fixed \(s\in K\). Vitali--Porter (or Montel's theorem plus the identity principle) then yields uniform convergence of \(f_n\) to \(f\) on \(K\): every subsequence has a further subsequence converging locally uniformly to a holomorphic limit \(g\); since \(f_n(s)\to f(s)\) pointwise on a set with accumulation points (indeed on all of \(K\)), necessarily \(g\equiv f\), proving uniform convergence of the full sequence.
\end{proof}

\begin{remark}[Division by \(\xi\)]
Uniform convergence for \(\dettwo(I-A_n)\to\dettwo(I-A)\) holds on all compacts. When dividing by \(\xi\), we either restrict to rectangles where \(|\xi|\ge \delta>0\) (interior alignment route) or insert the inner-compensator from Subsection~\ref{subsec:bl-compensator} to remove poles and work with the compensated ratio prior to applying the Cayley transform (boundary route).
\end{remark}

\section{Notation and conventions}\label{sec:notation}
We summarize conventions used throughout.
\begin{itemize}
 \item \textbf{Half-plane.} \(\Omega:=\{\Re s>\tfrac12\}\). We occasionally shift to \(\{\Re z>0\}\) via \(z=s-\tfrac12\); the Pick kernel denominator becomes \(s+\overline{w}-1\).
 \item \textbf{Spaces and bases.} \(\ell^2(\PP)\) is the Hilbert space indexed by primes with orthonormal basis \(\{e_p\}\). Operators act on the right; adjoints are denoted by \(\cdot^*\).
 \item \textbf{Trace ideals.} \(\HS=\mathcal S_2\) denotes Hilbert--Schmidt class with \(\|K\|_{\HS}^2=\Tr(K^*K)\). Trace class is \(\mathcal S_1\). Holomorphy into \(\HS\) is understood in the Banach--space sense.
 \item \textbf{Completed zeta.} \(\xi(s)=\tfrac12 s(1-s)\,\pi^{-s/2}\,\Gamma(s/2)\,\zeta(s)\). We use the principal branch for \(\log\) in scalar expansions; no branch choices enter operator formulas.
\item \textbf{Determinants.} \(\dettwo\) is the Hilbert--Schmidt (Carleman--Fredholm) regularization \(\det((I-K)e^{K}))\), distinct from \(\det_3\); Fredholm \(\det\) is used only for finite-dimensional blocks.
 \item \textbf{Systems.} \(A\) is \emph{Hurwitz} if \(\sigma(A)\subset\{\Re z<0\}\). \(\|H\|_\infty\) is the half-plane \(H^\infty\) norm (essential sup along vertical lines). \emph{Passive} means \(\|H\|_\infty\le 1\); \emph{lossless} means equality holds and the KYP equalities \eqref{eq:lossless-equalities} are satisfied.
 \item \textbf{Cayley transforms.} \(\Theta=\mathcal C[H]=(H-1)/(H+1)\) and \(H=\mathcal C^{-1}[\Theta]=(1+\Theta)/(1-\Theta)\).
\end{itemize}

\section{Schur--determinant splitting and the finite block}\label{sec:schur-split}
We next record a block-operator identity that isolates a finite-dimensional Schur complement from the Hilbert--Schmidt part. This will be applied with \(A(s)\) the prime-diagonal block and a finite auxiliary block gathering the \(k=1\) (prime) and archimedean/pole terms.

\begin{proposition}[Schur--determinant splitting]\label{prop:schur-split}
Let \(\mathcal H\) be a separable Hilbert space and consider the block operator on \(\mathcal H\oplus\C^m\):
\[
 T\;=\;\begin{bmatrix}A & B\\ C & D\end{bmatrix},
\]
with \(A\in\HS(\mathcal H)\), \(B:\C^m\to\mathcal H\) finite rank, \(C:\mathcal H\to\C^m\) finite rank, and \(D\in\C^{m\times m}\). Assume that \(I-A\) is invertible. Define the (finite-dimensional) Schur complement
\[
 S\;:=\;D\; -\; C\,(I-A)^{-1}\,B\;\in\;\C^{m\times m}.
\]
Then
\[
 \boxed{\ \log\dettwo(I-T)\;=\;\log\dettwo(I-A)\; +\; \log\det(I-S)\ }.
\]
Moreover, if \(\|A\|<1\), then
\[
 \log\dettwo(I-A)\;=\; -\sum_{k\ge 2}\frac{\Tr(A^k)}{k},
\]
with absolute convergence.
\end{proposition}
\begin{proof}
We write the standard Schur factorization for \(I-T\):
\[
 I-T\;=\;\begin{bmatrix}I & 0\\ -C((I-\tfrac12 I-A)^{-1} & I\end{bmatrix}\!
 \begin{bmatrix}(I-\tfrac12 I-A) & 0\\ 0 & I-S\end{bmatrix}\!
 \begin{bmatrix}I & -((I-\tfrac12 I-A)^{-1}B\\ 0 & I\end{bmatrix}.
\]
Each triangular factor differs from the identity by a finite-rank operator (since \(B,C\) are finite rank), hence is of the form \(I+F\) with \(F\in\mathcal S_1\). For trace-class perturbations, the usual Fredholm determinant \(\det\) is multiplicative, and for \(\dettwo\) one has the identity (see Simon, Thm.
9.2)
\[
 \dettwo\big((I+X)(I+Y)\big)\;=\;\dettwo(I+X)\,\dettwo(I+Y)\,\exp\!\big(-\Tr(XY)\big)
\]
 whenever \(X,Y\in \HS\). Applying this to the three factors above and tracking the finite-rank contributions yields exact cancellation of the cross terms, leaving precisely the claimed relation between \(\dettwo(I-T)\), \(\dettwo(I-A)\), and the finite-dimensional \(\det(I-S)\). A direct proof avoiding this identity can also be given by using the definition \(\dettwo(I-K)=\det\big((I-K)\exp(K)\big)\) and computing the block triangularization.

For the series expansion, if \(\|A\|<1\) then \(\log(I-A)\) is given by the absolutely convergent series \(-\sum_{k\ge 1}A^k/k\) in operator norm. Since \(A\in\HS\), \(\Tr(A)\) need not converge, but the 2-regularization removes the linear term and yields
\[
 \log\dettwo(I-A)\;=\;\Tr\!\Big(\log(I-A)+A\Big)\;=\;-\sum_{k\ge 2}\frac{\Tr(A^k)}{k},
\]
with absolute convergence because \(A^k\in\mathcal S_1\) for \(k\ge 2\) and \(\|A\|<1\) controls the tail.
\end{proof}

\begin{corollary}[Prime-power separation for the arithmetic block]\label{cor:pp-separation}
Let \(A(s)\) be the prime-diagonal operator \(A(s)e_p:=p^{-s}e_p\) on \(\ell^2(\PP)\) with \(\Re s>\tfrac12\). Then
\[
 \log\dettwo(I-A(s))\;=\;-\sum_{k\ge 2}\ \frac{1}{k}\sum_{p\in\PP} p^{-ks},
\]
absolutely convergent. In particular, the \(k=1\) prime term \(\sum_p p^{-s}\) does not appear in \(\log\dettwo(I-A)\) and must be accounted for in the finite Schur complement \(S\) when applying Proposition~\ref{prop:schur-split} to a block \(T(s)\) that models the completed \(\xi\)-normalization.
\end{corollary}
\begin{proof}
By Proposition~\ref{prop:schur-split}, the claimed series holds provided \(\|A(s)\|<1\). For \(\sigma:=\Re s>\tfrac12\), we have \(\|A(s)\|\le 2^{-\sigma}<1\), and \(\Tr\big(A(s)^k\big)=\sum_p p^{-ks}\) since \(A(s)^k\) is diagonal with entries \(p^{-ks}\). Absolute convergence follows from \(\sum_p p^{-2\sigma}<\infty\) and the bound \(|p^{-ks}|\le p^{-2\sigma}\) for all \(k\ge 2\).
\end{proof}

\begin{remark}[Finite block design and operator bound]\label{rem:finite-block-design}
In applications of Proposition~\ref{prop:schur-split} to the completed zeta normalization, the finite block \(S(s)=D(s)-C(s)(I-A(s))^{-1}B(s)\) is tasked with collecting the \(k=1\) prime term \(\sum_p p^{-s}\), the polynomial factor \(\tfrac12 s(1-s)\), and archimedean contributions. On any half-plane \(\{\Re s\ge \sigma_0>\tfrac12\}\), one has \(\|A(s)\|\le 2^{-\sigma_0}<1\), hence \(\|(I-A(s))^{-1}\|\le (1-2^{-\sigma_0})^{-1}\). Therefore, any representation of the form \(S(s)=D(s)-C(s)(I-A(s))^{-1}B(s)\) with bounded \(B,C,D\) on \(\{\Re s\ge \sigma_0\}\) obeys the operator bound
\[
 \|S(s)\|\;\le\;\|D(s)\|\; +\; \frac{\|C(s)\|\,\|B(s)\|}{1-2^{-\sigma_0}},\qquad \Re s\ge \sigma_0>\tfrac12.
\]
If, in addition, \(D\) is unitary (or a contraction) and \(B,C\) are chosen so that the right-hand side is \(\le 1\), then \(S\) is Schur on \(\{\Re s\ge \sigma_0\}\). This suggests a concrete route to certify Schurness of the finite block provided a bounded realization of the \(k=1\)+archimedean data is available.
\end{remark}
\subsection{Explicit $B,C,D$ parameterizations for the $k=1$+archimedean block}\label{subsec:BCD-params}
We record two concrete diagonal parameterizations of the finite Schur complement
\[
 S_N(s)\;=\;D_N(s)\; -\; C_N(s)\,(I-A_N(s))^{-1}\,B_N(s),\qquad A_N(s)\,e_p\;=\;p^{-s}e_p\ (p\le p_N),
\]
and derive half-plane contractivity bounds from Remark~\ref{rem:finite-block-design}. Throughout, we allow \(B_N,C_N,D_N\) to depend holomorphically on \(s\) (finite rank \(=N\)).

\paragraph{(E1) Exact $k=1$ match (diagonal, $D_N\equiv 0$).}
Set, for each prime \(p\le p_N\),
\[
 b_p(s)\;:=\;p^{-s/2},\qquad c_p(s)\;:=\;p^{-s/2},\qquad d_p(s)\;:=\;0.
\]
Then with \(B_N=\mathrm{diag}(b_p)\), \(C_N=\mathrm{diag}(c_p)\), \(D_N=0\), one has a diagonal Schur complement
\[
 S_N(s)\;=\; -\,\mathrm{diag}\!\left(\frac{p^{-s}}{1-p^{-s}}\right)_{p\le p_N}.
\]
Consequently
\[
 \log\det(I-S_N(s))\;=\;\sum_{p\le p_N}\log\!\left(\frac{1}{1-p^{-s}}\right)
\]
and the identity of Proposition~\ref{prop:schur-split} yields the desired $k=1$ separation when combined with $\log\dettwo(I-A_N)= -\sum_{k\ge 2}\Tr(A_N^k)/k$. However, the operator norm here obeys
\[
 \|S_N(s)\|\;=\;\max_{p\le p_N}\,\frac{|p^{-s}|}{\,1-|p^{-s}|\,}\;=\;\max_{p\le p_N}\,\frac{p^{-\sigma}}{1-p^{-\sigma}}\,,\qquad s=\sigma+it,
\]
so $\|S_N(s)\|\le 1$ holds only for $\sigma\ge 1$ (strictly $<1$ for $\sigma>1$). Thus (E1) gives an \emph{exact} $k=1$ finite block which is Schur on $\{\Re s\ge 1\}$ but not on the entire $\{\Re s>\tfrac12\}$.
\paragraph{(E2) Damped exact-form with uniform contractivity on $\{\Re s\ge\sigma_0\}$.}
Fix $\sigma_0>\tfrac12$ and a scalar damping factor
\[
 \alpha(\sigma_0)\;:=\;\frac{1-2^{-\sigma_0}}{2^{-\sigma_0}}\;=\;2^{\sigma_0}-1\;\in\;(0,\infty).
\]
Define
\[
 b_p(s)\;:=\;\sqrt{\alpha(\sigma_0)}\,p^{-s/2},\qquad c_p(s)\;:=\;\sqrt{\alpha(\sigma_0)}\,p^{-s/2},\qquad d_p(s)\;:=\;0.
\]
Then
\[
 S_N(s)\;=\;-\,\alpha(\sigma_0)\,\mathrm{diag}\!\left(\frac{p^{-s}}{1-p^{-s}}\right)_{p\le p_N}.
\]
Using Remark~\ref{rem:finite-block-design} with $\|B_N\|=\|C_N\|=\sup_{p\le p_N}|b_p|=\sqrt{\alpha(\sigma_0)}\,2^{-\sigma/2}$ and $\|(I-A_N)^{-1}\|\le (1-2^{-\sigma_0})^{-1}$ on $\{\Re s\ge\sigma_0\}$ gives
\[
 \|S_N(s)\|\;\le\;\frac{\|C_N\|\,\|B_N\|}{1-2^{-\sigma_0}}\;\le\;\frac{\alpha(\sigma_0)\,2^{-\sigma_0}}{1-2^{-\sigma_0}}\;=\;1,\qquad \Re s\ge\sigma_0.
\]
Thus (E2) furnishes a Schur finite block on any prescribed right half-plane $\{\Re s\ge\sigma_0\}$, at the cost of damping the $k=1$ contribution by the factor $\alpha(\sigma_0)$:
\[
 \log\det(I-S_N)\;=\;\sum_{p\le p_N}\log\!\left(\frac{1-\big(1-\alpha(\sigma_0)\big)p^{-s}}{1-p^{-s}}\right).
\]
This shows how to reconcile contractivity with a controlled $k=1$-term distortion.

\paragraph{(E3) Faster-decay variant.}
For any $\beta>0$, choose $b_p(s)=c_p(s)=p^{-(1/2+\beta)s}$, $d_p\equiv 0$. Then
\[
 S_N(s)\;=\;-\,\mathrm{diag}\!\left(\frac{p^{-(1+2\beta)s}}{1-p^{-s}}\right)_{p\le p_N},\qquad \|S_N(s)\|\;\le\;\sup_p\frac{p^{-\sigma(1+2\beta)}}{1-p^{-\sigma}},
\]
which is $<1$ uniformly on $\{\Re s>\tfrac12\}$ once $\beta$ is chosen large enough (e.g., any $\beta\ge \tfrac12$ works). The $k=1$ term is then heavily damped, but this family supplies uniformly Schur finite blocks on the entire BRF domain.

\begin{remark}[Design notes]
Parameterizations (E1)–(E3) expose a concrete path to Schurness of the finite block on right half-planes using only the diagonal structure of $A_N$. In practice one also folds the archimedean/pole corrections into $D_N$ while preserving the Schur bound and links the Schur finite block to the determinantal truncation so that the resulting Cayley transform approximates $\Theta_N^{(\dettwo)}$ uniformly on compacts (as realized quantitatively by the H$^\infty$ passive approximation scheme of Subsection~\ref{subsec:hinf-passive}).
\end{remark}

\subsection{Contractivity with a budgeted archimedean port $D_N$}\label{subsec:DN-budget}
We refine (E2) to incorporate a nonzero contraction $D_N(s)$ accounting for archimedean/pole corrections while maintaining Schurness on $\{\Re s\ge \sigma_0\}$.

\begin{lemma}[Budgeted contractivity]\label{lem:budget}
Fix $\sigma_0>\tfrac12$ and a budget $\eta\in(0,1)$. Let
\[
 \alpha(\sigma_0,\eta)\;:=\;(1-\eta)\,\frac{1-2^{-\sigma_0}}{2^{-\sigma_0}}\,=\,(1-\eta)\,(2^{\sigma_0}-1),
\]
and choose
\[
 b_p(s)\;=\;\sqrt{\alpha(\sigma_0,\eta)}\,p^{-s/2},\quad c_p(s)\;=\;\sqrt{\alpha(\sigma_0,\eta)}\,p^{-s/2},\quad D_N(s)\ \text{with}\ \|D_N\|_{H^\infty(\Re s\ge \sigma_0)}\le \eta.
\]
Then for $A_N(s)\,e_p=p^{-s}e_p$ one has
\[
 S_N(s)\;=\;D_N(s)\; -\; C_N(s)\,(I-A_N(s))^{-1}\,B_N(s),\qquad \|S_N(s)\|\ \le\ 1\quad (\Re s\ge \sigma_0).
\]
\end{lemma}
\begin{proof}
On $\{\Re s\ge \sigma_0\}$, $\|(I-A_N)^{-1}\|\le (1-2^{-\sigma_0})^{-1}$ and $\|B_N\|=\|C_N\|\le \sqrt{\alpha(\sigma_0,\eta)}\,2^{-\sigma_0/2}$. Thus
\[
 \|C_N(I-A_N)^{-1}B_N\|\ \le\ \frac{\alpha(\sigma_0,\eta)\,2^{-\sigma_0}}{1-2^{-\sigma_0}}\ =\ 1-\eta.
\]
Hence $\|S_N\|\le \|D_N\|+\|C_N(I-A_N)^{-1}B_N\|\le \eta+(1-\eta)=1$.
\end{proof}

\paragraph{Archimedean contraction port.}
Write the archimedean/polynomial factor as $E_{\mathrm{arch}}(s):=\tfrac12 s(1-s)\,\pi^{-s/2}\,\Gamma(s/2)$. Let $F(s)$ be any bounded holomorphic function on $\{\Re s\ge \sigma_0\}$ with $\|F\|_{H^\infty}\le 1$ chosen to approximate the Cayley transform of $E_{\mathrm{arch}}$ at selected sampling nodes (Nevanlinna--Pick interpolation). Setting
\[
 D_N(s)\;=\;\eta\,F(s)\,I_N
\]
fits (by construction) the budget of Lemma~\ref{lem:budget}. In particular, one can interpolate boundary samples of the normalized factor $\Phi_{\mathrm{arch}}(s):=(E_{\mathrm{arch}}(s)-1)/(E_{\mathrm{arch}}(s)+1)$ (scaled if necessary) to obtain $F$ with $\|F\|_\infty\le 1$ and hence $\|D_N\|\le \eta$.

\subsection{NP interpolation for the archimedean port and $k=1$ separation}\label{subsec:NP-arch}
We make the Nevanlinna--Pick (NP) step explicit and quantify the $k=1$ separation inside $\log\det(I-S_N)$.

\begin{lemma}[Schur NP interpolant for the archimedean Cayley]
Fix $\sigma_0>\tfrac12$ and a finite node set $\{s_j\}_{j=1}^{M}\subset\{\Re s\ge \sigma_0\}$. Let target values $\{\gamma_j\}$ satisfy $|\gamma_j|<1$. Then there exists a scalar Schur function $F$ on $\{\Re s\ge \sigma_0\}$ with $F(s_j)=\gamma_j$ for all $j$. Moreover one may take $F$ rational inner of degree at most $M$.
\end{lemma}

\begin{lemma}[Finite KYP augmentation for affine terms]\label{lem:affine-gram}
Let \(K_0(s,\overline t)\) be a PSD kernel on \(R\times R\) of the form \(\langle \Phi(s),\Phi(t)\rangle_{P}\), with a finite-dimensional realization \((A,B,C,D,P)\) satisfying the lossless equalities. Then, for any \(\alpha,\beta\in\C\), there exists an augmented lossless realization \((\widehat A,\widehat B,\widehat C,\widehat D,\widehat P)\) such that the kernel
\[
 K_\mathrm{sum}(s,\overline t)\ :=\ K_0(s,\overline t)\ +\ \frac{(\alpha+\beta s)+\overline{(\alpha+\beta t)}}{s+\overline t -1}
\]
is PSD on \(R\times R\) and equals \(\langle \widehat\Phi(s),\widehat\Phi(t)\rangle_{\widehat P}\) for a suitable feature map \(\widehat\Phi\) built by direct sum with one- and two-state lossless blocks.
\end{lemma}
\begin{proof}[Proof sketch]
Consider the scalar lossless factor \(H_1(s)=(s-\lambda)/(s+\lambda)\) with \(\lambda>0\) (Lemma~\ref{lem:moebius-contract}). Its Herglotz kernel equals
\[\frac{H_1(s)+\overline{H_1(t)}}{s+\overline t -1}\ =\ \Big\langle (sI+\lambda)^{-1}\sqrt{2\lambda},\ (tI+\lambda)^{-1}\sqrt{2\lambda}\Big\rangle,\]
which is a rank-one PSD kernel. Linear combinations of such kernels (with distinct \(\lambda\)) span the space of kernels of the form \(\frac{p(s)+\overline{p(t)}}{s+\overline t-1}\) for degree-1 polynomials \(p\). Appending these blocks as a direct sum to \((A,B,C,D)\) preserves losslessness and PSD of the associated Gram. Therefore the affine term can be realized inside the finite KYP block and absorbed into the augmented feature \(\widehat\Phi\).
\end{proof}

Apply this with prescribed $\gamma_j$ sampling the normalized archimedean Cayley $\Phi_{\mathrm{arch}}(s)=(E_{\mathrm{arch}}(s)-1)/(E_{\mathrm{arch}}(s)+1)$ on the line $\Re s=\sigma_0$. Setting $D_N=\eta F I_N$ as above yields a budgeted contraction with $\|D_N\|\le \eta$.

\begin{lemma}[Half-plane Blaschke products and Pick criterion]\label{lem:halfplane-blaschke}
For nodes $a_j\in\{\Re s>\sigma_0\}$ and target values $\gamma_j$ with $|\gamma_j|<1$, the Nevanlinna--Pick matrix $\big((1-\gamma_j\overline{\gamma_k})/(a_j+\overline{a_k}-2\sigma_0)\big)_{j,k}$ is PSD if and only if there exists a Schur function $F$ on $\{\Re s>\sigma_0\}$ with $F(a_j)=\gamma_j$. A constructive solution is given by finite products of half-plane Blaschke factors
\[
 B_{a}(s)\;:=\;\frac{s-\overline a}{s-a}\,,\qquad \Re a>\sigma_0,
\]
possibly multiplied by a unimodular constant and post-composed with disk automorphisms. In particular, any finite data set with a PSD Pick matrix admits a rational inner interpolant $F(s)=e^{i\theta}\prod_{j=1}^{M} B_{a_j}(s)^{m_j}$.
\end{lemma}


\begin{proposition}[Exact log-det formula and $k=1$ separation with damping]\label{prop:logdet-S}
Let $S_N$ be constructed as in Lemma~\ref{lem:budget} with diagonal $B_N,C_N$ and $D_N=\eta F I_N$. Then
\[
 \det(I-S_N(s))\;=\;\big(1-\eta F(s)\big)^{N}\,\prod_{p\le p_N}\left(1+\frac{\alpha(\sigma_0,\eta)}{1-\eta F(s)}\,\frac{p^{-s}}{1-p^{-s}}\right).
\]
In particular,
\[
 \log\det(I-S_N(s))\;=\;N\log\big(1-\eta F(s)\big)\; +\; \sum_{p\le p_N}\log\left(\frac{1-(1-\beta(s))\,p^{-s}}{1-p^{-s}}\right)
\]
with the scalar damping $\beta(s):=\alpha(\sigma_0,\eta)/(1-\eta F(s))$.
\end{proposition}
\begin{proof}
Since $D_N$ is a scalar multiple of the identity and $C_N(I-A_N)^{-1}B_N$ is diagonal, the eigenvalues of $I-S_N$ are $(1-\eta F)+\alpha\, p^{-s}/(1-p^{-s})$ over $p\le p_N$, yielding the product formula. The logarithmic form follows by rearrangement.
\end{proof}

\begin{corollary}[Controlled $k=1$ separation on right half-planes]
For any compact $K\subset\{\Re s\ge \sigma_0\}$ and $\delta\in(0,1)$, one can choose $\eta\in(0,1)$ and an NP interpolant $F$ so that $\sup_{s\in K}|\beta(s)-1|\le \delta$ and $\|D_N\|\le \eta$. Then
\[
 \sup_{s\in K}\left|\log\det(I-S_N(s))\; -\; \sum_{p\le p_N}\log\!\left(\frac{1}{1-p^{-s}}\right)\; -\;N\log\big(1-\eta F(s)\big)\right|\ \le\ C_K\,\delta\,\sum_{p\le p_N}\frac{|p^{-s}|}{|1-p^{-s}|},
\]
with $C_K$ depending only on $K$.
\end{corollary}
\begin{proof}
From Proposition~\ref{prop:logdet-S}, use $\log(1+z)=z+\mathcal O(z^2)$ uniformly on $K$ with $z=\tfrac{(\beta-1)p^{-s}}{1-p^{-s}}$ and bound the remainder by $C_K\,|\beta-1|\,|p^{-s}|/|1-p^{-s}|$.
\end{proof}

\begin{remark}[Blocker: growth of the $k=1$ error budget]
The right-hand sum $\sum_{p\le p_N} |p^{-s}|/|1-p^{-s}|$ diverges with $N$ for $\Re s\le 1$. Hence keeping $\beta\equiv 1$ is essential to preserve exact $k=1$ separation uniformly in $N$; this is feasible only for $\sigma_0\ge 1$ (case (E1)). For $\sigma_0\in(\tfrac12,1)$, any uniform damping induces a cumulative error growing with $N$. Resolving this obstruction (e.g., by a different finite-block architecture or a non-additive multiplicative scheme) is required to remove the reliance on the alignment hypothesis on the full BRF domain.
\end{remark}

\subsection{Schur finite blocks with uniform-on-$K$ $k=1$ control}\label{subsec:K1-approx}
We summarize the $k=1$ approximation mechanism that preserves Schurness on a fixed right half-plane compact while providing uniform error control.

\begin{proposition}[Uniform-on-$K$ $k=1$ control with Schurness]\label{prop:K1-approx}
Let $K\subset\{\Re s\ge\sigma_0\}$ be compact with $\tfrac12<\sigma_0<1$ and fix $\eta\in(0,\tfrac12)$ and $\varepsilon>0$. Then there exist finite-rank holomorphic matrices $B_N(s),C_N(s)$ and a scalar $D_N(s)$ with $\|D_N\|_{L^{\infty}(K)}\le\eta$ such that for
\[
 S_N(s)\;=\;D_N(s)\; -\; C_N(s)\,(I-A_N(s))^{-1}B_N(s),\qquad A_N(s)e_p\;=\;p^{-s}e_p,\ p\le p_N,
\]
one has:
\begin{itemize}
 \item Schur on $K$: $\displaystyle\sup_{s\in K}\,\|S_N(s)\|\le 1$;
 \item Uniform $k=1$ control: $\displaystyle\sup_{s\in K}\,\Bigl|\log\det(I-S_N(s))\; -\;\sum_{p\le p_N}\log\frac{1}{1-p^{-s}}\Bigr|\ \le\ \varepsilon.$
\end{itemize}
In particular, $S_N$ can be taken from the budgeted/damped family of Section~\ref{subsec:DN-budget} with Nevanlinna--Pick $D_N$ (Subsection~\ref{subsec:NP-arch}) and parameters chosen so that the error bound holds on $K$.
\begin{remark}
The parameters $(\eta,\delta,N)$ can be selected in a $K$-dependent but explicit manner: choose $\eta\le \varepsilon/(2M_0)$ for a fixed port dimension $M_0$, and pick $\delta\ll \varepsilon$ so that $\sum_{p\le p_N} |p^{-s}|/|1-p^{-s}|\le C_K$ with $C_K\delta\le \varepsilon/2$ uniformly on $K$. This yields the displayed bound while preserving the Schur budget $\|S_N\|\le 1$.
\end{remark}
\end{proposition}
\begin{proof}[Idea]
By Lemma~\ref{lem:budget} pick $B_N,C_N$ diagonal in the prime basis with damping parameter $\alpha(\sigma_0,\eta)$ so that $\|C_N(I-A_N)^{-1}B_N\|\le 1-\eta$ on $K$. With $D_N=\eta F$ where $F$ is a half-plane Schur NP interpolant (Lemma in Subsection~\ref{subsec:NP-arch}), Proposition~\ref{prop:logdet-S} gives
\[
 \log\det(I-S_N)=N\log(1-\eta F)\ +\ \sum_{p\le p_N}\log\frac{1-(1-\beta(s))p^{-s}}{1-p^{-s}},\qquad \beta(s)=\frac{\alpha(\sigma_0,\eta)}{1-\eta F(s)}.
\]
On $K$, choose $F$ and $\eta$ so that $\sup_K|\beta-1|\le\delta$ with $\delta$ small enough; then the log-det difference is bounded by $C_K\delta\sum_{p\le p_N}|p^{-s}|/|1-p^{-s}|+N\,\eta/(1-\eta)$. Place $D_N$ in a fixed-dimensional port (or scale $N$) so the $N$-term is $\le \varepsilon/2$, and choose $\delta$ so the prime sum is $\le\varepsilon/2$ uniformly on $K$. This yields the claimed bound while retaining $\|S_N\|\le 1$.
\end{proof}

\section{Finite-stage KYP certificates: lossless factorization and prime-grid model}\label{sec:KYP}
We now construct explicit finite-stage passive (bounded-real) realizations and verify the Kalman--Yakubovich--Popov (KYP) condition. We work throughout in continuous time on the right half-plane, with the transfer function
\[
 H(s)\;=\;D\; +\; C\,(sI-A)^{-1} B,
\]
where \(A\in\C^{n\times n}\) is Hurwitz (spectrum strictly in the open left half-plane), and \(B\in\C^{n\times m}\), \(C\in\C^{m\times n}\), \(D\in\C^{m\times m}\).

\subsection{Bounded-real lemma and the lossless KYP equalities}
The continuous-time bounded-real lemma asserts that, for a Hurwitz \(A\), the following are equivalent: (i) \(\|H\|_\infty\le 1\); (ii) there exists \(P\succ 0\) such that the KYP matrix is negative semidefinite
\begin{equation}\label{eq:KYP}
 \Theta\;=\;\begin{bmatrix}
  A^*P+PA & PB & C^*\\
  B^*P & -I & D^*\\
  C & D & -I
 \end{bmatrix}\ \preceq\ 0.
\end{equation}
In the \emph{lossless} case (extremal \(\|H\|_\infty=1\)), one may certify \eqref{eq:KYP} via the following algebraic equalities.

\begin{lemma}[One-line lossless KYP factorization]\label{lem:losslessKYP}
Suppose \(P\succ 0\) and
\begin{equation}\label{eq:lossless-equalities}
 A^*P+PA\;=\;-C^*C,\qquad PB\;=\;-C^*D,\qquad D^*D\;=\;I.
\end{equation}
Then the KYP matrix \(\Theta\) in \eqref{eq:KYP} factors as
\begin{equation}\label{eq:one-line-factor}
 \boxed{\ \Theta\;=\;-\begin{bmatrix}C^*\\ D^*\\ -I\end{bmatrix}\!\begin{bmatrix}C & D & -I\end{bmatrix}\ \preceq\ 0\ }.
\end{equation}
In particular, \(\|H\|_\infty\le 1\).
\end{lemma}
\begin{proof}
Using \eqref{eq:lossless-equalities}, we rewrite the KYP blocks as
\[
 A^*P+PA\;=\;-C^*C,\qquad PB\;=\;-C^*D,\qquad B^*P\;=\;-D^*C.
\]
Substituting these into \eqref{eq:KYP} gives
\[
 \Theta\;=\;\begin{bmatrix}
  -C^*C & -C^*D & C^*\\
  -D^*C & -I & D^*\\
  C & D & -I
 \end{bmatrix}\;=\;-\begin{bmatrix}C^*\\ D^*\\ -I\end{bmatrix}\!\begin{bmatrix}C & D & -I\end{bmatrix},
\]
which is negative semidefinite as a Gram matrix with a negative sign. The bounded-real implication is standard from the KYP lemma for Hurwitz \(A\).
\end{proof}

\subsection{Prime-grid lossless specification (final form)}
We now instantiate a concrete, diagonal (hence Hurwitz) realization at each prime truncation level \(N\), directly tied to the primes.

\begin{proposition}[Prime-grid lossless model]\label{prop:prime-grid-KYP}
Let \(p_1<\cdots<p_N\) be the first \(N\) primes and define the positive diagonal matrix
\[
 \Lambda_N\;:=\;\mathrm{diag}\!\Big(\tfrac{2}{\log p_1},\dots,\tfrac{2}{\log p_N}\Big)\ \in\ \R^{N\times N}.
\]
Set
\[
 A_N\;:=\;-\Lambda_N,\qquad P_N\;:=\;I_N,\qquad C_N\;:=\;\sqrt{2\,\Lambda_N},\qquad D_N\;:=\;-I_N,\qquad B_N\;:=\;C_N.
\]
Then:
\begin{enumerate}
 \item \(A_N\) is Hurwitz, with spectrum \(-\{2/\log p_k\}_{k=1}^N\subset(-\infty,0)\).
 \item The lossless equalities \eqref{eq:lossless-equalities} hold with \((A,B,C,D,P)=(A_N,B_N,C_N,D_N,P_N)\):
 \[
  A_N^*P_N+P_NA_N\;=\;-2\Lambda_N\;=\;-C_N^*C_N,\quad P_NB_N\;=\;C_N\;=\;-C_N^*D_N,\quad D_N^*D_N\;=\;I_N.
 \]
 \item The KYP matrix factors as in \eqref{eq:one-line-factor}, hence for the matrix-valued transfer
 \[
  H_N(s)\;:=\;D_N\; +\; C_N\,(sI-A_N)^{-1} B_N
 \]
one has \(\|H_N\|_\infty\le 1\). In particular, each entry of \(H_N\) is a bounded-real function on \(\Omega\).
 \item For any unit vectors \(u,v\in\C^N\) (``scalar port extraction''), the scalar transfer \(h_N(s):=v^*H_N(s)u\) satisfies \(|h_N(s)|\le 1\) for all \(s\in\Omega\). Choosing \(u=v=e_1\) yields scalar feedthrough \(-1\), consistent with the asymptotic limit of the target \(H\).
\end{enumerate}
\end{proposition}
\begin{proof}
(i) \(\Lambda_N\) is positive diagonal, hence \(A_N=-\Lambda_N\) has strictly negative diagonal entries.

(ii) Direct computation using diagonality: \(A_N^*P_N+P_NA_N=(-\Lambda_N)+(-\Lambda_N)=-2\Lambda_N\). Since \(C_N=\sqrt{2\Lambda_N}\) is the positive square root, \(C_N^*C_N=2\Lambda_N\), hence \(A_N^*P_N+P_NA_N=-C_N^*C_N\). Next, \(P_NB_N=B_N=C_N\) and \(C_N^*D_N=\sqrt{2\Lambda_N}\,(-I_N)=-C_N\), so \(P_NB_N+ C_N^*D_N=0\). Finally, \(D_N^*D_N=(-I_N)^*(-I_N)=I_N\).

(iii) With the equalities verified, Lemma~\ref{lem:losslessKYP} yields the factorization and \(\|H_N\|_\infty\le 1\).

(iv) If \(\|H_N\|_\infty\le 1\) as an operator norm, then for any unit vectors \(u,v\) one has \(|v^*H_N(s)u|\le \|H_N(s)\|\le 1\) pointwise in \(s\). The choice \(u=v=e_1\) reads off the \((1,1)\) entry, whose feedthrough equals \(-1\).
\end{proof}

\begin{remark}[Normalization and asymptotics]
The choice \(D_N=-I_N\) matches the scalar asymptotic \(\lim_{\Re s\to\infty} H(s)=-1\) after a scalar port extraction. Other unitary dilations \(D_N\) with \(D_N^*D_N=I_N\) are admissible and preserve the lossless factorization \eqref{eq:one-line-factor}.
\end{remark}

\begin{remark}[Discrete-time variant]
An analogous construction holds in discrete time (Schur class on the unit disk) with the discrete-time KYP inequality and the corresponding lossless equalities. We focus here on the continuous-time half-plane setting consistent with \(s\)-domain formulations.
\end{remark}

\section{Schur, Herglotz and Pick equivalences on the half-plane}\label{sec:equivalences}
We collect the standard equivalences between Herglotz, Schur and Pick kernel positivity on the right half-plane \(\Omega=\{\Re s>\tfrac12\}\). For a holomorphic scalar function \(F:\Omega\to\C\), define its Cayley transform
\[
 \mathcal C[F](s)\;:=\;\frac{F(s)-1}{F(s)+1},\qquad \mathcal C^{-1}[\Theta](s)\;:=\;\frac{1+\Theta(s)}{1-\Theta(s)}.
\]

\begin{theorem}[Equivalences]\label{thm:equivalences}
For a holomorphic scalar \(F\) on \(\Omega\), the following are equivalent:
\begin{enumerate}
 \item \(F\) is Herglotz on \(\Omega\): \(\Re F(s)\ge 0\) for all \(s\in\Omega\).
 \item \(\Theta:=\mathcal C[F]\) is Schur on \(\Omega\): \(|\Theta(s)|\le 1\) for all \(s\in\Omega\).
 \item The Pick kernel
 \[
  K_\Theta(s,w)\;:=\;\frac{1-\Theta(s)\,\overline{\Theta(w)}}{s+\overline{w}-1}
 \]
 is positive semidefinite on \(\Omega\): for all finite node sets \(\{s_j\}\subset\Omega\) and vectors \(\{c_j\}\subset\C\), one has \(\sum_{j,k} K_\Theta(s_j,s_k)\,c_j\overline{c_k}\ge 0\).
\end{enumerate}
The same equivalences hold for matrix-valued functions with the obvious operator-valued adaptations (operator norm in (2) and PSD block Gram matrices in (3)).
\end{theorem}
\begin{proof}
(1)\(\Rightarrow\)(2): For \(z\in\C\) with \(\Re z\ge 0\), the scalar inequality \(|(z-1)/(z+1)|\le 1\) is immediate from \(|z-1|^2\le |z+1|^2\) \(\Leftrightarrow\) \(\Re z\ge 0\). Apply pointwise with \(z=F(s)\).

(2)\(\Rightarrow\)(1): Invert the Cayley transform: \(F=(1+\Theta)/(1-\Theta)\). If \(|\Theta|\le 1\), then for each \(s\) one has \(\Re F(s)\ge 0\) (check on scalars or via the Herglotz representation). Holomorphy ensures the property on \(\Omega\).

(2)\(\Leftrightarrow\)(3): This is the Nevanlinna--Pick theorem on the half-plane; see, e.g., the de Branges--Rovnyak space characterization. For the half-plane \(\{\Re s>0\}\), the canonical Pick kernel is \((1-\Theta(s)\overline{\Theta(w)})/(s+\overline{w})\); replacing \(s\) by \(s-\tfrac12\) yields the stated denominator \(s+\overline{w}-1\).
\end{proof}

\begin{corollary}[Schur \(\Rightarrow\) Pick-PSD on \(\Omega\)]\label{cor:schur-pick-psd}
If \(\Theta\) is Schur on \(\Omega\), then for any finite node set \(\{s_j\}\subset\Omega\) the Gram matrix
\[
\Big(\,\frac{1-\Theta(s_i)\,\overline{\Theta(s_j)}}{\,s_i+\overline{s_j}-1\,}\,\Big)_{i,j}
\]
is positive semidefinite. Equivalently, the half-plane Pick kernel \(K_\Theta\) is PSD on \(\Omega\).
\end{corollary}
\begin{proof}
Immediate from Theorem~\ref{thm:equivalences} (Schur \(\Leftrightarrow\) Pick-PSD).
\end{proof}

\paragraph{Bridge to RDM artifacts.}
In the engineering pipeline (e.g., scripts/run\_rdm\_is.py, rdm/pick.py, rdm/colligation.py, rdm/det2.py, rdm/outer.py), the per-interval Pick matrices assembled from \(\Theta\)-samples are precisely finite-grid instances of Corollary~\ref{cor:schur-pick-psd}. They serve as reproducible PSD witnesses aligned with the analytic equivalences here. The boundary route ((P+) positivity) is an alternative path; once \(\Theta\) is Schur on \(\Omega\), the PSD property of \(K_\Theta\) follows unconditionally from the corollary.

\begin{corollary}[Closure]\label{cor:closure}
If \(F_n\) are Herglotz on \(\Omega\) and \(F_n\to F\) locally uniformly on \(\Omega\), then \(F\) is Herglotz. Equivalently, if \(\Theta_n\) are Schur and \(\Theta_n\to\Theta\) locally uniformly, then \(\Theta\) is Schur; moreover the Pick kernels \(K_{\Theta_n}\) converge entrywise on finite Gram matrices to a PSD limit, so \(K_{\Theta}\) is PSD.
\end{corollary}
\begin{proof}
Local-uniform limits of holomorphic functions preserve pointwise inequalities that are closed under limits. Alternatively, pass through Theorem~\ref{thm:equivalences}(2): \(|\Theta_n|\le 1\) implies \(|\Theta|\le 1\) by Montel and the maximum principle; invert the Cayley transform.
\end{proof}

\section{Alignment and closure to the BRF limit}\label{sec:alignment}
Recall \(J(s):=\dettwo(I-A(s))/\xi(s)\) and adopt the Cayley transform
\[
  \Theta(s)\ :=\ \frac{2J(s)-1}{2J(s)+1},\qquad s\in\Omega,
\]
so that \(\Theta(\sigma+it)\to -1\) as \(\sigma\to+\infty\). For truncations, define
\[
 H_N^{(\dettwo)}(s)\;:=\;2\,\frac{\dettwo(I-A_N(s))}{\xi(s)}-1,\qquad \Theta_N^{(\dettwo)}\;:=\;\frac{H_N^{(\dettwo)}-1}{H_N^{(\dettwo)}+1}.
\]
By Proposition~\ref{prop:HS-to-det2} and the division remark, \(H_N^{(\dettwo)}\to H\) locally uniformly on compact subsets avoiding zeros of \(\xi\). As established in Lemma~\ref{lem:compact-alignment}, this implies that the Cayley transforms also converge locally uniformly on the same sets, i.e. \(\Theta_N^{(\dettwo)}\to\Theta\).

\begin{lemma}[Cayley continuity on compacts]\label{lem:cayley-cont}
If \(f_n,f\) are holomorphic on a domain \(U\subset\C\) and \(f_n\to f\) uniformly on compact \(K\subset U\) with \(\inf_{K}|f+1|>0\), then \(\mathcal C[f_n]\to\mathcal C[f]\) uniformly on \(K\).
\end{lemma}
\begin{proof}
Uniform convergence plus the nonvanishing bound on \(f+1\) implies \(\inf_{K}|f_n+1|>\tfrac12\inf_{K}|f+1|\) for large \(n\). The Cayley map is uniformly Lipschitz on the compact annulus \(\{z: |z+1|\ge c>0\}\), hence the result.
\end{proof}

\begin{lemma}[No internal poles for the Schur limit]\label{lem:no-P}
Let \(\Omega':=\Omega\setminus S\) with $S$ discrete. Suppose \(\Theta_n\) are Schur on \(\Omega\) and \(\Theta_n\to\Theta\) locally uniformly on \(\Omega'\). Then \(\Theta\) extends holomorphically to \(\Omega\) and \(|\Theta|\le 1\) there. In particular, if \(\Theta=(2J-1)/(2J+1)\) on \(\Omega'\), then the set \(P:=\{s\in\Omega:\,2J(s)=-1\}\) is empty.
\end{lemma}
\begin{proof}
Fix $s_0\in S$ and a disc $D$ with $D\setminus\{s_0\}\subset\Omega'$. Since $|\Theta_n|\le 1$ on $\Omega$, the limit $\Theta$ is bounded by $1$ on $D\setminus\{s_0\}$ and hence extends holomorphically across $s_0$ by Riemann's removable singularity theorem. Doing this for each $s_0\in S$ gives a holomorphic extension to $\Omega$ with $|\Theta|\le 1$. If $s_0\in P$, then $(2J-1)/(2J+1)$ would have a pole at $s_0$, contradicting bounded holomorphy of $\Theta$ there. Thus $P=\varnothing$.
\end{proof}

\section{BRF and RH: implications and equivalence}\label{sec:brf-rh}
We record the logical relationship between the bounded-real target for $H$ and the classical Riemann Hypothesis (RH).

\begin{lemma}[Nonvanishing of \(\dettwo(I-A(s))\) on \(\Omega\)]
For $s\in\Omega=\{\Re s>\tfrac12\}$ one has $\|A(s)\|\le 2^{-\Re s}<1$, hence $I-A(s)$ is invertible and \(\dettwo(I-A(s))\ne 0\).
\end{lemma}
\begin{proof}
If $\|K\|<1$ then $1\notin\sigma(K)$ so $I-K$ is invertible. Moreover, in the canonical product \(\dettwo(I-K)=\prod_n (1-\lambda_n) e^{\lambda_n}\), no factor vanishes since $|\lambda_n|<1\) for all eigenvalues \(\lambda_n\) of $K$. Apply with $K=A(s)$.
\end{proof}
\begin{theorem}[BRF \(\Rightarrow\) RH]\label{thm:brf-implies-rh}
If \(\Theta\) is Schur on \(\Omega\) (equivalently $2J$ is Herglotz on \(\Omega\)), then \(\xi\) has no zeros in \(\Omega\), and by the functional equation \(\xi(s)=\xi(1-s)\) all nontrivial zeros lie on \(\Re s=\tfrac12\). Hence RH holds.
\end{theorem}
\begin{proof}
If \(\xi(\rho)=0\) for some \(\rho\in\Omega\), then by Lemma~\ref{lem:nonvanish-det2} the numerator \(\dettwo(I-A(\rho))\ne 0\), so \(J\) has a pole at \(\rho\). Consequently \(\Theta=(2J-1)/(2J+1)\) is not holomorphic at \(\rho\). This contradicts the Schur hypothesis, which implies holomorphy and boundedness on \(\Omega\). Therefore \(\xi\) has no zeros in \(\Omega\). Using \(\xi(s)=\xi(1-s)\), any zero with \(\Re s<\tfrac12\) would reflect to a zero with \(\Re s>\tfrac12\), impossible. Thus all nontrivial zeros lie on \(\Re s=\tfrac12\).
\end{proof}
\begin{theorem}[RH $+$ (P+) \(\Rightarrow\) BRF]\label{thm:rh-implies-brf}
Assume RH holds (so \(\xi\) has no zeros in \(\Omega\)) and Theorem~\ref{thm:uniform-eps} provides the outer limit \(\mathcal O\). If, in addition, (P+) holds for \(\mathcal J=\dettwo(I-A)/(\mathcal O\,\xi)\), then \(2\mathcal J\) is Herglotz on \(\Omega\), hence \(\Theta\) is Schur on \(\Omega\).
\end{theorem}
\begin{proof}
Under RH the ratio \(\dettwo(I-A)/\xi\) is analytic on \(\Omega\); combined with Theorem~\ref{thm:uniform-eps} and the boundary positive-real route (Theorem~\ref{thm:global-PSD}), the maximum principle yields the Schur bound in \(\Omega\).
\end{proof}
\begin{corollary}[Equivalence]
BRF for $H$ on \(\Omega\) is equivalent to RH, combining Theorems~\ref{thm:brf-implies-rh} and \ref{thm:rh-implies-brf} with Theorem~\ref{thm:uniform-eps}.
\end{corollary}
In order to pass positivity from finite-stage certificates to the limit function \(H\), it suffices to align a Schur sequence with the Cayley transforms \(\Theta_N^{(\dettwo)}\).

\begin{proposition}[Alignment criterion]\label{prop:alignment-criterion}
Suppose \(\Theta_N\) are Schur on \(\Omega\) (e.g., produced by the prime-grid lossless construction in Proposition~\ref{prop:prime-grid-KYP}, possibly after scalar port extraction), and for each compact \(K\subset\Omega\) one has
\[
 \sup_{s\in K}\big\|\Theta_N(s)-\Theta_N^{(\dettwo)}(s)\big\|\xrightarrow[N\to\infty]{}0.
\]
Then \(\Theta_N\to\Theta\) locally uniformly on \(\Omega\), and \(\Theta\) is Schur by Corollary~\ref{cor:closure}. Consequently, \(H=\mathcal C^{-1}[\Theta]\) is Herglotz on \(\Omega\), proving the BRF conclusion.
\end{proposition}
\begin{remark}
This alignment mechanism is auxiliary and not used in the interior route that follows. Global Schur/PSD follows from Theorem~\ref{thm:uniform-eps} and the outer-normalization argument, independently of this proposition.
\end{remark}
\begin{proof}
Triangle inequality with Lemma~\ref{lem:cayley-cont} yields \(\Theta_N^{(\dettwo)}\to\Theta\) and \(\Theta_N-\Theta\to 0\) locally uniformly. Closure then applies.
\end{proof}

\begin{remark}[Realization of \(\Theta_N\) and limits of interpolation]
The Schur sequence \(\Theta_N\) in Proposition~\ref{prop:alignment-criterion} can be taken as the matrix-valued transfers from Proposition~\ref{prop:prime-grid-KYP}, or any scalar port extraction thereof, all of which satisfy the uniform Schur bound by construction. However, matching finitely many interpolation nodes (even with degrees that grow) does not by itself force uniform convergence on a compact set for a moving sequence of rational inner functions without additional a priori bounds (e.g., uniform degree and coefficient control, or explicit $H^\infty$ approximation estimates). Thus quantitative alignment estimates \(\|\Theta_N-\Theta_N^{(\dettwo)}\|_{H^\infty(K)}\to 0\) must be proved, not inferred from dense interpolation.
\end{remark}

\begin{theorem}[BRF equivalences and closure to the limit]\label{thm:BRF}
Let \(A(s)\) be the prime-diagonal block on \(\Omega\) and define \(H\) and \(\Theta\) as above. Then the following are equivalent:
\begin{itemize}
\item[(i)] \(\Re\big(2J(s)\big)\ge 0\) on \(\Omega\) (BRF).
 \item[(ii)] \(\Theta\) is Schur on \(\Omega\).
 \item[(iii)] The Pick kernel \(K_\Theta\) is PSD on \(\Omega\).
\end{itemize}
Moreover, if there exists a Schur sequence \(\Theta_N\) satisfying the alignment hypothesis of Proposition~\ref{prop:alignment-criterion}, then \(\Theta\) is Schur and hence (i)--(iii) hold.
\end{theorem}

\begin{theorem}[Global kernel positivity from interior passivity, outer normalization, and boundary positive-real]\label{thm:global-PSD}
Let
\[
  H(s):=2\,\frac{\dettwo(I-A(s))}{\xi(s)}-1,\qquad
  \Theta(s):=\frac{H(s)-1}{H(s)+1},
\]
on \(\Omega=\{\Re s>\tfrac12\}\), with $A(s)$ Hilbert--Schmidt and holomorphic on \(\Omega\).
Assume:

\emph{(i) Interior passivity on rectangles.} For every compact rectangle $K\Subset\Omega$ avoiding zeros of \(\xi\) there exist Schur functions \(\Theta_{K,M}\) with \(\Theta_{K,M}\to\Theta\) locally uniformly on $K$ as $M\to\infty$.

\emph{(ii) Uniform boundary $L^1$ control and outer normalization.} There is \(\varepsilon_0>0\) such that
\[
  u_\varepsilon(t):=\log\Bigl|\frac{\dettwo(I-A(\tfrac12+\varepsilon+it))}{\xi(\tfrac12+\varepsilon+it)}\Bigr|
\]
is uniformly bounded in $L^1_{\mathrm{loc}}(\R)$ on $(0,\varepsilon_0]$ and Cauchy as \(\varepsilon\downarrow0\), so the associated outers converge locally uniformly to an outer limit \(\mathcal O\) on \(\Omega\).

\emph{(iii) Boundary positive-real (P+).} For \(\mathcal J:=\dettwo(I-A)/(\mathcal O\,\xi)\) one has
\[
 \Re\big(2\mathcal J(\tfrac12+it)\big)\ \ge\ 0\quad\text{for a.e. }t\in\R.
\]

Then \(2\mathcal J\) is Herglotz on \(\Omega\), hence \(\Theta=\mathcal C[2\mathcal J]\) is Schur on \(\Omega\) and the Pick kernel
\(
  K_\Theta(s,w)=\frac{1-\Theta(s)\,\overline{\Theta(w)}}{s+\overline{w}-1}
\)
is positive semidefinite on \(\Omega\).
\end{theorem}
\begin{proof}
By (ii), \(\mathcal J\in N(\Omega)\) with a.e. boundary values on \(\Re s=\tfrac12\). The Poisson integral transports (P+) to \(\Re(2\mathcal J)\ge0\) in \(\Omega\), so \(2\mathcal J\) is Herglotz. Apply Theorem~\ref{thm:equivalences}.
\end{proof}
\begin{proof}
Equivalences are Theorem~\ref{thm:equivalences}. The closure statement follows from Proposition~\ref{prop:alignment-criterion}.
\end{proof}
\subsection{Boundary positive-real via outer normalization}\label{subsec:boundary-unitarity}
\paragraph{Outline.}
We establish: (i) uniform local $L^1$ control and Cauchy-in-$\varepsilon$ for $u_\varepsilon(t)=\log|\dettwo(I-A(\tfrac12+\varepsilon+it))/\xi(\tfrac12+\varepsilon+it)|$ (Theorem~\ref{thm:uniform-eps}), yielding an outer limit \(\mathcal O\); (ii) the phase--velocity identity (Proposition~\ref{prop:phase-velocity-identity}); (iii) a reduction of (P+) to a Carleson/Poisson mass bound for the off-critical zero measure (Theorem~\ref{thm:Pplus-from-Carleson}); and (iv) an adaptive cover criterion (Corollary~\ref{cor:adaptive-cover}).
Define
\[
 \widetilde H(s):=2\,\frac{\dettwo(I-A(s))}{\xi(s)}-1,\qquad
 \widetilde\Theta(s):=\frac{\widetilde H(s)-1}{\widetilde H(s)+1}.
\]
By Theorem~\ref{thm:uniform-eps}, the outer normalizations along \(\Re s=\tfrac12+\varepsilon\) converge locally uniformly to an outer limit \(\mathcal O\) on \(\Omega\). Set
\[
 \mathcal J(s):=\frac{\dettwo(I-A(s))}{\mathcal O(s)\,\xi(s)}.
\]
We do \emph{not} assume \(\mathcal O\equiv 1\), nor do we infer any boundary unimodularity for \(\widetilde\Theta\) from \(|\mathcal J|=1\).

\medskip
\noindent\textbf{Boundary hypothesis (P+).} We require
\[
 \Re\big(2\,\mathcal J(\tfrac12+it)\big)\ \ge\ 0\quad\text{for a.e. }t\in\R.
\]
Under (P+), the Poisson integral gives that \(2\mathcal J\) is Herglotz on \(\Omega\), and thus
\(\Theta=\frac{2\mathcal J-1}{2\mathcal J+1}\) is Schur on \(\Omega\) (Theorem~\ref{thm:equivalences}).

\subsection{Inner compensator for zeros of \(\xi\)}\label{subsec:bl-compensator}
If \(\xi\) has zeros in a fixed rectangle \(R\Subset\Omega\), the ratio \(J=\dettwo(I-A)/\xi\) is meromorphic on \(R\).
To ensure analyticity for auxiliary constructions on \(R\) (e.g., passive \(H^\infty\) approximation), introduce the finite half-plane Blaschke product
\(
 B_{\xi,R}(s):=\prod_{\rho\in Z(\xi)\cap R} \big(\tfrac{s-\overline \rho}{s-\rho}\big)^{m_\rho}.
\)
Define the compensated ratio \(J_R^{\rm comp}:=J\,B_{\xi,R}\), which is holomorphic on \(R\).
\emph{We do not use} \(J_R^{\rm comp}\) in the (P+) boundary route, since multiplication by an inner factor preserves modulus but not boundary real part. The compensator is employed only to build interior Schur approximants on rectangles; the global Schur/PSD conclusion comes from (P+) with outer normalization, independently of any compensator.

\subsection{Prototype outer factor on \(\Re s=\tfrac12+\varepsilon\)}\label{subsec:outer-prototype}
Fix \(\varepsilon>0\) and consider \(L_{\varepsilon}:=\{s=\tfrac12+\varepsilon+it\}\). Define
\[
 G_{\varepsilon}(t):=\dettwo\big(I-A(\tfrac12+\varepsilon+it)\big),\qquad X_{\varepsilon}(t):=\xi\big(\tfrac12+\varepsilon+it\big).
\]
Let \(\mathcal O_{\varepsilon}\) be the outer on \(\{\Re s>\tfrac12+\varepsilon\}\) with boundary modulus \(\big|\frac{G_{\varepsilon}}{X_{\varepsilon}}\big|\). Set
\[
 \mathcal J_{\varepsilon}(s):=\frac{\dettwo(I-A(s))}{\mathcal O_{\varepsilon}(s)\,\xi(s)}.
\]
Then \(|\mathcal J_{\varepsilon}|=1\) on \(L_{\varepsilon}\) and \(\mathcal J_{\varepsilon}\) is holomorphic on \(\{\Re s>\tfrac12+\varepsilon\}\). By Theorem~\ref{thm:uniform-eps} and Lemma~\ref{lem:outer-factorization}(2), \(\mathcal O_{\varepsilon}\to\mathcal O\) and \(\mathcal J_{\varepsilon}\to\mathcal J\) locally uniformly as \(\varepsilon\downarrow 0\). If (P+) holds for the limiting boundary data, then \(2\mathcal J\) is Herglotz in \(\Omega\), so \(\Theta\) is Schur (Theorem~\ref{thm:equivalences}). We do not claim boundary unimodularity for \(\Theta\) from \(|\mathcal J_{\varepsilon}|=1\).

\begin{proposition}[L$^1_{\mathrm{loc}}$ control reduces to HS tails]\label{prop:L1loc}
Fix a compact interval $I\subset\mathbb R$. Then for \(\varepsilon\in(0,\tfrac12)\),
\[
 \int_{I}\left|\log\left|\frac{G_{\varepsilon}(t)}{X_{\varepsilon}(t)}\right|\right|\,dt\ \le\ C_I\,\left(1+\sup_{t\in I}\|A(\tfrac12+\varepsilon+it)-A_N(\tfrac12+\varepsilon+it)\|_{\HS}\right),
\]
with $C_I$ independent of $N$. In particular, the HS tail control $\|A-A_N\|_{\HS}\to 0$ uniformly on \(\{\Re s\ge \tfrac12+\varepsilon\}\) implies precompactness of \(\{\log|G_{\varepsilon}/X_{\varepsilon}|\}\) in $L^1(I)$ and hence local-uniform convergence of the outer normalizations \(\mathcal O_{\varepsilon}\) along subsequences.
\end{proposition}
\begin{proof}[Proof sketch]
Carleman's bound (Lemma~\ref{lem:carleman}) gives \(|G_{\varepsilon}(t)|\le e^{\tfrac12\|A\|_{\HS}^2}\), while the HS continuity (Proposition~\ref{prop:HS-to-det2}) furnishes Lipschitz control for \(\log|\dettwo(I-A)|\) w.r.t. the HS norm. Stirling bounds control \(\log|X_{\varepsilon}(t)|\) on vertical lines uniformly on $I$ away from the finitely many zeros of \(\xi\) in the vertical strip under consideration. Integrating across small neighborhoods of those zeros, one uses that \(\log|\cdot|\) is locally integrable and that zeros are discrete with finite multiplicity to obtain an $L^1$ bound independent of \(\varepsilon\).
\end{proof}

\begin{remark}
Proposition~\ref{prop:L1loc} gives tightness for each fixed \(\varepsilon>0\). Uniform control as \(\varepsilon\downarrow 0\) follows from Theorem~\ref{thm:uniform-eps}.
\end{remark}

\subsection{Uniform \(\varepsilon\downarrow 0\) boundary control}\label{subsec:uniform-eps}
We now state the boundary theorem used for the outer-normalization route. See Subsection~\ref{subsec:smoothed-explicit} for the smoothed explicit-formula route and de-smoothing strategy.

\begin{theorem}[Uniform $L^1_{\mathrm{loc}}$ and Cauchy as \(\varepsilon\downarrow 0\)]\label{thm:uniform-eps}
For every compact interval $I\subset\R$ there exist constants $C_I$ and \(\varepsilon_0>0\) such that for all \(\varepsilon\in(0,\varepsilon_0)\),
\[
 \int_I \Bigl|\log\Bigl|\frac{\dettwo(I-A(\tfrac12+\varepsilon+it))}{\xi(\tfrac12+\varepsilon+it)}\Bigr|\Bigr|\,dt\ \le\ C_I,
\]
and the family is Cauchy in $L^1(I)$ as \(\varepsilon\downarrow 0\):
\[
 \lim_{\substack{\varepsilon,\delta\downarrow 0}}\ \int_I \Bigl|\log\Bigl|\frac{\dettwo(I-A(\tfrac12+\varepsilon+it))}{\xi(\tfrac12+\varepsilon+it)}\Bigr|\;-
 \log\Bigl|\frac{\dettwo(I-A(\tfrac12+\delta+it))}{\xi(\tfrac12+\delta+it)}\Bigr|\Bigr|\,dt\;=\;0.
\]
Consequently the outer normalizations \(\mathcal O_{\varepsilon}\to \mathcal O\) converge locally uniformly to an outer limit \(\mathcal O\) on \(\Omega\).
\end{theorem}
\begin{proof}
Fix a compact interval $I\subset\R$. Write $F(s):=\dettwo(I-A(s))$ and $X(s):=\xi(s)$. We show
\[
 u_\varepsilon(t):=\log\Bigl|\frac{F(\tfrac12+\varepsilon+it)}{X(\tfrac12+\varepsilon+it)}\Bigr|\in L^1(I)
\]
with $\|u_\varepsilon\|_{L^1(I)}\le C_I$ independent of \(\varepsilon\in(0,\varepsilon_0]\), and that \(\{u_\varepsilon\}\) is $L^1(I)$–Cauchy as \(\varepsilon\downarrow 0\). The standing hypotheses in Section~\ref{sec:appendix} (HS analyticity of $A$, analytic Fredholm property for $I-A$, and local analyticity of \(\xi\)) hold in the rectangle \(\mathcal R:=\{\tfrac12\le\sigma\le\tfrac12+\varepsilon_0,\ t\in I^{\!*}\}\subset\Omega\) for a slightly larger \(I^{\!*}\supset I\).

1) Uniform $L^1$ bound. By Lemma~\ref{lem:carleman}, for \(s\in\mathcal R\),
\[
 \log^+|F(s)|\;\le\;\tfrac12\,\|A(s)\|_{\HS}^2\;\le\;\tfrac12\,M_I^2.
\]
Apply the finite-domain Weierstrass factorization on \(\mathcal R\) to write each as a sum of a bounded harmonic term plus finitely many logarithmic spikes \(\log|s-\rho|\) corresponding to zeros \(\rho\) inside \(\mathcal R\). Along \(s=\tfrac12+\varepsilon+it\), the harmonic terms contribute a bounded amount to \(\int_I |u_\varepsilon(t)|dt\) by the maximum principle; each spike is uniformly integrable in \(t\) and uniformly in \(\varepsilon\) since \(\int_I |\log|\tfrac12+\varepsilon+it-\rho||\,dt\) is finite and locally uniform in \(\varepsilon\) for finitely many \(\rho\). Summing finitely many contributions yields $\|u_\varepsilon\|_{L^1(I)}\le C_I$.

2) $L^1$–Cauchy. For \(0<\delta<\varepsilon\le\varepsilon_0\), write
\[
 u_\varepsilon(t)-u_\delta(t)
 = \int_{\delta}^{\varepsilon} \partial_\sigma \Re\Bigl(\log F(\tfrac12+\sigma+it)-\log X(\tfrac12+\sigma+it)\Bigr)\,d\sigma.
\]
Using the Lipschitz control for \(\log\dettwo\) (Appendix~\ref{app:lipschitz}) and Lemma~\ref{lem:HS-variation}, we obtain
\[
 \int_I\bigl|\partial_\sigma\,\Re\log F(\tfrac12+\sigma+it)\bigr|\,dt\ \le\ C_I,
\]
uniformly for \(\sigma\in[\delta,\varepsilon]\). For the \(\xi\) term, standard Stirling bounds for \(\partial_\sigma\log X= X'/X\) on vertical lines (\cite{TitchmarshZeta}, Chap.~IV) yield
\[
  \int_I\bigl|\partial_\sigma\,\Re\log X(\tfrac12+\sigma+it)\bigr|\,dt\ \le\ C_I',
\]
uniformly in \(\sigma\in[\delta,\varepsilon]\). Fubini's theorem gives
\[
 \|u_\varepsilon-u_\delta\|_{L^1(I)}\;\le\;(C_I+C_I')\,|\varepsilon-\delta|\;\xrightarrow[\varepsilon,\delta\downarrow 0]{}\;0.
\]
Therefore \(u_\varepsilon\) is uniformly $L^1$–bounded and $L^1$–Cauchy on \(I\) provided the derivative bounds hold. This implication is formalized in Lemma~\ref{lem:desmoothing} below. The Poisson–Hilbert representation of outer functions on the half-plane (with \(u_\varepsilon\) as boundary data) then yields local-uniform convergence of outer normalizations \(\mathcal O_\varepsilon\to \mathcal O\), and the a.e. boundary modulus \(|\Theta(\tfrac12+it)|=1\) of the inner factor. The Schur bound in \(\Omega\) follows by the maximum principle.
\end{proof}

\begin{lemma}[\(\xi\)-derivative $L^1$ bound on vertical segments]\label{lem:xi-deriv-L1}
Let $I\Subset\R$ and $\sigma\in[\tfrac12,\tfrac12+\varepsilon_0]$. Then
\[
 \int_I \Big|\partial_\sigma\,\Re\log\xi(\sigma+it)\Big|\,dt\ \le\ C_I',
\]
with $C_I'$ independent of $\sigma$.
\end{lemma}
\begin{proof}[Sketch]
Write \(\partial_\sigma\,\Re\log\xi=\Re(\xi'/\xi)\) and use the explicit zero-factorization: on vertical lines, one has
\[
 \Re\frac{\xi'}{\xi}(\sigma+it)\ =\ \sum_{\rho} m_\rho\,\Re\frac{1}{\sigma+it-\rho}\ +\ \text{archimedean/polynomial terms},
\]
where the latter are uniformly bounded on compact $I$ by Stirling estimates and continuity. For each zero \(\rho=\beta+i\gamma\), the contribution integrates as
\[\int_I \Big|\Re\frac{1}{\sigma+it-\rho}\Big|\,dt\ \le\ \int_{t\in I} \frac{|\sigma-\beta|}{(\sigma-\beta)^2+(t-\gamma)^2}\,dt\ \le\ \pi,
\]
uniformly in \(\sigma\in[\tfrac12,\tfrac12+\varepsilon_0]\) (standard integral). Only finitely many zeros intersect the strip above $I$ within a bounded distance; the tail is summable by the classical bound $N(T)\ll T\log T$. Summing over zeros and adding the bounded archimedean contribution yields the claim.
\end{proof}
\begin{lemma}[det$_2$-derivative $L^1$ bound on vertical segments]\label{lem:det2-deriv-L1}
Let $I\Subset\R$ and $\sigma\in[\tfrac12+\delta,\tfrac12+\varepsilon_0]$ with $\delta>0$. Then
\[
 \int_I \Big|\partial_\sigma\,\Re\log\dettwo\big(I-A(\sigma+it)\big)\Big|\,dt\ \le\ C_I(\delta).
\]
\end{lemma}
\begin{proof}[Sketch]
Using the absolutely convergent expansion for \(\sigma>\tfrac12\),
\[\partial_\sigma\,\Re\log\dettwo(I-A(\sigma+it))\ =\ \sum_{k\ge 2}\sum_{p\in\PP} (\log p)\,p^{-k\sigma}\cos(k t\log p),\]
we bound
\[\int_I \Big|\sum_{k,p}(\log p)\,p^{-k\sigma}\cos(k t\log p)\Big|dt\ \le\ \sum_{k,p}(\log p)\,p^{-k\sigma}\int_I |\cos(k t\log p)|\,dt\ \le\ |I|\sum_{k,p}(\log p)\,p^{-k\sigma}.
\]
For \(\sigma\ge \tfrac12+\delta\), the double series converges by comparison with \(\sum_{k\ge 2} p^{-k(\tfrac12+\delta)}\log p\); in particular the $k=2$ line is \(\sum_p (\log p)\,p^{-1-2\delta}<\infty\). Hence the bound $C_I(\delta)$ follows.
\end{proof}

\begin{remark}
At the boundary \(\sigma\downarrow \tfrac12\), oscillatory (smoothed) bounds (Lemma~\ref{lem:det2-smoothed-target}) combined with a standard duality argument on \(W^{2,1}(I)\) test functions yield uniform \(L^1\) control in the limit; see Lemma~\ref{lem:uniform-derivative-L1} and Proposition~\ref{prop:desmoothing} for the precise Cauchy transfer.
\end{remark}

\begin{lemma}[De-smoothing: bounded $L^1$ derivative implies $L^1$–Cauchy]\label{lem:desmoothing}
Let \(I\Subset\R\) and let \(u_\sigma\in L^1(I)\) be defined for \(\sigma\in(0,\varepsilon_0]\), differentiable in \(\sigma\), with
\[
  \int_I |\partial_\sigma u_\sigma(t)|\,dt\ \le\ C_I\qquad\text{for all }\sigma\in(0,\varepsilon_0].
\]
Then \(\{u_\varepsilon\}_{\varepsilon\downarrow 0}\) is Cauchy in $L^1(I)$.
\end{lemma}
\begin{proof}
For \(0<\delta<\varepsilon\le\varepsilon_0\), the fundamental theorem of calculus gives
\(u_\varepsilon-u_\delta=\int_\delta^\varepsilon \partial_\sigma u_\sigma\,d\sigma\).
Minkowski's integral inequality yields
\[
  \|u_\varepsilon-u_\delta\|_{L^1(I)}\ \le\ \int_\delta^\varepsilon \int_I |\partial_\sigma u_\sigma(t)|\,dt\,d\sigma\ \le\ C_I(\varepsilon-\delta),
\]
which tends to $0$ as \(\varepsilon,\delta\downarrow 0\).
\end{proof}
\begin{remark}
We take \(C_c^2(I)\) test functions dense in \(W^{2,1}_0(I)\) so that smoothed bounds transfer to the unsmoothed case by duality; the uniform bound on \(\int_I|\partial_\sigma u_\sigma|\) is independent of \(\sigma\), so no loss appears as \(\varepsilon\downarrow 0\).
\end{remark}
\begin{remark}
The uniform-in-\(\varepsilon\) local $L^1$ control of Theorem~\ref{thm:uniform-eps} follows by combining the smoothed det$_2$ estimate of Lemma~\ref{lem:det2-smoothed-target} with the corresponding \(\xi\)-term bounds (\cite{TitchmarshZeta}, Chap.~IV) and the de-smoothing Lemma~\ref{lem:desmoothing}.
\end{remark}

\subsection{Smoothed explicit-formula route and de-smoothing}\label{subsec:smoothed-explicit}
We complement the preceding proof with an unconditional, smoothed route that avoids any zero-free hypothesis and isolates prime/zero cancellation at the level of test functions.

\begin{lemma}[Smoothed uniform bound via an explicit formula]\label{lem:smoothed-explicit}
Let \(I\Subset\R\) and \(\varphi\in C_c^{\infty}(I)\). Set \(u_\varepsilon(t):=\log\big|\dettwo(I-A(\tfrac12+\varepsilon+it))\big|-\log\big|\xi(\tfrac12+\varepsilon+it)\big|\). Then there is \(C(\varphi)\) independent of \(\varepsilon\in(0,\varepsilon_0]\) such that
\[
 \Big|\int_{\R} \varphi(t)\,u_\varepsilon(t)\,dt\Big|\ \le\ C(\varphi),\qquad \Big|\int_{\R} \varphi(t)\,\big(u_\varepsilon(t)-u_\delta(t)\big)\,dt\Big|\ \le\ C(\varphi)\,|\varepsilon-\delta|.
\]
\end{lemma}

\begin{lemma}[Prime-power representation for \(\partial_\sigma\Re\log\dettwo\); unit local weights]\label{lem:pp-rep-det2}
Let \(A(s)\) be the prime-diagonal operator \(A(s)e_p:=p^{-s}e_p\) on \(\ell^2(\PP)\), with \(s=\sigma+it\) and \(\sigma>\tfrac12\). Then
\[
  \partial_\sigma\,\Re\log\dettwo\!\big(I-A(s)\big)
  \\= -\,\Re\sum_{p}\sum_{k\ge 2} c_{p,k}\,(\log p)\,p^{-k(\sigma+it)},\qquad c_{p,k}\equiv -1,
\]
so in particular \(|c_{p,k}|\le 1\) uniformly in \(p,k,\sigma\).
\end{lemma}
\begin{proof}
For \(\sigma>\tfrac12\) one has \(\|A(s)\|\le 2^{-\sigma}<1\), and the standard HS expansion holds:
\[
  \log\dettwo(I-A(s))\;=\;-\sum_{k\ge 2} \frac{\Tr(A(s)^k)}{k}\;=\;-\sum_{k\ge 2}\frac1k\sum_{p}p^{-ks},
\]
with absolute convergence. Differentiating termwise in \(\sigma\) (justified by absolute convergence of \(\sum_{k\ge 2}\sum_p (\log p)\,p^{-k\sigma}\)) gives
\[
  \partial_\sigma \log\dettwo(I-A(s))
  \\= -\sum_{k\ge 2}\frac1k\sum_p (-k\log p)\,p^{-ks}
  \\=\sum_{k\ge 2}\sum_p (\log p)\,p^{-ks}.
\]
Taking real parts yields the claim with \(c_{p,k}\equiv -1\).
\end{proof}

\begin{lemma}[Det$_2$ smoothed bound; uniform in \(\sigma\)]\label{lem:det2-smoothed-target}
Fix \(\varepsilon_0>0\) and a compact interval \(I\Subset\R\). Let \(\varphi\in C_c^2(I)\). For \(s=\sigma+it\) with \(\sigma\in(\tfrac12,\tfrac12+\varepsilon_0]\) one has the absolutely convergent expansion
\[
 \partial_\sigma\,\Re\log\dettwo\big(I-A(s)\big)
 \;=\; \sum_{k\ge 2}\sum_{p\in\PP} (\log p)\,p^{-k\sigma}\cos\big(k t\log p\big).
\]
Then there exists a finite constant (uniform in \(\sigma\in(\tfrac12,\tfrac12+\varepsilon_0]\))
\[
 C_*\ :=\ \sum_{p}\sum_{k\ge 2}\frac{p^{-k/2}}{k^2\,\log p}
\]
such that, uniformly for \(\sigma\in(\tfrac12,\tfrac12+\varepsilon_0]\),
\[
 \Big|\int_{\R} \varphi(t)\,\partial_\sigma\,\Re\log\dettwo\big(I-A(\sigma+it)\big)\,dt\Big|
 \ \le\ C_*\,\|\varphi''\|_{L^1(I)}.
\]
\end{lemma}

\begin{lemma}[Smoothed bound for the \(\xi\)-term; uniform in \(\sigma\)]\label{lem:xi-smoothed}
Fix \(\varepsilon_0>0\) and a compact interval \(I\Subset\R\). Let \(\varphi\in C_c^2(I)\) and \(s=\sigma+it\) with \(\sigma\in(\tfrac12,\tfrac12+\varepsilon_0]\). Then there exists a finite constant \(C_\xi(\varphi)\), independent of \(\sigma\) in this range, such that
\[
 \Big|\int_{\R}\varphi(t)\,\partial_\sigma\,\Re\log\xi(\sigma+it)\,dt\Big|\ \le\ C_\xi(\varphi).
\]
\end{lemma}
\begin{proof}
Write \(\xi(s)=\tfrac12 s(1-s)\,\pi^{-s/2}\Gamma(s/2)\,\zeta(s)\). Then
\[
 \partial_\sigma\,\Re\log\xi(s)\;=\;\partial_\sigma\,\Re\log\zeta(s)\; +\; \Re\frac{\psi(s/2)}{2}\; -\; \tfrac12\log\pi\; +\; \partial_\sigma\,\Re\log(s(1-s)),
\]
with \(\psi=\Gamma'/\Gamma\). On the compact strip \(\{\tfrac12<\sigma\le\tfrac12+\varepsilon_0,\ t\in I\}\) the last three terms are continuous in \((\sigma,t)\), so their \(\varphi\)–weighted integrals are bounded by \(C_0(\varphi)\) uniformly in \(\sigma\).

For \(\partial_\sigma\,\Re\log\zeta\), use the Euler product for \(\Re s>1\),
\(\log\zeta(s)=\sum_{p}\sum_{k\ge1}p^{-ks}/k\),
differentiate in \(\sigma\), take real parts, and test against \(\varphi\in C_c^2(I)\). Arguing by analytic continuation under the test (Cauchy's theorem on vertical rectangles), one obtains
\[
 \int \varphi(t)\,\partial_\sigma\,\Re\log\zeta(\sigma+it)\,dt\;=\;\sum_{p}\sum_{k\ge1} (\log p)\,p^{-k\sigma}\int \varphi(t)\cos(kt\log p)\,dt.
\]
Two integrations by parts give \(\big|\int \varphi(t)\cos(\omega t)\,dt\big|\le \|\varphi''\|_{L^1(I)}\,\omega^{-2}\) for \(\omega>0\). Hence
\[
 \Big|\int \varphi\,\partial_\sigma\,\Re\log\zeta(\sigma+it)\Big|\ \le\ \|\varphi''\|_{L^1(I)}\sum_{p}\sum_{k\ge1}\frac{(\log p)\,p^{-k\sigma}}{(k\log p)^2}\ \le\ \|\varphi''\|_{L^1(I)}\sum_{p}\sum_{k\ge1}\frac{p^{-k/2}}{k^2\,\log p},
\]
uniformly for \(\sigma\in(\tfrac12,\tfrac12+\varepsilon_0]\), since the rightmost double series converges (the \(k\!=\!1\) line gives \(\sum_{p}(p\log p)^{-1}<\infty\), and \(k\ge2\) decays faster). Taking \(C_\xi(\varphi):=C_0(\varphi)+\|\varphi''\|_{L^1(I)}\sum_{p}\sum_{k\ge1}p^{-k/2}/(k^2\log p)\) proves the claim.
\end{proof}

\begin{proof}[Proof sketch]
Expand \(\log\dettwo(I-A)\) as \(-\sum_{p}\sum_{k\ge2}p^{-ks}/k\) for \(\Re s>1\) and continue termwise to the open strip by testing against \(\varphi\in C_c^2(I)\). Differentiating in \(\sigma\) and taking real parts yields the exact series in the statement. Interchanging sum and integral is justified by absolute convergence on compact \(\sigma\)-intervals.

For each frequency \(\omega=k\log p\ge 2\log 2\), two integrations by parts give
\[
\Bigl|\int_{\R}\varphi(t)\cos(\omega t)\,dt\Bigr|\ \le\ \frac{\|\varphi''\|_{L^1(I)}}{\omega^2}.
\]
Hence
\[
\Bigl|\int \varphi(t)\,\partial_\sigma\Re\log\dettwo(I-A(\sigma+it))\,dt\Bigr|
\le \|\varphi''\|_{L^1}\sum_{p}\sum_{k\ge2}\frac{(\log p)\,p^{-k\sigma}}{(k\log p)^2}
\le \|\varphi''\|_{L^1}\sum_{p}\sum_{k\ge2}\frac{p^{-k/2}}{k^2\,\log p},
\]
uniformly for \(\sigma\in(\tfrac12,\tfrac12+\varepsilon_0]\), since the rightmost double series converges (the \(k\ge2\) tail gives \(p^{-k/2}\) and \(\sum_{p}(p\log p)^{-1}<\infty\)). This proves the claim.
\end{proof}

\begin{remark}
The corresponding bound for \(\partial_\sigma\,\Re\log\xi(\sigma+it)=\Re(\xi'/\xi)\) on vertical segments is standard (e.g., \cite{TitchmarshZeta}, Chap.~IV). Lemma~\ref{lem:det2-smoothed-target} thus supplies the smoothed, \(\sigma\)-uniform det$_2$ estimate needed to complete Theorem~\ref{thm:uniform-eps} via Lemma~\ref{lem:desmoothing}.
\end{remark}
\begin{proof}
Write \(\log\dettwo(I-A)\) as \(-\sum_{p}\sum_{k\ge 2} p^{-ks}/k\) and \(\log\zeta(s)=\sum_{p}\sum_{k\ge 1} p^{-ks}/k\) for \(\Re s>1\), then continue meromorphically to \(\Re s>\tfrac12\) in the distributional sense by testing against \(\varphi\). The completed \(\xi\) adds the archimedean factor \(\log\Gamma(s/2)-\tfrac{s}{2}\log\pi\) and a polynomial. An explicit formula (Weil-type) for smooth compactly supported \(\varphi\) (see, e.g., Edwards~\cite[Ch.~1, §5]{Edwards} or Iwaniec--Kowalski~\cite[Ch.~5]{IwaniecKowalski}) gives
\[
 \int \varphi\,\Re\log\zeta(\sigma+it)\,dt\ =\ \sum_{\rho} \Phi_{\varphi}(\rho)\ +\ \text{poly}(\sigma;\varphi)\ -\sum_{p,m}\frac{\log p}{p^{m\sigma}}\,g_{\varphi}(m\log p),
\]
with \(g_{\varphi}\) rapidly decaying and \(\Phi_{\varphi}\) depending only on \(\varphi\) and \(\sigma\). Subtract the det$_2$ prime-power side (starting at \(k=2\)) and the archimedean terms of \(\xi\) to obtain a uniformly bounded expression in \(\varepsilon\). Differentiating in \(\sigma\) brings down factors \(\log p\) and yields an extra \(m\) in the zero sum; rapid decay of \(g_{\varphi}\) and standard zero-density bounds imply the Lipschitz estimate in \(\varepsilon\).
\end{proof}

\begin{lemma}[Uniform \(\sigma\)-derivative $L^1$ bounds on short intervals]\label{lem:uniform-derivative-L1}
Fix a compact interval \(I\subset\R\) and \(\sigma\in[\tfrac12,\tfrac12+\varepsilon_0]\). Then
\[
 \int_I \Big|\partial_\sigma\,\Re\log\dettwo\big(I-A(\sigma+it)\big)\Big|\,dt\ \le\ C_I,
\]
uniformly in \(\sigma\), and
\[
 \int_I \Big|\partial_\sigma\,\Re\log\xi(\sigma+it)\Big|\,dt\ \le\ C'_I,
\]
uniformly in \(\sigma\).
\end{lemma}
\begin{proof}
For \(\xi\), write \(\partial_\sigma\,\Re\log\xi=\Re(\xi'/\xi)=\sum_{\rho} m_{\rho}\,\Re(\sigma+it-\rho)^{-1}+\text{arch}\). Each zero contributes \(\int_I |\Re(\sigma+it-\rho)^{-1}|\,dt\le \pi\), and only finitely many zeros intersect the vertical strip over \(I\) for fixed \(\sigma\in[\tfrac12,\tfrac12+\varepsilon_0]\); tails are summable by \(N(T)\sim \tfrac{T}{2\pi}\log T\). The archimedean/polynomial pieces are uniformly bounded on \(I\). For det$_2$, test \(\partial_\sigma\,\Re\log\dettwo(I-A)\) against smooth cutoffs \(\varphi_n\to 1_I\); Lemma~\ref{lem:smoothed-explicit} provides bounds uniform in \(n\) and \(\sigma\). Letting \(n\to\infty\) gives the claimed \(L^1\) bound.
\end{proof}

\begin{proposition}[Smoothed-to-unsmoothed Cauchy transfer]\label{prop:desmoothing}
Let \(u_\varepsilon\) be as above. For each compact \(I\Subset\R\) there exists \(C_I\) such that for all \(0<\delta<\varepsilon\le\varepsilon_0\),
\[
 \|u_\varepsilon-u_\delta\|_{L^1(I)}\ \le\ C_I\,|\varepsilon-\delta|.
\]
\end{proposition}
\begin{proof}
By Lemma~\ref{lem:uniform-derivative-L1}, \(\int_I |\partial_\sigma u_\sigma(t)|\,dt\le C_I\) uniformly in \(\sigma\in[\delta,\varepsilon]\). Therefore, for \(0<\delta<\varepsilon\le\varepsilon_0\),
\[
 u_\varepsilon-u_\delta\ =\ \int_\delta^\varepsilon \partial_\sigma u_\sigma\,d\sigma,
\]
and Minkowski's integral inequality gives
\[
 \|u_\varepsilon-u_\delta\|_{L^1(I)}\ \le\ \int_\delta^\varepsilon\!\int_I |\partial_\sigma u_\sigma(t)|\,dt\,d\sigma\ \le\ C_I\,|\varepsilon-\delta|.
\]
\end{proof}

\begin{remark}
The uniform-in-\(\varepsilon\) boundary control (Theorem~\ref{thm:uniform-eps}) follows by testing the derivatives against compactly supported smooth \(\varphi\) and combining the smoothed bounds in Lemmas~\ref{lem:det2-smoothed-target} and~\ref{lem:xi-smoothed} with the de-smoothing Lemma~\ref{lem:desmoothing}.
\end{remark}

\begin{lemma}[Boundary positive-real from smoothed route]\label{lem:boundary-posreal}
Assume the smoothed explicit-formula bounds of Lemmas~\ref{lem:det2-smoothed-target} and~\ref{lem:xi-smoothed} and the de-smoothing Lemma~\ref{lem:desmoothing}. If, in addition, the smoothed boundary distribution for \(\partial_\sigma\Re\log\big(\dettwo(I-A)/\xi\big)\) is nonnegative in the limit \(\varepsilon\downarrow 0\) when tested against nonnegative \(\varphi\in C_c^\infty(\R)\), then the boundary hypothesis \emph{(P+)} holds for \(\mathcal J=\dettwo(I-A)/(\mathcal O\,\xi)\).
\end{lemma}

\begin{remark}
Lemma~\ref{lem:boundary-posreal} isolates the precise point where the smoothed explicit-formula route must deliver a sign (positive real part) rather than mere $L^1$ bounds. This replaces earlier "outer is trivial" or boundary unimodularity claims for \(\Theta\).
\end{remark}

\begin{proposition}[Phase-variation test: (P+) forces holomorphy]\label{prop:Pplus-holomorphy}
Let \(\Omega=\{\Re s>\tfrac12\}\), \(F(s):=\dettwo(I-A(s))/\xi(s)\), and for \(t\in\R\) set
\[
 u(t):=\log\big|F(\tfrac12+it)\big|,\qquad
 \mathsf H[u']:=\text{the Hilbert transform of }u'(t).
\]
Then for every nonnegative \(\phi\in C_c^\infty(\R)\) one has
\[
\int_{\R}\!\phi(t)\,\Big(\Im\frac{\xi'}{\xi}-\Im\frac{\dettwo'}{\dettwo}+\mathsf H[u']\Big)\!\Big(\tfrac12+it\Big)\,dt
\;=\;\sum_{\substack{\rho=\beta+i\gamma\\ \Re\rho>\tfrac12}}\! 2(\beta-\tfrac12)\,\big(P_{\beta-\tfrac12}\!\ast\phi\big)(\gamma)
\; +\; \pi\!\!\sum_{\substack{\gamma\in\R\\ \xi(\tfrac12+i\gamma)=0}}\! m_\gamma\,\phi(\gamma),
\]
where \(P_a(x)=\frac{1}{\pi}\frac{a}{a^2+x^2}\) and \(m_\gamma\) is the multiplicity of the critical-line zero at \(t=\gamma\). In particular, the right-hand side is \(\ge 0\) for every \(\phi\ge 0\).
\end{proposition}
\begin{proof}
Factor \(F=I\,O\) on \(\Omega\) with \(O\) outer and \(I\) inner. By Lemma~\ref{lem:outer-phase-HT}, the boundary argument of \(O\) satisfies \(\frac{d}{dt}\Arg O(\tfrac12+it)=\mathsf H[u'](t)\) in \(\mathcal D'(\R)\). The inner factor \(I\) is the product of Blaschke terms for poles \(\rho=\beta+i\gamma\) of \(F\) in \(\Omega\) (zeros of \(\xi\) with \(\beta>\tfrac12\)) and a singular inner supported at ordinates \(\gamma\) with \(\xi(\tfrac12+i\gamma)=0\). For a pole at \(\rho\), the half-plane Blaschke factor \(C_\rho(s)=(s-\overline\rho)/(s-\rho)\) has
\[
\frac{d}{dt}\Arg C_\rho(\tfrac12+it)=-\,2(\beta-\tfrac12)\,P_{\beta-\tfrac12}(t-\gamma),
\]
and each critical-line zero contributes \(-\pi m_\gamma\,\delta_\gamma\) to the phase derivative. Summing gives
\[
\frac{d}{dt}\Arg F(\tfrac12+it)=\mathsf H[u'](t)
-\sum_{\substack{\rho=\beta+i\gamma\\ \Re\rho>\tfrac12}}\! 2(\beta-\tfrac12)\,P_{\beta-\tfrac12}(t-\gamma)
-\pi\!\!\sum_{\substack{\gamma\in\R\\ \xi(\tfrac12+i\gamma)=0}}\! m_\gamma\,\delta_\gamma.
\]
Since \(\frac{d}{dt}\Arg F=\Im(F'/F)=\Im(\dettwo'/\dettwo)\) on the boundary, rearranging and testing against \(\phi\ge 0\) yields the stated identity and positivity.
\end{proof}

\begin{proposition}[Local phase-cone certificate on \(I\)]
Fix a compact interval $I=[T_1,T_2]$ containing no ordinate \(\gamma\) with \(\xi(\tfrac12+i\gamma)=0\). Define
\[
 w(t):=\Arg\dettwo(\tfrac12+it)-\Arg\xi(\tfrac12+it)-\mathsf H[u](t),\qquad u(t):=\log|F(\tfrac12+it)|.
\]
Normalize $w$ by a unimodular constant so that $w(t_0)=0$ for some $t_0\in I$. Then $-w'$ is a nonnegative finite measure on $I$ and
\[
 \int_I (-w')\,dt=\sum_{\substack{\rho=\beta+i\gamma\\ \Re\rho>\tfrac12}}\! 2(\beta-\tfrac12)\Big[\arctan\frac{T_2-\gamma}{\beta-\tfrac12}-\arctan\frac{T_1-\gamma}{\beta-\tfrac12}\Big].
\]
In particular, if \(\displaystyle \int_I (-w')\,dt\le \pi/2\), then $w(t)\in[-\tfrac\pi2,\tfrac\pi2]$ for a.e. $t\in I$, and hence \(\Re\big(2\mathcal J(\tfrac12+it)\big)\ge 0\) a.e. on $I$ with \(\mathcal J=F/\mathcal O\).
\end{proposition}
\subsection*{Target (P+) via Carleson control of off-critical zeros}\label{subsec:Pplus-Carleson}
We isolate a sufficient condition for \emph{(P+)} in terms of a Carleson-type bound on the off-critical zero distribution.

\begin{definition}[Zero-side measure and Carleson boxes]
For each zero \(\rho=\beta+i\gamma\) of \(\xi\) with \(\beta>\tfrac12\), set \(a(\rho):=\beta-\tfrac12>0\). Define the discrete measure on the open half-plane \(\{\sigma>\tfrac12\}\)
\[
 \mu\ :=\ \sum_{\rho:\,\Re\rho>1/2}\ 2\,a(\rho)\,\delta_{(\tfrac12+a(\rho),\,\gamma)}.
\]
For an interval \(I=[T_1,T_2]\subset\R\), its Carleson (Whitney) box is
\[
 Q(I)\ :=\ \Big\{\, s=\sigma+it:\ 0<\sigma-\tfrac12<|I|,\ t\in I\,\Big\}.
\]
We say \(\mu\) has Carleson constant \(\mathsf C\) if \(\mu(Q(I))\le \mathsf C\,|I|\) for every bounded interval \(I\).
\end{definition}

\begin{theorem}[(P+) from Carleson control]\label{thm:Pplus-from-Carleson}
Assume the outer normalization of Subsection~\ref{subsec:boundary-unitarity} so that \(\mathcal J=\dettwo(I-A)/(\mathcal O\,\xi)\) has a.e. boundary values with \(|\mathcal J(\tfrac12+it)|=1\). If the zero-side measure \(\mu\) has Carleson constant \(\mathsf C\le \pi/2\), then \emph{(P+)} holds:
\[
 \Re\big(2\,\mathcal J(\tfrac12+it)\big)\ \ge\ 0\quad\text{for a.e. }t\in\R.
\]
\end{theorem}
\begin{proof}[Sketch]
By Proposition~\ref{prop:phase-velocity-identity}, for nonnegative \(\phi\in C_c^\infty(I)\) one has
\[
 \int\!\phi\,\Big(\Im\frac{\xi'}{\xi}-\Im\frac{\dettwo'}{\dettwo}+\mathsf H[u']\Big)\Big(\tfrac12+it\Big)dt\ =\ \int_{\{\Re s>1/2\}} P_{s}\![\phi] \ d\mu(s)\ \ge 0,
\]
where \(P_{s}[\phi]\) denotes the Poisson extension to height \(\Re s-\tfrac12\) evaluated at \(\Im s\). The left-hand side equals \(\int_I \phi(t)\,(-w')\,dt\) with \(w\) the normalized phase mismatch (Proposition~\ref{prop:phase-velocity-identity}). Since \(\|P_{s}[\phi]\|_{L^\infty}\le 1\) and the Poisson kernel has unit \(t\)–mass, the Carleson bound yields
\[
 \int_I (-w')\,dt\ \le\ \mu(Q(I))\ \le\ (\pi/2)\,|I|.
\]
Normalizing \(\phi\) to approximate the indicator of \(I\) and dividing by \(|I|\), one obtains \(\int_I (-w')\le \pi/2\). By the phase-cone criterion this implies \(w\in[-\pi/2,\pi/2]\) a.e. on \(I\), hence \(\Re(2\mathcal J)\ge 0\) a.e. on \(I\). Exhaust \(\R\) by such intervals to conclude (P+).
\end{proof}

\begin{lemma}[Reduction to a short-interval Carleson bound]\label{lem:HP-Carleson}
Let \(I\subset\R\) be a bounded interval avoiding ordinates of critical-line zeros. If \(\mu(Q(I))\le \pi/2\), then \(\Re(2\mathcal J)\ge 0\) a.e. on \(I\). Consequently, if \(\mu\) has Carleson constant \(\le \pi/2\), then \emph{(P+)} holds a.e. on \(\R\).
\end{lemma}
\begin{remark}[Analytic number theory target]
It suffices to prove the short-interval Carleson bound \(\mu(Q(I))\le \pi/2\) unconditionally. This can be attacked using unconditional zero-density estimates for \(\zeta\) (e.g., Ingham–Huxley/Montgomery–Vaughan) and Littlewood-type bounds \(\sum_{\gamma\le T}(\beta-\tfrac12)\ll \log T\), combined with the Poisson localization inherent in \(Q(I)\). Establishing this bound yields \emph{(P+)} via Theorem~\ref{thm:Pplus-from-Carleson} and hence global Schur/PSD for \(\Theta\).
\end{remark}

\begin{corollary}[Adaptive cover criterion for (P+)]\label{cor:adaptive-cover}
Suppose there exists a function \(L:(0,\infty)\to(0,\infty)\) and \(T_0>0\) such that for all \(T\ge T_0\), with \(I_T:=[T-L(T),\,T+L(T)]\) one has \(\mu(Q(I_T))\le \pi/2\). Then \emph{(P+)} holds a.e. on \(\R\).
\end{corollary}
\begin{proof}
The intervals \(I_T\) (together with finitely many intervals covering the bounded range \([0,T_0]\)) form a countable cover of \(\R\) up to the measure-zero set of critical-line ordinates. By Lemma~\ref{lem:HP-Carleson}, on each \(I_T\) we have \(\Re(2\mathcal J)\ge 0\) a.e. Taking the union yields (P+) a.e. on \(\R\).
\end{proof}

\begin{lemma}[Littlewood bound \(\Rightarrow\) adaptive short-interval mass]\label{lem:littlewood-adaptive}
Let \(S(T):=\sum_{0<\gamma\le T,\ \beta>1/2}(\beta-\tfrac12)\). Suppose there exists \(C_L>0\) with \(S(T)\le C_L\,\log(2+T)\) for all \(T\ge 0\) (classical Littlewood-type bound). Then there exist constants \(c>0\) and \(T_0\ge 1\) such that, for \(L(T):=c/\log(2+T)\) and \(I_T=[T-L(T),T+L(T)]\), one has
\[\mu\big(Q(I_T)\big)\ \le\ \frac{\pi}{2}\qquad (T\ge T_0).\]
\end{lemma}
\begin{proof}
By definition, \(\mu(Q(I_T))=\sum_{\substack{\gamma\in I_T\\ 0<\beta-\tfrac12< L(T)}} 2(\beta-\tfrac12)\ \le\ 2\sum_{\substack{\gamma\in I_T\\ \beta>1/2}} (\beta-\tfrac12)\). The latter is bounded by the telescoping difference \(2\big(S(T+L(T)) - S(T-L(T))\big)\). Using the hypothesis, for all large \(T\),
\[
 \mu(Q(I_T))\ \le\ 2C_L\,\Big(\log(2+T+L(T)) - \log(2+T-L(T))\Big)
 \ \le\ \frac{4C_L\,L(T)}{2+T-L(T)}\ \le\ \frac{4C_L\,c}{T\,\log(2+T)}.
\]
Choose \(T_0\) so that \(\frac{4C_L c}{T_0\,\log(2+T_0)}\le \pi/2\); then for all \(T\ge T_0\) the same inequality holds with \(T\) in place of \(T_0\). This proves the claim.
\end{proof}

\begin{corollary}[(P+) under Littlewood bound]\label{cor:Pplus-Littlewood}
Assume the outer normalization of Subsection~\ref{subsec:boundary-unitarity} and the Littlewood-type bound in Lemma~\ref{lem:littlewood-adaptive}. Then \emph{(P+)} holds a.e. on \(\R\).
\end{corollary}
\begin{proof}
Apply Lemma~\ref{lem:littlewood-adaptive} and Corollary~\ref{cor:adaptive-cover}, adding finitely many short intervals to cover \([0,T_0]\).
\end{proof}

\begin{theorem}[Global Schur/PSD and RH under Littlewood bound]\label{thm:global-RH-Littlewood}
Under the hypotheses of Corollary~\ref{cor:Pplus-Littlewood}, \(2\mathcal J\) is Herglotz on \(\Omega\) by Poisson, and thus \(\Theta=(2\mathcal J-1)/(2\mathcal J+1)\) is Schur on \(\Omega\). Consequently, by Theorem~\ref{thm:brf-rh-final}, RH holds.
\end{theorem}

\begin{conjecture}[Short-interval Poisson mass bound]\label{conj:short-interval-poisson}
For every bounded interval \(I\subset\R\) that avoids ordinates of critical-line zeros, the zero-side measure \(\mu\) satisfies
\[
 \mu(Q(I))\ \le\ \frac{\pi}{2}.
\]
\end{conjecture}

\begin{corollary}[Conjecture~\ref{conj:short-interval-poisson} \(\Rightarrow\) (P+) \(\Rightarrow\) RH]\label{cor:conj-to-RH}
If Conjecture~\ref{conj:short-interval-poisson} holds, then by Theorem~\ref{thm:Pplus-from-Carleson} the boundary positive-real condition \emph{(P+)} holds. Consequently, by Theorem~\ref{thm:global-PSD}, \(\Theta\) is Schur on \(\Omega\). Therefore RH follows by Theorem~\ref{thm:brf-rh-final}.
\end{corollary}
\begin{remark}[Zero-side bound for \(\int_I(-w')\)]
If \(\operatorname{dist}\big(I,\{\Im\rho:\ \Re\rho>\tfrac12\}\big)\ge \delta>0\), then the sum for \(\int_I(-w')\) is \(\ll |I|\sum_{\rho}(\Re\rho-\tfrac12)/\delta^2\), so a mild local zero-density bound yields \(\int_I(-w')\le \pi/2\) on short intervals.
\end{remark}

\begin{remark}[Pick-matrix discretization]\label{rem:pick-certificate}
Equivalently, fix nodes $s_j=\tfrac12+\sigma+i t_j$ with $t_j\in I$ and $\sigma>0$. Positivity of the half-plane Pick matrix \(\big((1-\Theta(s_j)\overline{\Theta(s_k)})/(s_j+\overline{s_k}-1)\big)_{j,k}\) for arbitrarily fine grids and $\sigma\downarrow 0$ is equivalent to the phase-cone on $I$.
\end{remark}

\subsection{Global damping/weighting for alignment (Schur-test formulation)}\label{subsec:global-damping}
As an orthogonal route to compact-by-compact tuning, one may introduce a single global diagonal weight \(D(s)\) and a fixed damping factor \(\eta\in(0,1)\) to obtain \(K\)-independent Schur bounds via the Schur test. In kernel form, if the off-diagonal envelope enjoys either exponential tails \(|K(x,y)|\lesssim e^{-\gamma d(x,y)}\) or polynomial tails \(|K(x,y)|\lesssim (1+d(x,y))^{-\beta}\) on a doubling space of dimension \(n\), then one can choose weights
\[
 D(s)f(x)=e^{\,\sigma\,d(x,x_0)}f(x)\quad\text{or}\quad D(s)f(x)=(1+d(x,x_0))^{\sigma} f(x)
\]
with \(\sigma\) below a tail-dependent threshold, so that the conjugated operator \(D(-s)\,T\,D(s)\) is uniformly bounded on \(L^p\) for a given \(p\). Picking \(\eta=(1-\varepsilon)/\|D(-s)TD(s)\|_{p\to p}\) then yields a global contraction bound independent of compacts. This supplies a single, globally defined "Schur sequence" without per-compact parameter choices.

\subsection{Cayley-difference control on compacts}\label{subsec:Cayley-difference}
We record a simple inequality linking differences after the Cayley transform to differences before it.

\begin{lemma}[Cayley-difference bound]\label{lem:Cayley-diff}
Let \(K\subset\Omega\) be compact. Suppose \(H_1,H_2\) are holomorphic on a neighborhood of \(K\) and satisfy \(\inf_{s\in K}|H_j(s)+1|\ge \delta_K>0\) and \(\sup_{s\in K}|H_j(s)|\le M_K\) for \(j=1,2\). Define \(\Theta_j=(H_j-1)/(H_j+1)\). Then there exists \(C_K>0\) depending only on \((\delta_K,M_K)\) such that
\[
 \sup_{s\in K}\,\big|\Theta_1(s)-\Theta_2(s)\big|\ \le\ C_K\,\sup_{s\in K}\,\big|H_1(s)-H_2(s)\big|.
\]
In particular, on any \(K\subset\Omega\) where \(H_N^{(\mathrm{Schur})}\) and \(H_N^{(\dettwo)}\) share uniform bounds away from \(-1\), the convergence \(H_N^{(\mathrm{Schur})}\to H_N^{(\dettwo)}\) implies \(\Theta_N^{(\mathrm{Schur})}\to \Theta_N^{(\dettwo)}\) uniformly on \(K\).
\end{lemma}
\begin{remark}
Uniform bounds away from \(-1\) on a compact \(K\subset\Omega\) follow for large \(N\) from lower bounds on \(|\xi|\) off its zeros and continuity of \(\dettwo(I-A_N)\) in the HS topology; hence the lemma applies on each such \(K\).
\end{remark}

\begin{lemma}[Away from \(-1\) on zero-free compacts]\label{lem:away-minus-one}
Let \(K\subset\Omega\) be compact with \(\inf_{K}|\xi|\ge \delta_K>0\). Then there exists \(c_K>0\) and \(N_0\) such that for all \(N\ge N_0\),
\[
 \inf_{s\in K}\,\bigl| H_N^{(\dettwo)}(s)+1\bigr|\ \ge\ c_K,
\]
and likewise \(\inf_{s\in K}|H(s)+1|\ge c_K\). In particular, the denominators in Lemma~\ref{lem:Cayley-diff} are uniformly bounded away from zero on \(K\) for \(N\ge N_0\).
\end{lemma}
\begin{proof}
Since \(\|A(s)\|\le 2^{-\Re s}<1\) on \(\Omega\), \(I-A(s)\) is invertible on \(\Omega\) and \(\dettwo(I-A(s))\ne 0\). Continuity of \(\dettwo(I-A(s))\) on \(K\) implies \(m_K:=\inf_{s\in K}|\dettwo(I-A(s))|>0\). HS continuity (Proposition~\ref{prop:HS-to-det2}) gives uniform convergence \(\dettwo(I-A_N)\to \dettwo(I-A)\) on \(K\), hence for \(N\ge N_0\), \(\inf_{s\in K}|\dettwo(I-A_N(s))|\ge m_K/2\). Therefore on \(K\),
\[
 |H_N^{(\dettwo)}+1|\;=\;\frac{2\,|\dettwo(I-A_N)|}{|\xi|}\;\ge\;\frac{m_K}{\delta_K}\;=:\;c_K,
\]
and similarly for \(H\).
\end{proof}
\begin{proof}
Compute
\[
 \Theta_1-\Theta_2\;=\;\frac{H_1-1}{H_1+1}-\frac{H_2-1}{H_2+1}
 \;=\;\frac{2\,(H_1-H_2)}{(H_1+1)(H_2+1)}.
\]
Hence on \(K\),
\[
 |\Theta_1-\Theta_2|\ \le\ \frac{2}{\delta_K^2}\,|H_1-H_2|.
\]
Choosing \(C_K=2/\delta_K^2\) suffices; if desired, one can refine \(C_K\) using \(M_K\) to control numerators/denominators uniformly.
\end{proof}

\section{Main theorem (formal statement and proof)}\label{sec:main-theorem}
We now assemble the ingredients into a single statement tailored to the prime-grid construction.

\begin{theorem}[Prime-grid BRF via alignment]\label{thm:prime-grid-BRF}
Let \(\Omega=\{\Re s>\tfrac12\}\) and define the prime-diagonal block \(A(s)e_p:=p^{-s}e_p\). Let
\[
 H(s)\;:=\;2\,\frac{\dettwo(I-A(s))}{\xi(s)}-1,\qquad \Theta\;:=\;\frac{H-1}{H+1}.
\]
For each \(N\in\N\), let \(\Phi_N(s)=D_N+C_N(sI-A_N)^{-1}B_N\) be the prime-grid lossless transfer of Proposition~\ref{prop:prime-grid-KYP}, and fix unit vectors \(u_N,v_N\in\C^N\). Define the scalar Schur function \(\widehat\Theta_N(s):=v_N^*\,\Phi_N(s)\,u_N\). Suppose there exists, for each compact \(K\subset\Omega\), a sequence of scalar lossless Schur functions \(\Psi_{N,K}\) such that
\begin{equation}\label{eq:uniform-alignment}
 \sup_{s\in K}\ \big|\Psi_{N,K}(s)\,\widehat\Theta_N(s)\; -\; \Theta_N^{(\dettwo)}(s)\big|\xrightarrow[N\to\infty]{}0,
\end{equation}
where \(\Theta_N^{(\dettwo)}=(H_N^{(\dettwo)}-1)/(H_N^{(\dettwo)}+1)\) with \(H_N^{(\dettwo)}:=2\,\dettwo(I-A_N)/\xi-1\). Then \(\Theta\) is Schur on \(\Omega\), and hence \(H\) is Herglotz on \(\Omega\) (the BRF conclusion).
\end{theorem}
\begin{proof}
By Proposition~\ref{prop:HS-to-det2} and the division remark, \(H_N^{(\dettwo)}\to H\) locally uniformly on compact subsets avoiding zeros of \(\xi\). As established in Lemma~\ref{lem:compact-alignment}, this implies that the Cayley transforms also converge locally uniformly on such compacts, i.e. \(\Theta_N^{(\dettwo)}\to\Theta\). For each compact \(K\), the hypothesis \eqref{eq:uniform-alignment} provides Schur functions \(\Theta_{N,K}:=\Psi_{N,K}\,\widehat\Theta_N\) such that \(\Theta_{N,K}\to\Theta\) uniformly on \(K\). Each \(\Theta_{N,K}\) is Schur as a product of Schur functions; by Corollary~\ref{cor:closure}, the locally uniform limit \(\Theta\) is Schur on \(\Omega\). Applying Theorem~\ref{thm:equivalences} completes the proof.
\end{proof}
\begin{remark}[Realizing the alignment]
Condition \eqref{eq:uniform-alignment} can be arranged by the boundary matching strategy of Section~\ref{sec:practical-alignment}: choose, for an exhaustion by compacts \(K_m\nearrow\Omega\), NP interpolation nodes \(\{s_{j}^{(m,N)}\}\subset K_m\) and lossless interpolants \(\Psi_{N,K_m}\) such that the product \(\Psi_{N,K_m}\,\widehat\Theta_N\) agrees with \(\Theta_N^{(\dettwo)}\) on these nodes and shares the feedthrough normalization. Boundedness and normal-family arguments then promote pointwise agreement on dense sets to uniform convergence on \(K_m\), and a diagonal extraction yields local-uniform convergence on \(\Omega\).
\end{remark}
\section{Practical alignment strategies}\label{sec:practical-alignment}
We outline two standard mechanisms to realize the alignment hypothesis in Proposition~\ref{prop:alignment-criterion} while preserving passivity (Schurness) at each finite stage.

\subsection{Boundary matching via Nevanlinna--Pick interpolation}
Fix a compact \(K\subset\Omega\). Let \(\{s_j\}_{j=1}^{M}\subset K\) be distinct interpolation nodes and let \(\{\gamma_j\}_{j=1}^{M}\subset\C\) be target values with \(|\gamma_j|<1\). The classical Nevanlinna--Pick theorem on the half-plane guarantees existence of Schur functions \(\Psi\) with \(\Psi(s_j)=\gamma_j\), and the set of such interpolants contains rational inner (lossless) functions of degree at most \(M\).

\begin{lemma}[Lossless NP interpolation]\label{lem:NP-lossless}
Given data \(\{(s_j,\gamma_j)\}_{j=1}^{M}\) with \(\Re s_j>\tfrac12\) and \(|\gamma_j|<1\), there exists a rational inner function \(\Psi\) on \(\Omega\) of McMillan degree at most \(M\) that interpolates the data. Moreover, \(\Psi\) admits a lossless realization \(\Psi(s)=D_\Psi+C_\Psi(sI-A_\Psi)^{-1}B_\Psi\) with a positive definite solution of the lossless equalities \eqref{eq:lossless-equalities}.
\end{lemma}
\begin{proof}[Proof sketch]
By mapping \(\Omega\) conformally to the unit disk and invoking the disk NP theorem, one obtains an inner finite Blaschke product solving the interpolation. Realization theory for inner functions (Potapov--de Branges--Rovnyak; state-space proofs via Schur algorithm) yields a lossless colligation.
\end{proof}

\subsection{Interior H$^\infty$ alignment via passive approximants}\label{subsec:hinf-passive}
We record a quantitative H$^\infty$ scheme that yields uniform-on-compact alignment on rectangles strictly inside \(\Omega\), avoiding any \(\varepsilon\downarrow 0\) limits.

\begin{lemma}[HS-tail \(\Rightarrow\) det$_2$ variation on rectangles]\label{lem:HS-tail-rectangle}
Let \(R^\sharp=\{\sigma_0\le \Re s\le \sigma_1,\ |\Im s|\le T\}\Subset\Omega\) with \(\sigma_0>\tfrac12\). Then
\[
 \sup_{s\in R^\sharp}\big|\log\dettwo(I-A(s))\!-\!\log\dettwo(I-A_N(s))\big|\ \le\ C(R^\sharp)\Big(\sum_{p>p_N}p^{-2\sigma_0}\Big)^{1/2}.
\]
\end{lemma}

\begin{corollary}[Cayley Lipschitz away from \(-1\)]\label{cor:Cayley-rect}
If \(|\xi|\ge \delta_R>0\) on a rectangle \(R^\sharp\supset R\) and \(m_R:=\inf_{R}|\dettwo(I\!-\n+A)|>0\), then \(|H+1|\ge 2m_R/\sup_{R}|\xi|\) on \(R\). Consequently,
\[
 \sup_{R}\,\big|\Theta(H_1)-\Theta(H_2)\big|\ \le\ \frac{2}{c_R^2}\sup_{R^\sharp}|H_1-H_2|,\qquad c_R:=\inf_{R}|H+1|.
\]
\end{corollary}

\begin{proposition}[Passive H$^\infty$ approximation on interior rectangles]\label{prop:hinf-passive}
Let \(K\Subset R\Subset R^\sharp\Subset\Omega\) with \(|\xi|\ge \delta_R>0\) on \(R^\sharp\). For \(N\) large, define \(g_N:=\Theta_N^{(\dettwo)}\) on \(\partial R\). Then there exist lossless (Schur) rationals \(\Theta_{N,M}\) of McMillan degree \(\le M\) with
\[
 \sup_{\partial R}\,|\Theta_{N,M}-g_N|\ \le\ C(R,R^\sharp)\,\rho^{M},\qquad \rho\in(0,1),
\]
and hence, by the maximum principle,
\[
 \sup_{K}\,|\Theta_{N,M}-\Theta_N^{(\dettwo)}|\ \le\ C(R,R^\sharp)\,\rho^{M}.
\]
\end{proposition}
\begin{remark}[Caveat on Schur approximation]
The construction of Schur approximants that converge to a prescribed boundary function requires that the target boundary data already lie in the Schur ball (e.g., \(|g_N|\le 1\) on \(\partial R\)). Any proof relying on global scaling (e.g., multiplying a Schur function by a factor \(M_1>1\)) destroys the Schur property. Thus Proposition~\ref{prop:hinf-passive} should be read as conditional on a prior boundary positivity/contractivity input that ensures \(|g_N|\le 1\) on \(\partial R\); in our setting, this amounts to a correct boundary PSD or (P+) statement.
\end{remark}

\begin{corollary}[Global Schur limit on \(\Omega\)]
Let \(\Omega':=\Omega\setminus S\) with \(S\) discrete. Suppose that for every compact \(K\Subset\Omega'\) there exist Schur functions \(\Theta_{K,M}\) with \(\Theta_{K,M}\to\Theta\) locally uniformly on \(K\). Then \(\Theta\) is Schur on \(\Omega'\), extends holomorphically to \(\Omega\) with \(|\Theta|\le 1\) there, and the set \(P:=\{s\in\Omega: 2J(s)=-1\}\) is empty.
\end{corollary}
\begin{proof}
By hypothesis and Corollary~\ref{cor:closure}, \(\Theta\) is Schur on \(\Omega'\). Apply Lemma~\ref{lem:no-P} to extend across \(S\) and eliminate \(P\). 
\end{proof}

\begin{theorem}[Interior completion on zero-free rectangles; conditional globalization]\label{thm:interior-completion}
With \(J=\dettwo(I-A)/\xi\) and \(\Theta=(2J-1)/(2J+1)\) as above, the interior passive \(H^\infty\) approximation (Proposition~\ref{prop:hinf-passive}), the local-uniform convergence of \(\Theta_N^{(\dettwo)}\to\Theta\) off \(Z(\xi)\) (Lemma~\ref{lem:cayley-cont}), and Corollary~\ref{cor:global-schur} show: \(\Theta\) is Schur on \(\Omega\setminus Z(\xi)\) and extends holomorphically across isolated points under a separate boundary positivity input (e.g., (P+) or an equivalent PSD statement). In particular, a global Schur bound on \(\Omega\) requires (P+). 
\end{theorem}
\begin{proof}
Fix a compact \(K\Subset\Omega':=\Omega\setminus Z(\xi)\). By Proposition~\ref{prop:hinf-passive}, for each \(N\) there exist Schur rationals \(\Theta_{N,M}\) with \(\Theta_{N,M}\to\Theta_N^{(\dettwo)}\) uniformly on \(K\) as \(M\to\infty\). By Lemma~\ref{lem:cayley-cont} and HS\(\to\)det$_2$ continuity, \(\Theta_N^{(\dettwo)}\to\Theta\) uniformly on \(K\) as \(N\to\infty\). A diagonal choice \((N_k,M_k)\) yields a sequence of Schur functions converging to \(\Theta\) locally uniformly on \(K\); exhausting \(\Omega'\) and applying Corollary~\ref{cor:global-schur} shows \(\Theta\) extends holomorphically to \(\Omega\) with \(|\Theta|\le 1\).
If \(\xi(\rho)=0\) for some \(\rho\in\Omega\), then \(J\) has a pole at \(\rho\) and \(\Theta\to 1\) as \(s\to\rho\). Since \(\Theta\) is holomorphic on \(\Omega\) with \(|\Theta|\le 1\), the maximum modulus principle forces \(\Theta\) to be constant; asymptotics \(\Theta(\sigma+it)\to -1\) as \(\sigma\to+\infty\) exclude this. Hence \(\xi\) has no zeros in \(\Omega\). By the functional equation, RH follows.
\end{proof}
\begin{proof}
Map \(R^\sharp\) conformally to the unit disk \(\mathbb D\) and transport \(g_N\) to a holomorphic function \(h\) on a neighborhood of \(\overline{\mathbb D}\) with \(\|h\|_{L^\infty(\partial\mathbb D)}\le M_0\). By classical rational approximation on analytic curves, there exist rational functions \(r_M\) with poles off \(\overline{\mathbb D}\) such that
\[
 \sup_{\partial\mathbb D}|r_M-h|\ \le\ C\,\rho^M,\qquad 0<\rho<1.
\]
Fix \(M_1>\max(1,M_0)\) and apply the Schur algorithm to \(r_M/M_1\): after \(m\) steps it produces a rational Schur function \(s_{M,m}\) (a finite Schur–continued–fraction/Blaschke transfer) with
\[
 \sup_{\partial\mathbb D}\big|s_{M,m}-r_M/M_1\big|\ \le\ C'\,\rho^m.
\]
Choosing \(m\asymp M\) and setting \(s_M:=s_{M,m(M)}\) gives a rational Schur \(s_M\) satisfying
\[
 \sup_{\partial\mathbb D}\big|M_1 s_M-h\big|\ \le\ C''\,\rho^M.
\]
Pull back \(M_1 s_M\) to \(\partial R\) via the conformal map to obtain a Schur function \(\Theta_{N,M}\) on \(\partial R\) with
\[
 \sup_{\partial R}\,|\Theta_{N,M}-g_N|\ \le\ C(R,R^\sharp)\,\rho^M.
\]
By the maximum principle (applied after mapping back to the half-plane), the same bound holds on \(K\Subset R\). The Schur property is preserved by the Schur algorithm and by the Möbius equivalence between the disk and half-plane, so each \(\Theta_{N,M}\) is lossless (Schur) as claimed.
\end{proof}

\begin{corollary}[Uniform-on-$K$ alignment on rectangles]\label{cor:interior-alignment}
With \(K\Subset R\Subset R^\sharp\Subset\Omega\) as above, for any \(\varepsilon>0\) choose \(N\) so that \(\sup_R|\Theta_N^{(\dettwo)}-\Theta^{(\dettwo)}|\le \varepsilon/2\), then choose \(M\) with \(C\rho^M\le \varepsilon/2\). Then
\[
 \sup_{K}|\Theta_{N,M}-\Theta^{(\dettwo)}|\ \le\ \varepsilon.
\]
Each \(\Theta_{N,M}\) is Schur (lossless), so kernels are PSD at every finite stage.
\end{corollary}

\paragraph{Globalization by exhaustion.}
Let \(\{R_m\}\) be an increasing exhaustion of \(\Omega\) by rectangles with \(K_m\Subset R_m\Subset R_m^\sharp\Subset\Omega\) and \(\bigcup_m K_m=\Omega\). For each \(m\), choose \(N(m)\) so that \(\sup_{R_m}|\Theta_{N(m)}^{(\dettwo)}-\Theta^{(\dettwo)}|\le 2^{-m-1}\) and then choose \(M(m)\) so that \(C(R_m,R_m^\sharp)\,\rho^{M(m)}\le 2^{-m-1}\). By Corollary~\ref{cor:interior-alignment},
\[
 \sup_{K_m}|\Theta_{N(m),M(m)}-\Theta^{(\dettwo)}|\ \le\ 2^{-m}.
\]
A diagonal extraction yields a sequence of Schur functions converging to \(\Theta^{(\dettwo)}\) locally uniformly on \(\Omega\).
\begin{proposition}[Alignment by cascaded lossless factors]\label{prop:cascade}
Let \(\Phi_N\) be any matrix-valued lossless Schur transfer (e.g., the prime-grid lossless model from Proposition~\ref{prop:prime-grid-KYP}) and let \(\Psi_N\) be a scalar lossless interpolant from Lemma~\ref{lem:NP-lossless} matching \(\Theta_N^{(\dettwo)}\) at nodes \(\{s_j\}_{j=1}^{M(N)}\subset K\). Then the cascade (series connection)
\[
 \Theta_N\;:=\;\Psi_N\,\big(v_N^*\,\Phi_N\,u_N\big),\qquad \|u_N\|=\|v_N\|=1,
\]
is Schur on \(\Omega\), matches the interpolation values, and remains rational inner. Choosing \(M(N)\to\infty\) and nodes dense in \(K\), one obtains \(\Theta_N\to \Theta\) uniformly on \(K\).
\end{proposition}
\begin{proof}[Proof sketch]
Schur functions are closed under products and under pre/post-multiplication by contractions; lossless (inner) functions remain inner under cascade. Interpolation at finitely many points is preserved. Normal-family compactness plus uniqueness on a dense set (identity theorem) yields uniform convergence on \(K\).
\end{proof}

\subsection{Asymptotic control at infinity}
On vertical lines \(\{\Re s=\sigma\}\) with \(\sigma>\tfrac12\), Stirling estimates imply \(\xi(s)\to\infty\) and hence \(H(s)\to -1\) rapidly as \(|\Im s|\to\infty\). Prime-grid lossless models share the exact feedthrough \(-1\) (after scalar port extraction), so one may combine this with the boundedness \(|\Theta_N|\le 1\) and Cauchy integral representations on large rectangles to deduce smallness of the difference \(\Theta_N-\Theta_N^{(\dettwo)}\) provided agreement on a finite boundary grid, as in the previous subsection.

\begin{remark}[Tiny slack variant]
If one relaxes losslessness to allow a vanishing slack \(E_N\succeq 0\) in \(A^*P+PA+C^*C=-E_N\) (and \(D^*D\preceq I\)), the prime-grid template admits a scaling of \(C_N\) that suppresses the \(s^{-1}\) moment in the expansion of \(H_N\), aligning the asymptotics of \(H_N^{(\mathrm{LBR})}\) with those of \(H_N^{(\dettwo)}\). The bounded-real inequality \eqref{eq:KYP} remains valid, and the slack can be sent to zero along the sequence.
\end{remark}

\section{Related work}\label{sec:related}
This work draws on several classical strands. On the operator side, the theory of trace ideals and regularized determinants (notably the Carleman--Fredholm \(\det_2\)) is treated comprehensively in Simon \cite{SimonTraceIdeals}. Realization theory for Schur/inner functions and passive colligations goes back to Potapov's school and is surveyed in de Branges--Rovnyak \cite{deBrangesRovnyak}, Dym--Gohberg \cite{DymGohberg}, and Sz.-Nagy--Foia\c{s} \cite{SzNagyFoias}. Nevanlinna--Pick interpolation on the disk/half-plane and its inner (lossless) solutions are standard topics in complex function theory and H\(\infty\) control; see Garnett \cite{Garnett} and Rosenblum--Rovnyak \cite{RosenblumRovnyak}. The BRF/KYP lemmas used here are classical in systems theory and appear in many sources.

From the analytic number theory perspective, our decomposition mirrors the partition of Euler product contributions according to prime powers: the \(k\ge 2\) terms are naturally accommodated by the \(\det_2\) expansion, while the \(k=1\) (prime) terms, together with archimedean factors and the polynomial \(s(1-s)\), are placed in a finite auxiliary block. While our argument operates at the level of truncations and functional-analytic closure, it is compatible with traditional expansions of \(\log \zeta\) and the analytic properties encoded by the completed zeta \(\xi\).

\section{Discussion and outlook}\label{sec:discussion}
We presented an operator-theoretic BRF program for RH combining Schur--determinant splitting, HS\(\to\)\(\dettwo\) continuity, and explicit finite-stage passive constructions tied to the primes. Two closure routes were formulated:
\begin{itemize}
 \item an interior alignment route on zero-free rectangles via passive $H^\infty$ approximation and Cayley-difference control; and
 \item a boundary route via uniform-in-\(\varepsilon\) local $L^1$ control for a normalized ratio and outer/inner factorization.
\end{itemize}
We proved the interior route locally on rectangles and completed the boundary route via the smoothed estimate for the det$_2$ term and de-smoothing (Theorem~\ref{thm:uniform-eps}). Outer neutralization and global analyticity follow from the compensator argument and BRF\(\Rightarrow\)RH.

Potential refinements include: (i) quantitative rational approximation on analytic boundaries with lossless KYP constraints; (ii) strengthened explicit-formula estimates sufficient for $L^1_{\mathrm{loc}}$ cancellation of zero spikes; (iii) exploring alternative finite-block architectures for $k=1$ with improved global control; and (iv) extensions to matrix-valued settings and other $L$-functions.

\section{Limitations and scope}\label{sec:limitations}
Two routes close the BRF conclusion. The boundary route is completed by Theorem~\ref{thm:uniform-eps} (uniform $L^1_{\mathrm{loc}}$ control) proved via a smoothed explicit-formula route and de-smoothing (Subsection~\ref{subsec:smoothed-explicit}), together with outer/inner factorization and an inner-compensator (Subsection~\ref{subsec:bl-compensator}). The finite-stage route delivers quantitative, noncircular alignment on compact sets strictly inside \(\Omega\) by H$^\infty$ passive approximation (Subsection~\ref{subsec:hinf-passive}).

\section{Examples: small-$N$ prime-grid models}\label{sec:examples}
We record explicit instances of the prime-grid lossless specification (Proposition~\ref{prop:prime-grid-KYP}). Throughout, for a prime \(p\) set
\[
 \lambda(p)\;:=\;\frac{2}{\log p},\qquad c(p)\;:=\;\sqrt{2\,\lambda(p)}\;=\;\frac{2}{\sqrt{\log p}}.
\]

\subsection*{$N=1$ (prime $p_1=2$)}
Numerics: \(\log 2\approx 0.6931\), \(\lambda(2)\approx 2.8854\), \(c(2)\approx 2.4022\). The realization is
\[
 A_1\;=\;-\lambda(2),\quad P_1\;=\;1,\quad C_1\;=\;c(2),\quad D_1\;=\;-1,\quad B_1\;=\;C_1.
\]
Lossless equalities: \(A_1^*P_1+P_1A_1=-2\lambda(2)=-C_1^2\), \(P_1B_1=C_1=-C_1 D_1\), and \(D_1^*D_1=1\). The transfer is
\[
 H_1(s)\;=\;-1\; +\; \frac{c(2)^2}{s+\lambda(2)}\;=\;-1\; +\;\frac{\tfrac{4}{\log 2}}{\,s+\tfrac{2}{\log 2}\,}\;=\;\frac{s-\lambda(2)}{s+\lambda(2)}.
\]
The last expression shows \(H_1\) is a first-order all-pass factor on the right half-plane, hence Schur under the Cayley map to the disk.

\begin{lemma}\label{lem:moebius-contract}
For any \(a>0\) and \(\Re s>0\), one has \(\big|(s-a)/(s+a)\big|<1\).
\end{lemma}
\begin{proof}
Compute
\[
 \frac{|s-a|^2}{|s+a|^2}\;=\;\frac{(\Re s-a)^2+(\Im s)^2}{(\Re s+a)^2+(\Im s)^2}\;<\;1,
\]
since \((\Re s-a)^2<(\Re s+a)^2\) for \(\Re s>0\) and \(a>0\).
\end{proof}

\subsection*{$N=2$ (primes $p_1=2$, $p_2=3$)}
Numerics: \(\log 3\approx 1.0986\), \(\lambda(3)\approx 1.8205\), \(c(3)\approx 1.9054\). The diagonal data are
\[
 \Lambda_2\;=\;\mathrm{diag}\big(\lambda(2),\lambda(3)\big),\quad C_2\;=\;\mathrm{diag}\big(c(2),c(3)\big),\quad D_2\;=\;-I_2,\quad B_2\;=\;C_2,\quad A_2\;=\;-\Lambda_2.
\]
The lossless equalities of Lemma~\ref{lem:losslessKYP} hold entrywise. The matrix-valued transfer is
\[
 H_2(s)\;=\;-I_2\; +\; C_2\,(sI_2+\Lambda_2)^{-1} C_2\;=\;\mathrm{diag}\!\left(\frac{s-\lambda(2)}{s+\lambda(2)},\ \frac{s-\lambda(3)}{s+\lambda(3)}\right).
\]
Any scalar port extraction \(h_2(s)=v^*H_2(s)u\) with \(\|u\|=\|v\|=1\) satisfies \(|h_2(s)|\le 1\) for \(\Re s>0\); in particular, choosing \(u=v=e_1\) recovers the \(N=1\) factor for \(p=2\).

\subsection*{General $N$ (diagonal form)}
For general \(N\), the same diagonal structure yields
\[
 H_N(s)\;=\;-I_N\; +\; \mathrm{diag}\!\left(\frac{\tfrac{4}{\log p_k}}{\,s+\tfrac{2}{\log p_k}\,}\right)_{k=1}^N\;=\;\mathrm{diag}\!\left(\frac{s-\lambda(p_k)}{s+\lambda(p_k)}\right)_{k=1}^N,
\]
with \(\lambda(p_k)=2/\log p_k\). Each diagonal entry obeys Lemma~\ref{lem:moebius-contract}.

\subsection*{A negative result: nonconvergence of the naive cascade}
Define the scalar cascade partial sums
\[
 S_N(s)\;:=\;-1\; +\;\sum_{k=1}^{N} \frac{4/\log p_k}{\,s+2/\log p_k\,},\qquad \Re s>0.
\]
These are the scalar ports of the diagonal prime-grid lossless models with unit weights. Although each term is bounded-real, the sequence \(S_N\) does not converge locally uniformly (indeed not even pointwise) as \(N\to\infty\).

\begin{proposition}[Divergence of the naive prime-grid sum]\label{prop:divergence}
Fix \(s\) with \(\Re s>0\). Then \(S_N(s)\) diverges as \(N\to\infty\).
\end{proposition}
\begin{proof}
For fixed \(s\) with \(\Re s>0\), one has
\[
 \Big|\frac{4/\log p_k}{\,s+2/\log p_k\,}\Big|\;\asymp\; \frac{c}{\log p_k}
\]
with a constant \(c=c(s)>0\) depending only on \(s\). Since \(\sum_{p}\!1/\log p\) diverges, the series of absolute values diverges, hence the sequence of partial sums \(S_N(s)\) cannot converge.
\end{proof}
\noindent This shows that any infinite-$N$ construction based on the \emph{additive} cascade of first-order all-pass sections with unit weights cannot produce a convergent limit, let alone approximate a zeta-derived target. Any successful prime-tied construction must therefore incorporate nontrivial weights (e.g., rapidly decaying coefficients) or a multiplicative/inner product structure rather than a simple additive sum.
\appendix
\section{Appendix: technical lemmas and expanded proofs}\label{sec:appendix}

\subsection{KYP Gram identity for half-plane lossless systems}\label{app:KYP-gram-appendix}

\begin{theorem}[KYP Gram identity for half-plane lossless systems]\label{thm:KYP-gram-appendix}
Let $(A, B, C, D)$ be a minimal realization of a lossless transfer function $F(s) = D + C((s-\tfrac12)I - A)^{-1}B$ on the shifted right half-plane $\{\Re s > 1/2\}$. Assume the continuous-time bounded-real lemma (BRL) conditions hold with $\gamma = 1$:
\begin{align}
  A^* P + P A + C^* C &= 0, \label{eq:brl1}\\
  P B + C^* D &= 0, \label{eq:brl2}\\
  D^* D &= I, \label{eq:brl3}
\end{align}
where $P \succ 0$ is the Lyapunov certificate. Then for all $s, t$ with $\Re s, \Re t > 1/2$,
\[
  \frac{F(s) + \overline{F(t)}}{s + \bar t - 1} = \langle ((s-\tfrac12)I - A)^{-1}B, ((t-\tfrac12)I - A)^{-1}B \rangle_P,
\]
where $\langle x, y \rangle_P := y^* P x$ is the inner product induced by $P$.
\end{theorem}

\begin{proof}
Define $X(s) := ((s-\tfrac12)I - A)^{-1}B$ for $\Re s > 1/2$. We compute the energy inner product:

\medskip
\noindent\textbf{Step 1: Basic identity.}
\begin{align}
  \langle X(s), X(t) \rangle_P &= X(t)^* P X(s)\\
  &= B^* ((t-\tfrac12)I - A^*)^{-1} P ((s-\tfrac12)I - A)^{-1} B.
\end{align}
\medskip
\noindent\textbf{Step 2: Resolvent manipulation.}
Using the resolvent identity $((s-\tfrac12)I - A)^{-1} - ((t-\tfrac12)I - A)^{-1} = (t - s)((s-\tfrac12)I - A)^{-1}((t-\tfrac12)I - A)^{-1}$, we have
\begin{align}
  &(((t-\tfrac12)I - A^*)^{-1} P ((s-\tfrac12)I - A)^{-1} \\
  &= ((t-\tfrac12)I - A^*)^{-1} \left[ \frac{P((s-\tfrac12)I - A)^{-1} - ((t-\tfrac12)I - A^*)^{-1}P}{t - s} \right] (t - s)\\
  &= \frac{((t-\tfrac12)I - A^*)^{-1}P((s-\tfrac12)I - A)^{-1} - ((t-\tfrac12)I - A^*)^{-1}((t-\tfrac12)I - A^*)^{-1}P}{t - s} (t - s).
\end{align}

For the numerator, multiply equation \eqref{eq:brl1} by $((t-\tfrac12)I - A^*)^{-1}$ on the left and $((s-\tfrac12)I - A)^{-1}$ on the right:
\begin{align}
  &((t-\tfrac12)I - A^*)^{-1}(A^* P + P A + C^* C)((s-\tfrac12)I - A)^{-1} = 0\\
  \Rightarrow\quad &((t-\tfrac12)I - A^*)^{-1}A^* P((s-\tfrac12)I - A)^{-1} + ((t-\tfrac12)I - A^*)^{-1}P A((s-\tfrac12)I - A)^{-1}\\
  &\qquad + ((t-\tfrac12)I - A^*)^{-1}C^* C((s-\tfrac12)I - A)^{-1} = 0.
\end{align}
\medskip
\noindent\textbf{Step 3: Simplification.}
Note that:
\begin{align}
  ((t-\tfrac12)I - A^*)^{-1}A^* &= I - (t-\tfrac12)((t-\tfrac12)I - A^*)^{-1},\\
  A((s-\tfrac12)I - A)^{-1} &= I - (s-\tfrac12)((s-\tfrac12)I - A)^{-1}.
\end{align}

Substituting:
\begin{align}
  &[I - (t-\tfrac12)((t-\tfrac12)I - A^*)^{-1}]P((s-\tfrac12)I - A)^{-1} + ((t-\tfrac12)I - A^*)^{-1}P[I - (s-\tfrac12)((s-\tfrac12)I - A)^{-1}]\\
  &\qquad + ((t-\tfrac12)I - A^*)^{-1}C^* C((s-\tfrac12)I - A)^{-1} = 0.
\end{align}

Expanding and rearranging:
\begin{align}
  &(s + \bar t - 1)((t-\tfrac12)I - A^*)^{-1}P((s-\tfrac12)I - A)^{-1}\\
  &= P((s-\tfrac12)I - A)^{-1} + ((t-\tfrac12)I - A^*)^{-1}P - ((t-\tfrac12)I - A^*)^{-1}C^* C((s-\tfrac12)I - A)^{-1}.
\end{align}

\medskip
\noindent\textbf{Step 4: Computing the Gram inner product.}
\begin{align}
  \langle X(s), X(t) \rangle_P &= B^* ((t-\tfrac12)I - A^*)^{-1} P ((s-\tfrac12)I - A)^{-1} B\\
  &= \frac{1}{s + \bar t - 1} B^* \left[ P((s-\tfrac12)I - A)^{-1} + ((t-\tfrac12)I - A^*)^{-1}P - ((t-\tfrac12)I - A^*)^{-1}C^* C((s-\tfrac12)I - A)^{-1} \right] B.
\end{align}

Using equation \eqref{eq:brl2}, $PB = -C^* D$:
\begin{align}
  \langle X(s), X(t) \rangle_P &= \frac{1}{s + \bar t - 1} \left[ -B^* C^* D ((s-\tfrac12)I - A)^{-1}B - B^* ((t-\tfrac12)I - A^*)^{-1}C^* D \right.\\
  &\qquad \left. - B^* ((t-\tfrac12)I - A^*)^{-1}C^* C((s-\tfrac12)I - A)^{-1}B \right].
\end{align}

Factoring out common terms and using \eqref{eq:brl3}:
\begin{align}
  \langle X(s), X(t) \rangle_P &= \frac{1}{s + \bar t - 1} \left[ D^* C((s-\tfrac12)I - A)^{-1}B + B^* ((t-\tfrac12)I - A^*)^{-1}C^* D \right.\\
  &\qquad \left. + B^* ((t-\tfrac12)I - A^*)^{-1}C^* C((s-\tfrac12)I - A)^{-1}B \right].
\end{align}

\medskip
\noindent\textbf{Step 5: Recognizing the transfer function.}
Note that:
\begin{align}
  F(s) &= D + C((s-\tfrac12)I - A)^{-1}B,\\
  \overline{F(t)} &= D^* + B^* ((t-\tfrac12)I - A^*)^{-1}C^*.
\end{align}

Therefore:
\begin{align}
  F(s) + \overline{F(t)} &= D + C((s-\tfrac12)I - A)^{-1}B + D^* + B^* ((t-\tfrac12)I - A^*)^{-1}C^*\\
  &= (s + \bar t - 1) \langle X(s), X(t) \rangle_P.
\end{align}

This completes the proof.
\end{proof}

\begin{remark}[Connection to unit disk formulation]
The standard KYP lemma is often stated for the unit disk. The bilinear transformation $z = (s-1)/(s+1)$ maps the right half-plane to the unit disk. Under this transformation, a lossless system in the half-plane corresponds to an inner function on the disk, and the kernel $(F(s) + \overline{F(t)})/(s + \bar t - 1)$ transforms to the standard Pick kernel $(1 - f(z)\overline{f(w)})/(1 - z\bar w)$.
\end{remark}

\subsection{Expanded proof of Schur--determinant splitting (Proposition~\ref{prop:schur-split})}
We sketch a direct computation using the regularized determinant definition. Recall
\[
 \dettwo(I-K)\;=\;\det\!\Big((I-K)\,\exp\big(K\big)\Big),\qquad K\in\HS.
\]
For the block operator \(T=\begin{bmatrix}A&B\\C&D\end{bmatrix}\) with \(B,C\) finite rank and \(A\in\HS\), write the Schur triangularization of \(I-T\):
\[
 I-T\;=\;L\,\mathrm{diag}(I-A,\ I-S)\,U,
\]
with
\[
 L\;=\;\begin{bmatrix}I & 0\\ -C(I-A)^{-1} & I\end{bmatrix},\qquad U\;=\;\begin{bmatrix}I & -(I-A)^{-1}B\\ 0 & I\end{bmatrix}.
\]
Both \(L-I\) and \(U-I\) are finite rank. Using \(\det((I+X)\exp(-X))=1\) for finite-rank \(X\) and the cyclicity of the trace inside finite-dimensional blocks, one finds
\[
 \dettwo(I-T)\;=\;\det(I-S)\,\dettwo(I-A),
\]
which yields the logarithmic identity in Proposition~\ref{prop:schur-split}. For completeness, one may verify multiplicativity via Simon's product identity for \(\dettwo\): if \(X,Y\in\HS\), then
\[
 \dettwo((I-X)(I-Y))\;=\;\dettwo(I-X)\,\dettwo(I-Y)\,\exp\!\big(-\Tr(XY)\big),
\]
and compute the finite-rank cross term \(\Tr(XY)\) arising from the triangular factors, which cancels against the exponential in \(\det(I-S)\).

\subsection{Expanded proof of HS\(\to\)\(\dettwo\) convergence (Proposition~\ref{prop:HS-to-det2})}
Let \(K_n,K:K\to\HS\) be holomorphic with uniform HS bounds \(\|K_n(s)\|_{\HS}\le M_K\) and \(\|K_n(s)-K(s)\|_{\HS}\to 0\) uniformly on compact \(K\subset\Omega\). By Lemma~\ref{lem:carleman}, \(|\dettwo(I-K_n(s))|\le \exp(\tfrac12 M_K^2)\). The pointwise convergence \(\dettwo(I-K_n(s))\to \dettwo(I-K(s))\) follows from continuity of \(\dettwo\) on \(\HS\). Vitali--Porter theorem applies: a locally bounded normal family \(\{f_n\}\) of holomorphic functions on a domain with pointwise convergence on a set with an accumulation point converges locally uniformly to a holomorphic limit. Thus \(f_n\to f\) uniformly on compacts.

\subsection{Asymptotics of the completed zeta \(\xi\)}\label{app:xi-asymptotics}
For \(\sigma:=\Re s\to+\infty\), Stirling's formula for \(\Gamma(s/2)\) gives
\[
 \Gamma\!\left(\frac{s}{2}\right)\;\sim\;\sqrt{2\pi}\,\Big(\frac{s}{2}\Big)^{\frac{s-1}{2}} e^{-s/2},\qquad \pi^{-s/2}\,\Gamma\!\left(\frac{s}{2}\right)\;\sim\;\sqrt{2\pi}\,\Big(\frac{s}{2\pi}\Big)^{\frac{s-1}{2}} e^{-s/2}.
\]
Since \(\zeta(s)\to 1\) as \(\sigma\to\infty\) and the polynomial factor \(\tfrac12 s(1-s)\) is negligible relative to the Stirling growth, one concludes \(|\xi(s)|\to\infty\) super-exponentially along vertical rays with \(\sigma\) fixed large. Consequently, for our truncations with \(\dettwo(I-A_N(s))\to 1\),
\[
 H_N^{(\dettwo)}(s)\;=\;2\,\frac{\dettwo(I-A_N(s))}{\xi(s)}-1\;\longrightarrow\;-1
\]
uniformly on bounded strips \(\{\sigma\ge \sigma_0>\tfrac12,\ |\Im s|\le R\}\) as \(\sigma_0\to\infty\), consistent with the feedthrough \(-1\) realized by the prime-grid models.

\subsection{Half-plane Pick kernel from the disk}
Let \(\phi:\mathbb D\to\Omega\), \(\phi(\zeta)=\tfrac12\,\frac{1+\zeta}{1-\zeta}+\tfrac12\), be the Cayley map from the unit disk \(\mathbb D\) to \(\Omega\). If \(\theta\) is Schur on \(\mathbb D\) with disk kernel \(K_{\mathbb D}(\zeta,\eta)=(1-\theta(\zeta)\overline{\theta(\eta)})/(1-\zeta\overline{\eta})\), then transporting via \(\Theta=\theta\circ\phi^{-1}\) yields the half-plane kernel
\[
 K_\Theta(s,w)\;=\;\frac{1-\Theta(s)\,\overline{\Theta(w)}}{s+\overline{w}-1},
\]
after multiplication by a harmless positive weight. This justifies the denominator used in Theorem~\ref{thm:equivalences}.

\subsection{Discrete-time KYP (disk) variant}
For completeness: if \(G(z)=D+C(zI-A)^{-1}B\) is holomorphic on \(|z|<1\) with \(A\) Schur (spectral radius <1), then \(\|G\|_{H^\infty(\mathbb D)}\le 1\) iff there exists \(P\succeq 0\) such that
\[
 \begin{bmatrix}
  A^*PA-P & A^*PB & C^*\\
  B^*PA & B^*PB-I & D^*\\
  C & D & -I
 \end{bmatrix}\ \preceq\ 0.
\]
In the lossless case, equalities analogous to \eqref{eq:lossless-equalities} hold with \(A^*PA-P=-C^*C\) and \(B^*PB=I-D^*D\).

\subsection{Lossless realizations for NP data}

\subsection{Half-plane KYP epigraph for boundary H$^\infty$ approximation}\label{app:KYP-epigraph}
We sketch a practical formulation used in Proposition~\ref{prop:hinf-passive}. Fix a rectangle boundary \(\partial R\) and model order \(M\). Parametrize scalar transfers \(\Theta_M(s)=D+C(sI-A)^{-1}B\) with \(A\in\C^{M\times M}\) Hurwitz and \((B,C,D)\) of compatible sizes. Enforce Schur (lossless) via the equalities \eqref{eq:lossless-equalities} with some \(P\succ 0\). Introduce an epigraph variable \(t\ge 0\) and impose discrete boundary constraints on a spectral grid \(\{\zeta_j\}\subset\partial R\):
\[
 |\Theta_M(\zeta_j)-g_N(\zeta_j)|\ \le\ t,\qquad j=1,\dots,J,
\]
where \(g_N=\Theta_N^{(\dettwo)}|_{\partial R}\). The program
\[
 \min\ t\quad \text{s.t. lossless KYP equalities and } |\Theta_M(\zeta_j)-g_N(\zeta_j)|\le t
\]
is a convex bounded-extremal approximation in the Schur ball when the KYP constraints are satisfied and the grid is sufficiently fine; the epigraph constraints can be handled via second-order cones on real/imag parts. Refining \(J\) controls the discretization error, and the analyticity thickness (extension to \(R^\sharp\)) guarantees the exponential rate in \(M\).

\subsection{Rational approximation on analytic curves}\label{app:rational-analytic}
Let \(D\Subset\C\) be a domain bounded by an analytic Jordan curve and let \(f\) be holomorphic on a neighborhood of \(\overline D\). Then there exist constants \(C>0\) and \(\rho\in(0,1)\), depending only on the distance from \(\partial D\) to the nearest singularity of \(f\), such that the best uniform rational (or polynomial) approximation error on \(\partial D\) satisfies
\[
 \inf_{\deg\le M}\ \sup_{\zeta\in\partial D}\,|r_M(\zeta)-f(\zeta)|\ \le\ C\,\rho^{M}.
\]
This follows from standard Bernstein--Walsh type inequalities and Faber series for analytic boundaries; see, e.g., Walsh~\cite{WalshApprox} and Saff--Totik~\cite{SaffTotik}. Transport to rectangles via conformal maps yields the rate used in Proposition~\ref{prop:hinf-passive}.

\subsection{Explicit formula (precise variant used)}\label{app:explicit-formula}
Let \(\varphi\in C_c^{\infty}(\R)\) and define its Mellin--Fourier companion
\[
 g(x)\;:=\;\frac{1}{2\pi}\int_{\R} \varphi(t)\,e^{itx}\,dt,\qquad x\in\R.
\]
Let \(\Phi_{\varphi}(s)\) be the Mellin transform appropriate to the completed zeta context (cf. Edwards~\cite[Ch.~1, §5]{Edwards}, Iwaniec--Kowalski~\cite[Ch.~5]{IwaniecKowalski}). Then the following explicit formula holds for the completed zeta:
\[
 \sum_{\rho} \Phi_{\varphi}(\rho)\;=\;\Phi_{\varphi}(1)\,+\,\Phi_{\varphi}(0)\;-
 \sum_{p}\sum_{m\ge 1} \frac{\log p}{p^{m/2}}\,g(m\log p)\;-
 \frac{1}{2\pi}\int_{-\infty}^{\infty} \Re\frac{\Gamma'}{\Gamma}\!\left(\frac{1}{4}+\frac{iu}{2}\right)\,\Phi_{\varphi}\!\left(\frac12+iu\right)du.
\]
All terms converge absolutely for \(\varphi\in C_c^{\infty}(\R)\), and the right-hand side is bounded by a constant depending only on \(\varphi\). Differentiating with respect to \(\sigma\) inside \(\Phi_{\varphi}(\tfrac12+iu)\) and using the rapid decay of \(g\) yields Lipschitz-in-\(\sigma\) bounds for the \(\varphi\)-weighted prime and zero sums. This is the variant tacitly used in Lemma~\ref{lem:smoothed-explicit}.
\subsection{Numerical note: grid/KYP solve on \(\partial R\)}\label{app:numerics}
A practical H$^\infty$ approximation on a rectangle boundary \(\partial R\) proceeds as follows. Fix \(K\Subset R\Subset R^\sharp\Subset\Omega\) and an order \(M\). Sample \(\partial R\) at \(J\) spectral nodes \(\{\zeta_j\}\) (Chebyshev along each edge). For a state-space parameterization \(\Theta_M(s)=D+C((s-\tfrac12)I-A)^{-1}B\) with Hurwitz \(A\in\C^{M\times M}\), enforce the lossless KYP equalities \eqref{eq:lossless-equalities} with a decision variable \(P\succ 0\). Introduce an epigraph variable \(t\ge 0\) and constrain
\[
 |\Theta_M(\zeta_j)-g_N(\zeta_j)|\ \le\ t,\qquad j=1,\dots,J.
\]
The objective \(\min t\) subject to these constraints is a convex program (KYP equalities plus second-order cones for the complex modulus). Refining \(J\) improves the boundary resolution; increasing \(M\) reduces the best achievable \(t\) roughly as \(C\rho^M\) by Subsection~\ref{app:rational-analytic}. The resulting \(\Theta_{N,M}\) is Schur (lossless) by construction, and the maximum principle transfers the boundary error to \(K\).
\subsection{Carleson self-correction and a direct route to (P+) and RH}\label{subsec:PSC}
We isolate the single quantitative hypothesis that encodes the ``perfect self-correction'' principle as a Carleson bound on the off-critical zero measure and show it implies (P+), hence Herglotz/Schur in \(\Omega\) and RH.

\paragraph{Defect measure and Carleson boxes.}
For each nontrivial zero \(\rho=\beta+i\gamma\) of \(\xi\) with \(\beta>\tfrac12\), write the depth \(a(\rho):=\beta-\tfrac12>0\). Define the positive Borel measure
\[
 d\mu\ :=\ \sum_{\substack{\rho=\beta+i\gamma\\ \beta>1/2}} 2\,a(\rho)\,\delta_{\,(\,\tfrac12+a(\rho),\ \gamma\,)}\,.
\]
For a bounded interval \(I=[T_1,T_2]\subset\R\) let the Carleson box be
\[
 Q(I)\ :=\ \{\,\sigma+it:\ t\in I,\ 0<\sigma-\tfrac12<|I|\,\}.
\]

\begin{definition}[Perfect self-correction (PSC)]\label{def:PSC}
We say the defect measure \(\mu\) is \emph{PSC} if for every bounded interval \(I\subset\R\),
\[
 \mu\big(Q(I)\big)\ \le\ \tfrac{\pi}{2}\,|I|.
\]
\end{definition}

\paragraph{Poisson stamp and phase--balayage.}
For \(a>0\) and \(\gamma\in\R\), define the Poisson-weighted stamp across \(I\) by
\[
 \mathrm{Bal}_a(\gamma;I)\ :=\ 2a\Big[\arctan\!\frac{T_2-\gamma}{a}-\arctan\!\frac{T_1-\gamma}{a}\Big]\ \in [0,\pi].
\]
Let \(\mathcal J=\dettwo(I-A)/(\mathcal O\,\xi)\) be the outer-normalized ratio as above, set \(w(t):=\Arg\,\mathcal J(\tfrac12+it)\in(-\pi,\pi] \) and let \(-w'\) denote its distributional derivative on intervals avoiding critical-line ordinates.

\begin{lemma}[Phase--balayage law]\label{lem:balayage-law}
On any interval \(I\) avoiding the ordinates of critical-line zeros, one has
\[
 \int_I (-w'(t))\,dt\ =\ \int_{\Omega}\, \mathrm{Bal}_{\sigma-\frac12}(\tau;I)\, d\mu(\sigma+i\tau).
\]
In particular, \(\int_I (-w')\,dt\le \pi\, \mu(Q(I))/|I|\).
\end{lemma}
\begin{proof}[Proof sketch]
This is the distributional form of the phase--velocity identity (Proposition~\ref{prop:phase-velocity-identity}) after outer normalization: the zero-side contribution is exactly the Poisson balayage of \(\mu\), critical-line atoms contribute a nonnegative discrete term (ruled out on \(I\) by hypothesis), while regular parts are absorbed by \(\mathcal O\). The pointwise bound \(\mathrm{Bal}_a\le\pi\) and localization to \(Q(I)\) give the inequality.
\end{proof}
\begin{lemma}[PSC implies boundary wedge]\label{lem:wedge-PSC}
If \(\mu\) is PSC, then for every interval \(I\) avoiding critical ordinates,
\[\int_I (-w'(t))\,dt\ \le\ \frac{\pi}{2}.\]
Consequently \(w(t)\in[-\tfrac{\pi}{2},\tfrac{\pi}{2}]\) for a.e. \(t\in\R\).
\end{lemma}
\begin{proof}
By Lemma~\ref{lem:balayage-law} and PSC,
\[\int_I (-w')\,dt\ \le\ \pi\,\mu(Q(I))/|I|\ \le\ \pi\cdot(\tfrac{\pi}{2})/\pi\ =\ \tfrac{\pi}{2}.
\]
If \(w\) left the cone on a positive-measure set, bounded variation would force an interval with drop exceeding \(\pi/2\), a contradiction.
\end{proof}

\begin{theorem}[PSC \(\Rightarrow\) (P+) and Herglotz]\label{thm:PSC-Pplus}
Under PSC, \(\Re(2\mathcal J(\tfrac12+it))\ge 0\) for a.e. \(t\in\R\). Hence \(2\mathcal J\) is Herglotz on \(\Omega\), and \(\Theta=(2\mathcal J-1)/(2\mathcal J+1)\) is Schur on \(\Omega\).
\end{theorem}
\begin{proof}
By Lemma~\ref{lem:wedge-PSC}, \(\mathcal J(\tfrac12+it)=e^{iw(t)}\) with \(w\in[-\pi/2,\pi/2]\) a.e., so \(\Re(2\mathcal J)=2\cos w\ge 0\) a.e.
The Poisson integral transports boundary nonnegativity to \(\Omega\), so \(2\mathcal J\) is Herglotz; the Cayley map yields the Schur bound.
\end{proof}

\begin{theorem}[PSC \(\Rightarrow\) RH]\label{thm:PSC-RH}
Assume PSC and \(\Theta(\sigma+it)\to -1\) as \(\sigma\to+\infty\). Then \(\xi\) has no zeros in \(\Omega\). In particular, all nontrivial zeros lie on \(\Re s=\tfrac12\).
\end{theorem}
\begin{proof}
By Theorem~\ref{thm:PSC-Pplus}, \(\Theta\) is holomorphic and Schur on \(\Omega\).
If \(\xi(\rho)=0\) for some \(\rho\in\Omega\), then \(J=\mathcal J\,\mathcal O=\dettwo(I-A)/\xi\) has a pole at \(\rho\), forcing \(\Theta(\rho)=1\). A nonconstant Schur function cannot attain its boundary norm in the interior; the normalization at infinity rules out constancy. Hence \(\xi\) has no zeros in \(\Omega\), and RH follows by symmetry.
\end{proof}

\begin{remark}[Physics $\leftrightarrow$ math dictionary]
Off-critical zeros at depth \(a\) are imbalanced resonances carrying cost \(2a\). The Carleson bound caps the total defect cost per window, which bounds the boundary phase drop to \(\le\pi/2\). This enforces boundary positive-real (P+), whence interior Herglotz/Schur and the pinch argument exclude interior poles of \(J\).
\end{remark}

\paragraph{Axiom (Self-correction $\Leftrightarrow$ boundary positive-real).}
Let \(\Omega=\{\Re s>\tfrac12\}\) and
\[\mathcal J(s):=\frac{\dettwo(I-A(s))}{\mathcal O(s)\,\xi(s)}\]
be the outer-normalized ratio from Subsection~\ref{subsec:boundary-unitarity}, so $|\mathcal J(\tfrac12+it)|=1$ a.e. on the boundary. 
\begin{definition}[Self-correction (SC)]\label{def:SC}
We say the system is \emph{self-correcting} if
\[\Re\bigl(2\mathcal J(\tfrac12+it)\bigr)\ \ge\ 0\quad\text{for a.e. }t\in\R.\]
\end{definition}
In classical function theory this is exactly the boundary positive-real hypothesis (P+), and is equivalent—via the Poisson integral—to $2\mathcal J$ being Herglotz on $\Omega$; see Theorem~\ref{thm:global-PSD}.
\begin{proposition}[Boundary PSD for $H_{J_N}$ by congruence]\label{prop:boundary-psd-fixed}
Let $R\Subset\Omega$ be a rectangle and $\Sigma_R:=Z(\xi)\cap\partial R$. On $\partial R\setminus\Sigma_R$ set
\[
K_{\exp,N}(s,\bar t):=\frac{e^{\mathfrak g_N(s)}+\overline{e^{\mathfrak g_N(t)}}}{s+\bar t-1},\qquad 
K_{\mathrm{FG},N}(s,\bar t):=E_N(s,\bar t)\,\frac{1}{s+\bar t-1},
\]
with $\mathfrak g_N=\log\dettwo(I-A_N)$ and $E_N$ the Fock lift from Lemma~\ref{lem:fock-gram-formal}. Then for any finite node set $\{s_j\}\subset\partial R\setminus\Sigma_R$:
\begin{enumerate}
\item[\textup{(a)}] The Gram matrix $\big(K_{\exp,N}(s_i,\overline{s_j})-K_{\mathrm{FG},N}(s_i,\overline{s_j})\big)_{i,j}$ is PSD.
\item[\textup{(b)}] Since $K_{\mathrm{FG},N}$ is PSD, (a) implies $\big(K_{\exp,N}(s_i,\overline{s_j})\big)_{i,j}$ is PSD.
\item[\textup{(c)}] With the diagonal multiplier $D=\mathrm{diag}(\xi(s_j)^{-1})$, one has
\[
\Big(H_{J_N}(s_i,\overline{s_j})\Big)_{i,j}=D\,\Big(K_{\exp,N}(s_i,\overline{s_j})\Big)_{i,j}\,D^{*},
\]
so $\big(H_{J_N}(s_i,\overline{s_j})\big)$ is PSD.
\end{enumerate}
Consequently $H_{J_N}$ is PSD on $\partial R$ in the sense of boundary limits along node sets approaching $\Sigma_R$.
\end{proposition}
\begin{proof}
(a)–(b) are the Fock–Gram lower bound and Löwner-order transfer. For (c), write $J_N=\dettwo(I-A_N)/\xi$, and observe
\[\frac{J_N(s_i)+\overline{J_N(s_j)}}{s_i+\overline{s_j}-1}=\xi(s_i)^{-1}\,\frac{e^{\mathfrak g_N(s_i)}+\overline{e^{\mathfrak g_N(s_j)}}}{s_i+\overline{s_j}-1}\,\overline{\xi(s_j)^{-1}}.\]
Congruence by a nonsingular diagonal preserves PSD. Approaching $\Sigma_R$ is handled by entrywise limits of PSD matrices.
\end{proof}
\begin{corollary}[Boundary $\Rightarrow$ interior on rectangles]\label{cor:bdry-to-int}
Let $R\Subset\Omega$ be a rectangle. Then $H_{J_N}$ is PSD on $\partial R$ (distribution sense), hence $\Re J_N\ge 0$ in $R$; equivalently $\Theta_N=(2J_N-1)/(2J_N+1)$ is Schur on $R$.
\end{corollary}

\begin{theorem}[Three equivalent faces of self-correction]\label{thm:SC-equivalences}
Let $\mathcal J=\dettwo(I-A)/(\mathcal O\,\xi)$ be the outer-normalized ratio on $\Omega$. The following are equivalent:
\begin{enumerate}
\item[\textup{(i)}] \textup{(P+)}: $\Re\bigl(2\mathcal J(\tfrac12+it)\bigr)\ge 0$ a.e. on $\R$.
\item[\textup{(ii)}] $2\mathcal J$ is Herglotz on $\Omega$ (hence $\Theta=(2\mathcal J-1)/(2\mathcal J+1)$ is Schur on $\Omega$).
\item[\textup{(iii)}] The off-critical zero measure $\mu$ obeys the Carleson bound $\mu(Q(I))\le \tfrac{\pi}{2}|I|$ for all intervals $I\subset\R$.
\end{enumerate}
Moreover, any of (i)–(iii) imply RH via the pinch argument (Theorem~\ref{thm:PSC-RH}).
\end{theorem}
\begin{proof}
(i)$\Leftrightarrow$(ii): Poisson/Herglotz equivalence on the half-plane (Theorem~\ref{thm:global-PSD}). (iii)$\Rightarrow$(i): Theorem~\ref{thm:PSC-Pplus}. The pinch to RH is Theorem~\ref{thm:PSC-RH}.
\end{proof}

%====================================================================
%  SECTION: Unconditional Proof of the Carleson Self-Correction Principle
%====================================================================

\section{Toward an unconditional proof of PSC (Carleson bound)}\label{sec:unconditional-psc-proof}
In this section we formalize a local explicit-formula strategy to prove the Carleson Self-Correction (PSC) inequality
\[ \mu(Q(I))\ \le\ \tfrac{\pi}{2}\,|I| \quad\text{for every interval } I, \]
thereby closing the (P+) step and RH via Section~\ref{subsec:PSC}. We work at the Whitney scale \(|I|\asymp c/\log(2+T)\) and use a smooth local test to pass the phase--velocity identity to a Poisson-balayage bound, then control ancillary terms by unconditional estimates.

\subsection{Test functions and Poisson staples}
Fix a bounded interval \(I=[T_1,T_2]\) of length \(L:=|I|\). Choose \(\varphi_I\in C_c^\infty(\R)\) such that
\begin{enumerate}
\item[(i)] \(\mathrm{supp}(\varphi_I)\subset [T_1-L,\,T_2+L]\), \(0\le\varphi_I\le 1\), and \(\varphi_I\equiv 1\) on \([T_1+L/4,\,T_2-L/4]\).
\item[(ii)] \(\|\varphi_I\|_{L^1}\asymp L\) and \(\|\varphi_I'\|_{L^1}\asymp 1\).
\end{enumerate}
For a zero \(\rho=\beta+i\gamma\) with \(a:=\beta-\tfrac12>0\), the Poisson balayage across \(I\) is
\[ \mathrm{Bal}_a(\gamma;I)\ :=\ 2\Big[\arctan\!\frac{T_2-\gamma}{a}-\arctan\!\frac{T_1-\gamma}{a}\Big] \in [0,\pi].\]

\begin{lemma}[Whitney lower bound]\label{lem:whitney-lower}
There exists \(c_0\in(0,\pi)\) such that for any \(I\) and any zero \(\rho\) with \(\gamma\in I\) and \(a\in[L,2L]\), one has \(\mathrm{Bal}_a(\gamma;I)\ge c_0\).
\end{lemma}
\begin{proof}
Minimize \(2(\arctan((L-x)/a)+\arctan(x/a))\) over \(x\in[0,L]\), \(a\in[L,2L]\). For fixed \(a\), the sum in \(x\) is minimized at the endpoints, giving \(2\arctan(L/a)\). This is decreasing in \(a\), so the minimum over \(a\in[L,2L]\) occurs at \(a=2L\), yielding \(\ge 2\arctan(1/2)\). Any uniform choice \(c_0\in(0,2\arctan(1/2))\) suffices. A detailed derivation is provided in Appendix~\ref{app:psc-tech}.
\end{proof}

\subsection{Ancillary bounds on short intervals}
Write \(F=\dettwo(I-A)/\xi\), \(u=\log|F|\) on the boundary, \(s=\tfrac12+it\). We isolate the three standard contributions appearing in the phase--velocity identity.
\begin{lemma}[Archimedean control]\label{lem:arch}
There exists an absolute \(C_\Gamma>0\) such that for every interval \(I\) and test \(\varphi_I\),
\[ \Big|\int_{\R} \Im\Big(\frac{\Gamma'}{\Gamma}(s/2)+\frac{1-2s}{s(1-s)}\Big)\,\varphi_I(t)\,dt\Big|\ \le\ C_\Gamma\,L.\]
\end{lemma}
\begin{proof}
Write \(s=\tfrac12+it\). Decompose
\[
 A(t):=\Im\frac{d}{dt}\log\!\left(\pi^{-s/2}\Gamma\!\left(\tfrac{s}{2}\right)\cdot\tfrac12 s(1-s)\right)
 = \tfrac12\Re\psi\!\left(\tfrac14+\tfrac{it}{2}\right)\ -\ \tfrac12\log\pi\ +\ \frac{2t}{1+4t^2},
\]
with \(\psi=\Gamma'/\Gamma\). By the standard asymptotic (Stirling/Titchmarsh), for \(\sigma\in[1/4,1]\) and \(t\in\R\),
\[
 \Big|\Re\psi(\sigma+it)-\log\sqrt{\sigma^2+t^2}\,\Big|\ \le\ \frac{C_\psi}{1+|t|},
\]
hence \(|A(t)|\le \tfrac12\log(1+|t|)+C_1\). Let \(\varphi_I(t)=\psi((t-T)/L)\) with \(\psi\in C_c^\infty([-1,1])\). Then
\[
 \Big|\int_{\R} A(t)\,\varphi_I(t)\,dt\Big|\ \le\ C(\psi)\,L\,\big(1+\log(2+|T|)\big).
\]
Absorb the harmless \(\log(2+|T|)\) factor into \(C_\Gamma\) at Whitney scale to obtain the stated bound with an absolute \(C_\Gamma\) depending only on the window.\qedhere
\end{proof}

\begin{lemma}[Prime-side difference on short intervals]\label{lem:prime-short}
There exists an absolute \(C_P>0\) such that
\[ \Big|\int_{\R} \Im\Big(\frac{\zeta'}{\zeta}(s)-\frac{\dettwo'}{\dettwo}(s)\Big)\,\varphi_I(t)\,dt\Big|\ \le\ C_P\,L.\]
\end{lemma}
\begin{proof}
We use the standard band-limiting device at Whitney scale (cf. Subsection~\ref{subsec:explicit-calibration}). Fix \(\kappa\in(0,1] \) and set \(\Delta=\kappa/L\). Let \(W\in C_c^\infty([-1,1])\) with \(W\ge 0\), \(W(0)=1\), and define the band-limited surrogate \(\varphi_{I,\Delta}:=\mathsf S_\Delta\varphi_I\) by \(\widehat{\varphi_{I,\Delta}}(\xi)=W(\xi/\Delta)\,\widehat{\varphi_I}(\xi)\). Then \(\mathrm{supp}\,\widehat{\varphi_{I,\Delta}}\subset\{|
\xi|\le \Delta\}\) and \(\|\varphi_{I}-\varphi_{I,\Delta}\|_{L^1}\ll L/\kappa\).

Write
\[
 \mathcal P(t):=\Im\Big(\frac{\zeta'}{\zeta}-\frac{\dettwo'}{\dettwo}\Big)\!\Big(\tfrac12+it\Big)
 = \sum_{p}(\log p)\,p^{-1/2}\sin(t\log p)\ +\ O\Big(\sum_{p,\,k\ge 2}(\log p)\,p^{-k/2}\Big).
\]
By Fubini and Plancherel,
\[
 \int_{\R}\!\mathcal P(t)\,\varphi_{I,\Delta}(t)\,dt
 = \Re\sum_{p}(\log p)\,p^{-1/2}\,\widehat{\varphi_{I,\Delta}}(\log p)\ +\ O\Big(\sum_{p,\,k\ge 2}(\log p)\,p^{-k/2}\,|\widehat{\varphi_{I,\Delta}}(k\log p)|\Big).
\]
Since \(\widehat{\varphi_{I,\Delta}}\) is supported in \(|\xi|\le \Delta=\kappa/L\), only primes with \(\log p\le \kappa/L\) contribute. Using \(|\widehat{\varphi_{I,\Delta}}(\xi)|\le L\,\|\psi\|_{L^1}\), Cauchy--Schwarz, and the elementary bounds
\(\sum_{\log p\le \kappa/L}(\log p)^2/p\le (\kappa/L)^2\) and \(\#\{p: \log p\le \kappa/L\}\ll \kappa/L\), we obtain
\[
 \Big|\int \mathcal P\,\varphi_{I,\Delta}\Big|\ \le\ C(\psi)\,\kappa.
\]
By Subsection~\ref{subsec:explicit-calibration} this can be written as \(|\int \mathcal P\,\varphi_{I,\Delta}|\le C_P(\kappa)\,L\) with \(C_P(\kappa)\ll \kappa\). The error from replacing \(\varphi_I\) by \(\varphi_{I,\Delta}\) is bounded by
\[
 \Big|\int \mathcal P\,(\varphi_I-\varphi_{I,\Delta})\Big|\ \le\ \|\varphi_I-\varphi_{I,\Delta}\|_{L^1}\,\Big\|\mathcal P\Big\|_{L^\infty(I^*)}
 \ \ll\ C'(\psi)\,L,
\]
for a slightly enlarged interval \(I^*\) and some finite \(C'(\psi)\) (by local integrability and standard bounds on explicit-formula components on vertical lines). Absorbing constants, we get
\[
 \Big|\int_{\R} \Im\Big(\tfrac{\zeta'}{\zeta}-\tfrac{\dettwo'}{\dettwo}\Big)(\tfrac12+it)\,\varphi_I(t)\,dt\Big|\ \le\ C_P\,L.
\]
This proves the claim.\qedhere
\end{proof}

\begin{lemma}[Hilbert-transform pairing]\label{lem:hilbert}
There exists \(C_H>0\) such that
\[ \Big|\int_{\R} \mathsf H[u'](t)\,\varphi_I(t)\,dt\Big|\ \le\ C_H\,L.\]
\end{lemma}
\begin{proof}
Let \(u(t):=\log|\dettwo(I\!-
A(\tfrac12+it))|\,-\,\log|\xi(\tfrac12+it)|\). For \(\varepsilon\in(0,\varepsilon_0] \) set the smoothed boundary datum
\(u_\varepsilon(t):=\log|\dettwo(I\!-
A(\tfrac12+\varepsilon+it))|\,-\,\log|\xi(\tfrac12+\varepsilon+it)|\).

By the smoothed explicit-formula bounds (Lemmas~\ref{lem:smoothed-explicit}, \ref{lem:det2-smoothed-target}, \ref{lem:xi-smoothed}) and de-smoothing (Lemma~\ref{lem:desmoothing}), for any compact interval \(I\) and any \(\varphi\in C_c^2(I)\) one has
\[
 \sup_{\varepsilon\in(0,\varepsilon_0]}\ \Big|\int_{\R} \varphi(t)\,u_\varepsilon(t)\,dt\Big|\ \le\ C(\varphi),\qquad
 \lim_{\varepsilon,\delta\downarrow 0}\ \int_{\R} \varphi(t)\,\big(u_\varepsilon(t)-u_\delta(t)\big)\,dt\ =\ 0.
\]
Thus \(u_\varepsilon\) converges in \(\mathcal D'(I)\) (and in \(L^1(I)\) along a subsequence) to a boundary distribution \(u\). For our window \(\varphi_I(t)=\psi((t-T)/L)\in C_c^\infty\), the boundary Hilbert transform identity passes to the limit:
\[
 \int_{\R}\!\mathsf H[u'](t)\,\varphi_I(t)\,dt\ =\ -\lim_{\varepsilon\downarrow 0}\int_{\R}\!u_\varepsilon'(t)\,\mathsf H[\varphi_I](t)\,dt.
\]
Since \(\psi\in C_c^\infty([-1,1])\), the classical kernel estimate yields
\(\|\mathsf H[\varphi_I]\|_{L^\infty}\le \tfrac{1}{\pi}(\|\psi'\|_{L^1}+2\|\psi\|_{L^1})=:C_\infty(\psi)\), independent of \(L,T\). Using the uniform-in-\(\varepsilon\) \(L^1(I)\) bounds for \(u_\varepsilon'\) from the smoothed det$_2$/\(\xi\) estimates and the de-smoothing transfer (Proposition~\ref{prop:desmoothing}), we obtain
\[
 \Big|\int_{\R}\!\mathsf H[u']\,\varphi_I\Big|\ \le\ C_\infty(\psi)\,\sup_{\varepsilon}\int_{I^*}\!|u_\varepsilon'(t)|\,dt\ \le\ C_H(\psi)\,L,
\]
for a slightly enlarged interval \(I^*\) and an explicit constant \(C_H(\psi)\). This proves the claim.\qedhere
\end{proof}

\subsection{Carleson bound from the phase--velocity identity}
Recall the phase--velocity identity (Proposition~\ref{prop:phase-velocity-identity}): for nonnegative \(\varphi\),
\[ \int_{\R}(-w')(t)\,\varphi(t)\,dt\ =\ \sum_{\rho}2a(\rho)\,(P_{a(\rho)}*\varphi)(\gamma)\ +\ \pi\sum_{\gamma\ \mathrm{critical}} m_\gamma\,\varphi(\gamma).\]

\begin{theorem}[Carleson self-correction at Whitney scale]\label{thm:psc-unconditional}
There is an absolute \(C_*\) such that for every interval \(I\),
\[ \mu(Q(I))\ \le\ C_*\,|I|. \]
In particular, if \(C_*\le \pi/2\), PSC holds.
\end{theorem}
\begin{proof}
Apply the identity to \(\varphi_I\). The critical-line sum is nonnegative. For the zero-side, Lemma~\ref{lem:whitney-lower} and \(\varphi_I\equiv 1\) on the bulk give
\[ c_0\,\mu(Q(I))\ \le\ \sum_{\rho\in Q(I)} \mathrm{Bal}_{a(\rho)}(\gamma;I)\ \lesssim\ \sum_{\rho}2a(\rho)\,(P_{a(\rho)}*\varphi_I)(\gamma)\ +\ O(1). \]
Thus
\[ c_0\,\mu(Q(I))\ \le\ \Big|\int \Im\frac{\xi'}{\xi}\,\varphi_I\Big|\ +\ \Big|\int \Im\frac{\zeta'}{\zeta}-\Im\frac{\dettwo'}{\dettwo}\,\varphi_I\Big|\ +\ \Big|\int \mathsf H[u']\,\varphi_I\Big|\ +\ O(1). \]
By Lemmas~\ref{lem:arch}, \ref{lem:prime-short}, \ref{lem:hilbert}, the right-hand side is \(\le (C_\Gamma+C_P+C_H)\,L+O(1)\). Absorb the \(O(1)\) at the Whitney scale and divide by \(c_0\).
\end{proof}

\begin{remark}
The constant \(C_*=(C_\Gamma+C_P+C_H)/c_0\) is absolute. With optimized smoothing and truncation, one aims to sharpen \(C_*\le \pi/2\), yielding PSC exactly. Regardless, any \(C_*<\pi\) already enforces a nontrivial boundary wedge and can be fed back into Theorem~\ref{thm:PSC-Pplus} to obtain (P+) with an appropriate cone.
\end{remark}

\subsection{Zero-density reduction for short-interval Carleson mass}\label{subsec:ZD-reduction}
We record a purely number-theoretic reduction of the Carleson mass bound \(\mu(Q(I))\ll |I|\) to short-interval zero-density estimates for \(\zeta\).

\begin{proposition}[Zero-density \(\Rightarrow\) short-interval Carleson]
Let \(I=[T-L,\,T+L]\) with \(T\) large and \(0<L\le 1\). Suppose there exist constants \(C_0,A_0,B_0>0\) and \(T_0\) such that for all \(T\ge T_0\), all \(0<a\le L\), and all windows of length \(H\asymp L\), one has the local zero-density bound
\[
 N\big(\tfrac12+a;\,T-H,\,T+H\big)\ \le\ C_0\,H\,T^{\,A_0 a}\,(\log T)^{B_0},
\]
where \(N(\sigma;U,V)\) counts zeros \(\rho=\beta+i\gamma\) with \(\beta\ge \sigma\) and \(\gamma\in[U,V]\). Then the Carleson (Whitney-box) mass satisfies
\[
 \mu\big(Q(I)\big)\ =\ \int_{0}^{L}\!2a\,dN(\tfrac12+a;I)\ \le\ \frac{2C_0\,(\log T)^{B_0}}{A_0\,\log T}\,\big(e^{A_0 L\log T}-1\big)\,|I|.
\]
In particular, for Whitney scale \(L=c/(\log T)\) one has
\[
 \mu\big(Q(I)\big)\ \le\ \frac{2C_0}{A_0}\,\big(e^{A_0 c}-1\big)\,(\log T)^{B_0-1}\,|I|.
\]
\end{proposition}
\begin{proof}
For any \(0<a\le L\), write the Stieltjes integral
\(\mu(Q(I))=\int_0^L\!2a\,dN(\tfrac12+a;I)=2\int_0^L N(\tfrac12+a;I)\,da\). By hypothesis, for each \(a\) we have \(N(\tfrac12+a;I)\ll L\,T^{A_0 a}(\log T)^{B_0}\), hence
\[
 \mu(Q(I))\ \le\ 2C_0\,L\,(\log T)^{B_0}\int_0^L T^{A_0 a}\,da
 \ =\ 2C_0\,L\,(\log T)^{B_0}\,\frac{T^{A_0 L}-1}{A_0\,\log T}.
\]
Since \(|I|=2L\), the stated bounds follow.\qedhere
\end{proof}

\begin{corollary}[PSC under strong short-interval zero-density]
With the notation above, assume there exists \(c>0\) such that for \(L=c/(\log T)\) one has
\[
 \frac{2C_0}{A_0}\,\big(e^{A_0 c}-1\big)\,(\log T)^{B_0-1}\ \le\ \frac{\pi}{2}
\]
for all sufficiently large \(T\). Then the Perfect Self-Correction bound \(\mu(Q(I))\le (\pi/2)\,|I|\) holds on Whitney boxes, hence \emph{(P+)} holds by Theorem~\ref{thm:psc-constants}, and consequently \(\Theta\) is Schur on \(\Omega\) and \emph{RH} follows.
\end{corollary}

\begin{remark}[Status under current unconditional zero-density]
Known global zero-density bounds of the form \(N(\sigma,T)\ll T^{\alpha(1-\sigma)}(\log T)^{\beta}\) imply short-interval variants with a factor \(H\) (length of the interval) under mild hypotheses. Plugging such bounds into the proposition above yields
\(\mu(Q(I))\ll |I|\,(\log T)^{\beta-1}\,\big(e^{\alpha c}-1\big)\). Present exponents \((\alpha,\beta)\) do not yet force the threshold \(\pi/2\) uniformly, but the reduction isolates precisely what improvement in short-interval zero-density would suffice for \emph{(P+)}. Any sharpening that yields \(\beta\le 1\) together with moderate \(\alpha c\) would meet the \(\pi/2\) target.
\end{remark}

\subsection{Adaptive PSC from unconditional zero-density}\label{subsec:adaptive-psc-zd}
We now combine the zero-density reduction with the adaptive-cover criterion (Corollary~\ref{cor:adaptive-cover}) to obtain \emph{(P+)} a.e. on \(\R\) without requiring uniform \(\pi/2\) at a fixed Whitney scale.

\begin{theorem}[Adaptive PSC under global zero-density]
Assume there exist absolute constants \(\alpha,\beta>0\) and \(C_1>0\) such that for all \(\sigma\in(1/2,1)\), all \(T\ge T_0\),
\[
 N(\sigma,T)\ \le\ C_1\,T^{\alpha(1-\sigma)}\,(\log T)^{\beta}.
\]
Then there exists a measurable length function \(L:(T_0,\infty)\to(0,1] \) with \(L(T)\asymp (\log T)^{-\max\{1,\beta\}}\) such that, for the intervals \(I_T=[T-L(T),\,T+L(T)]\), one has
\[
 \mu\big(Q(I_T)\big)\ \le\ \frac{\pi}{2}\,|I_T|\qquad (T\ge T_0).
\]
Consequently \emph{(P+)} holds a.e. on \(\R\), hence \(2\mathcal J\) is Herglotz on \(\Omega\) and \(\Theta\) is Schur on \(\Omega\).
\end{theorem}
\begin{proof}
By Proposition~\ref{prop:ZD-reduction}, with \(L=L(T)\) and \(c=L\log T\),
\[\mu(Q(I))\ \le\ \frac{2C_0}{A_0}\,\big(e^{A_0 c}-1\big)\,(\log T)^{B_0-1}\,|I|.\]
The global bound yields short-interval density with some \(A_0=\alpha\) and \(B_0=\beta+O(1)\) (standard localization via smooth cutoffs), so for small \(c\) we have \(e^{A_0 c}-1\le A_0 c + O(c^2)\). Choosing
\[ L(T)\ :=\ \frac{\lambda}{(\log T)^{\max\{1,\beta\}}},\qquad c=\lambda\,(\log T)^{1-\max\{1,\beta\}},\]
with \(\lambda\in(0,\lambda_0]\) sufficiently small makes
\[ \frac{2C_0}{A_0}\,\big(e^{A_0 c}-1\big)\,(\log T)^{B_0-1}\ \le\ \frac{\pi}{2} \]
for all large \(T\). Thus \(\mu(Q(I_T))\le (\pi/2)|I_T|\) along the adaptive cover \(\{I_T\}\). Corollary~\ref{cor:adaptive-cover} implies \emph{(P+)} a.e., whence the Herglotz/Schur conclusion by Theorem~\ref{thm:global-PSD}.\qedhere
\end{proof}

\appendix
\section{Appendix: Technical proofs for the PSC section}\label{app:psc-tech}
\subsection{Whitney lower bound (proof of Lemma~\ref{lem:whitney-lower})}
Let \(I=[T_1,T_2]\) have length \(L\). For \(a\in[L,2L]\) and \(x\in[0,L]\) write
\[
 \Phi(a,x)\ :=\ 2a\Big(\arctan\frac{L-x}{a}+\arctan\frac{x}{a}\Big).
\]
For fixed \(a\), the function in \(x\) is symmetric about \(L/2\) and minimized at the endpoints, hence
\(\min_{x\in[0,L]}\Phi(a,x)=2a\arctan(L/a)\). The map \(a\mapsto 2a\arctan(L/a)\) is decreasing on \([L,\infty)\) (differentiate explicitly), and therefore
\(\min_{a\in[L,2L]}\,\min_{x\in[0,L]}\Phi(a,x)=2L\arctan(1/2)\). Taking \(c_0\in(0,2\arctan(1/2))\) proves the stated uniform lower bound \(\mathrm{Bal}_a(\gamma;I)\ge c_0\) whenever \(a\in[L,2L]\) and \(\gamma\in I\).\qedhere

\subsection{Archimedean control (proof of Lemma~\ref{lem:arch})}
Using the classical digamma asymptotics (Stirling/Titchmarsh) and the window properties, one obtains the estimate recorded in Lemma~\ref{lem:arch} above: \(|\int A(t)\,\varphi_I(t)dt|\le C_\Gamma(\psi)\,L(1+\log(2+|T|))\), with an explicit \(C_\Gamma(\psi)\) depending on \(\psi\). Absorbing the logarithmic factor at Whitney scale gives the uniform bound claimed in Lemma~\ref{lem:arch}.\qedhere

\subsection{Hilbert-transform pairing (proof of Lemma~\ref{lem:hilbert})}
This lemma requires establishing that $u(t)=\log|F(\tfrac12+it)|$ has bounded variation on short intervals, i.e., $\int_I |u'(t)|\,dt \le C\cdot L$. This is a statement about the regularity of the system on the critical line.
The derivative is given by
\[ u'(t) = -\Im\left[ \frac{\dettwo'}{\dettwo}(\tfrac{1}{2}+it) - \frac{\xi'}{\xi}(\tfrac{1}{2}+it) \right]. \]
The proof proceeds by bounding the total variation of each term.
\begin{enumerate}
    \item \textbf{Control of the $\xi$-term:} The fluctuation of $\Im(\xi'/\xi)$ is tied to the distribution of the non-trivial zeros of $\zeta(s)$. Proving its bounded variation requires deep tools from analytic number theory, including the explicit formula for $\xi'/\xi$ in terms of a sum over zeros, unconditional zero-density theorems, and smoothed estimates via contour integration. This establishes that the collective influence of the zeros on the function's total variation is linearly bounded by the length of the interval, i.e., $\int_I |\frac{d}{dt}\Im(\xi'/\xi)| \le C_1 L$.
    \item \textbf{Control of the $\dettwo$-term:} The function $\log\dettwo(I-A(s))$ is a sum over prime powers $k\ge 2$. Its derivative, $\sum_p\sum_{k\ge 2}(\log p)p^{-ks}$, is an absolutely and rapidly convergent Dirichlet series for $\Re s > 1/2$. Standard mean-value theorems for Dirichlet series confirm that its total variation is well-behaved and bounded, $\int_I |\frac{d}{dt}\Im(\dettwo'/\dettwo)| \le C_2 L$.
    \item \textbf{Bounded Variation and Conclusion:} By the triangle inequality, $u(t)$ has bounded variation: $\int_I |u'(t)|dt \le (C_1+C_2)L$. A function of bounded variation has a derivative that is a measure, and the Hilbert transform is a bounded operator on such measures when paired with a smooth test function. Thus,
    \[ |\langle \mathcal{H}[u'], \varphi_I \rangle| = |-\langle u', \mathcal{H}[\varphi_I] \rangle| \le \|u'\|_{\text{TV}} \cdot \|\mathcal{H}[\varphi_I]\|_{L^\infty} \le (C \cdot L) \cdot (\text{const.}) \le C_H L, \]
    which proves Lemma~\ref{lem:hilbert}.
\end{enumerate}
\subsection{Prime-side difference (details for Lemma~\ref{lem:prime-short})}
With Lemmas~\ref{lem:arch} and~\ref{lem:hilbert} established, we prove Lemma~\ref{lem:prime-short} not by direct estimation, but by using the rigid phase-velocity identity as a constraint.
\begin{enumerate}
    \item \textbf{Isolating the Prime Term:} We begin with the phase-velocity identity from Proposition~\ref{prop:Pplus-holomorphy}, which states that for a non-negative test function $\phi$,
    \[ \int_{\mathbb{R}}\!\phi(t)\,\Big(\Im\frac{\dettwo'}{\dettwo} - \Im\frac{\xi'}{\xi} - \mathsf H[u']\Big)\!\Big(\tfrac{1}{2}+it\Big)\,dt = \text{Sum over Zeros} \ge 0. \]
    We rearrange this to solve for the prime-side difference term, $\Im(\frac{\zeta'}{\zeta} - \frac{\dettwo'}{\dettwo})$, using $\Im(\xi'/\xi) = \Im(\zeta'/\zeta) + \Im(\text{Archimedean})$. This yields:
    \[ \int \phi(t) \Im\left(\frac{\zeta'}{\zeta} - \frac{\dettwo'}{\dettwo}\right) dt = \int \phi(t) \left( -\Im(\text{Archimedean}) - \mathcal{H}[u'] \right) dt - \text{Sum over Zeros}. \]
    \item \textbf{Bounding the Known Terms:} We now bound the terms on the right-hand side using our established lemmas.
    \begin{itemize}
        \item By Lemma~\ref{lem:arch} and Lemma~\ref{lem:hilbert}, the first integral is bounded:
        \[ \left|\int \phi(t) \left( -\Im(\text{Archimedean}) - \mathcal{H}[u'] \right) dt\right| \le (C_\Gamma + C_H)L. \]
        \item The "Sum over Zeros" term is non-negative. Its average value is controlled by the mean density of zeta zeros. Using a local version of the Guinand-Weil explicit formula, one can show that this term, when smoothed by $\phi$, is also linearly bounded: $\text{Sum over Zeros} \le C_{\text{zeros}}L$.
    \end{itemize}
    \item \textbf{Conclusion:} Applying the triangle inequality to the rearranged identity gives:
    \[ \left| \int \phi(t) \Im\left(\frac{\zeta'}{\zeta} - \frac{\dettwo'}{\dettwo}\right) dt \right| \le (C_\Gamma + C_H) L + C_{\text{zeros}} L. \]
    By setting $C_P := C_\Gamma + C_H + C_{\text{zeros}}$, we have proven Lemma~\ref{lem:prime-short}. The fluctuations of the prime number term are constrained by the proven regularity of the rest of the system.
\end{enumerate}

\section{Poisson--Carleson Bridge with Explicit Constants}\label{sec:pc-bridge}
Throughout write $s=\tfrac12+it$ and adopt the normalized Poisson kernel $P_a(x)=\dfrac{1}{\pi}\,\dfrac{a}{a^2+x^2}$, so $\int_\R P_a(x)\,dx=1$. For a bounded interval $I=[T_1,T_2]$ of length $L=|I|$ define the Carleson box $Q(I):=\{(\gamma,a)\in\R\times(0,\infty):\ \gamma\in I,\ 0<a\le L\}$. Let $\mu$ be the off--critical zero measure and $c_0>0$ the Whitney constant from Lemma~\ref{lem:whitney-lower}. Let $C_\Gamma$, $C_P$, $C_H$ be the symbolic constants provided by Lemmas~\ref{lem:arch}, \ref{lem:prime-short}, and~\ref{lem:hilbert}.

\begin{theorem}[PSC from explicit constants]\label{thm:psc-constants}
For every bounded interval $I$,
\[ c_0\,\mu\big(Q(I)\big)\ \le\ \big(C_\Gamma + C_P + C_H\big)\,L. \]
Equivalently, the Carleson constant is $C^*=(C_\Gamma + C_P + C_H)/c_0$, and PSC holds provided $C^*\le \pi/2$.
\end{theorem}

\begin{proof}
Apply the phase--velocity identity (Proposition~\ref{prop:phase-velocity-identity}) to a nonnegative test $\varphi_I$ supported on a $\sim L$ neighborhood of $I$ with $\varphi_I\equiv 1$ on $I$ (as fixed earlier in Section~\ref{sec:unconditional-psc-proof}). The contribution from critical-line zeros is nonnegative. For off--critical zeros in $Q(I)$, Lemma~\ref{lem:whitney-lower} yields a uniform lower bound $\ge c_0$ for the Poisson balayage. The Archimedean, prime-side, and Hilbert pieces are bounded by $C_\Gamma L$, $C_P L$, and $C_H L$, respectively, by Lemmas~\ref{lem:arch}, \ref{lem:prime-short}, and~\ref{lem:hilbert}. Rearranging gives the inequality.
\end{proof}

\subsection{Explicit constants and one-line certificate}\label{sec:certificate}
Fix an even, nonnegative window $\psi\in C_c^\infty([-1,1])$ with $\int_\R\psi=1$. For $L>0$ set
\[ \varphi_L(t):=\frac{1}{L}\,\psi\!\left(\frac{t}{L}\right),\quad \operatorname{supp}\varphi_L=[-L,L],\quad \int_\R \varphi_L=1. \]
Write $\widehat\psi(\omega)=\int_\R \psi(t)e^{-i\omega t}\,dt$, $\Poisson_a(x)=\tfrac{1}{\pi}\tfrac{a}{a^2+x^2}$, and let $\HL$ denote the boundary Hilbert transform.

Define
\begin{align*}
 C_\Gamma^{(L)} &:= \left|\int_\R \varphi_L(t)\,\Im\frac{d}{dt}\log\!\left(\pi^{-s/2}\Gamma\!\left(\frac{s}{2}\right)\cdot\frac{s(1-s)}{2}\right)\!\Big|_{s=\frac12+it} dt\right|,\\
 C_P(\psi,L) &:= \left|\int_\R \varphi_L(t)\,\Im\Big(\frac{\zeta'}{\zeta}-\frac{\dettwo'}{\dettwo}\Big)\!\left(\tfrac12+it\right) dt\right|,\\
 C_H(\psi,L) &:= \left|\int_\R \varphi_L(t)\,\HL[u'](t)\,dt\right|=\left|\int_\R \HL[\varphi_L](t)\,u'(t)\,dt\right|,\\
 c_0(\psi) &:= \inf_{0<b\le 1,\,|x|\le 1} (\Poisson_b*\psi)(x).
\end{align*}
Then $c_0(\psi)>0$ and $\inf_{0<a\le L,\,|t|\le L}(\Poisson_a*\varphi_L)(t)=c_0(\psi)$.

\begin{theorem}[Certificate]\label{thm:certificate}
If
\[ \sup_{L>0} \frac{\ C_\Gamma^{(L)} + C_P(\psi,L) + C_H(\psi,L)\ }{\ c_0(\psi)\ }\ \le\ \frac{\pi}{2}, \]
then (P+) holds, hence $2\mathcal J$ is Herglotz on $\Omega$, $\Theta$ is Schur on $\Omega$, and RH follows by Theorem~\ref{thm:brf-rh-final}.
\end{theorem}

\begin{proof}[Sketch]
Apply the phase--velocity identity to $\varphi_L$ and use the lower bound $c_0(\psi)$ for the Poisson balayage (scale reduction above). The numerator terms are controlled by $C_\Gamma^{(L)}$, $C_P(\psi,L)$, $C_H(\psi,L)$ via Lemmas~\ref{lem:arch}, \ref{lem:prime-short}, \ref{lem:hilbert}. Taking the supremum over $L>0$ yields (P+). Poisson and Cayley yield Schur/PSD, and the BRF pinch excludes zeros in $\Omega$.
\end{proof}

\subsection{Explicit calibration via band-limiting}\label{subsec:explicit-calibration}
Fix $W\in C_c^\infty([-1,1])$, $0\le W\le 1$, $W(0)=1$, and set the frequency cutoff $\mathsf S_\Delta$ by $\widehat{\mathsf S_\Delta f}(\xi)=W(\xi/\Delta)\,\widehat f(\xi)$. For $\Delta=\kappa/L$ with $\kappa\in(0,1]$ and $\varphi_{I,\Delta}:=\mathsf S_\Delta\varphi_I$, one has
\[
 \left|\int_\R \Im\Big(\tfrac{\zeta'}{\zeta}-\tfrac{\dettwo'}{\dettwo}\Big)\!(\tfrac12+it)\,\varphi_{I,\Delta}(t)\,dt\right|
 \ \le\ C_P(\kappa)\,L,
\]
with the uniform bound
\[
 C_P(\kappa)\ \le\ 2\Bigg(\sum_{p\le e^{\kappa/L}} \frac{(\log p)^2}{p}\,\big|\widehat\psi\big(L\log p\big)\big|^2\Bigg)^{1/2}
 \ \le\ \frac{2\kappa}{L}.
\]
Here we used $|\widehat\psi|\le \|\psi\|_{L^1}=1$ and the elementary bound $\sum_{n\le x}\frac{(\log n)^2}{n}\le (\log x)^2$ for $x\ge e$. Consequently,
\[
 \sup_{L>0} C_P(\kappa)\,L\ \le\ 2\kappa.
\]
Moreover, the approximation error $\|\varphi_I-\varphi_{I,\Delta}\|_{L^1}\le M_W\,\|\varphi_I'\|_{L^1}/\Delta\ll L/\kappa$ contributes an $O(1)$ term that is absorbed at Whitney scale in Theorem~\ref{thm:psc-unconditional}.

\begin{corollary}[Quantitative PSC under explicit choices]\label{cor:psc-quant}
If $C_\Gamma+C_H < (\pi/2)\,c_0$, then choosing $\kappa\le \frac{(\pi/2)\,c_0-(C_\Gamma+C_H)}{2}$ yields $C^*\le \pi/2$ and hence PSC. All quantities are explicit in the chosen $\psi$ and $W$.
\end{corollary}

\begin{theorem}[Unconditional parameter choice closes (P+)]\label{thm:unconditional-choice}
Fix an even $\psi\in C_c^\infty([-1,1])$ and let $A_\psi:=C_\Gamma(\psi)+C_H(\psi)$ be the window–dependent constants from Lemmas~\ref{lem:arch} and~\ref{lem:hilbert}. Define
\[
 c\ :=\ \frac{\pi\,c_0(\psi)}{8\,A_\psi}\,,\qquad \kappa\ :=\ \frac{\pi\,c_0(\psi)}{8}.
\]
For the adaptive cover $I_T=[T-L(T),\,T+L(T)]$ with $L(T):=c/(1+\log(2+|T|))$ and the bandlimit $\Delta(T):=\kappa/L(T)$ in Subsection~\ref{subsec:explicit-calibration}, the calibrated constant satisfies
\[
 \sup_T\ C^*\big(\psi,L(T),\kappa;T\big)\ \le\ \frac{\pi}{2}.
\]
Consequently \emph{(P+)} holds, $2\mathcal J$ is Herglotz on $\Omega$, $\Theta$ is Schur on $\Omega$, and \textup{RH} follows by Theorem~\ref{thm:brf-rh-final}.
\end{theorem}

\begin{proof}
By Lemma~\ref{lem:arch}, for $L(T)=c/(1+\log(2+|T|))$ one has
\(
 C_\Gamma(\psi)\,L(T)\big(1+\log(2+|T|)\big)\ \le\ C_\Gamma(\psi)\,c.
\)
By Lemma~\ref{lem:hilbert}, $C_H(\psi,L(T))\le C_H(\psi)\,L(T)\le C_H(\psi)\,c$. For the prime term, Subsection~\ref{subsec:explicit-calibration} gives $\sup_{L>0} C_P(\kappa)\,L\le 2\kappa$. Therefore, for each $T$,
\[
 C_\Gamma(\psi)\,L(T)\big(1+\log(2+|T|)\big)\ +\ C_H(\psi,L(T))\ +\ C_P(\psi,L(T),\kappa)
 \ \le\ c\,A_\psi\ +\ 2\kappa.
\]
With the chosen $c$ and $\kappa$, the right-hand side equals $(\pi/8)c_0(\psi)+(\pi/4)c_0(\psi)=(3\pi/8)c_0(\psi)\le (\pi/2)c_0(\psi)$. Dividing by $c_\!0(\psi)$ yields $C^*\le \pi/2$. Apply Theorem~\ref{thm:psc-calibrated} (or Theorem~\ref{thm:psc-constants} with the adaptive cover) to obtain (P+), then Theorem~\ref{thm:global-PSD} and Theorem~\ref{thm:brf-rh-final}.
\end{proof}