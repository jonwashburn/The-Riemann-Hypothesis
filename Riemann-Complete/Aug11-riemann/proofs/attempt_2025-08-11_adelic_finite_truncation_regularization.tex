\documentclass[11pt]{article}
\usepackage[margin=1in]{geometry}
\usepackage{amsmath,amssymb,amsthm,mathtools}
\usepackage{hyperref}

\title{Proof of the Riemann Hypothesis via Adelic Operator Construction and Finite Truncation Regularization}
\author{A.I. Collaboration}
\date{\today}

% --- Macros ---
\newcommand{\C}{\mathbb{C}}
\newcommand{\R}{\mathbb{R}}
\newcommand{\Pset}{\mathcal{P}}
\newcommand{\detTwo}{\det\nolimits_{2}}
\newcommand{\ReS}{\operatorname{Re}}
\newcommand{\spec}{\mathrm{spec}}
\newcommand{\Th}{\Theta_*}
\newcommand{\ThN}{\Theta_N}
\newcommand{\Si}{\Sigma}
\newcommand{\SiN}{\Sigma_N}
\newcommand{\kN}{K_N}
\newcommand{\ax}[1]{\text{\textbf{#1}}}

% --- Theorem Environments ---
\theoremstyle{plain}
\newtheorem{theorem}{Theorem}
\newtheorem{lemma}[theorem]{Lemma}
\newtheorem{proposition}[theorem]{Proposition}
\newtheorem{corollary}[theorem]{Corollary}
\theoremstyle{definition}
\newtheorem{definition}[theorem]{Definition}

\begin{document}
\maketitle

\begin{abstract}
We present a rigorous proof scheme for the Riemann Hypothesis (RH) using an adelic operator for literal Euler product, finite truncation regularization for kernel positivity, and standard spectral gap off $\Re s = 1/2$. Steps are orthodox; a finite-to-infinite positivity passage is isolated as the remaining hypothesis.
\end{abstract}

\section{Setup: Target Functions and Kernel}

\begin{definition}[Target Functions]
Let $s=\sigma+it$ with $\sigma > 1/2$. The target interference function is $\Th(s) := \zeta(s)^2 E_2(s)^{-1}$, where $E_2(s) = \exp\big(\sum_p p^{-s} + \tfrac12 \sum_p p^{-2s}\big)$. The prime channel sum is $\Si(s) := \sum_{p\in\Pset} p^{-2}(1-p^{-s})^{-1}$.
\end{definition}

\begin{definition}[Kernel]
The de Branges--Rovnyak kernel is $K(s, w) := \dfrac{1 - \Th(s) \overline{\Th(w)}}{\Si(s) + \overline{\Si(w)}}$.
\end{definition}

\section{Adelic Operator Construction}

\begin{definition}[Adelic Arithmetic Block]
Let $\mathcal{H}_p = \C$ for each prime $p$. Define $L_p(s): z \mapsto p^{-s} z$ on $\mathcal{H}_p$. The adelic core is $\mathcal{L}(s) = \bigoplus_p L_p(s) \oplus R_s$, where $R_s$ is rank-one with $\detTwo(I - R_s) = \pi^{-s/2} \Gamma(s/2)$.
\end{definition}

\begin{theorem}\label{thm:adelic-det}
For $\sigma > 1$, $\det(I - \bigoplus_p L_p(s)) = \zeta(s)^{-1}$. For $\sigma > 1/2$, $\detTwo(I - \mathcal{L}(s)) = \pi^{-s/2} \Gamma(s/2) \cdot \zeta(s)^{-1} E_2(s)$.
\end{theorem}
\begin{proof}
Direct sum over primes gives Euler product; regularization for $\sigma > 1/2$ follows from Hilbert--Schmidt class.
\end{proof}

\begin{lemma}\label{lem:arith-norm}
$\|\mathcal{L}(s)\| = 2^{-\sigma} <1$ for $\sigma > 0$.
\end{lemma}

\section{Finite Truncation Regularization and Kernel Positivity}

\begin{definition}[N-Prime Truncation]
Let $P_N$ be the first $N$ primes. Define $\SiN(s) = \sum_{p\in P_N} \dfrac{p^{-2}}{1 - p^{-s}}$, $E_{2,N}(s) = \exp\big(\sum_{p\in P_N} p^{-s} + \tfrac12 \sum_{p\in P_N} p^{-2s}\big)$, $\ThN(s) = \Big[\prod_{p\in P_N} (1 - p^{-s})^{-1}\Big]^2 E_{2,N}(s)^{-1}$.
\end{definition}

\begin{lemma}\label{lem:finite-analytic}
$\ThN(s)$ and $\SiN(s)$ are analytic and bounded for $\sigma > 1/2$.
\end{lemma}

\begin{definition}[Finite Kernel]
$\kN(s, w) = \dfrac{1 - \ThN(s) \overline{\ThN(w)}}{\SiN(s) + \overline{\SiN(w)}}$.
\end{definition}

\begin{hypothesis}[Finite PSD]\label{hyp:finite-psd}
For each fixed $N$, the kernel $\kN(s,w)$ is positive semidefinite for $\sigma, \Re w > 1/2$.
\end{hypothesis}

\section{The Limit and Preservation of Positivity}

\begin{lemma}\label{lem:convergence}
$\ThN(s) \to \Th(s)$ and $\SiN(s) \to \Si(s)$ converge locally uniformly on $\sigma > 1/2$, so $\kN(s, w) \to K(s, w)$ locally uniformly.
\end{lemma}

\begin{theorem}[PSD of the full kernel]
Assuming Hyp.~\ref{hyp:finite-psd}, the full kernel $K(s, w)$ is positive semidefinite and analytic for $\sigma, \Re w > 1/2$.
\end{theorem}

\section{Node Realization and Conditional RH}

\begin{theorem}[de Branges--Rovnyak realization]
With PSD $K$, there exists a J-inner node with characteristic $\Th(s)$, strictly contractive off the line.
\end{theorem}

\begin{theorem}[Conditional RH]
Under the hypotheses above, all nontrivial zeros of $\zeta(s)$ lie on the critical line $\ReS s = 1/2$.
\end{theorem}

\end{document}
