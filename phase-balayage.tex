% Phase–Balayage on the Half–Plane: Measure, Kernel, Bounds, and Neutralization
% Ready to \input into the manuscript; assumes theorem environments are defined.

\section*{Phase--balayage on the half--plane (identity and neutralization)}

\paragraph{Setting and notation.}
Work in the upper half--plane $\mathbb{H}=\{z=x+iy:y>0\}$ with boundary $\mathbb{R}$. Let $F$ be meromorphic in $\mathbb{H}$, of bounded type, with non--tangential boundary values a.e. on $\mathbb{R}$. Write
\[
\log F = U+iV \qquad (U=\log|F|,\; V=\arg F)
\]
on simply connected subsets of $\mathbb{H}$ that avoid zeros and poles. For $\varphi\in C_c^\infty(\mathbb{R})$, let $\mathcal{P}[\varphi]$ denote its Poisson extension and $\mathcal H[\varphi]$ its Hilbert transform. Write $w(t):=\Arg F(t)$ and $u(t):=\log|F(t)|$.

\paragraph{Zero--pole measure and Bal kernel.}
Let $Z(F)$ (resp. $P(F)$) be the multisets of zeros (resp. poles) with multiplicities. Define the signed atomic measure
\[
\mu\;:=\;\sum_{p\in P(F)} m(p)\,\delta_p\;-
\;\sum_{a\in Z(F)} m(a)\,\delta_a\qquad\text{on }\mathbb{H},
\]
so poles contribute positively and zeros negatively. For a bounded interval $I=(a,b)\subset\mathbb{R}$ and $z=x+iy\in\mathbb{H}$, define
\[
\mathrm{Bal}(z;I)\;:=\;2\pi\,\mathcal{P}[\mathbf{1}_I](z)
\;=\;\int_{a}^{b}\frac{2y}{(t-x)^2+y^2}\,dt
\;=\;2\Big(\arctan\tfrac{b-x}{y}-\arctan\tfrac{a-x}{y}\Big),
\]
with $0\le \mathrm{Bal}\le 2\pi$, increasing in $I$, and $\mathrm{Bal}(z;\mathbb{R})=2\pi$.

\begin{lemma}[Far--field bound for the Bal kernel]\label{lem:farfield}
Let $I=(a,b)$ with length $L=b-a$ and center $c=\tfrac{a+b}{2}$. Then, for $z=x+iy\in\mathbb{H}$,
\[
\mathrm{Bal}(z;I)\ \le\ \frac{4Ly}{(x-c)^2+y^2-\tfrac{L^2}{4}}\quad\text{whenever }(x-c)^2+y^2>\tfrac{L^2}{4}.
\]
In particular, if $z\notin Q(\alpha I):=\{x+iy: x\in \alpha I,\ 0<y\le \alpha L\}$ with $\alpha>1$, then
\[
\mathrm{Bal}(z;I)\ \le\ \frac{C_\alpha\,L\,y}{\mathrm{dist}(z,I)^2+y^2},
\]
with $C_\alpha$ depending only on $\alpha$.
\end{lemma}

\begin{theorem}[Phase--balayage, test--function form]\label{thm:balayage-test}
Assume $F$ is meromorphic in $\mathbb{H}$ of bounded type and $u=\log|F|\in L^1_{\mathrm{loc}}(\mathbb{R})$. Then for every $\varphi\in C_c^\infty(\mathbb{R})$,
\[
\boxed{\quad
\int_{\mathbb{R}} (-w'(t))\,\varphi(t)\,dt
\;=\;
2\pi\int_{\mathbb{H}} \mathcal{P}[\varphi](z)\,d\mu(z)\;-
\;\int_{\mathbb{R}}\mathcal{H}[u'](t)\,\varphi(t)\,dt.
\quad}
\]
Equivalently, $\langle -w',\varphi\rangle=\int_{\mathbb{H}} 2\pi\,\mathcal{P}[\varphi]\,d\mu-\langle \mathcal{H}[u'],\varphi\rangle$.
\end{theorem}

\begin{proof}
Use Poisson--Jensen for $U$, take harmonic conjugates for $V$, differentiate tangentially, and pass to non--tangential boundary limits. The Poisson identity
$\int \tfrac{2y}{(t-x)^2+y^2}\,\varphi(t)\,dt=2\pi\,\mathcal P[\varphi](x+iy)$ converts each atom to the Poisson balayage. Summing with signs yields the identity.
\end{proof}

\begin{corollary}[Phase--balayage, interval form]\label{cor:balayage-interval}
Under the hypotheses of Theorem~\ref{thm:balayage-test}, for any bounded open interval $I=(a,b)$ whose endpoints are points of approximate continuity for $u$ and $w$,
\[
\boxed{\quad
\int_{I} (-w'(t))\,dt
\;=\;\int_{\mathbb{H}}\mathrm{Bal}(z;I)\,d\mu(z)
\;-
\big(\mathcal{H}[u](b)-\mathcal{H}[u](a)\big).
\quad}
\]
\end{corollary}

\begin{proof}
Approximate $\mathbf{1}_I$ by smooth $\varphi_n$ and pass to the limit in Theorem~\ref{thm:balayage-test} using dominated convergence for the Poisson term and a.e. boundary values for $\mathcal H[u]$ at the endpoints.
\end{proof}

\paragraph{Neutralization on a Carleson box.}
For a bounded interval $I$ and aperture $\alpha>1$, set
\[
Q(I)=\{x+iy : x\in I,\ 0<y\le |I|\},\qquad Q(\alpha I)=\{x+iy : x\in \alpha I,\ 0<y\le \alpha|I|\}.
\]
Define the local Blaschke neutralizer on $Q(\alpha I)$ by
\[
B_I(z):=\prod_{a\in Z(F)\cap Q(\alpha I)} \Big(\tfrac{z-a}{z-\overline{a}}\Big)^{m(a)}\,
\prod_{p\in P(F)\cap Q(\alpha I)} \Big(\tfrac{z-\overline{p}}{z-p}\Big)^{m(p)}.
\]
Then $B_I$ is inner: $|B_I(t)|=1$ for a.e. $t\in\mathbb{R}$. Let $\widetilde F:=F/B_I$. By construction, $\widetilde F$ has no zeros or poles in $Q(\alpha I)$ and $\mathcal H[\log|\widetilde F|]=\mathcal H[\log|F|]$ a.e.

\begin{corollary}[Neutralized phase--balayage on $I$]\label{cor:neutralized}
With the above notation,
\[
\int_I (-w_F')\,dt
\;=\; \int_{Q(\alpha I)} \mathrm{Bal}(z;I)\,d\mu(z)
\;+
\int_{\mathbb{H}\setminus Q(\alpha I)} \mathrm{Bal}(z;I)\,d\mu(z)
\;-
\big(\mathcal{H}[u](b)-\mathcal{H}[u](a)\big),
\]
so that
\[
\int_I (-w_{\widetilde F}')\,dt
\;=\;\int_{\mathbb{H}\setminus Q(\alpha I)} \mathrm{Bal}(z;I)\,d\mu_{\mathrm{far}}(z)
\;-
\big(\mathcal{H}[u](b)-\mathcal{H}[u](a)\big).
\]
Moreover,
\[
\int_I(-w_{B_I}')\,dt\;=\;\int_{Q(\alpha I)} \mathrm{Bal}(z;I)\,d\mu_{\mathrm{near}}(z).
\]
\end{corollary}

\begin{lemma}[Uniform bounds after neutralization]\label{lem:uniform-bounds}
For $z\notin Q(\alpha I)$,
\[
0\ \le\ \mathrm{Bal}(z;I)\ \le\ 2\pi\ \min\!\left\{1,\ \frac{C_\alpha\,|I|\,\Im z}{\mathrm{dist}(z,I)^2+(\Im z)^2}\right\}.
\]
Consequently,
\[
\left|\int_{\mathbb{H}\setminus Q(\alpha I)} \mathrm{Bal}(z;I)\,d\mu(z)\right|
\;\le\; 2\pi\,|\mu|(\mathbb{H}\setminus Q(\alpha I)),
\]
with refined decay available from Lemma~\ref{lem:farfield} when $\mathrm{dist}(z,I)\gg |I|$.
\end{lemma}

\begin{remark}
The identity separates two mechanisms:
\[
\int_I (-w')= \underbrace{\int \mathrm{Bal}\,d\mu}_{\text{discrete interior data: poles $-$ zeros}} - \underbrace{\big(\mathcal{H}[u](b)-\mathcal{H}[u](a)\big)}_{\text{outer (boundary--driven) term}}.
\]
The Bal term is purely geometric ($0\le \mathrm{Bal}\le 2\pi$); after neutralization it becomes a controlled far--field sum. The outer term depends only on the boundary modulus and is the only source of sign changes unrelated to zeros/poles; it is handled by fixed-aperture $H^1$–BMO/Carleson control against smooth windows.
\end{remark}
