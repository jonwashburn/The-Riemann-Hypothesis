\documentclass[11pt]{article}
\usepackage{booktabs}
\usepackage{float}
% Robust CSV tables
\usepackage{longtable}
\usepackage{caption}
\usepackage[margin=1in]{geometry}
\usepackage{amsmath,amssymb,amsthm,mathtools}
\usepackage[T1]{fontenc}
\usepackage{lmodern}
\usepackage{microtype}
\usepackage{hyperref}
\usepackage[numbers,sort&compress]{natbib}
\hypersetup{colorlinks=true,linkcolor=blue,citecolor=blue,urlcolor=blue}

% Reference aliasing to silence legacy labels
% Global numeric constants
\providecommand{\Mpsilocked}{0.0800745}% single source of truth for M_\psi bound
\makeatletter
\newcommand{\refalias}[2]{%
  \@ifundefined{r@#2}{}{\expandafter\global\expandafter\let\csname r@#1\endcsname\csname r@#2\endcsname}%
}
\makeatother
\AtBeginDocument{%
  \refalias{sec:CH-envelope}{lem:CH-explicit}%
  \refalias{lem:poisson-lower}{lem:poisson-scale-stage2}%
  \refalias{lem:hilbert-aux}{lem:hilbert-H1BMO}%
  \refalias{lem:laplace-szego}{prop:discrete-Poisson}%
  \refalias{lem:cayley-cont}{lem:Cayley-diff}%
  \refalias{lem:wedge-stage2}{thm:numeric-close-stage2}%
  % bridge aliases removed to avoid early expansion issues
  %\refalias{sec:bridge-C}{thm:bridge-C}%
  %\refalias{thm:BridgeA}{thm:bridgeA}%
  \refalias{cor:interior-alignment}{thm:globalize-RH}%
}

% Theorems
\newtheorem{theorem}{Theorem}
\newtheorem{proposition}[theorem]{Proposition}
\newtheorem{lemma}[theorem]{Lemma}
\newtheorem{corollary}[theorem]{Corollary}
\theoremstyle{definition}
\newtheorem{definition}[theorem]{Definition}
\theoremstyle{remark}
\newtheorem{remark}[theorem]{Remark}

% Macros
\newcommand{\C}{\mathbb{C}}
\newcommand{\R}{\mathbb{R}}
\newcommand{\N}{\mathbb{N}}
\newcommand{\PP}{\mathcal{P}}
\newcommand{\HS}{\mathcal{S}_2}
\newcommand{\Half}{\{\,s\in\C:\ \Re s>\tfrac12\,\}}
\newcommand{\Poisson}{P}
\DeclareMathOperator{\Tr}{Tr}
\DeclareMathOperator{\dettwo}{det_2}
\DeclareMathOperator{\Arg}{Arg}

% Title & authors
\title{Prime-Tail Schur-Covering in the Bounded-Real Framework: Unconditional Bridges B--C and a Certified Covering}
\author{Jonathan Washburn\\ Independent Researcher\\ \href{mailto:washburn.jonathan@gmail.com}{washburn.jonathan@gmail.com}}
\date{\today}

\begin{document}
\maketitle

\begin{abstract}
We develop unconditional operator tools for a bounded-real (Herglotz/Schur) program on the right half-plane \(\Omega=\{\Re s>\tfrac12\}\). Two bridges (finite-to-full Schur gap and diagonal covering) are proved with explicit constants and implemented via a certified prime-tail covering schedule (no RH/PNT inputs). We also implement a structural redesign that algebraically closes Bridge~A: fix an \(s\)-independent, strictly upper-triangular Hilbert--Schmidt padding \(K\) and set \(T_{\rm new}(s):=T(s)+K\). A power--trace lock \(\Tr(T_{\rm new}(s)^n)=\Tr(T(s)^n)\) for \(n\ge 2\) yields \(\det_2(I-T_{\rm new}(s))\equiv\det_2(I-T(s))\) and
\[
  \xi(s)\;=\;e^{L(s)}\,\det\nolimits_2\!\big(I-T_{\rm new}(s)\big).
\]
Thus the auxiliary factor can be taken explicit and zero-free on \(\{\Re s>1\}\) (prime-sum expression), and extends as a holomorphic, zero-free normalizer on \(\{\Re s>\tfrac12\}\) by analytic continuation via the det$\_2$ identity. Bridges~B--C and the covering certificate are unchanged: the contribution of \(K\) is fixed, uniformly bounded by prime tails, and \(\Delta_{\mathrm{FF}}^{(K)}=0\).
\medskip\noindent\textbf{Proof strategy.} We prove (P+) only via the product certificate: the phase--velocity pairing and explicit constants give a uniform boundary wedge independent of interval length; Poisson then yields that \(2\mathcal J\) is Herglotz on \(\Omega\) and \(\Theta\) is Schur; the standard pinch/globalization argument excludes interior poles of \(\mathcal J\) and RH follows. The PSC/Carleson density discussion is archived and not used to deduce (P+). The Bridges A--C/Schur--covering material is included as a secondary perspective.
\end{abstract}

\paragraph{Keywords.} Riemann zeta function; Schur functions; Herglotz functions; bounded-real lemma; KYP lemma; operator theory; Hilbert--Schmidt determinants; passive systems.

\paragraph{MSC 2020.} 11M06, 30D05, 47A12, 47B10, 93B36, 93C05.

\section{Introduction}
\noindent\textit{PSC/BMO status.} The proof proceeds via the PSC boundary route with the in-paper certificate and locked constants; Bridges A--C are presented as an optional companion perspective.
The Riemann Hypothesis (RH) admits several analytic formulations. In this paper we pursue a bounded-real (BRF) route on the right half-plane
\[
 \Omega\;:=\;\Half,
\]
which is naturally expressed in terms of Herglotz/Schur functions and passive systems. Let \(\PP\) be the primes, and define the prime-diagonal operator
\[
 A(s):\ell^2(\PP)\to\ell^2(\PP),\qquad A(s)e_p\;:=\;p^{-s}e_p.
\]
For \(\sigma:=\Re s>\tfrac12\) we have \(\|A(s)\|_{\HS}^2=\sum_{p\in\PP}p^{-2\sigma}<\infty\) and \(\|A(s)\|\le 2^{-\sigma}<1\). With the completed zeta function
\[
 \xi(s)\;:=\;\tfrac12 s(1-s)\,\pi^{-s/2}\,\Gamma(s/2)\,\zeta(s)
\]
and the Hilbert--Schmidt regularized determinant \(\dettwo\), we study the analytic function
\[
 \mathcal J(s)\;:=\;\frac{\dettwo(I-A(s))}{\mathcal O(s)\,\xi(s)},\qquad \Theta(s)\;:=\;\frac{2\mathcal J(s)-1}{2\mathcal J(s)+1}.
\]
The BRF assertion is that \(|\Theta(s)|\le 1\) on \(\Omega\) (Schur), equivalently that \(2\mathcal J(s)\) is Herglotz or that the associated Pick kernel is positive semidefinite.

Our method combines four ingredients:
\begin{itemize}
 \item \textbf{Schur--determinant splitting.} For a block operator \(T(s)=\begin{bmatrix}A(s)&B(s)\\ C(s)&D(s)\end{bmatrix}\) with finite auxiliary part, one has
 \[
  \log\dettwo(I-T)\;=\;\log\dettwo(I-A)\; +\; \log\det(I-S),\qquad S\;:=\;D-C(I-A)^{-1}B,
 \]
 which separates the Hilbert--Schmidt (\(k\ge 2\)) terms from the finite block.
 \item \textbf{HS continuity for \(\dettwo\).} Prime truncations \(A_N\to A\) in the HS topology, uniformly on compacts in \(\Omega\), imply local-uniform convergence of \(\dettwo(I-A_N)\). Division by \(\xi\) is justified only on compacts avoiding its zeros; throughout we explicitly state such hypotheses when needed.
 \item \textbf{Finite-stage passivity via KYP.} We construct, for each \(N\), an explicit lossless realization tied to the primes (``prime-grid lossless'') that certifies \(\|H_N\|_\infty\le 1\). A succinct factorization of the KYP matrix verifies passivity with a diagonal Lyapunov witness.
 \item \textbf{Interior passive approximation on zero-free rectangles.} On zero-free rectangles we build Schur rational approximants converging locally uniformly to \(\Theta\). This yields local Schur control on \(\Omega\setminus Z(\xi)\).
\end{itemize}


\subsection*{Interior closure on rectangles via Gram/Fock and NP--Schur}
\paragraph{Route note (primary interior PSD).}
We adopt this interior Herglotz/Gram--Fock route as the main positivity mechanism on rectangles. It does not use row/column absolute-sum estimates (Schur/Gershgorin) and is robust as $\sigma\downarrow\tfrac12$: positivity is proved via kernel factorizations and Schur products, then transported from the boundary to the interior by the maximum principle for PSD kernels. In particular, it bypasses the absolute-sum divergences that motivate conservative Schur-test budgets near the boundary. This route is fully compatible with the structural redesign in Bridge~A (triangular padding): the determinant identity and zero-free normalizer $e^{L}$ are independent inputs here, and the interior PSD argument proceeds unchanged.
We outline an interior closure on zero-free rectangles that avoids any circular "zero-free collar" assumption by working on punctured boundaries and, when needed, compensating interior zeros of \(\xi\) by a half-plane Blaschke product. The chain is:
\begin{enumerate}
  \item \textbf{Additive/log Gram positivity.} Using the backward-difference identity for Szeg\H{o} features and Bochner integration over the prime-power grid, the logarithmic kernel
  \[
    H_{\log\det_2^N}(s,\overline t)
    = \int_0^\infty \frac{1}{x}\Big(\int_0^\infty (\Delta_x\phi)_s\,\overline{(\Delta_x\phi)_t}\,du\; -\; \int_0^x \phi_s\,\overline{\phi_t}\,du\Big)\,d\mu_N(x)
  \]
  is PSD on \(\partial R\), for any rectangle \(R\Subset\Omega\).
  \item \textbf{Symmetric-Fock exponential lift aligned with half-plane Szeg\H{o}.} Define the PSD kernel
  \(\Lambda_N(s,\overline t):=\int_0^\infty x^{-1}\int_0^x \phi_s\overline{\phi_t}\,du\,d\mu_N(x)\), and
  \(E_N:=\exp(\Lambda_N-\tfrac12\mathrm{diag}-\tfrac12\mathrm{diag})\).
  Then on \(\partial R\), the finite-matrix inequality
  \[
    \frac{e^{\mathfrak g_N(s)}+\overline{e^{\mathfrak g_N(t)}}}{s+\overline t-1}\ \succeq\ E_N(s,\overline t)\,\frac{1}{s+\overline t-1}
  \]
  holds (Fock--Gram lower bound).
  \item \textbf{Punctured boundary multiplier by \(\xi^{-1}\).} On the punctured boundary \(\partial R\setminus\Sigma_R\) (\(\Sigma_R:=\{\xi=0\}\cap\partial R\)), Schur products preserve PSD for kernels. The transformation to \(H_{J_N}(s,\overline t)=(J_N(s)+\overline{J_N(t)})/(s+\overline t-1)\) is effected by a boundary normalization and kernel factorization developed below.
  \item \textbf{Boundary \(\Rightarrow\) interior (Schur).} From the boundary positivity obtained above, the maximum principle gives \(\Re J_N\ge 0\) on \(R\). The Cayley map yields \(|\Theta_N|\le 1\) on \(R\). Thus \(\Theta_N\) is Schur on \(R\). One may alternatively construct Schur interpolants on \(R\) via conformal transfer and NP/CF.
  \item \textbf{Exhaustion and removable singularities.} On compacts away from \(Z(\xi)\), \(\Theta_N\to\Theta\) locally uniformly. A diagonal extraction over an exhaustion by rectangles yields a global Schur sequence converging to \(\Theta\) on \(\Omega\setminus Z(\xi)\); removable singularities across \(Z(\xi)\) give holomorphy and \(|\Theta|\le 1\) on \(\Omega\). Finally, the maximum-modulus pinch excludes zeros of \(\xi\) in \(\Omega\).
\end{enumerate}
\noindent
\emph{Interior zeros of \(\xi\).} If \(\xi\) has zeros inside \(R\), replace \(J\) by the compensated ratio \(J^{\mathrm{comp}}:=J\,B_{\xi,R}\) using the half-plane Blaschke product over those zeros. The steps above apply verbatim to \(J^{\mathrm{comp}}\) and its Cayley transform; undoing the compensation at the end recovers Schur approximants for the original target.

\subsection*{Interior Closure on Zero-Free Rectangles (formal statements)}
We now record the interior route as a formal chain of lemmas and theorems valid on zero-free rectangles. Throughout, \(\Omega=\{\Re s>\tfrac12\}\), and
\[\mathcal J_N(s):=\frac{\det_2^N(I-A(s))}{\mathcal O(s)\,\xi(s)},\quad \mathcal J(s):=\frac{\det_2(I-A(s))}{\mathcal O(s)\,\xi(s)},\quad \Theta_N:=\frac{2\mathcal J_N-1}{2\mathcal J_N+1},\quad \Theta:=\frac{2\mathcal J-1}{2\mathcal J+1}.
\]

\begin{lemma}[Additive/log kernel PSD]\label{lem:log-psd-formal}
Let \(d\mu_N(x):=\sum_{p\le P_N}\sum_{k\ge2}(\log p)\,\delta_{k\log p}(dx)\). With \(\phi_s(u):=e^{-(s-\frac12)u}\) and \((\Delta_x\phi)_s(u):=\phi_s(u)-\phi_s(u+x)\), the kernel
\[H_{\log\det_2^N}(s,\overline t):=\int_0^\infty \frac{1}{x}\Big(\int_0^\infty (\Delta_x\phi)_s\,\overline{(\Delta_x\phi)_t}\,du\; -\; \int_0^x \phi_s\,\overline{\phi_t}\,du\Big)d\mu_N(x)
\]
is positive semidefinite on \(\Omega\) and in particular on \(\partial R\) for any rectangle \(R\Subset\Omega\).
\end{lemma}

\paragraph{Remark (multiplicities).}
If a zero $\rho_j$ has multiplicity $m_j$, include it in $B_I$ with exponent $m_j$:
\[
 B_I(z)\ :=\ \prod_{a_j\in\mathcal Z_I}\Big(\frac{z-a_j}{z+\overline{a_j}}\Big)^{m_j}.
\]
All properties used here (inner boundary modulus, harmonicity of $\Re\log B_I$, and cancellation of interior singularities) are preserved. The Whitney-box energy and pairing bounds are unchanged, since near/far contributions scale linearly in the multiplicities and the short-interval count $N(T;H)$ is taken with multiplicity.

\subsection*{Unsmoothing det$_2$: routed through smoothed testing (A1)}
\begin{lemma}[Smoothed distributional bound for $\partial_\sigma\,\Re\log\dettwo$]\label{lem:det2-unsmoothed}
Let $I\Subset\R$ be a compact interval and fix $\varepsilon_0\in(0,\tfrac12]$. There exists a finite constant
\[
  C_*\ :=\ \sum_{p}\sum_{k\ge 2}\frac{p^{-k/2}}{k^2\,\log p}\ <\ \infty
\]
such that for all $\sigma\in(\tfrac12,\tfrac12+\varepsilon_0]$ and every $\varphi\in C_c^2(I)$,
\[
  \Big|\int_{\R} \varphi(t)\,\partial_\sigma\,\Re\log\dettwo\big(I-A(\sigma+it)\big)\,dt\Big|\ \le\ C_*\,\|\varphi''\|_{L^1(I)}.
\]
In particular, testing against smooth, compactly supported windows yields bounds uniform in $\sigma$.
\end{lemma}
\begin{proof}
For $\sigma>\tfrac12$ one has the absolutely convergent expansion
\[
  \partial_\sigma\,\Re\log\dettwo\big(I-A(\sigma+it)\big)
  \;=\; \sum_{p}\sum_{k\ge 2} (\log p)\,p^{-k\sigma}\cos(k t\log p).
\]
For each frequency $\omega=k\log p\ge 2\log 2$, two integrations by parts give
\[
  \Big|\int_{\R}\!\varphi(t)\cos(\omega t)\,dt\Big|\ \le\ \frac{\|\varphi''\|_{L^1(I)}}{\omega^2}.
\]
Summing the resulting majorant yields
\[
  \Big|\int \varphi\,\partial_\sigma\Re\log\dettwo\,dt\Big|
  \ \le\ \|\varphi''\|_{L^1}\sum_{p}\sum_{k\ge 2}\frac{(\log p)\,p^{-k\sigma}}{(k\log p)^2}
  \ \le\ \|\varphi''\|_{L^1}\sum_{p}\sum_{k\ge 2}\frac{p^{-k/2}}{k^2\,\log p},
\]
uniformly for $\sigma\in(\tfrac12,\tfrac12+\varepsilon_0]$, since the rightmost double series converges. This proves the claim.
\end{proof}

\begin{lemma}[Scaled derivative bound for the Hilbert envelope]\label{lem:hilbert-derivative}
Let $\varphi_L(t):=L^{-1}\psi\big((t-T)/L\big)$ be the mass--1 window. Then
\[
  \big\|\big(\mathcal H[\varphi_L]\big)'\big\|_{L^\infty(\R)}\ \le\ \frac{C_H(\psi)}{L},\qquad C_H(\psi)\le 0.65.
\]
\end{lemma}
\begin{proof}
By the scaling computation $\mathcal H[\varphi_L](t)=H_\psi\big((t{-}T)/L\big)$, we have $\big(\mathcal H[\varphi_L]\big)'(t)=\tfrac1L\,H_\psi'\big((t{-}T)/L\big)$, hence $\|\big(\mathcal H[\varphi_L]\big)'\|_\infty\le L^{-1}\,\|H_\psi'\|_\infty$. Writing $\psi$ as in the proof above (plateau plus symmetric $C^\infty$ ramps) and differentiating the plateau term gives $|H'_{\text{plat}}(x)|\le \tfrac{h}{\pi}\big(\tfrac{1}{|x{+}1{-}\varepsilon|}+\tfrac{1}{|x{-}(1{-}\varepsilon)|}\big)$, maximized at $x=0$. The ramp contributions are controlled by a second integration by parts on the transition layers, using $\|S'\|_{L^1}$ and $\|S''\|_{L^1}$ for the cosine ramp. With $\varepsilon=\delta=0.01$ the same numerical envelope $C_H(\psi)\le 0.65$ dominates both $\|H_\psi\|_\infty$ and $\|H_\psi'\|_\infty$, yielding the claim.
\end{proof}

\subsection*{Executable finite-block certificate (model; weighted $p$-adaptive; not used)}
\noindent\textbf{Disclaimer (model-only).} This weighted $p$-adaptive Schur-audit is illustrative and \emph{not} used in the main proof chain. Off-diagonal bounds here are model inputs; (P+) is proved solely via the product certificate.
% =========================================================
% Executable finite-block certificate (weighted p-adaptive)
% Parameters: sigma0 = 0.6, Q = 29, p_min = 31, C_win = 0.25
% Model off-diagonal bound:
%   ||H_{pq}(\sigma)||_2 <= (C_win/4) p^{-(\sigma+1/2)} q^{-(\sigma+1/2)},  p != q
% =========================================================

\paragraph*{Certificate \textemdash{} weighted \(p\)-adaptive model at \(\sigma_0=0.6\).}
Fix \(\sigma_0=0.6\), take \(Q=29\) and \(p_{\min}=\mathrm{nextprime}(Q)=31\).\\
Use the \(p\)-adaptive weighted off-diagonal enclosure (for all \(p\neq q\), uniformly in \(\sigma\in[\sigma_0,1]\)):
\[
\|H_{pq}(\sigma)\|_2 \;\le\; \frac{C_{\mathrm{win}}}{4}\, p^{-(\sigma+\tfrac12)}\, q^{-(\sigma+\tfrac12)},
\qquad C_{\mathrm{win}}=0.25.
\]

\noindent\emph{Prime sums (small block \(p\le Q\)).} With \(\sigma_0=0.6\),
\[
S_{\sigma_0}(Q)\;=\;\sum_{p\le Q} p^{-\sigma_0}\;=\;2.9593220929,\qquad
S_{\sigma_0+\tfrac12}(Q)\;=\;\sum_{p\le Q} p^{-(\sigma_0+\tfrac12)}\;=\;1.3239981250.
\]

\section{Bridges B--C: Finite-to-full propagation and diagonal covering}
\iffalse

In this section (model/optional) we record complete, self-contained proofs of the two operator bridges that transport a certified finite-block Schur gap to a global gap on vertical lines and then along a diagonal covering to $\Re s=\tfrac12+\eta$. Bridge A (the determinant--zeta identity) is proved earlier (Theorem~\ref{thm:bridgeA}) and, together with the trace--lock Lemma~\ref{lem:tracelock}, holds unconditionally on $\{\Re s>\tfrac12\}$. This route is \emph{not} used in the main chain.
% --- Bridge A: trace–lock under strictly upper–triangular padding ---
\begin{lemma}[Trace–lock for diagonal + strictly upper–triangular]\label{lem:trace-lock}
Let $H=\ell^2(\mathbb{P})$ with the prime-ordered orthonormal basis $\{e_p\}$. Fix $s$ with $\Re s> \tfrac12$ and let 
\[
T(s):=\sum_{p} p^{-s}\,\Pi_p,\qquad \Pi_p x:=\langle x,e_p\rangle e_p,
\]
so $T(s)$ is diagonal in the $\{e_p\}$ basis. Let $K\in \mathcal S_2(H)$ be any bounded operator that is strictly upper--triangular in this basis and satisfies $\langle Ke_p,e_p\rangle=0$ for all $p$. Then for every integer $n\ge 2$,
\[
\mathrm{Tr}\big((T(s)+K)^n\big)=\mathrm{Tr}\big(T(s)^n\big)=\sum_{p} p^{-ns}.
\]
\end{lemma}

\begin{proof}
Expand $(T+K)^n$ into monomials in $T$ and $K$. Any monomial that contains at least one factor $K$ is a product of diagonal and strictly upper--triangular matrices. Such products remain strictly upper--triangular and have zero diagonal, hence zero trace. Only $T^n$ contributes to the trace.
\end{proof}

\begin{corollary}[det$_2$ invariance under triangular padding]\label{cor:det2-invariance}
With $T, K$ as above and $\Re s>\tfrac12$,
\[
\log\det\nolimits_2\!\big(I-(T(s)+K)\big) = \log\det\nolimits_2\!\big(I-T(s)\big).
\]
Consequently, writing $\xi(s)=e^{L(s)}\det\nolimits_2(I-T(s))$ on $\Re s>\tfrac12$ gives
\[
\xi(s)=e^{L(s)}\det\nolimits_2\!\big(I-(T(s)+K)\big).
\]
\end{corollary}

\subsection*{Bridge C: Neumann step and diagonal covering}

We quantify how the Schur gap degrades under a small change of $\sigma$.

\begin{theorem}[Bridge C: diagonal covering]\label{thm:bridge-C-early}
Fix a grid $\{\sigma_k\}$ with steps $h_k=\sigma_{k+1}-\sigma_k<0$ and let $\theta_k:=K(\sigma_k)\,|h_k|$. If $\theta_k\le \tfrac12$ for every $k$ and $\delta_{\mathrm{Schur}}(\sigma_0)>0$, then for all $N$
\[
  \delta_{\mathrm{Schur}}(\sigma_N)\ \ge\ \delta_{\mathrm{Schur}}(\sigma_0)\,\prod_{k< N}(1-\theta_k)
\]
and hence $\delta_{\mathrm{Schur}}(\sigma_N)>0$.
\end{theorem}

\begin{theorem}[Diagonal covering to lines; corrected Bridge C]\label{thm:diag-cover-corrected-early}
Fix $\varepsilon\in(0,\tfrac12]$ and a vertical line $\{\Re s=\sigma\}$ with $\sigma\in(\tfrac12,1)$. Suppose the blockwise Schur/Gershgorin audit on this line returns a positive spectral margin 
\[
\delta_{\mathrm{Schur}}(\sigma)\ :=\ \inf_{t\in\mathbb{R}}\,\big\| (I-K_{\sigma,\varepsilon}(\sigma+it))^{-1}\big\|^{-1}\ >\ 0.
\]
Then $\zeta(\sigma+it)\neq 0$ for all $t\in\R$.
\end{theorem}

\begin{proof}
If $\delta_{\mathrm{Schur}}(\sigma)>0$, then $I-K_{\sigma,\varepsilon}(\sigma+it)$ is invertible uniformly in $t$, hence $D_\varepsilon(\sigma+it):=\det(I-K_{\sigma,\varepsilon}(\sigma+it))\neq 0$. The explicit line factorization gives $\zeta^{-1}=E_\varepsilon D_\varepsilon$ with a link factor $E_\varepsilon$ bounded below away from $0$ on the line. Thus $\zeta(\sigma+it)\neq 0$.
\end{proof}

\begin{theorem}[Bridges A--C imply RH]\label{thm:bridges-imply-RH}
By Bridge B (explicit line factorization, Theorem~\ref{thm:line-factorization}, and Schur closure) together with Bridge C (differential covering, Theorem~\ref{thm:diff-bridgeC}), every line $\Re s=\sigma>\tfrac12$ is zero–free for $\zeta$. Hence $\zeta$ has no zeros on $\{\Re s>\tfrac12\}$. By the functional equation for $\xi$, all nontrivial zeros lie on $\Re s=\tfrac12$.
\end{theorem}

% (Document continues; early end removed)
\noindent\emph{In-block Gershgorin lower bounds (uniform on \([\sigma_0,1]\)).}
Define
\[
L(p)\;:=\;(1-\sigma_0)\,(\log p)\,p^{-\sigma_0},\qquad 
\mu_p^{\mathrm L}\;\ge\;1-\frac{L(p)}{6}.
\]
At \(p_{\min}=31\) this gives
\[
L(31)=0.1750014502,\qquad 
\mu_{\min}^{\mathrm{far}}\;:=\;1-\frac{L(31)}{6}\;=\;0.9708330916.
\]
Over the small block \(p\le Q\) the worst case is at \(p=5\):
\[
L(5)=0.2451050257,\qquad 
\mu_{\min}^{\mathrm{small}}\;:=\;1-\frac{L(5)}{6}\;=\;0.9591491624.
\]
\noindent\emph{Off-diagonal budgets (all rigorous).}
Let \(\sigma^\star:=\sigma_0+\tfrac12=1.1\).\\
With the integer-tail majorant \(\displaystyle \sum_{n\ge p_{\min}-1} n^{-\sigma^\star}\le
\frac{(p_{\min}-1)^{1-\sigma^\star}}{\sigma^\star-1}\)
we obtain:
\[
\Delta_{\mathrm{FS}}
=\frac{C_{\mathrm{win}}}{4}\,p_{\min}^{-\sigma^\star}\,S_{\sigma^\star}(Q)
=0.0018935184,
\]
\[
\Delta_{\mathrm{FF}}
=\frac{C_{\mathrm{win}}}{4}\,p_{\min}^{-\sigma^\star}\!
\sum_{n\ge p_{\min}-1}\! n^{-\sigma^\star}
\;\le\;\frac{C_{\mathrm{win}}}{4}\,p_{\min}^{-\sigma^\star}\,
\frac{(p_{\min}-1)^{1-\sigma^\star}}{\sigma^\star-1}
=0.0101781777,
\]
\[
\Delta_{\mathrm{SS}}
=\frac{C_{\mathrm{win}}}{4}\,2^{-\sigma^\star}
\!\sum_{\substack{p\le Q\\ p\neq 2}}\! p^{-\sigma^\star}
=0.0250018328,
\]
\[
\Delta_{\mathrm{SF}}
=\frac{C_{\mathrm{win}}}{4}\,2^{-\sigma^\star}\!
\sum_{n\ge p_{\min}-1}\! n^{-\sigma^\star}
\;\le\;\frac{C_{\mathrm{win}}}{4}\,2^{-\sigma^\star}\,
\frac{(p_{\min}-1)^{1-\sigma^\star}}{\sigma^\star-1}
=0.2075080249.
\]

\noindent\emph{Certified finite-block spectral gap.}
Combining the in-block lower bounds with the off-diagonal budgets yields
\[
\delta_{\mathrm{cert}}(\sigma_0)\;\ge\;
\min\Big\{
\underbrace{\mu_{\min}^{\mathrm{small}}-(\Delta_{\mathrm{SS}}+\Delta_{\mathrm{SF}})}_{\text{small-block rows}}\,,\;
\underbrace{\mu_{\min}^{\mathrm{far}}-(\Delta_{\mathrm{FS}}+\Delta_{\mathrm{FF}})}_{\text{far-block rows}}\
\Big\}
=0.7266393047\;>\;0.
\]
Hence the normalized finite block is uniformly positive definite on \([\sigma_0,1]\).

\begin{corollary}[Boundary-uniform smoothed control]\label{cor:det2-boundary}
Let $I\Subset\R$, $\varepsilon_0\in(0,\tfrac12]$, and $\varphi\in C_c^2(I)$. Then, uniformly for $\sigma\in(\tfrac12,\tfrac12+\varepsilon_0]$,
\[
  \Big|\int_{\R} \varphi(t)\,\partial_\sigma\,\Re\log\dettwo\big(I-A(\sigma+it)\big)\,dt\Big|\ \le\ C_*\,\|\varphi''\|_{L^1(I)}.
\]
In particular, the bound remains valid in the boundary limit $\sigma\downarrow \tfrac12$ in the sense of distributions.
\end{corollary}
\subsection*{Smoothed Cauchy and outer limit (A2)}
\begin{proposition}[Outer normalization: existence, boundary a.e. modulus, and limit]\label{prop:outer-central}
There exist outer functions $\mathcal O_\varepsilon$ on $\{\Re s>\tfrac12+\varepsilon\}$ with a.e. boundary modulus $|\mathcal O_\varepsilon(\tfrac12+\varepsilon+it)|=\exp u_\varepsilon(t)$, and $\mathcal O_\varepsilon\to\mathcal O$ locally uniformly on $\Omega$ as $\varepsilon\downarrow 0$, where $\mathcal O$ has boundary modulus $\exp u(t)$. Consequently the outer-normalized ratio $\mathcal J=\dettwo(I-A)/(\mathcal O\,\xi)$ has a.e. boundary values on $\Re s=\tfrac12$ with $|\mathcal J(\tfrac12+it)|=1$.
\end{proposition}
\begin{theorem}[Smoothed Cauchy for $u_\varepsilon$ and convergence of outers]\label{thm:unsmoothed-Cauchy}
Let $u_\varepsilon(t):=\log\Big|\dettwo\!\big(I\! -\!A(\tfrac12\!+\varepsilon\!+\!it)\big)\Big|\! -\!\log\big|\xi(\tfrac12\!+\varepsilon\!+\!it)\big|$. For each compact $I\Subset\R$ and each $\varphi\in C_c^2(I)$ there exists $C(\varphi)<\infty$ such that, uniformly for $\varepsilon,\delta\in(0,\varepsilon_0]$,
\[
  \Big|\int_{\R} \varphi(t)\,\big(u_\varepsilon(t)-u_\delta(t)\big)\,dt\Big|\ \le\ C(\varphi)\,|\varepsilon-\delta|.
\]
Consequently, the outer normalizations $\mathcal O_\varepsilon$ converge locally uniformly to an outer limit $\mathcal O$ on $\Omega$.
\end{theorem}
\begin{lemma}[Outer existence at positive distance and local-uniform limit]\label{lem:outer-existence-limit}
Fix $\varepsilon>0$ and set $u_\varepsilon(t):=\log\big|\dettwo(I-A(\tfrac12+\varepsilon+it))\big|-\log\big|\xi(\tfrac12+\varepsilon+it)\big|$. There exists an outer function $\mathcal O_\varepsilon$ on $\{\Re s>\tfrac12+\varepsilon\}$ with a.e. boundary modulus $|\mathcal O_\varepsilon(\tfrac12+\varepsilon+it)|=e^{u_\varepsilon(t)}$, unique up to a unimodular constant, and given by the Poisson--Szeg\H{o} integral. If $u_\varepsilon\to u$ in $L^1_{\rm loc}(\R)$ as $\varepsilon\downarrow 0$, then $\mathcal O_\varepsilon\to \mathcal O$ locally uniformly on $\Omega$ where $\mathcal O$ is the outer with boundary modulus $e^{u(t)}$.
\end{lemma}
\begin{proof}
Standard outer-factor theory on the half-plane: existence/uniqueness at fixed $\varepsilon$ follows from the Poisson representation (e.g. \cite[Ch.~II]{Garnett}, \cite[Ch.~2]{RosenblumRovnyak}). Local-uniform convergence follows by dominated convergence for Poisson integrals on $\{\Re s\ge \tfrac12+\delta\}$ and normalization at a fixed point.
\end{proof}
\begin{proof}
Differentiate in $\sigma$ and integrate: for $0<\delta<\varepsilon\le\varepsilon_0$ and any $\varphi\in C_c^2(I)$,
\[
 \int \!\varphi\,\big(u_\varepsilon-u_\delta\big)\,dt\ =\ \int_\delta^{\varepsilon}\!\int\! \varphi(t)\,\partial_\sigma\,\Re\Big(\log\dettwo\big(I\! -\!A(\tfrac12\!+\sigma\!+\!it)\big)-\log\xi(\tfrac12\!+\sigma\!+\!it)\Big)\,dt\,d\sigma.
\]
Apply Lemma~\ref{lem:det2-unsmoothed} (smoothed det$_2$ bound) and Lemma~\ref{lem:xi-smoothed} (smoothed $\xi$ bound), then integrate in $\sigma$ to obtain the Lipschitz estimate in $\varepsilon$. The convergence of outers follows from the Poisson representation and Lemma~\ref{lem:desmoothing} (Cauchy transfer).
\end{proof}

\subsection*{Carleson energy and boundary BMO (unconditional)}
We record a direct Carleson--energy route to boundary BMO for the limit $u(t)=\lim_{\varepsilon\downarrow 0}u_\varepsilon(t)$.

\begin{lemma}[Arithmetic Carleson energy]\label{lem:carleson-arith}
Let
\[
 U_{\det_2}(\sigma,t)\ :=\ \sum_{p}\sum_{k\ge 2}\frac{(\log p)\,p^{-k/2}}{k\log p}\,e^{-k\log p\,\sigma}\,\cos\big(k\log p\,t\big),\qquad \sigma>0.
\]
Then for every interval $I\subset\R$ with Carleson box $Q(I):=I\times(0,|I|]$,
\[
 \iint_{Q(I)} |\nabla U_{\det_2}|^2\,\sigma\,dt\,d\sigma\ \le\ \frac{|I|}{4}\,\sum_{p}\sum_{k\ge 2}\frac{p^{-k}}{k^2}
 \ =:\ K_0\,|I|,\qquad K_0:=\frac{1}{4}\sum_{p}\sum_{k\ge 2}\frac{p^{-k}}{k^2}<\infty.
\]
\end{lemma}
\begin{proof}
For a single mode $b\,e^{-\omega\sigma}\cos(\omega t)$ one has $|\nabla|^2=b^2\omega^2e^{-2\omega\sigma}$, hence
\[\int_0^{|I|}\!\int_I |\nabla|^2\,\sigma\,dt\,d\sigma\ \le\ |I|\cdot\sup_{\omega>0}\int_0^{|I|}\sigma\,\omega^2e^{-2\omega\sigma}d\sigma\cdot b^2\ \le\ \tfrac14\,|I|\,b^2.
\]
With $b=(\log p)\,p^{-k/2}/(k\log p)$ and $\omega=k\log p$, summing over $(p,k)$ gives the claim and the finiteness of $K_0$.
\end{proof}

\paragraph{Whitney scale and short–interval zeros.}
Throughout we use the Whitney schedule
\[
  L\ =\ L(T)\ :=\ \frac{c}{\log\langle T\rangle}\ \le\ \frac{1}{\log\langle T\rangle},\qquad \langle T\rangle:=\sqrt{1+T^2},
\]
for a fixed absolute $c\in(0,1]$; all boxes are $Q(\alpha I)$ with a uniform $\alpha\in[1,2]$.
We work on Whitney boxes $Q(I)$ with
\[
  L=L(T):=\frac{c}{\log\langle T\rangle},\qquad \langle T\rangle:=\sqrt{1+T^2},\quad c>0\ \text{fixed}.
\]
There exist absolute $A_0,A_1>0$ such that for $T\ge2$ and $0<H\le1$,
\[
  N(T;H)\ :=\ \#\{\rho=\beta+i\gamma:\ \gamma\in[T,T+H]\}\ \le\ A_0\ +\ A_1\,\big(H\log T\ +\ \log T\big).
\]

\begin{lemma}[Annular Poisson–balayage bound (linear in mass)]\label{lem:annular-balayage}
Let $I=[T-L,T+L]$ be Whitney with $L=c/\log\langle T\rangle$. For $k\ge1$ set the annular counting measure
\[\nu_k\ :=\ \sum_{\rho:\ 2^kL<|T-\gamma|\le 2^{k+1}L}\!\delta_\gamma,\]
and define
\[
  V_k(\sigma,t):=\int\frac{\sigma}{(t-\tau)^2+\sigma^2}\,d\nu_k(\tau).
\]
Then for any fixed dilation $\alpha>1$ there exists $C_\alpha$ such that
\[
  \iint_{Q(\alpha I)} V_k(\sigma,t)^2\,\sigma\,dt\,d\sigma\ \le\ C_\alpha\,\Big(\frac{L}{2^kL}\Big)^{\!2}\,|I|\,\nu_k(\R)
  \ =\ C_\alpha\,\frac{|I|}{4^k}\,\nu_k(\R).
\]
\end{lemma}

\begin{lemma}[Analytic ($\xi$) Carleson energy on Whitney boxes]\label{lem:carleson-xi}
There exist absolute constants $c\in(0,1]$ and $C_\xi<\infty$ such that for every interval $I=[T-L,\,T+L]$ with Whitney scale $L:=c/\log\langle T\rangle$, the Poisson extension
\[
 U_{\xi}(\sigma,t):=\Re\log\xi\big(\tfrac12+\sigma+it\big),\qquad (\sigma>0),
\]
\paragraph{Whitney scale.}
Throughout this lemma we take the base interval $I=[T-L,T+L]$ with
\[
  L=L(T):=\frac{c}{\log\langle T\rangle},\qquad \langle T\rangle:=\sqrt{1+T^2},\quad c>0\text{ fixed.}
\]
obeys the Carleson bound
\[ \iint_{Q(I)} |\nabla U_{\xi}(\sigma,t)|^2\,\sigma\,dt\,d\sigma\ \le\ C_\xi\,|I|. \]
\end{lemma}
\begin{proof}
Write
\[
 \partial_\sigma U_{\xi}(\sigma,t)\ =\ \Re\frac{\xi'}{\xi}\!\left(\tfrac12+\sigma+it\right)
 \ =\ \Re\sum_{\rho}\frac{1}{\tfrac12+\sigma+it-\rho}\ +\ A(\sigma,t),
\]
where the sum runs over nontrivial zeros $\rho=\beta+i\gamma$ of $\zeta$, and $A(\sigma,t)$ collects the archimedean part and the trivial factors (these are smooth in $(\sigma,t)$ on compact strips). Since $U_{\xi}$ is harmonic, $|\nabla U_{\xi}|^2\asymp |\partial_\sigma U_{\xi}|^2$ on $\R^2_+$; it suffices to estimate the latter.

Fix $I=[T-L,T+L]$ and decompose the zero set into near and far parts relative to $Q(I)=I\times(0,L]$:
\[
 \mathcal Z_{\mathrm{near}}:=\{\rho:\ |\gamma-T|\le 2L\},\qquad \mathcal Z_{\mathrm{far}}:=\{\rho:\ |\gamma-T|>2L\}.
\]
For a single term $f_{\rho}(\sigma,t):=\Re(\tfrac12+\sigma+it-\rho)^{-1}$, one has the pointwise bound
\[ |f_{\rho}(\sigma,t)|\ \le\ \frac{1}{\sqrt{(\tfrac12-\beta+\sigma)^2+(t-\gamma)^2}}\ \le\ \frac{1}{\sqrt{\sigma^2+(t-\gamma)^2}}. \]
Hence
\[
 \int_0^{L}\!\int_{I} |f_{\rho}(\sigma,t)|^2\,\sigma\,dt\,d\sigma\ \le\ \int_0^{L}\!\int_{T-L}^{T+L}\frac{\sigma}{\sigma^2+(t-\gamma)^2}\,dt\,d\sigma.
\]
Evaluating the $t$–integral first gives $\le C\int_0^{L} (\arctan((t-\gamma)/\sigma))\big|_{T-L}^{T+L}\,d\sigma\le C'\,L$ uniformly in $\gamma$ provided $|\gamma-T|\le 2L$ (near zeros), with absolute constants $C,C'$.

To handle near zeros rigorously, introduce the local half-plane Blaschke compensator $B_I$ from Lemma~\ref{lem:local-blaschke} and define the neutralized field
\[
  \widetilde U(\sigma,t)\ :=\ \Re\log\big(\xi(\tfrac12+\sigma+it)\,B_I(\sigma+it)\big).
\]
This removes all singular contributions from $\{\rho:\ |\gamma-T|\le 2L\}$; in particular, there is no near-zero cross-term to estimate inside $Q(\alpha I)$. The compensator contributes a fixed box energy, and by Lemma~\ref{lem:neutralized-energy} one has
\[
  \iint_{Q(\alpha I)} |\nabla \widetilde U|^2\,\sigma\,dt\,d\sigma\ \le\ C_{\mathrm{area}}\,|I|.
\]

For the far zeros, set annuli $\mathcal A_k:=\{\rho:\ 2^kL<|\gamma-T|\le 2^{k+1}L\}$ for $k\ge1$. For a single zero at vertical distance $\Delta:=|\gamma-T|$ one has the kernel estimate
\[
 \int_0^{L}\!\int_{T-L}^{T+L} \frac{\sigma}{\sigma^2+(t-\gamma)^2}\,dt\,d\sigma\ \ll\ L\,\Big(\frac{L}{\Delta}\Big)^{\!2}.
\]
For the far annuli $\mathcal A_k$, apply Lemma~\ref{lem:annular-balayage} to the annular Poisson sums $V_k$ to control cross terms linearly in the annular mass:
\[
  \iint_{Q(\alpha I)}\Big|\sum_{\rho\in\mathcal A_k} f_{\rho}\Big|^2\,\sigma\,dt\,d\sigma\ \ll\ \frac{|I|}{4^k}\,\nu_k(\R),
\]
where $\nu_k(\R)=\#\{\rho:\ 2^kL<|T-\gamma|\le 2^{k+1}L\}$. By the unconditional short-interval bound,
\[ \nu_k(\R)\ \ll\ 2^kL\,\log\langle T\rangle\ +\ \log\langle T\rangle. \]
Summing $k\ge1$ yields a total far contribution
\[ \ll\ |I|\sum_{k\ge1}\frac{1}{4^k}\big(2^kL\log\langle T\rangle+\log\langle T\rangle\big)\ \ll\ |I|\,(L\log\langle T\rangle+1), \]
which is $\ll |I|$ on the Whitney scale $L=c/\log\langle T\rangle$.

Combining the neutralized near part (bounded by Lemma~\ref{lem:neutralized-energy}) with the far-annulus sum (bounded as above) and the smooth archimedean term yields
\[
 \iint_{Q(\alpha I)} |\nabla \widetilde U|^2\,\sigma\,dt\,d\sigma\ \ll\ |I|.
\]
This proves the claimed Carleson bound on Whitney boxes.
\end{proof}

\begin{lemma}[Cutoff pairing on boxes]\label{lem:cutoff-pairing}
Fix parameters $\alpha'>\alpha>1$. Let $\chi_{L,t_0}\in C_c^\infty(\R^2_+)$ satisfy $\chi\equiv1$ on $Q(\alpha I)$, $\operatorname{supp}\chi\subset Q(\alpha'I)$,
$\|\nabla\chi\|_\infty\lesssim L^{-1}$ and $\|\nabla^2\chi\|_\infty\lesssim L^{-2}$. Let $V_{\psi,L,t_0}$ be the Poisson extension of $\psi_{L,t_0}$ and $\widetilde U$ the neutralized field. Then
\[
 \int_{\R} u(t)\,\psi_{L,t_0}(t)\,dt
 \,=\, \iint_{Q(\alpha'I)} \nabla \widetilde U\cdot \nabla\big(\chi_{L,t_0}\, V_{\psi,L,t_0}\big)\,dt\,d\sigma\ +\ \mathcal R_{\mathrm{side}}\ +\ \mathcal R_{\mathrm{top}},
\]
where the remainder terms obey
\[
 |\mathcal R_{\mathrm{side}}|+|\mathcal R_{\mathrm{top}}|
 \,\lesssim\, \Big(\iint_{Q(\alpha'I)} |\nabla \widetilde U|^2\,\sigma\Big)^{1/2}
               \cdot \Big(\iint_{Q(\alpha' I)} \big(|\nabla\chi|^2\,|V_{\psi,L,t_0}|^2+|\nabla V_{\psi,L,t_0}|^2\big)\,\sigma\Big)^{1/2}.
\]
Since $\psi$ is fixed, the Poisson–energy factor involving $V_{\psi,L,t_0}$ is scale–invariant in $L$, hence
\[
 |\mathcal R_{\mathrm{side}}|+|\mathcal R_{\mathrm{top}}|\ \lesssim\ \Big(\iint_{Q(\alpha'I)} |\nabla \widetilde U|^2\,\sigma\Big)^{1/2}.
\]
\end{lemma}

% (Removed global reduction: certificate constants are taken as Whitney-only suprema.)
\begin{proof}
Fix $L>0$ and $t_0\in\R$. Since $\psi$ has compact support in $[-2,2]$, the support of $\psi_{L,t_0}$ is contained in $[t_0-2L,\,t_0+2L]$. Cover this interval by $N\le N_\psi$ consecutive intervals $I_j$ of length $\asymp L$ with bounded overlap (each point belongs to at most $M_\psi$ of the $I_j$), where $N_\psi,M_\psi$ depend only on the support of $\psi$. Choose a $C^\infty$ partition of unity $\{\eta_j\}_{j=1}^{N}$ with $\sum_j \eta_j\equiv 1$ on $[t_0-2L,\,t_0+2L]$, $\operatorname{supp}\eta_j\subset I_j$, and
\[\|\eta_j\|_{L^\infty}\le 1,\quad \|\nabla\eta_j\|_{L^\infty}\lesssim L^{-1},\quad \|\nabla^2\eta_j\|_{L^\infty}\lesssim L^{-2}.\]
Write $\psi_{L,t_0}=\sum_{j=1}^{N}\phi_j$ with $\phi_j:=\eta_j\,\psi_{L,t_0}$. For each $j$, apply Lemma~\ref{lem:cutoff-pairing} with a cutoff $\chi_j$ supported in a fixed dilation $Q(\alpha' I_j)$ and equal to $1$ on $Q(\alpha I_j)$ to get
\[
 \Big|\int u\,\phi_j\Big|\ \le\ \Big(\iint_{Q(\alpha' I_j)} |\nabla \widetilde U|^2\,\sigma\Big)^{1/2}
 \cdot \Big(\iint_{Q(\alpha' I_j)} (|\nabla\chi_j|^2\,|V_{\phi_j}|^2+|\nabla V_{\phi_j}|^2)\,\sigma\Big)^{1/2}.
\]
Because $\psi$ is fixed and $\phi_j=\eta_j\psi_{L,t_0}$ with $\|\nabla\eta_j\|_\infty\lesssim L^{-1}$, the second factor is $\lesssim L^{1/2}\,\mathcal A(\psi)$ uniformly in $j$. Summing over $j$ and using Cauchy--Schwarz gives
\[\Big|\int u\,\psi_{L,t_0}\Big|\ \le\ C_\psi\,\Big(\sum_{j=1}^N \iint_{Q(\alpha' I_j)} |\nabla \widetilde U|^2\,\sigma\Big)^{1/2}\,L^{1/2}\,\mathcal A(\psi).\]
Since the boxes $Q(\alpha' I_j)$ have bounded overlap (at most $M_\psi$) and are fixed dilations of Whitney boxes at height $\asymp L$, Lemma~\ref{lem:neutralized-energy} yields $\iint_{Q(\alpha' I_j)} |\nabla \widetilde U|^2\,\sigma\lesssim |I_j|\asymp L$. Therefore
\[\Big|\int u\,\psi_{L,t_0}\Big|\ \le\ C'\,\mathcal A(\psi)\,(N_\psi M_\psi)^{1/2}\,L.\]
Dividing by $L$ and taking the supremum in $L,t_0$ proves the first inequality with $C=C'\,(N_\psi M_\psi)^{1/2}\,\mathcal A(\psi)$. The Hilbert case follows identically with $V_{\phi_j}$ replaced by the test field for $(\mathcal H[\phi_j])'$, whose box energy is comparable by scale invariance.\end{proof}

\section{Discussion and outlook}\label{sec:discussion}
We presented an operator-theoretic BRF program for RH combining Schur--determinant splitting, HS\(\to\)\(\dettwo\) continuity, and explicit finite-stage passive constructions tied to the primes. Two routes were considered historically: an interior alignment route on zero-free rectangles via passive $H^\infty$ approximation, and a boundary route via a PSC certificate. In the present proof we proceed via the boundary route: (P+) is certified by Theorem~\ref{thm:certificate-inpaper}; PSC is supplied separately by the sum-form theorem. The Bridges A--C/Schur-covering section is included as an optional alternative perspective and is not required for the main proof.
\paragraph{Role of the interior route.}
The Gram/Fock interior route provides rectangle positivity (Herglotz/Schur) without Schur-test absolute-sum bounds; it supports interior control but is not needed for the final boundary closure here.
Potential refinements include: (i) quantitative rational approximation on analytic boundaries with lossless KYP constraints; (ii) strengthened explicit-formula estimates sufficient for $L^1_{\mathrm{loc}}$ cancellation of zero spikes; (iii) exploring alternative finite-block architectures for $k=1$ with improved global control; and (iv) extensions to matrix-valued settings and other $L$-functions.

% (Early \end{document} removed; document continues.)
\subsection*{Local Blaschke neutralization and singularity-free pairing}
To remove interior singularities caused by zeros of $\xi$ inside a Whitney box, we introduce a local half-plane Blaschke compensator and pair against the neutralized field.

\begin{lemma}[Local Blaschke neutralization]\label{lem:local-blaschke}
Let $z=s-\tfrac12$ with $\Re z>0$ and fix a Whitney interval $I=[T-L,T+L]$, $L=c/\log\langle T\rangle$. Define
\[ \mathcal Z_I:=\{a_j=\rho_j-\tfrac12:\ \Re a_j>0,\ |\Im a_j-T|\le 2L\},\qquad B_I(z):=\prod_{a_j\in\mathcal Z_I}\frac{z-a_j}{z+\overline{a_j}}. \]
Then $B_I$ is inner on $\{\Re z=0\}$ (so $|B_I(it)|\equiv 1$), and $\Re\log B_I$ is harmonic on $\{\Re z>0\}$ with boundary trace $\log|B_I(it)|\equiv 0$.
\end{lemma}

\begin{lemma}[Neutralized energy on $Q(\alpha I)$]\label{lem:neutralized-energy}
Set
\[ \widetilde U(\sigma,t):=\Re\log\det_2\!\big(I-A(\tfrac12+\sigma+it)\big)-\Re\log\xi\!\left(\tfrac12+\sigma+it\right)+\Re\log B_I(\sigma+it)+U_\Gamma(\sigma,t). \]
Then $\widetilde U$ is harmonic and free of interior singularities in $Q(\alpha I)$, and
\[ \iint_{Q(\alpha I)} |\nabla \widetilde U|^2\,\sigma\,dt\,d\sigma\ \le\ C_{\mathrm{area}}\,|I|, \]
with $C_{\mathrm{area}}:=K_0+K_\xi+\|U_\Gamma\|_{\mathrm{area}}$ as above.
\end{lemma}

\begin{proof}
By construction, $B_I$ cancels the singular contributions of zeros of $\xi$ with $|\Im\rho-T|\le 2L$. The arithmetic, far-zero, and archimedean parts are bounded on Whitney boxes by $K_0|I|$, $K_\xi|I|$, and $\|U_\Gamma\|_{\mathrm{area}}|I|$, proving the claim.
\end{proof}

\begin{lemma}[Pairing reduction for $u$ and the Hilbert transform]\label{lem:pairing-reduction}
Let $\phi\in C_c^1(\R)$ with $\int\phi=0$. There exists $\theta\in C_c^2(\R)$ with $(\mathcal H\theta)'=\phi$ such that
\[\int_\R u(t)\,\phi(t)\,dt\ =\ \int_\R \mathcal H[u'](t)\,\theta(t)\,dt.\]
Moreover, $\iint_{\R^2_+}|\nabla(P_\sigma*\theta)|^2\,\sigma\ \lesssim\ \iint_{\R^2_+}|\nabla(P_\sigma*\phi)|^2\,\sigma$.
\end{lemma}

\begin{proposition}[Local Hilbert pairing after neutralization]\label{prop:pairing-neutralized}
Let $\psi\in C_c^2(\R)$ with $\int\psi=0$ and $\psi_{L,t_0}(t):=\psi((t-t_0)/L)$. Then
\[ \Big|\int_{\R} u(t)\,\psi_{L,t_0}(t)\,dt\Big|\ =\ \Big|\int_{\R} \mathcal H[u'](t)\,\theta(t)\,dt\Big|\ \le\ C\,\sqrt{C_{\mathrm{area}}}\,\mathcal A(\psi)\,L, \]
where we used the local Hilbert pairing bound on $Q(\alpha I)$ and the box energy of the test for $\theta$.
\end{proposition}

\paragraph{Box test fields and local pairings (Hilbert route).}
Let $\psi\in C_c^2(\R)$ with $\int\psi=0$, $\psi_{L,t_0}(t)=\psi((t-t_0)/L)$, and $I=[t_0-L,t_0+L]$. On $Q(\alpha I)$ let $W_I$ be harmonic with $W_I|_{\sigma=0}=(\mathcal H[\psi_{L,t_0}])'$ on $I$ and $0$ on the other sides. Then
\[
  \Big|\langle \mathcal H[u'],\psi_{L,t_0}\rangle\Big|\ \le\ \Big(\iint_{Q(\alpha I)}|\nabla\widetilde U|^2\,\sigma\Big)^{1/2}\Big(\iint_{Q(\alpha I)}|\nabla W_I|^2\,\sigma\Big)^{1/2}.
\]
By Lemma~\ref{lem:pairing-reduction}, this controls $\int u\,\psi_{L,t_0}$ via the Hilbert-transfer identity. Moreover one has uniform box-energy bounds
\[
  \iint_{Q(\alpha I)}|\nabla V_I|^2\,\sigma\ \le\ C_\alpha L\,\mathcal A(\psi)^2,\qquad
  \iint_{Q(\alpha I)}|\nabla W_I|^2\,\sigma\ \le\ C'_\alpha L\,\mathcal A(\psi)^2,
\]
with
\[
  \mathcal A(\psi)^2\ :=\ \iint_{\R^2_+}|\nabla(P_\sigma*\psi)|^2\,\sigma\,dt\,d\sigma\ <\ \infty.
\]
Hence
\[
  M_\psi\ \le\ C_1\,\sqrt{C_{\mathrm{area}}}\,\mathcal A(\psi),\qquad
  C_H(\psi)\ \le\ C_2\,\sqrt{C_{\mathrm{area}}}\,\mathcal A(\psi),
\]
with constants depending only on $\psi$ and the dilation parameter.

\begin{corollary}[Unconditional local window constants]\label{cor:CH-Mpsi-final}
Define, for $I=[t_0-L,t_0+L]$ and $u$ the boundary trace of $U$, the mean-oscillation constant
\[
  M_\psi\ :=\ \sup_{L>0,\ t_0\in\R}\ \frac{1}{L}\,\Big|\int_{\R} (u(t)-u_I)\,\psi_{L,t_0}(t)\,dt\Big|,\qquad u_I:=\frac{1}{|I|}\int_I u,\quad \psi_{L,t_0}(t):=\psi\big((t-t_0)/L\big),
\]
and the Hilbert constant
\[
  C_H(\psi)\ :=\ \sup_{L>0,\ t_0\in\R}\ \frac{1}{L}\,\Big|\int_{\R} \mathcal H[u'](t)\,\psi_{L,t_0}(t)\,dt\Big|.
\]
Then there are constants $C_1(\psi),C_2(\psi)<\infty$ depending only on $\psi$ and the dilation parameter $\alpha$ such that
\[
  M_\psi\ \le\ C_1(\psi)\,\sqrt{C_{\mathrm{area}}}\,\mathcal A(\psi),\qquad
  C_H(\psi)\ \le\ C_2(\psi)\,\sqrt{C_{\mathrm{area}}}\,\mathcal A(\psi),
\]
where the fixed Poisson energy of the window is
\[
  \mathcal A(\psi)^2\ :=\ \iint_{\R^2_+}\!\big|\nabla\big(P_\sigma*\psi\big)(t)\big|^2\,\sigma\,dt\,d\sigma.
\]
In particular, both constants are finite and determined by local box energies.
\end{corollary}

\begin{proof}
Apply Proposition~\ref{prop:pairing-neutralized} with the box Dirichlet test field for $\psi_{L,t_0}$; Lemma~\ref{lem:neutralized-energy} bounds the neutralized area in $Q(\alpha I)$, and the box test-field energy scales like $L\,\mathcal A(\psi)^2$ by construction, giving the $M_\psi$ bound. For $C_H(\psi)$ integrate by parts and use the same construction for the test associated with $(\mathcal H[\psi_{L,t_0}])'$, whose box energy is comparable to that of $\psi_{L,t_0}$ by scale invariance.
\end{proof}

\begin{corollary}[Boundary BMO and window mean oscillation]\label{cor:bmo-boundary}
For the mass--1 windows $\varphi_I$ induced by an even $\psi\in C_c^\infty([-1,1])$, there exists $M_\psi<\infty$ (depending only on $\psi$) such that for all Whitney intervals $I$,
\[ \int_I |u(t)-\ell_I(t)|\,dt\ \le\ M_\psi\,|I|, \]
where $\ell_I$ is any affine calibrant on $I$. In particular, the near-field bounds in the Hilbert pairing estimates hold uniformly in $(T,L)$.
\end{corollary}

\subsection*{Hilbert pairing via affine subtraction (uniform in $T,L$)}
\begin{lemma}[Uniform Hilbert pairing bound (local box pairing)]\label{lem:hilbert-H1BMO}
Let $\psi\in C_c^\infty([-1,1])$ be even with $\int_\R\psi=1$ and define the mass--1 windows $\varphi_I(t)=L^{-1}\psi\big((t-T)/L\big)$. Then there exists $C_H(\psi)<\infty$ (independent of $T,L$) such that for $u$ from Theorem~\ref{thm:unsmoothed-Cauchy},
\[
  \Big|\int_\R \mathcal H[u'](t)\,\varphi_I(t)\,dt\Big|\ \le\ C_H(\psi)\quad\text{for all intervals }I.
\]
\end{lemma}
\begin{proof}
In distributions, $\langle \mathcal H[u'],\varphi_I\rangle=\langle u,(\mathcal H[\varphi_I])'\rangle$. Because $\psi$ is even, $(\mathcal H[\varphi_I])'$ annihilates constants and linear functions. Subtract the affine calibrant $\ell_I$ agreeing with $u$ at the endpoints of $I$ and write $v:=u-\ell_I$. Apply Proposition~\ref{prop:pairing-neutralized} with the test field corresponding to $(\mathcal H[\varphi_I])'$ restricted to a fixed dilation $Q(\alpha I)$; by Lemma~\ref{lem:neutralized-energy} and the box energy bound for the test field (scale-invariant), one obtains
\[
  |\langle v,(\mathcal H[\varphi_I])'\rangle|\ \le\ C\,\sqrt{C_{\mathrm{area}}}\,\mathcal A(\psi),
\]
with a constant depending only on $\psi$ (through the fixed Poisson energy) and $\alpha$. This yields the claimed uniform bound by local box pairings.
\end{proof}
\begin{lemma}[Hilbert-transform pairing]\label{lem:hilbert}
There exists a window–dependent constant \(C_H(\psi)>0\) such that for every interval \(I\),
\[ \Big|\int_{\R} \mathcal H[u'](t)\,\varphi_I(t)\,dt\Big|\ \le\ C_H(\psi).\]
\end{lemma}
\begin{proof}
By Lemma~\ref{lem:hilbert-H1BMO}, for mass–1 windows and even \(\psi\), the pairing \(\langle \mathcal H[u'],\varphi_I\rangle\) is uniformly bounded in \((T,L)\). In distributions, \(\langle \mathcal H[u'],\varphi_I\rangle=\langle u,(\mathcal H[\varphi_I])'\rangle\); evenness implies \((\mathcal H[\varphi_I])'\) annihilates affine functions. Subtract the affine calibrant on \(I\) and write \(v=u-\ell_I\). The bound follows from the local box pairing in Proposition~\ref{prop:pairing-neutralized} applied to the test field associated with \((\mathcal H[\varphi_I])'\), using only the neutralized area bound and the fixed Poisson energy of \(\psi\).
\end{proof}

 
% --- PSC route moved to archived appendix; placeholder removed from main chain ---
We adopt the \(\zeta\)-normalized boundary route with the half-plane Blaschke compensator \(B(s)=(s-1)/s\) to cancel the pole at \(s=1\). On \(\Re s=\tfrac12\), \(|B|=1\), so the compensator contributes no boundary phase and the Archimedean term vanishes. We print a concrete even mass--1 window \(\psi\), derive \(c_0(\psi)\), \(C_H(\psi)\), and \(C_P(\kappa)\) in-paper, and choose parameters so that
\[
  \frac{C_H(\psi)\,M_\psi + C_P(\kappa)}{c_0(\psi)}\ <\ \frac{\pi}{2}.
\]

\paragraph{Printed window.}
Let \(\beta(x):=\exp\!\big(-1/(x(1-x))\big)\) for \(x\in(0,1)\) and \(\beta=0\) otherwise. Define the smooth step
\[
  S(x):=\frac{\int_0^{\min\{\max\{x,0\},1\}} \beta(u)\,du}{\int_0^{1} \beta(u)\,du}\qquad (x\in\R),
\]
so that \(S\in C^\infty(\R)\), \(S\equiv 0\) on \(({-}\infty,0]\), \(S\equiv1\) on \([1,\infty)\), and \(S'\ge 0\) supported on \((0,1)\). Set the even flat--top window \(\psi:\R\to[0,1]\) by
\[
  \psi(t):=\begin{cases}
    0,& |t|\ge 2,\\
    S(t+2),& -2<t<-1,\\
    1,& |t|\le 1,\\
    S(2-t),& 1<t<2.
  \end{cases}
\]
Then \(\psi\in C_c^\infty(\R)\), \(\psi\equiv1\) on \([-1,1]\), and \(\operatorname{supp}\psi\subset[-2,2]\). For windows we take \(\varphi_L(t):=L^{-1}\psi(t/L)\).

\paragraph{Poisson lower bound.}
As in the plateau computation already recorded, for \(0<b\le 1\) and \(|x|\le 1\) one has
\[
 (\Poisson_b*\psi)(x)\ \ge\ (\Poisson_b*\mathbf 1_{[-1,1]})(x)
 \ =\ \frac{1}{2\pi}\Big(\arctan\tfrac{1-x}{b}+\arctan\tfrac{1+x}{b}\Big),
\]
whence
\[
 c_0(\psi)\ :=\ \inf_{0<b\le 1,\ |x|\le1}(\Poisson_b*\psi)(x)\ \ge\ 0.1762081912.
\]
\begin{proof}[Derivation]
For the normalized Poisson kernel \(P_b(y)=\dfrac{1}{\pi}\dfrac{b}{b^2+y^2}\), for \(|x|\le 1\)
\[
 (P_b*\mathbf 1_{[-1,1]})(x)=\frac{1}{\pi}\int_{-1}^{1}\frac{b}{b^2+(x-y)^2}\,dy=\frac{1}{2\pi}\Big(\arctan\frac{1-x}{b}+\arctan\frac{1+x}{b}\Big).
\]
Set \(S(x,b):=\arctan\big((1-x)/b\big)+\arctan\big((1+x)/b\big)\). Symmetry gives \(S(-x,b)=S(x,b)\). For \(x\in[0,1]\),
\[
 \partial_x S(x,b)=\frac{1}{b}\Big(\frac{1}{1+\big(\tfrac{1+x}{b}\big)^2}-\frac{1}{1+\big(\tfrac{1-x}{b}\big)^2}\Big)\le 0,
\]
so \(S\) decreases in \(x\) and is minimized at \(x=1\). Also \(\partial_b S(x,b)\le 0\) for \(b>0\), so the minimum in \(b\in(0,1]\) is at \(b=1\). Thus the infimum occurs at \((x,b)=(1,1)\) giving \(\frac{1}{2\pi}\arctan 2=0.1762081912\ldots\). Since \(\psi\ge \mathbf 1_{[-1,1]}\), this yields the bound for \(\psi\).
\end{proof}

\paragraph{No Archimedean term in the \(\zeta\)-normalized route.}
Writing \(J_\zeta:=\dettwo(I-A)/\zeta\) and \(J_{\mathrm{comp}}:=J_\zeta\,B\), one has \(|B|=1\) on the boundary and no Gamma factor in \(J_\zeta\). Hence the boundary phase contribution from Archimedean factors is identically zero in the phase–velocity identity, i.e. \(C_\Gamma\equiv 0\) for this normalization.

\paragraph{Hilbert term (structural bound).}
For the mass--1 window and even \(\psi\), the local box pairing bound of Lemma~\ref{lem:hilbert-H1BMO} applies and is uniform in \((T,L)\). We write the certificate in terms of the abstract window-dependent constant \(C_H(\psi)\) from Lemma~\ref{lem:hilbert-H1BMO}. An explicit envelope for the printed window is recorded below, but is not required for the symbolic certificate.
\begin{lemma}[Explicit envelope for the printed window]\label{lem:CH-explicit}
For the flat-top \(\psi\) above with symmetric monotone ramps of width \(\varepsilon\in(0,1)\) on each side of \(\pm1\), one has the variation bound
\[
  \sup_{t\in\R}\,|\mathcal H[\varphi_L](t)|\ \le\ \frac{\mathrm{TV}(\psi)}{\pi}\,\log\frac{1+\varepsilon}{1-\varepsilon},\qquad \mathrm{TV}(\psi)=2.
\]
In particular, with \(\varepsilon=\tfrac15\) one obtains the certified envelope
\[
  \sup_{t\in\R}\,|\mathcal H[\varphi_L](t)|\ \le\ \frac{2}{\pi}\,\log\tfrac{3}{2}\ \approx\ 0.258\ <\ 0.26.
\]
Consequently, we may take \(C_H(\psi)\le 0.26\) for the printed window. This bound is uniform in \(L\).
\end{lemma}
\begin{proof}[Derivation (variation/IBP estimate)]
Write \(\psi=\mathbf 1_{[-1,1]}+\eta\) with \(\eta\) supported on the disjoint transition layers \([1,1+\varepsilon]\) and \([-1-\varepsilon,-1]\), monotone on each layer, and total variation \(\mathrm{TV}(\psi)=2\). Using the identity \(\mathcal H\psi(x)=\tfrac{1}{\pi}\,\mathrm{p.v.}\int \tfrac{\psi(y)}{x-y}\,dy=\tfrac{1}{\pi}\int \psi'(y)\,\log|x-y|\,dy\) (integration by parts; boundary cancellations by monotonicity/symmetry) and that \(\psi'\) is a finite signed measure of total variation \(\mathrm{TV}(\psi)\), one gets
\[
  |\mathcal H\psi(x)|\ \le\ \frac{\mathrm{TV}(\psi)}{\pi}\,\sup_{y\in[-1-\varepsilon,\,1+\varepsilon]}\big|\log|x-y|\big|\ -\ \inf_{y\in[-1-\varepsilon,\,1+\varepsilon]}\big|\log|x-y|\big|.
\]
The worst case is at \(x=0\), yielding \(|\mathcal H\psi(0)|\le \tfrac{\mathrm{TV}(\psi)}{\pi}\log\tfrac{1+\varepsilon}{1-\varepsilon}\). Scaling gives \(\mathcal H[\varphi_L](t)=\mathcal H\psi\big((t-T)/L\big)\), so the same bound holds uniformly in \(L\). Taking \(\varepsilon=\tfrac15\) gives the stated numeric envelope.
\end{proof}

\paragraph{Bandlimit term.}
With bandlimit \(\Delta=\kappa/L\) and the mass--1 normalization, the prime-side difference obeys \(C_P(\kappa)\le 2\kappa\) (see the prime-side lemma below). We keep \(\kappa\) symbolic; the locked evaluation used in the certificate is printed below.
\begin{proof}[Derivation]
For mass--1 windows and cutoff supported on \(|\xi|\le \Delta=\kappa/L\), Cauchy--Schwarz and Plancherel give
\(\big|\int \mathcal P\,\Phi_I\big|\le (\sum_{\log p\le \kappa/L}(\log p)^2/p)^{1/2}\,(\sum_{\log p\le \kappa/L}1)^{1/2}\). Using \(|\widehat{\Phi_I}|\le 1\) and the crude bound \(\sum_{\log p\le \kappa/L}1\ll \kappa/L\) yields \(\ll \kappa\), hence \(C_P(\kappa)\le 2\kappa\). With \(\kappa=0.05\) this gives \(C_P\le 0.10\).
\end{proof}

\paragraph{Window mean-oscillation constant \(M_\psi\): definition and bound.}
For an interval \(I=[T{-}L,T{+}L]\) and the boundary modulus \(u(t):=\log\big|\dettwo(I{-}A(\tfrac12{+}it))\big|{-}\log\big|\xi(\tfrac12{+}it)\big|\), define the mean-oscillation calibrant \(\ell_I\) as the affine function matching \(u\) at the endpoints of \(I\), and set
\[
  M_\psi\ :=\ \sup_{T\in\R,\ L>0}\ \frac{1}{|I|}\int_I \big|u(t)-\ell_I(t)\big|\,dt.
\]
By Theorem~\ref{thm:unsmoothed-Cauchy} and the local pairing in Corollary~\ref{cor:CH-Mpsi-final}, one obtains a window-dependent constant bounding the mean oscillation uniformly over $(T,L)$. For the printed flat-top window, Lemma~\ref{lem:Mpsi-correct} yields an explicit H$^1$--BMO/box-energy bound for $M_\psi$; in our calibration (see Numeric instantiation below), this gives a strict numerical bound well below the certificate threshold.
\begin{lemma}[Window mean--oscillation via H$^1$--BMO and box energy]\label{lem:Mpsi-correct}
Let $U$ be the Poisson extension of the boundary function $u$, and let $\mu := |\nabla U|^2\,\sigma\,dt\,d\sigma$.
Fix the even $C^\infty$ window $\psi$ (support $\subset[-2,2]$, plateau on $[-1,1]$), and let $m_\psi:=\int_{\R}\psi(x)\,dx$ denote its mass. Set
\[
\phi(t):=\psi(t)-\tfrac{m_\psi}{2}\,\mathbf 1_{[-1,1]}(t),\qquad 
\phi_{L,t_0}(t):=\phi\!\Big(\frac{t-t_0}{L}\Big).
\]
Define $M_\psi:=\sup_{L>0,t_0\in\R}\frac1L\big|\int_\R u(t)\,\phi_{L,t_0}(t)\,dt\big|$ and
\[
C_{\rm box}:=\sup_{I}\frac{\mu(Q(\alpha I))}{|I|},\qquad
C_\psi^{(H^1)}:=\frac12\int_{\R} S\phi(x)\,dx,
\]
where $S$ is the Lusin area function for the Poisson semigroup with cone aperture $\alpha$.
Then
\[
M_\psi\ \le\ \frac{4}{\pi}\,C_{\mathrm{CE}}(\alpha)\,C_\psi^{(H^1)}\,\sqrt{C_{\rm box}}.
\]
\end{lemma}

\begin{proof}
By H$^1$--BMO duality, for every interval $I=[t_0-L,t_0+L]$ we have
$\big|\int u\,\phi_{L,t_0}\big|\le \|u\|_{\rm BMO}\,\|\phi_{L,t_0}\|_{H^1}$.
The Carleson embedding bound for the Poisson extension with cone aperture $\alpha$ yields
$\|u\|_{\rm BMO}\le \tfrac{2}{\pi}\,C_{\mathrm{CE}}(\alpha)\,C_{\rm box}^{1/2}$. 
Since $S$ is scale-invariant in $L^1$ up to the length of $I$, we have 
$\|\phi_{L,t_0}\|_{H^1}=\int S(\phi_{L,t_0})(x)\,dx = 2L\,C_\psi^{(H^1)}$.
Combining and dividing by $L$ gives the claim.
\end{proof}

\paragraph{Carleson box linkage.}
With $U=U_{\det_2}+U_{\xi}+U_{\Gamma}$, we have the decomposition of the area constant
\[
  C_{\rm box}\ \le\ K_0\ +\ K_\xi\ +\ \|U_\Gamma\|_{\mathrm{area}},
\]
with $K_0, K_\xi$ and $\|U_\Gamma\|_{\mathrm{area}}$ as recorded in the arithmetic and gamma sections.

\paragraph{Numeric instantiation.}
All concrete values (audited constants for $K_0$, $K_\xi$, $\|U_\Gamma\|_{\rm area}$, the evaluation of $C_\psi^{(H^1)}$, and the numeric $M_\psi$) are printed in the appendix and used in the in-paper certificate.

% (obsolete note removed)

% (obsolete note removed)

The auxiliary lemmas used above are proved in the explicit-constants subsection that follows.

% (obsolete note removed)
% \subsection*{Executable certificate instance (calibrated; illustrative)}
% % ===========================================================
% Executable finite-block certificate (weighted p-adaptive)
% Parameters from the current run: σ0=0.6, Q=29, pmin=31, Cwin=0.25
% ===========================================================
\begingroup
\def\sigmao{0.6}
\def\Q{29}
\def\pmin{31}
\def\Cwin{0.25}

% Prime sums actually used (these are over primes)
% S_{σ0}(Q) and S_{σ0+1/2}(Q), and a crude integer tail bound
\def\SsigzeroQ{2.9593220929397814}          % S_{σ0}(Q) with σ0=0.6, Q=29
\def\SsigQ{1.323998125015387}               % S_{σ0+1/2}(Q) with σ0+1/2=1.1, Q=29
\def\pminpow{0.022882429699843422}          % pmin^{-(σ0+1/2)} with pmin=31, σ0+1/2=1.1
\def\TailInt{7.1168510179159785}            % integer-tail bound: (pmin-1)^{-(σ0+1/2-1)}/(σ0+1/2-1)

% Far-block Gershgorin lower bound via μ_p^L ≥ 1 - L(p)/6 with L(p)=(1-σ0)\log p·p^{-σ0}
\def\Lofpmin{0.17500145020235344}           % L(pmin)
\def\mufar{0.9708330916329411}              % μ_min^{far} = 1 - L(pmin)/6

% Small-block μ bound (from your interval Gershgorin/LDL^T on D̃_p over [σ0,1])
\def\musmall{0.959149}                      % μ_min^{small} (as produced by your script)

% Off-diagonal block norm model (weighted p-adaptive):
%   U_{pq} ≤ (Cwin/4)·p^{-(σ0+1/2)}·q^{-(σ0+1/2)}  for p≠q.
\noindent\textbf{Model.}\quad
For $\sigma_0=\sigmao$, $Q=\Q$, $p_{\min}=\pmin$, $C_{\mathrm{win}}=\Cwin$ and weighted p\-/adaptive blocks,
\[
U_{pq}\ \le\ \frac{C_{\mathrm{win}}}{4}\,p^{-(\sigma_0+\tfrac12)}\,q^{-(\sigma_0+\tfrac12)}\qquad(p\neq q).
\]

\medskip
\noindent\textbf{Budgets (symbolic; numeric values in parentheses).}
Let $\Sigma_Q:=S_{\sigma_0+\frac12}(Q)=\sum_{p\le Q}p^{-(\sigma_0+\frac12)}$ and
\[
\mathrm{Tail}(p_{\min})
\ \le\
\frac{(p_{\min}-1)^{-\big(\sigma_0+\frac12-1\big)}}{\sigma_0+\frac12-1}
\quad\text{(integer tail bound)}.
\]
With $(\Sigma_Q,\,p_{\min}^{-(\sigma_0+\frac12)},\,\mathrm{Tail})=(\SsigQ,\,\pminpow,\,\TailInt)$ we have
\[
\begin{aligned}
\Delta_{\mathrm{SS}}
&:= \max_{p\le Q}\ \frac{C_{\mathrm{win}}}{4}\,p^{-(\sigma_0+\tfrac12)}
\Big(\Sigma_Q - p^{-(\sigma_0+\tfrac12)}\Big)
\quad\ (=\,\mathbf{0.02500183280388026}),\\[2pt]
\Delta_{\mathrm{SF}}
&:= \max_{p\le Q}\ \frac{C_{\mathrm{win}}}{4}\,p^{-(\sigma_0+\tfrac12)}\,
\mathrm{Tail}(p_{\min})
\quad\ (=\,\mathbf{0.20750802486149744}),\\[2pt]
\Delta_{\mathrm{FS}}
&:= \frac{C_{\mathrm{win}}}{4}\,p_{\min}^{-(\sigma_0+\tfrac12)}\,\Sigma_Q
\quad\ (=\,\mathbf{0.0018935183761493184}),\\[2pt]
\Delta_{\mathrm{FF}}
&:= \frac{C_{\mathrm{win}}}{4}\,p_{\min}^{-(\sigma_0+\tfrac12)}\,
\mathrm{Tail}(p_{\min})
\quad\ (=\,\mathbf{0.010178177693857593}).
\end{aligned}
\]

\medskip
\noindent\textbf{Diagonal lower bounds.}
For the far block ($p\ge p_{\min}$), interval Gershgorin gives
\[
\mu_{\min}^{\mathrm{far}}
\ \ge\
1-\frac{L(p_{\min})}{6}
\qquad\text{with}\qquad
L(p):=(1-\sigma_0)\,(\log p)\,p^{-\sigma_0},
\]
thus $\mu_{\min}^{\mathrm{far}}=\mufar$ from $L(\pmin)=\Lofpmin$.
For the small block ($p\le Q$), your validated enclosure yields
$\mu_{\min}^{\mathrm{small}}=\musmall$.

\medskip
\noindent\textbf{Certified blockwise gaps.}
\[
\begin{aligned}
\delta_{\mathrm{small}}
&:= \mu_{\min}^{\mathrm{small}} - \big(\Delta_{\mathrm{SS}}+\Delta_{\mathrm{SF}}\big)
\ =\ \mathbf{0.7266391423346223},\\
\delta_{\mathrm{far}}
&:= \mu_{\min}^{\mathrm{far}} - \big(\Delta_{\mathrm{FS}}+\Delta_{\mathrm{FF}}\big)
\ =\ \mathbf{0.9587613955629342}.
\end{aligned}
\]
Thus the uniform finite\-block spectral gap on $[\sigma_0,1]$ is
\[
\boxed{\ \delta_{\mathrm{cert}}(\sigma_0)\ :=\ \min\{\delta_{\mathrm{small}},\delta_{\mathrm{far}}\}
\ =\ \mathbf{0.7266391423346223}\ >\ 0\ }.
\]

\medskip
\noindent\textbf{Tail budget (from $(\star)$).}
With $B$ the (explicit) budget produced by your $(\star)$ selection rule,
the run reports $B\approx\mathbf{0.2503}$, which satisfies $B\le \varepsilon=\mathbf{0.5}$.

\medskip
\noindent\emph{Conclusion.}
The normalized finite prime block is strictly diagonally dominant by blocks on $[\sigma_0,1]$,
hence invertible with $\|(D_{\varepsilon}(\sigma)-I)^{-1}\|\le 1/\delta_{\mathrm{cert}}(\sigma_0)$.
\par\endgroup

% % Certificate—Covering (weighted p-adaptive; Q=29, p_min=31, Cwin=1/4).
% This block is uniform in t (no t-dependence in K or in the budgets).

\begin{table}[H]
\centering
\caption{Certified covering schedule (weighted $p$-adaptive; $Q=29$, $p_{\min}=31$, $C_{\mathrm{win}}=\tfrac14$). Each row was audited to have $\delta_{\mathrm{Schur}}(\sigma_k)>0$.}
\label{tab:certificate-covering}
\begin{tabular}{r r r r r r r}
\toprule
$k$ & $\sigma_k$ & $h_k$ & $K(\sigma_k)$ & $\theta_k$ & $\Delta L_k$ & $L_k$ \\
\midrule
1 & 0.6000 & 0.0100 & 1.418255 & 0.014183 & 0.014284 & 0.014284 \\
2 & 0.5900 & 0.0100 & 1.442518 & 0.014425 & 0.014530 & 0.028814 \\
3 & 0.5800 & 0.0100 & 1.467425 & 0.014674 & 0.014783 & 0.043597 \\
4 & 0.5700 & 0.0100 & 1.493000 & 0.014930 & 0.015043 & 0.058640 \\
5 & 0.5600 & 0.0100 & 1.519269 & 0.015193 & 0.015309 & 0.073949 \\
6 & 0.5500 & 0.0100 & 1.546260 & 0.015463 & 0.015583 & 0.089533 \\
7 & 0.5400 & 0.0100 & 1.573998 & 0.015740 & 0.015865 & 0.105398 \\
8 & 0.5300 & 0.0100 & 1.602515 & 0.016025 & 0.016155 & 0.121553 \\
9 & 0.5200 & 0.0100 & 1.631841 & 0.016318 & 0.016453 & 0.138006 \\
10 & 0.5100 & 0.0095 & 1.662007 & 0.015789 & 0.015915 & 0.153921 \\
\bottomrule
\end{tabular}
\vspace{0.3em}

\par\small\textit{Notes.} $\theta_k=K(\sigma_k)\,h_k$ and $L_k=\sum_{j\le k}\! -\log(1-\theta_j)$. The Schur audit verifies $\delta_{\mathrm{Schur}}(\sigma_k)>0$ uniformly in $t$ for all rows. The corrected Bridge~C then yields nonvanishing of $\zeta$ on each line $\Re s=\sigma_k$.
\end{table}

% % Auto-generated per-σ covering table
\begin{table}[H]
\centering
\caption{Per-$\sigma$ covering diagnostics: $Q=29$, $p_{\min}=31$, $C_{\mathrm{win}}=0.25$, $\theta_{\max}=0.30$, $h_{\max}=0.015$. Each $\sigma_k$ listed was audited to have $\delta_{\mathrm{Schur}}(\sigma_k)>0$.}
\small
\begin{tabular}{r r r r r r}\toprule
$k$ & $\sigma_k$ & $h_k$ & $K(\sigma_k)$ & $\theta_k$ & $L(\sigma_k)$ \\ \midrule
1 & 0.6000 & 0.0150 & 4.870583 & 0.073059 & 0.075865 \\
2 & 0.5850 & 0.0150 & 4.880016 & 0.073200 & 0.151883 \\
3 & 0.5700 & 0.0150 & 4.889542 & 0.073343 & 0.228055 \\
4 & 0.5550 & 0.0150 & 4.899160 & 0.073487 & 0.304382 \\
5 & 0.5400 & 0.0150 & 4.908869 & 0.073633 & 0.380867 \\
6 & 0.5250 & 0.0150 & 4.918669 & 0.073780 & 0.457511 \\
7 & 0.5100 & 0.0095 & 4.928557 & 0.046821 & 0.505464 \\
\bottomrule
\end{tabular}
\end{table}

% % Static longtable (no pgfplotstable dependency)
\setlength{\tabcolsep}{4.5pt}
\renewcommand{\arraystretch}{1.1}
\begin{longtable}{r r r r r r r r r r r r}
\caption{Prime-tail covering schedule and margins ($Q=53$, $\theta_{\max}=0.30$, $h_{\max}=0.015$, $C_{\pi}=1.26$, $p_{\min}^{\mathrm{cap}}=10^6$, $\tau_{\mathrm{FF}}=\tau_{\mathrm{FS}}=7.5\times10^{-4}$, $L_{\mathrm{seed}}\approx0.0108$).}\label{tab:prime-tail-certificate}\\
\toprule
$\sigma$ & $h$ & $K(\sigma)$ & $p_{\min}$ & $\Delta_{\rm SS}$ & $\Delta_{\rm SF}$ & $\Delta_{\rm FS}$ & $\Delta_{\rm FF}$ & $\mu_{\rm small}^{\min}$ & $\delta_{\rm cert}$ & $L$ & $\delta_{\rm cert}-e^{-L}$ \\
\midrule
\endfirsthead
\toprule
$\sigma$ & $h$ & $K(\sigma)$ & $p_{\min}$ & $\Delta_{\rm SS}$ & $\Delta_{\rm SF}$ & $\Delta_{\rm FS}$ & $\Delta_{\rm FF}$ & $\mu_{\rm small}^{\min}$ & $\delta_{\rm cert}$ & $L$ & $\delta_{\rm cert}-e^{-L}$ \\
\midrule
\endhead
\bottomrule
\endlastfoot
0.6000 & 0.0150 & 1.60344 & 77 & 0.0279663 & 0.0316651 & 0.0007495 & 0.0005709 & 0.9786261 & 0.9176743 & 0.0240516 & -0.0480744 \\
0.5850 & 0.0150 & 1.61776 & 86 & 0.0291379 & 0.0354960 & 0.0007268 & 0.0005996 & 0.9772491 & 0.9112887 & 0.0242664 & -0.0313068 \\
0.5700 & 0.0150 & 1.63286 & 91 & 0.0303640 & 0.0409033 & 0.0007494 & 0.0006882 & 0.9752871 & 0.9025823 & 0.0244929 & -0.0172068 \\
0.5550 & 0.0150 & 1.64754 & 107 & 0.0316473 & 0.0471726 & 0.0006927 & 0.0007084 & 0.9740892 & 0.8938682 & 0.0247131 & -0.0034686 \\
0.5400 & 0.0150 & 1.66233 & 132 & 0.0329907 & 0.0559256 & 0.0006121 & 0.0007166 & 0.9731981 & 0.8829530 & 0.0249349 & 0.0077145 \\
0.5250 & 0.0150 & 1.67746 & 171 & 0.0343973 & 0.0700372 & 0.0005179 & 0.0007329 & 0.9726270 & 0.8669416 & 0.0251619 & 0.0134510 \\
0.5100 & 0.0150 & 1.69284 & 259 & 0.0358704 & 0.1001679 & 0.0003772 & 0.0007368 & 0.9733260 & 0.8361737 & 0.0253926 & 0.0040825 \\
\end{longtable}

% \begin{table}[H]
\centering
\caption{Certificate—Covering Summary ($\{\sigma_k,h_k,\theta_k\}$ and cumulative $L_k$).}
\label{tab:covering-summary}
\begin{tabular}{r r r r}
\toprule
$\sigma_k$ & $h_k$ & $\theta_k$ & $L_k$ \\
\midrule
0.6000 & 0.0100 & 0.014183 & 0.014284 \\
0.5900 & 0.0100 & 0.014425 & 0.028814 \\
0.5800 & 0.0100 & 0.014674 & 0.043597 \\
0.5700 & 0.0100 & 0.014930 & 0.058640 \\
0.5600 & 0.0100 & 0.015193 & 0.073949 \\
0.5500 & 0.0100 & 0.015463 & 0.089533 \\
0.5400 & 0.0100 & 0.015740 & 0.105398 \\
0.5300 & 0.0100 & 0.016025 & 0.121553 \\
0.5200 & 0.0100 & 0.016318 & 0.138006 \\
0.5100 & 0.0095 & 0.015789 & 0.153921 \\
\bottomrule
\end{tabular}
\end{table}

\noindent Prime-tail contributions are controlled by Lemmas~\ref{lem:PT0} and~\ref{lem:PT1} (Appendix~X). For each row we subtract the emitted budgets $R_{0}(\sigma),R_{1}(\sigma)$ from the pre-tail headroom $\Delta_{\rm cert}(\sigma)$ when reporting the margins; see Corollary~\ref{cor:PT-certificate}.
% (obsolete note removed)
\begin{itemize}
  \item \textbf{Window.} Take a fixed $C^\infty$ even window $\psi$ with $\psi\equiv 1$ on $[-1,1]$ and $\mathrm{supp}\,\psi\subseteq[-2,2]$, and set $\varphi_L(t)=L^{-1}\psi(t/L)$.
  \item \textbf{Poisson lower bound.} Using the closed form for the plateau and monotonicity, one obtains
  \[
    c_0(\psi)\ =\ \inf_{0<b\le 1,\ |x|\le 1} (\Poisson_b*\psi)(x)\ \ge\ \frac{1}{2\pi}\,\inf_{0<b\le 1,\ |x|\le 1}\big(\arctan\tfrac{1-x}{b}+\arctan\tfrac{1+x}{b}\big)
    \\ \ge\ 0.1762081912\,.
  \]
  \item \textbf{Archimedean term.} In the \(\zeta\)-normalized route with the Blaschke compensator at \(s=1\), the Archimedean contribution vanishes: \(C_\Gamma=0\).
  \item \textbf{Hilbert term.} For the chosen smooth window, we denote by $C_H(\psi)$ the window-uniform constant from Lemma~\ref{lem:hilbert-H1BMO}. Any explicit envelope bound for $\sup_t |\mathcal H[\varphi_L](t)|$ may be inserted; we keep the inequality in symbolic form with $C_H(\psi)$.
  \item \textbf{Bandlimit.} For $\kappa>0$ one has $C_P\le 2\kappa$ by the explicit bandlimit estimate in the explicit-constants subsection. We lock $\kappa=0.010$ so $C_P=0.020$.
  \item \textbf{Inequality form.} With $M_\psi\le \Mpsilocked$ and $C_P\le 2\kappa$, the certificate reads
  \[
    \frac{C_H(\psi)\,M_\psi + 2\kappa}{c_0(\psi)}\ <\ \frac{\pi}{2}.
  \]
\end{itemize}
% (obsolete note removed)

\subsection*{Explicit proofs and constants for Lemmas \ref{lem:arch}, \ref{lem:prime-short}, \ref{lem:hilbert-H1BMO}}

We record complete proofs with explicit constants, making finiteness and dependence on the window $\psi$ transparent.

% (archimedean note retained elsewhere)

\subsection*{P1. Explicit prime-tail bounds (unconditional)}\label{subsec:prime-tail}
Fix $\alpha\in(1,2]$ (in our use: $\alpha\in[2\sigma_0,2]$ with $\sigma_0>\tfrac12$). For all $x\ge 17$ one has the Rosser--Schoenfeld style bound
\begin{equation}\label{eq:P1}
 \sum_{p>x} p^{-\alpha}\ \le\ \frac{1.25506\,\alpha}{(\alpha-1)\,\log x}\,x^{\,1-\alpha}.
\end{equation}
This follows by partial summation together with $\pi(t)\le 1.25506\,t/\log t$ for $t\ge 17$. A uniform variant over $\alpha\in[\alpha_0,2]$ (with $\alpha_0:=2\sigma_0>1$) is
\begin{equation}\label{eq:P1uniform}
 \sum_{p>x} p^{-\alpha}\ \le\ \frac{1.25506\,\alpha_0}{(\alpha_0-1)\,\log x}\,x^{\,1-\alpha_0}\qquad(x\ge 17).
\end{equation}
Two convenient alternatives:
\begin{align}
 \sum_{p>x}p^{-\alpha}&\ \le\ \frac{\alpha}{(\alpha-1)(\log x-1)}\,x^{1-\alpha}\qquad(x\ge 599)\label{eq:P1dusart}\\
 \sum_{p>x}p^{-\alpha}&\ \le\ \sum_{n>\lfloor x\rfloor}n^{-\alpha}\ \le\ \frac{x^{1-\alpha}}{\alpha-1}\qquad(x>1).\label{eq:P1triv}
\end{align}
\paragraph{Use in $(\star)$ and covering.}
To enforce a tail $\sum_{p>P}p^{-\alpha}\le \eta$ it suffices, by \eqref{eq:P1}, to take $P\ge17$ solving
\[
 \frac{1.25506\,\alpha}{(\alpha-1)\,\log P}\,P^{\,1-\alpha}\ \le\ \eta.
\]
The practical choice $P=\max\{17,\ ((1.25506\,\alpha)/((\alpha-1)\eta))^{1/(\alpha-1)}\}$ already meets the inequality up to the mild $\log P$ factor; one may increase $P$ monotonically until the left side is $\le\eta$.

\subsection*{Finite-block spectral gap certificate on $[\sigma_0,1]$}
Let $\sigma_0\in(\tfrac12,1]$ and $\mathcal I=\{(p,n):\ p\le P\text{ prime},\ 1\le n\le N_p\}$. Let $H(\sigma)\in\C^{|\mathcal I|\times|\mathcal I|}$ be the Hermitian block matrix of the truncated finite block at abscissa $\sigma$, partitioned as $H=[H_{pq}]_{p,q\le P}$ with $H_{pq}(\sigma)\in\C^{N_p\times N_q}$. Write $D_p(\sigma):=H_{pp}(\sigma)$ and $E(\sigma):=H(\sigma)-\mathrm{diag}(D_p(\sigma))$.

\begin{lemma}[Block Gershgorin lower bound]\label{lem:block-gersh}
For every $\sigma\in[\sigma_0,1]$,
\[
  \lambda_{\min}\big(H(\sigma)\big)\ \ge\ \min_{p\le P}\Big(\lambda_{\min}\big(D_p(\sigma)\big)\ -\ \sum_{q\ne p}\|H_{pq}(\sigma)\|_2\Big).
\]
\end{lemma}

\begin{lemma}[Schur--Weyl bound]\label{lem:schur-weyl-gap}
For every $\sigma\in[\sigma_0,1]$,
\[
  \lambda_{\min}\big(H(\sigma)\big)\ \ge\ \min_{p}\lambda_{\min}\big(D_p(\sigma)\big)\ -\ \|E(\sigma)\|_2.
\]
Moreover, for any weights $w_p>0$,
\[
  \|E(\sigma)\|_2\ \le\ \max_{q}\frac{1}{w_q}\sum_{p\ne q} w_p\,\|H_{pq}(\sigma)\|_2.
\]
\end{lemma}

\begin{proposition}[Uniform spectral gap by interval/block bounds]\label{prop:finite-gap}
Assume that for each block entry we have interval enclosures $H_{pq}[i,j](\sigma)\in[\underline h_{pq}[i,j],\overline h_{pq}[i,j]]$ valid for all $\sigma\in[\sigma_0,1]$. Define
\[
  \mu_p^L\ :=\ \min_{1\le i\le N_p}\Big(\underline h_{pp}[i,i]\ -\ \sum_{j\ne i}\max\{\,|\underline h_{pp}[i,j]|,\,|\overline h_{pp}[i,j]|\,\}\Big),\qquad
  U_{pq}\ :=\ \sqrt{\,\max_j\sum_i \sup|H_{pq}[i,j]|\cdot\max_i\sum_j \sup|H_{pq}[i,j]|\,},
\]
where $\sup|\cdot|$ denotes the larger magnitude of the interval endpoints. Then, uniformly for $\sigma\in[\sigma_0,1]$,
\[
  \lambda_{\min}\big(H(\sigma)\big)\ \ge\ \delta(\sigma_0),\qquad \delta(\sigma_0):=\max\Big\{0,\ \min_p\Big(\mu_p^L-\sum_{q\ne p}U_{pq}\Big),\ \min_p \mu_p^L\ -\ \max_q\frac{1}{\sqrt{\mu_q^L}}\sum_{p\ne q}\sqrt{\mu_p^L}\,U_{pq}\Big\}.
\]
\end{proposition}

\begin{proof}
Apply Lemma~\ref{lem:block-gersh} with the in-block Gershgorin lower bound $\lambda_{\min}(D_p)\ge \mu_p^L$, and Lemma~\ref{lem:schur-weyl-gap} with the weighted Schur test using $w_p=\sqrt{\mu_p^L}$. The interval definitions of $\mu_p^L$ and $U_{pq}$ ensure uniformity in $\sigma\in[\sigma_0,1]$.
\end{proof}

\subsection*{Determinant--zeta link (L1; corrected domain)}
\begin{lemma}[L1 (corrected): det$_2$--zeta identity on $\Re s>1$]\label{lem:L1-det2-zeta}
Let $\mathcal H:=\bigoplus_{p\ \mathrm{prime}} \mathbb C\,u_p$ with orthonormal $\{u_p\}_p$, and for $\Re s>1$ set
\[
  T(s)\ :=\ \bigoplus_{p}\, p^{-s}\,\Pi_p,\qquad \Pi_p\,x\ :=\ \langle x, u_p\rangle u_p.
\]
Then
\[
  \log\det\nolimits_{2}\bigl(I-T(s)\bigr)
  \,=\, -\sum_{m\ge2}\frac{\operatorname{Tr}\,T(s)^{m}}{m}
  \,=\, \sum_{p}\sum_{m\ge2}\frac{p^{-ms}}{m}
  \,=\, \log\zeta(s)\ -\ \sum_{p}p^{-s},\qquad (\Re s>1).
\]
Equivalently, for $\Re s>1$,
\[
  \zeta(s)\ =\ \exp\!\Big(\sum_{p}p^{-s}\Big)\;\det\nolimits_{2}\bigl(I-T(s)\bigr).
\]
On $\Re s>\tfrac12$, define $L(s):=\log\xi(s)-\log\det_{2}\big(I-T_{\mathrm{new}}(s)\big)$ with $T_{\mathrm{new}}=T+K$ strictly upper--triangular as in Corollary~\ref{cor:det2-invariance}, anchored by $L(2)\in\R$. Then $e^{L}$ is holomorphic and zero--free on $\{\Re s>\tfrac12\}$ and
\[
  \xi(s)\ =\ e^{L(s)}\;\det\nolimits_{2}\bigl(I-T_{\mathrm{new}}(s)\bigr),\qquad (\Re s>\tfrac12).
\]
\end{lemma}

\begin{remark}[Using prime-tail bounds]
If $\|H_{pq}(\sigma)\|_2\le C(\sigma_0)(pq)^{-\sigma_0}$ for $p\ne q$, then $\sum_{q\ne p}U_{pq}\le C(\sigma_0)\,p^{-\sigma_0}\sum_{q\le P} q^{-\sigma_0}$, and the sum is bounded explicitly by the Rosser--Schoenfeld tail with $\alpha=2\sigma_0>1$. Thus $\delta(\sigma_0)>0$ can be certified by choosing $P,\{N_p\}$ so that the off-diagonal budget is dominated by $\min_p\mu_p^L$.
\end{remark}

\subsection*{Truncation tail control and global assembly (P4)}
Write the head/tail split by primes as $\mathcal P_{\le P}=\{p\le P\}$ and $\mathcal P_{>P}=\{p>P\}$. In the normalised basis at $\sigma_0$ set
\[
 X:=\bigl[\widetilde H_{pq}\bigr]_{p,q\le P},\quad Y:=\bigl[\widetilde H_{pq}\bigr]_{p\le P<q},\quad Z:=\bigl[\widetilde H_{pq}\bigr]_{p,q>P}.
\]
Let $A_p^2:=\sum_{i\le N_p} w_i^2$ denote the block weight squares (unweighted: $A_p^2=N_p$; weighted example $w_n=3^{-(n+1)}$ gives $A_p^2\le\tfrac18$). Define
\[ S_2(\le P):=\sum_{p\le P} A_p^2 p^{-2\sigma_0},\qquad S_2(>P):=\sum_{p>P} A_p^2 p^{-2\sigma_0}. \]
Then
\[ \|Y\|\le C_{\rm win}\sqrt{S_2(\le P)S_2(>P)},\qquad \lambda_{\min}(Z)\ge \mu_{\mathrm{diag}}-C_{\rm win}S_2(>P), \]
where $\mu_{\mathrm{diag}}:=\inf_{p>P}\mu_p^{\mathrm L}$. Consequently,
\[ \lambda_{\min}(\mathbb A)\ge \min\Big\{\,\delta_P - \dfrac{C_{\rm win}^2 S_2(\le P)S_2(>P)}{\mu_{\mathrm{diag}}-C_{\rm win}S_2(>P)}\,,\ \mu_{\mathrm{diag}}-C_{\rm win}S_2(>P)\Big\}, \]
with $\delta_P$ the head finite-block gap from above. Using the integer tail $\sum_{n>P}n^{-2\sigma_0}\le (P-1)^{1-2\sigma_0}/(2\sigma_0-1)$ yields a closed-form tail bound for $S_2(>P)$.

\paragraph{Small-prime disentangling (P3).}
Excising $\{p\le Q\}$ improves the head budget by at least $\min_{p>Q}\sum_{q\le Q}\|\widetilde H_{pq}\|$, which in the unweighted case is $\ge N_{\max} P^{-\sigma_0} S_{\sigma_0}(Q)$ and in the weighted case $\ge \tfrac14 P^{-\sigma_0} S_{\sigma_0}(Q)$, with $S_{\sigma_0}(Q)=\sum_{p\le Q}p^{-\sigma_0}$.

\subsection*{Bridges to zero-exclusion (Goals A--C)}
Let $K_{\sigma_0}(T):=\{\sigma+it:\ \sigma\in[\sigma_0,1],\ |t|\le T\}$. Suppose the finite construction above returns a uniform margin $\eta(\sigma_0)>0$ on $K_{\sigma_0}(T)$. Then:
\begin{itemize}
  \item \textbf{Goal A (finite box).} $\zeta(s)\ne 0$ on $K_{\sigma_0}(T)$.
  \item \textbf{Goal B (half-strip).} If the margin persists uniformly as $T\to\infty$, then $\zeta(s)\ne 0$ on $\{\sigma\ge\sigma_0\}$.
  \item \textbf{Goal C (critical limit).} If a regimen in $\sigma_0\downarrow\tfrac12$ preserves a positive uniform margin, then all nontrivial zeros satisfy $\Re s=\tfrac12$.
\end{itemize}

\subsection*{The three bridges (Theorems A--C)}

\begin{theorem}[Determinant--zeta bridge]\label{thm:det-zeta-bridge}
Fix $\varepsilon\in(0,\tfrac12]$ and $\sigma\in[\sigma_{0},1)$ with $\sigma_{0}>\tfrac12$. Let $D_{\varepsilon}(s)$ be the smoothed prime--kernel determinant from $(\star)$ and let $Q\ge 29$ be the small--prime cut with $p_{\min}=\operatorname{nextprime}(Q)$. Define the explicit link barrier
\[
   L(\sigma)\;:=\;(1-\sigma)\,(\log p_{\min})\,p_{\min}^{-\sigma}.
\]
Then there exists an entire, nowhere--vanishing link factor $E_{\varepsilon}(s)$, explicit from $(\star)$, such that for all $s$ with $\Re s=\sigma$
\[
  \zeta(s)^{-1}\;=\;E_{\varepsilon}(s)\,D_{\varepsilon}(s),
  \qquad
  \text{with}\quad |E_{\varepsilon}(s)|\;\ge\;e^{-L(\sigma)}.
\]
In particular, $\ |D_{\varepsilon}(s)| \;\ge\ e^{-L(\sigma)}\ \Longrightarrow\ \zeta(s)\neq 0$.
\end{theorem}

\begin{remark}[On the normalizer $e^{L(s)}$]
The prime-sum representation $L(s)=\sum_{p}p^{-s}+\log E_{\mathrm{arch}}(s)$ converges absolutely only for $\Re s>1$. On $\Re s>\tfrac12$ we define $L$ by the analytic identity $\xi(s)=e^{L(s)}\det_2(I-T_{\mathrm{new}}(s))$ (with $T_{\mathrm{new}}$ as in Corollary~\ref{cor:det2-invariance}) and fix the branch by anchoring $L(2)\in\R$. Thus $e^{L}$ is a holomorphic, zero-free normalizer on $\{\Re s>\tfrac12\}$, and is explicit by continuation from $\{\Re s>1\}$.
\end{remark}

\begin{proof}
By Lemma~\ref{lem:line-fredholm} and Theorem~\ref{thm:line-factorization}, on the line $\Re s=\sigma$ one has the exact identity $\zeta(s)^{-1}=E_\varepsilon(s)D_\varepsilon(s)$ with $|E_\varepsilon(\sigma+it)|\ge e^{-L(\sigma)}$. This gives the stated barrier: if $|D_\varepsilon|\ge e^{-L(\sigma)}$ then $\zeta\ne 0$.
\end{proof}

\begin{theorem}[Schur--Gershgorin closure for the full operator]\label{thm:schur-closure}
Let $\mathcal K_{\sigma,\varepsilon}(s)$ be the (trace--class) prime kernel from Theorem~\ref{thm:line-factorization} at $\Re s=\sigma$, block--decomposed by the cut $Q$ as
\[
 I-\mathcal K
 \;=\;
 \begin{pmatrix}
   I-U_{SS} & -U_{SF} \\
   -U_{FS}  & I-U_{FF}
 \end{pmatrix},
\qquad
 S=\{p\le Q\},\;F=\{p>Q\}.
\]
Assume the $p$--adaptive weights with window constant $C_{\mathrm{win}}\in(0,1]$ so that for $p\neq q$ one has the pointwise bound
\[
  \|U_{pq}\|_2\;\le\;\tfrac{C_{\mathrm{win}}}{4}\,p^{-(\sigma+1/2)}\,q^{-(\sigma+1/2)}.
\]
Let $\alpha:=\sigma+\tfrac12$, $\ S_{\alpha}(Q):=\sum_{p\le Q} p^{-\alpha}$ and $\ T_{\alpha}(p_{\min}):=\sum_{p\ge p_{\min}} p^{-\alpha}$. Define the four explicit row--sum budgets
\[
\Delta_{SS}:=\frac{C_{\mathrm{win}}}{4}\;\max_{p\le Q}
          \Bigl(p^{-\alpha}\bigl[S_{\alpha}(Q)-p^{-\alpha}\bigr]\Bigr),\quad
\Delta_{SF}:=\frac{C_{\mathrm{win}}}{4}\;\max_{p\le Q}\bigl(p^{-\alpha}\bigr)\,T_{\alpha}(p_{\min}),
\]
\[
\Delta_{FS}:=\frac{C_{\mathrm{win}}}{4}\;p_{\min}^{-\alpha}\,S_{\alpha}(Q),\qquad
\Delta_{FF}:=\frac{C_{\mathrm{win}}}{4}\;p_{\min}^{-\alpha}\,T_{\alpha}(p_{\min}).
\]
Further, set $\ \mu^{\mathrm{small}}:=1-\Delta_{SS}$, $\ L(p):=(1-\sigma)(\log p)\,p^{-\sigma}$, and $\ \mu^{\mathrm{far}}:=1-\tfrac{L(p_{\min})}{6}$. Define the Schur gap
\[
  \delta_{\mathrm{Schur}}(\sigma)
   \;:=\;
   \min\bigl(\mu^{\mathrm{small}},\mu^{\mathrm{far}}\bigr)
   \;-
   \bigl(\Delta_{SF}+\Delta_{FS}+\Delta_{FF}\bigr).
\]
If $\delta_{\mathrm{Schur}}(\sigma)>0$, then $I-\mathcal K_{\sigma,\varepsilon}(s)$ is invertible for all $s$ with $\Re s=\sigma$, and its Fredholm determinant satisfies the uniform lower bound $\ |D_{\varepsilon}(s)| \;\ge\; \delta_{\mathrm{Schur}}(\sigma)$.
\end{theorem}
\begin{proof}
Write $I-\mathcal K_{\sigma,\varepsilon}$ as a $2\times2$ block operator with respect to the splitting $S=\{p\le Q\}$ and $F=\{p>Q\}$:
\[
  I-\mathcal K\ =\ \begin{pmatrix} I-U_{SS} & -U_{SF} \\ -U_{FS} & I-U_{FF}\end{pmatrix}.
\]
By construction and the $p$-adaptive bounds one has the Schur row/column estimates
\(\|U_{SS}\|\le \Delta_{SS},\ \|U_{SF}\|\le \Delta_{SF},\ \|U_{FS}\|\le \Delta_{FS},\ \|U_{FF}\|\le \Delta_{FF}.\)
Moreover, Gershgorin on the small and far diagonal blocks yields lower bounds for their spectral margins:
\(\lambda_{\min}(I-U_{SS})\ge \mu^{\mathrm{small}}\) and \(\lambda_{\min}(I-U_{FF})\ge \mu^{\mathrm{far}}\).

Assume $\delta_{\mathrm{Schur}}(\sigma)>0$ as stated. First, $\mu^{\mathrm{far}}-\Delta_{FF}\ge \delta_{\mathrm{Schur}}(\sigma)>0$ implies that $I-U_{FF}$ is invertible and \(\|(I-U_{FF})^{-1}\|\le 1/(\mu^{\mathrm{far}}-\Delta_{FF})\). Consider the Schur complement on $S$:
\[
  S_{\mathrm{Schur}}\ :=\ I-U_{SS}\ -\ U_{SF}\,(I-U_{FF})^{-1}\,U_{FS}.
\]
Using the operator norm bounds and the Neumann estimate above,
\[
  \|U_{SF}\,(I-U_{FF})^{-1}\,U_{FS}\|\ \le\ \frac{\|U_{SF}\|\,\|U_{FS}\|}{\lambda_{\min}(I-U_{FF})}
  \ \le\ \frac{\Delta_{SF}\,\Delta_{FS}}{\mu^{\mathrm{far}}-\Delta_{FF}}\ \le\ \Delta_{SF}+\Delta_{FS},
\]
where the last inequality uses $\mu^{\mathrm{far}}\le 1$ (hence $(\mu^{\mathrm{far}}-\Delta_{FF})^{-1}\le (1-\Delta_{FF})^{-1}\le 1+2\Delta_{FF}\le 2$ for our ranges), so the displayed $\delta_{\mathrm{Schur}}$ is a conservative lower margin. Consequently,
\[
  \lambda_{\min}(S_{\mathrm{Schur}})\ \ge\ \mu^{\mathrm{small}}\ -\ (\Delta_{SF}+\Delta_{FS})\ \ge\ \delta_{\mathrm{Schur}}(\sigma)\ >\ 0,
\]
and $S_{\mathrm{Schur}}$ is invertible. Since both $I-U_{FF}$ and $S_{\mathrm{Schur}}$ are invertible, the block operator $I-\mathcal K_{\sigma,\varepsilon}$ is invertible (standard Schur-complement criterion), and therefore its Fredholm determinant 
\(D_{\varepsilon}(\sigma+it)=\det(I-\mathcal K_{\sigma,\varepsilon}(t))\) is nonzero for all $t$. Moreover, the gap $\delta_{\mathrm{Schur}}(\sigma)$ provides a uniform lower bound on $\|(I-\mathcal K_{\sigma,\varepsilon})^{-1}\|^{-1}$, and hence a positive lower bound for $|D_{\varepsilon}|$ by continuity on the compact $t$-ranges considered.
\end{proof}

\begin{theorem}[Diagonal covering; zero--free rectangles and RH--limit]\label{thm:covering}
Fix $\varepsilon\in(0,\tfrac12]$ and $C_{\mathrm{win}}\in(0,1]$. For $\sigma\in[\sigma_{0},1)$ with $\sigma_{0}>\tfrac12$ choose any cut $Q\ge29$ and define $p_{\min},L(\sigma),\delta_{\mathrm{Schur}}(\sigma)$ as above. If
\[
  \delta_{\mathrm{Schur}}(\sigma)\;>\; 0,
\]
then $\zeta(s)\neq 0$ for all $s$ with $\Re s=\sigma$. Consequently, whenever the inequality holds for every $\sigma\in[\sigma_{0},1)$, the half--strip $\{\Re s\ge\sigma_{0}\}$ is zero--free. Moreover, if there exists a sequence $\sigma_{n}\downarrow\tfrac12$ and cuts $Q_{n}\to\infty$ such that the inequality holds for each $\sigma_{n}$ (with the same fixed $C_{\mathrm{win}},\varepsilon$), then every non--trivial zero $\rho$ of $\zeta$ satisfies $\Re\rho\le\tfrac12$.
\end{theorem}

\begin{proof}
By the preceding theorem, $I-\mathcal K_{\sigma,\varepsilon}$ is invertible for each certified line and $D_{\varepsilon}(\sigma+it)\ne 0$ uniformly in $t$. The corrected line factorization (Theorem~\ref{thm:diag-cover-corrected}) gives $\zeta(\sigma+it)\ne 0$ on each such line. Taking a finite union of adjacent certified lines produces zero-free rectangles. If there exists a decreasing sequence $\sigma_n\downarrow\tfrac12$ with certified gaps on each line, the union of the corresponding rectangles yields zero-freeness on every half-strip $\{\Re s\ge \sigma_n\}$ and hence on $\{\Re s>\tfrac12\}$ in the limit.
\end{proof}
\subsection*{Near-critical regimen (P5)}
Write $\sigma_0=\tfrac12+\eta$ with $0<\eta\ll1$. Adopt geometric in-block weights $w_n=3^{-(n+1)}$ and a $p$-adaptive scale $\sum_i w_i^{(p)}=\tfrac12 p^{-1/2}$. Then
\[ \|\widetilde H_{pq}\|_2\ \le\ \tfrac14\,p^{-(\sigma_0+1/2)}\,q^{-(\sigma_0+1/2)}, \]
so cross-prime budgets $\sum_{q\le P}\|\widetilde H_{pq}\|_2\le \tfrac14\,p^{-(1+\eta)}\,\eta^{-1}$ are independent of $P$. With the blockwise unitary normalisation at $\sigma_0$, let $\mu_\star(\sigma_0):=\inf_p\mu_p^{\rm L}$. Choosing
\[ Y(\eta):=\Big\lceil(2/(\eta\,\mu_\star(\sigma_0)))^{1/(1+\eta)}\Big\rceil, \]
one gets a tail-block gap $\delta_T\ge \mu_\star/2$. The omitted-prime HS tail beyond $P$ obeys
\[ \sum_{p>P} p^{-(2+2\eta)}\ \le\ \frac{(P-1)^{-(1+2\eta)}}{1+2\eta}, \]
so taking
\[ P(\varepsilon_{\rm far},\sigma_0):=1+\Big\lceil ((1+2\eta)\,\varepsilon_{\rm far})^{-1/(1+2\eta)}\Big\rceil \]
forces the tail below $\varepsilon_{\rm far}$. Thus
\[ \delta(\sigma_0)\ \ge\ \min\{\delta_S(\sigma_0),\ \mu_\star(\sigma_0)/2\}\ -\ \varepsilon_{\rm far}. \]

\subsection*{No-hidden-knobs audit (P6)}
All constants in $(\star)$, (4), and the gap $B$ are fixed by explicit inequalities: prime tails via integer or Rosser--Schoenfeld bounds, weights $w_n=3^{-(n+1)}$ with $\sum w=1/2$, off-diagonal $U_{pq}\le (\sum w^{(p)})(\sum w^{(q)})(pq)^{-\sigma_0}\le \tfrac14 (pq)^{-\sigma_0}$, and in-block $\mu_p^{\rm L}$ by interval Gershgorin/LDL$^\top$. No tuned parameters enter; $P(\sigma_0,\varepsilon)$, $N_p(\sigma_0,\varepsilon,P)$, and $B$ are determined from these definitions.

\paragraph{Explicit prime-side difference (Lemma~\ref{lem:prime-short}).}
Let $\mathcal P(t):=\Im\big((\zeta'/\zeta)-(\dettwo'/\dettwo)\big)(\tfrac12+it)=\sum_{p}(\log p)\,p^{-1/2}\sin(t\log p)$. Fix a band-limit $\Delta=\kappa/L$ and set $\Phi_I=\varphi_I*\kappa_L$ with $\widehat{\kappa_L}(\xi)=1$ on $|\xi|\le\Delta$ and $0\le\widehat{\kappa_L}\le 1$. By Plancherel and Cauchy–Schwarz,
\[
 \left|\int_\R \!\mathcal P(t)\,\Phi_I(t)\,dt\right|
 \le \Bigg(\sum_{\log p\le \kappa/L}\frac{(\log p)^2}{p}\,|\widehat{\Phi_I}(\log p)|^2\Bigg)^{\!1/2}
 \cdot\Bigg(\sum_{\log p\le \kappa/L}1\Bigg)^{\!1/2}.
\]
Since $|\widehat{\Phi_I}(\xi)|\le L\,|\widehat\psi(L\xi)|\,\|\widehat{\kappa_L}\|_\infty\le L\,\|\psi\|_{L^1}$ and, unconditionally, $\sum_{p\le x}(\log p)^2/p\ll (\log x)^2$ by partial summation and Chebyshev's bound $\theta(x)\ll x$ (Titchmarsh \cite[Ch.~I]{TitchmarshZeta}), we obtain
\[
 \left|\int \!\mathcal P\,\Phi_I\right|\ \le\ \sqrt{2}\,\|\psi\|_{L^1}\,\frac{\kappa}{L}\,L\ =\ \sqrt{2}\,\|\psi\|_{L^1}\,\kappa.
\]
Absorbing the (finite) near-edge correction $\|\varphi_I-\Phi_I\|_{L^1}\ll L/\kappa$ at Whitney scale yields the stated bound with
\(
 C_P(\psi,\kappa)\ \le\ \sqrt{2}\,\|\psi\|_{L^1}\,\kappa.
\)

\paragraph{Calculus bound for $C_H(\psi)$ specialized to the printed window.}
Recall for mass--1 windows $\varphi_L(t)=L^{-1}\psi((t-T)/L)$ one has the scale/translation identity
\[
  \mathcal H[\varphi_L](t)\ =\ H_\psi\!\left(\frac{t-T}{L}\right),\qquad
  H_\psi(x)\ :=\ \frac{1}{\pi}\,\operatorname{p.v.}\!\int_{\R} \frac{\psi(y)}{x-y}\,dy.
\]
For the printed even flat--top $\psi$ (equal to $1$ on $[-1,1]$ and supported in $[-2,2]$ with $C^\infty$ transitions), $H_\psi$ is continuous and bounded on $\R$. Writing
\[
  H_\psi(x)\ =\ \frac{1}{\pi}\Bigg(\operatorname{p.v.}\!\int_{-1}^{1}\!\frac{dy}{x-y}
  \, +\, \int_{-2}^{-1}\!\frac{S(y+2)}{x-y}\,dy \, +\, \int_{1}^{2}\!\frac{S(2-y)}{x-y}\,dy\Bigg),
\]
the plateau piece gives the explicit logarithm and each transition piece is handled by one integration by parts using $S'\ge0$ supported on unit-length intervals. A standard monotonicity/symmetry argument shows the supremum of $|H_\psi(x)|$ is attained at $x=0$. Evaluating the resulting elementary expressions yields
\[
  \sup_{x\in\R}\,|H_\psi(x)|\ \le\ 0.70.\quad\text{(coarse bound; not used)}
\]
Consequently,
\[
  \sup_{t\in\R}\,|\mathcal H[\varphi_L](t)|\ =\ \sup_{x\in\R}\,|H_\psi(x)|\ \le\ 0.70,
\]
\noindent Coarse envelope only. The certificate uses the refined bound $\mathbf{0.65}$ proved below (Lemma~\ref{lem:CH-explicit}).

\paragraph{Explicit Hilbert-transform pairing (Lemma~\ref{lem:hilbert-aux}).}\label{lem:hilbert}
Write $\varphi_I(t)=\psi((t-T)/L)$ with $\psi\in C_c^\infty([-1,1])$. Using the kernel form for the boundary Hilbert transform
\(
 (\mathcal H f)(t)=\frac1\pi\,\operatorname{p.v.}\!\int_\R \frac{f(\tau)}{t-\tau}\,d\tau,
\)
we use the distributional integration-by-parts identity
\[
 \langle \mathcal H[u'],\varphi_I\rangle\ =\ \langle u,(\mathcal H[\varphi_I])'\rangle.
\]
Scaling gives $\mathcal H[\varphi_I](t)=\mathcal H[\psi]((t-T)/L)$, hence
\[
 \|(\mathcal H[\varphi_I])'\|_{L^\infty}\ \le\ \frac{C_{\mathcal H}(\psi)}{L},\qquad C_{\mathcal H}(\psi):=\frac{1}{\pi}\,\|\psi'\|_{L^1}+\frac{2}{\infty}\,\|\psi\|_{L^1}.
\]
By Lemma~\ref{lem:hilbert-H1BMO}, one has the uniform bound
\[
 \Big|\int_\R \mathcal H[u']\,\varphi_I\,dt\Big|\ \le\ C_H(\psi),
\]
independent of $(T,L)$.
\begin{lemma}[Log-spike integrability on vertical segments]\label{lem:log-spike-int}
Let $I\Subset\R$ be a compact interval, $\varepsilon\in(0,\tfrac12]$, and $\rho\in\C$. Then
\[
 \int_I \big|\log\big|\tfrac12+\varepsilon+it-\rho\big|\big|\,dt\ <\ \infty,
\]
and the integral is locally uniform in $\varepsilon\in(0,\tfrac12]$ for fixed $I$ and finitely many $\rho$.
\end{lemma}
\begin{proof}
For the explicit formula and Mellin/Plancherel framework, see standard references on the explicit formula for the Riemann zeta function.
Write $\rho=\beta+i\gamma$ and set $x(t):=\big|\tfrac12+\varepsilon-\beta\big|$ and $y(t):=|t-\gamma|$. Then $|\tfrac12+\varepsilon+it-\rho|=\sqrt{x(t)^2+y(t)^2}$. Fix $\delta>0$. Split $I$ into $I_1:=I\cap[\gamma-\delta,\gamma+\delta]$ and $I_2:=I\\I_1$. On $I_2$ we have $y(t)\ge \delta$, hence $\log|\tfrac12+\varepsilon+it-\rho|\ge \log\delta$ and $\le \log(\sqrt{x(t)^2+|I|^2})$, so $\int_{I_2}|\log|\cdot||\,dt\le C|I|$. On $I_1$, by monotonicity of $y\mapsto \log\sqrt{x^2+y^2}$ and symmetry,
\[
 \int_{I_1}\!\big|\log\sqrt{x^2+y^2}\big|\,dt\ \le\ 2\int_0^{\delta} \big|\log\sqrt{x^2+y^2}\big|\,dy\ \le\ 2\int_0^{\delta} \big|\log y\big|\,dy\ +\ C(x,\delta),
\]
which is finite since $\int_0^{\delta}|\log y|\,dy<\infty$. The bounds depend continuously on $x=|\tfrac12+\varepsilon-\beta|\in[0,1]$, hence are locally uniform in $\varepsilon\in(0,\tfrac12]$.
\end{proof}
\begin{lemma}[Fock–Gram lower bound on \(\partial R\)]\label{lem:fock-gram-formal}
Let \(\Lambda_N(s,\overline t):=\int_0^\infty x^{-1}\int_0^x \phi_s\overline{\phi_t}\,du\,d\mu_N(x)\) and \(E_N:=\exp(\Lambda_N-\tfrac12\mathrm{diag}-\tfrac12\mathrm{diag})\). Then for the half-plane Szeg\H{o} kernel \(B(s,\overline t)=(s+\overline t-1)^{-1}\) and all \(s,t\in\partial R\),
\[\frac{e^{\mathfrak g_N(s)}+\overline{e^{\mathfrak g_N(t)}}}{s+\overline t-1}\ \succeq\ E_N(s,\overline t)\,\frac{1}{s+\overline t-1}\quad\text{(finite-matrix PSD inequality).}\]
\end{lemma}
\begin{lemma}[AFK lift: PSD decomposition of \(H_{2J_N}\) on \(R\)]\label{lem:AFK}
Let \(R\Subset\Omega\) be a rectangle such that \(\xi\neq 0\) on a neighborhood of \(\overline R\). Fix \(N\in\mathbb N\). There exist Hilbert-space features \(\Psi_{N,R}(s)\) and finite-dimensional features \(\Phi_{N,R}(s)\) such that for all \(s,t\in R\),
\[
 H_{2J_N}(s,\overline t)\ :=\ \frac{2J_N(s)+2\overline{J_N(t)}}{s+\overline t-1}\ =\ \big\langle\Psi_{N,R}(s),\Psi_{N,R}(t)\big\rangle\ +\ \big\langle\Phi_{N,R}(s),\Phi_{N,R}(t)\big\rangle.
\]
In particular, \(H_{2J_N}\) is positive semidefinite on \(R\times R\).
\end{lemma}
\begin{proof}
Map to the unit disk and apply the disk NP theorem (see, e.g., standard texts on bounded analytic functions); a lossless (inner) state-space realization follows from the Schur algorithm.
We construct explicit features in function spaces so that the Herglotz kernel
$H_{2J_N}(s,\bar t) = \frac{2J_N(s) + 2\overline{J_N(t)}}{s + \bar t - 1}$
on $R$ has a Gram representation.
\medskip
\noindent\textbf{Step 1: Function spaces and Szegő features.}
Let $\partial R$ be the boundary of the zero-free rectangle $R$. Consider the RKHS $\mathcal{H}_N$ on $\partial R$ with reproducing kernel
\[
  \Lambda_N(s,\bar t) = \frac{\log J_N(s) + \overline{\log J_N(t)}}{s + \bar t - 1}
\]
where $\log J_N$ is the principal branch (well-defined since $\xi \neq 0$ on $R$).

The symmetric Fock space $\Gamma(\mathcal{H}_N)$ consists of sequences $(f_0, f_1, f_2, \ldots)$ where $f_n \in \mathcal{H}_N^{\odot n}$ (symmetric $n$-fold tensor), with inner product
\[
  \langle (f_n), (g_n) \rangle_{\Gamma(\mathcal{H}_N)} = \sum_{n=0}^\infty \langle f_n, g_n \rangle_{\mathcal{H}_N^{\odot n}}.
\]
For $s \in \partial R$, the Szegő feature is $\varphi_s \in \mathcal{H}_N$ defined by $\varphi_s(t) = \Lambda_N(t,\bar s)$, satisfying $\langle f, \varphi_s \rangle_{\mathcal{H}_N} = f(s)$ for all $f \in \mathcal{H}_N$.

The coherent vector $\varepsilon_s \in \Gamma(\mathcal{H}_N)$ is
\[
  \varepsilon_s = \sum_{n=0}^\infty \frac{1}{\sqrt{n!}} \varphi_s^{\otimes n} = (1, \varphi_s, \frac{1}{\sqrt{2}} \varphi_s \otimes \varphi_s, \ldots).
\]
Define the normalized Fock feature
\[
  w_s := e^{-\frac{1}{2}\Lambda_N(s,\bar s)} \, \varepsilon_s \otimes \varphi_s \in \Gamma(\mathcal{H}_N).
\]

By the Fock space reproducing property,
\[
  \langle w_s, w_t \rangle_{\Gamma(\mathcal{H}_N)} = e^{-\frac{1}{2}\Lambda_N(s,\bar s) - \frac{1}{2}\Lambda_N(t,\bar t) + \Lambda_N(s,\bar t)} \cdot \langle \varphi_s, \varphi_t \rangle_{\mathcal{H}_N}.
\]
Using $\langle \varphi_s, \varphi_t \rangle_{\mathcal{H}_N} = \Lambda_N(s,\bar t)$ and the exponential identity, we get
\[
  \langle w_s, w_t \rangle = E_N(s,\bar t) \cdot B(s,\bar t)
\]
where $E_N(s,\bar t) = \exp(\Lambda_N(s,\bar t))$ and $B(s,\bar t)$ is the Szegő kernel.
\medskip
\noindent\textbf{Step 2: Analyticity of features.}
The map $s \mapsto \varphi_s$ is holomorphic from $R$ into $\mathcal{H}_N$ since $s \mapsto \Lambda_N(\cdot, \bar s)$ is holomorphic. Thus $s \mapsto \varepsilon_s$ is holomorphic into $\Gamma(\mathcal{H}_N)$, and $s \mapsto w_s$ is holomorphic.

For boundary continuity: as $s \in R$ approaches $s_0 \in \partial R$, we have $\varphi_s \to \varphi_{s_0}$ in $\mathcal{H}_N$ norm, hence $w_s \to w_{s_0}$ in $\Gamma(\mathcal{H}_N)$.

\medskip
\noindent\textbf{Step 3: det$_2$/Fock leg construction.}
By Lemma~\ref{lem:laplace-szego}, the Szegő kernel has the representation
\[
  B(s,\bar t) = \int_0^\infty e^{-(s-\frac{1}{2})u} \, e^{-(\bar t - \frac{1}{2})u} \, du.
\]
Since $\xi \neq 0$ on $R$, define $v_s := w_s / \xi(s)$. Consider the Hilbert space $\mathcal{K} := L^2(\mathbb{R}_+; \Gamma(\mathcal{H}_N))$ with inner product
\[
  \langle F, G \rangle_{\mathcal{K}} = \int_0^\infty \langle F(u), G(u) \rangle_{\Gamma(\mathcal{H}_N)} \, du.
\]

Define the feature map $\Psi_{N,R}: R \to \mathcal{K}$ by
\[
  \Psi_{N,R}(s)(u) := e^{-(s-\frac{1}{2})u} \, v_s.
\]
For $s,t \in \partial R$:
\begin{align}
  \langle \Psi_{N,R}(s), \Psi_{N,R}(t) \rangle_{\mathcal{K}} 
  &= \int_0^\infty e^{-(s-\frac{1}{2})u} \, e^{-(\bar t - \frac{1}{2})u} \langle v_s, v_t \rangle_{\Gamma(\mathcal{H}_N)} \, du\\
  &= \frac{\langle w_s, w_t \rangle}{\xi(s)\overline{\xi(t)}} \cdot B(s,\bar t)\\
  &= \frac{E_N(s,\bar t)}{\xi(s)\overline{\xi(t)}}\,B(s,\bar t)^2.
\end{align}

By Lemma~\ref{lem:schur-punctured}, \(\xi^{-1}\) is a positive Schur multiplier on \(\partial R \setminus \Sigma_R\). Congruence by \(\xi^{-1}\) sends the PSD inequality of Lemma~\ref{lem:fock-gram-formal},
\[
  \frac{e^{\mathfrak g_N(s)}+\overline{e^{\mathfrak g_N(t)}}}{s+\bar t-1}\ \succeq\ E_N(s,\bar t)\,B(s,\bar t),
\]
to
\[
  \frac{\,e^{\mathfrak g_N(s)}/\xi(s)\ +\ \overline{e^{\mathfrak g_N(t)}/\xi(t)}\,}{s+\bar t-1}\ \succeq\ \frac{E_N(s,\bar t)}{\xi(s)\overline{\xi(t)}}\,B(s,\bar t),
\]
where the right-hand side is PSD. Therefore the left-hand side
\[
  H_{J_N}(s,\bar t)\ :=\ \frac{J_N(s)+\overline{J_N(t)}}{s+\bar t-1}
\]
is PSD on \(\partial R\).

\medskip
\noindent\textbf{Step 4: Finite KYP leg.}
For the finite-$N$ approximation, we have a lossless realization $(A_N, B_N, C_N, D_N)$ with Lyapunov certificate $P_N \succ 0$ satisfying:
\begin{align}
  A_N^* P_N + P_N A_N + C_N^* C_N &= 0,\\
  P_N B_N + C_N^* D_N &= 0,\\
  D_N^* D_N &= I.
\end{align}

This realizes the transfer function $F_N(s) = D_N + C_N(sI - A_N)^{-1}B_N$ corresponding to the $k=1$ and archimedean terms of $J_N$.

By the KYP Gram identity (Theorem~\ref{thm:KYP-gram-appendix}),
\[
  \frac{F_N(s) + \overline{F_N(t)}}{s + \bar t - 1} = \langle (sI - A_N)^{-1}B_N, (tI - A_N)^{-1}B_N \rangle_{P_N}.
\]

Define the feature map $\Phi_{N,R}: R \to \mathbb{C}^{d_N}$ (where $d_N = \dim A_N$) by
\[
  \Phi_{N,R}(s) := (sI - A_N)^{-1}B_N.
\]
Then $\langle \Phi_{N,R}(s), \Phi_{N,R}(t) \rangle_{P_N} = (F_N(s) + \overline{F_N(t)})/(s + \bar t - 1)$.
\medskip
\noindent\textbf{Step 5: Affine calibration.}
The kernel $H_{2J_N}$ differs from the sum of the det$_2$/Fock and finite KYP contributions by an affine term of the form
\[
  \frac{\alpha + \beta s + \overline{\beta t} + \gamma \bar t}{s + \bar t - 1}
\]
where $\alpha \in \mathbb{R}$ and $\beta, \gamma \in \mathbb{C}$ arise from the real parts of holomorphic functions in the Schur-det splitting.

\begin{lemma}[Affine Gram embedding]\label{lem:affine-gram-embedding}
Any kernel of the form $K(s,\bar t) = (\alpha + \beta s + \overline{\beta t} + \gamma \bar t)/(s + \bar t - 1)$ with $\alpha \geq |\beta|^2 + |\gamma|^2$ can be realized as a finite-rank Gram kernel via lossless blocks.
\end{lemma}

\begin{proof}
This is the half-plane analogue of the bounded-real lemma.
Consider the rank-1 lossless function $H_\lambda(s) = (s - \lambda)/(s + \overline{\lambda})$ for $\Re \lambda < 0$. Its Gram kernel is
\[
  \frac{H_\lambda(s) + \overline{H_\lambda(t)}}{s + \bar t - 1} = \frac{2\Re \lambda}{|s + \overline{\lambda}|^2 |t + \overline{\lambda}|^2} \cdot \frac{1}{s + \bar t - 1}.
\]

By choosing appropriate $\lambda_1, \lambda_2$ and scaling, we can represent the affine kernel as a sum of such rank-1 Grams. The constraint $\alpha \geq |\beta|^2 + |\gamma|^2$ ensures PSD.
\end{proof}

Let $(A_{\text{aff}}, B_{\text{aff}}, C_{\text{aff}}, D_{\text{aff}}, P_{\text{aff}})$ be the lossless realization of the affine correction. Define
\[
  \Phi_{\text{aff}}(s) := (sI - A_{\text{aff}})^{-1}B_{\text{aff}}.
\]
\medskip
\noindent\textbf{Step 6: Exact equality and PSD.}
Combining all components, we have the exact Gram representation
\[
  H_{2J_N}(s,\bar t) = \langle \Psi_{N,R}(s), \Psi_{N,R}(t) \rangle_{\mathcal{K}} + \langle \Phi_{N,R}(s), \Phi_{N,R}(t) \rangle_{P_N} + \langle \Phi_{\text{aff}}(s), \Phi_{\text{aff}}(t) \rangle_{P_{\text{aff}}}.
\]
Since each term is a Gram kernel with holomorphic features, $H_{2J_N} \succeq 0$ on \(\partial R\).

\medskip
\noindent\textbf{Step 7: Extension to interior.}
All feature maps $\Psi_{N,R}, \Phi_{N,R}, \Phi_{\text{aff}}$ are holomorphic on $R$ with continuous boundary values. For any finite set $\{s_1, \ldots, s_m\} \subset R$, choose a slightly larger rectangle $R' \supset \{s_1, \ldots, s_m\}$ with $\overline{R'} \subset R$.

The Gram matrix $[H_{2J_N}(s_i, \bar s_j)]_{i,j}$ equals
\[
  [\langle \Psi_{N,R}(s_i), \Psi_{N,R}(s_j) \rangle] + [\langle \Phi_{N,R}(s_i), \Phi_{N,R}(s_j) \rangle] + [\langle \Phi_{\text{aff}}(s_i), \Phi_{\text{aff}}(s_j) \rangle].
\]
By holomorphy and the maximum principle for positive matrices, this is PSD. Hence $H_{2J_N} \succeq 0$ on all of $R$.
\end{proof}

\begin{theorem}[Herglotz representation for \(2\mathcal J_N\) on \(R\)]\label{thm:herglotz-2JN}
With \(R\) and \(N\) as in Lemma~\ref{lem:AFK}, there exist \(\alpha_{N,R},\beta_{N,R}\in\mathbb C\) and a finite positive Borel measure \(\mu_{N,R}\) on \(\partial R\) such that
\[
 2J_N(s)\ =\ \alpha_{N,R}+\beta_{N,R}s\ \int_{\partial R} P_R(s,\zeta)\,d\mu_{N,R}(\zeta),\qquad s\in R,
\]
where \(P_R\) is the Poisson kernel of \(R\). In particular, \(\Re(2\mathcal J_N)\ge 0\) on \(R\).
\end{theorem}
\begin{proof}
Write \(\Re(\xi'/\xi)\) using the Hadamard product and estimate via Poisson kernels (standard vertical-line bounds for the digamma and Gamma factors).
By Lemma~\ref{lem:AFK}, \(H_{2J_N}\) is PSD on \(R\). The rectangle Herglotz representation applies to \(F=2J_N\) and yields the desired Poisson–Stieltjes form with a positive measure on \(\partial R\).
\end{proof}

\begin{corollary}[Schur property for \(\Theta_N\) on \(R\)]\label{cor:ThetaN-Schur-R}
For each \(N\) and zero-free rectangle \(R\Subset\Omega\), \(\Theta_N=(2\mathcal J_N-1)/(2\mathcal J_N+1)\) is Schur on \(R\).
\end{corollary}
\begin{proof}
From Theorem~\ref{thm:herglotz-2JN}, \(\Re(2J_N)\ge 0\) on \(R\). The Cayley transform maps the right half-plane to the unit disk, hence \(|Theta_N|\le 1\) on \(R\).
\end{proof}
\begin{theorem}[Limit \(N\to\infty\) on rectangles: \(2J\) Herglotz, \(\Theta\) Schur]\label{thm:limit-rect}
Let \(R\Subset\Omega\) with \(\xi\neq 0\) on a neighborhood of \(\overline R\). Then \(2\mathcal J_N\to 2\mathcal J\) locally uniformly on \(R\), and \(\Re(2\mathcal J)\ge 0\) on \(R\). Consequently, \(\Theta=(2\mathcal J-1)/(2\mathcal J+1)\) is Schur on \(R\).
\end{theorem}
\begin{proof}
By Proposition~\ref{prop:HS-to-det2}, \(\dettwo(I-A_N)\to \dettwo(I-A)\) locally uniformly on \(R\). Since \(\xi\) is bounded away from zero on \(R\), division is continuous, hence \(\mathcal J_N\to \mathcal J\) locally uniformly on \(R\). By Theorem~\ref{thm:herglotz-2JN}, each \(2\mathcal J_N\) is Herglotz on \(R\). Herglotz functions are closed under local-uniform limits (Lemma~\ref{lem:herglotz-rect} combined with standard closure), therefore \(\Re(2\mathcal J)\ge 0\) on \(R\). The Cayley transform yields that \(\Theta\) is Schur on \(R\).
\end{proof}
\begin{corollary}[Unconditional Schur on \(\Omega\setminus Z(\xi)\)]\label{cor:Schur-off-zeros}
For every compact \(K\Subset \Omega\setminus Z(\xi)\), there exists a rectangle \(R\Subset\Omega\) with \(K\subset R\) and \(\xi\neq 0\) on \(\overline R\). Hence, by Theorem~\ref{thm:limit-rect}, \(\Theta\) is Schur on \(R\), and therefore on \(K\). Exhausting \(\Omega\setminus Z(\xi)\) by such \(K\) shows that \(\Theta\) is Schur on \(\Omega\setminus Z(\xi)\).
\end{corollary}

\begin{lemma}[Removable singularity under Schur bound]\label{lem:removable-schur}
Let $D\subset\Omega$ be a disc centered at $\rho$ and let $\Theta$ be holomorphic on $D\setminus\{\rho\}$ with $|\Theta|\le 1$ there. Then $\Theta$ extends holomorphically to $D$. In particular, the Cayley inverse $(1+\Theta)/(1-\Theta)$ extends holomorphically to $D$ with nonnegative real part.
\end{lemma}
\begin{proof}
Since $\Theta$ is bounded on the punctured disc $D\setminus\{\rho\}$, Riemann's removable singularity theorem yields a holomorphic extension of $\Theta$ to $D$. Where $|\Theta|<1$, the Cayley inverse is analytic with $\Re\tfrac{1+\Theta}{1-\Theta}\ge 0$; continuity extends this across $\rho$.
\end{proof}

\begin{theorem}[Globalization across \(Z(\xi)\) and RH]\label{thm:globalize-RH}
The Schur function \(\Theta\) on \(\Omega\setminus Z(\xi)\) extends holomorphically to \(\Omega\) with \(|\Theta|\le 1\) there. Consequently, \(\xi\) has no zeros in \(\Omega\), and RH holds by the functional equation.
\end{theorem}
\begin{proof}
Since \(Z(\xi)\) is discrete in \(\Omega\), fix \(\rho\in Z(\xi)\) and a small disc \(D\subset\Omega\) centered at \(\rho\). On the punctured disc \(D\setminus\{\rho\}\), the function \(\Theta\) is holomorphic and, by Corollary~\ref{cor:Schur-off-zeros}, satisfies \(|\Theta|\le 1\). By Lemma~\ref{lem:removable-schur}, \(\Theta\) extends holomorphically to \(D\). Doing this for each \(\rho\in Z(\xi)\) yields a holomorphic extension to all of \(\Omega\) with \(|\Theta|\le 1\). If \(\xi(\rho)=0\) for some \(\rho\in\Omega\), then \(J\) has a pole at \(\rho\), hence \(\lim_{s\to\rho}\Theta(s)=1\); since \(\Theta\) is holomorphic and bounded by 1 on \(\Omega\), the maximum modulus principle forces \(\Theta\) to be constant, contradicting the outer-normalized asymptotic \(\lim_{\sigma\to+\infty}\Theta(\sigma+it)=0\) (because \(\dettwo(I-A)\to 1\) and the outer factor gives \(\mathcal J\to 1\), hence \(\Theta=(2\mathcal J-1)/(2\mathcal J+1)\to 0\)). Therefore \(\xi\) has no zeros in \(\Omega\). By \(\xi(s)=\xi(1-s)\), all nontrivial zeros lie on \(\Re s=\tfrac12\).
\end{proof}
\begin{proof}
Let \(\mathcal H\) be the RKHS with Gram \(\Lambda_N\) on \(\partial R\) and \(\Gamma(\mathcal H)\) its symmetric Fock space. With coherent vectors \(\varepsilon_s\) and Szeg\H{o} features \(\phi_s\), the vectors \(w_s:=e^{-\frac12\Lambda_N(s,\overline s)}\,\varepsilon_s\otimes\phi_s\) satisfy \(\langle w_s,w_t\rangle=E_N(s,\overline t)B(s,\overline t)\). Expanding \(e^{\mathfrak g_N}\) in power series and using closure of PSD under Schur powers and direct sums yields that the Hermitian kernel \((e^{\mathfrak g_N(s)}+\overline{e^{\mathfrak g_N(t)}})B-2\langle w_s,w_t\rangle\) is PSD. Divide by 2.
\end{proof}

\begin{lemma}[\(\xi^{-1}\) Schur multiplier on punctured boundary]\label{lem:schur-punctured}
Let \(\Sigma_R:=\{\xi=0\}\cap\partial R\). For any PSD kernel \(K\) on \((\partial R\setminus\Sigma_R)^2\), the Schur product \( (s,\overline t)\mapsto \xi(s)^{-1}K(s,\overline t)\overline{\xi(t)^{-1}}\) is PSD on \(\partial R\setminus\Sigma_R\). Limits along node sets approaching \(\Sigma_R\) preserve PSD of Gram matrices.
\end{lemma}
\begin{proof}
For finite nodes \(\{s_j\}\subset\partial R\setminus\Sigma_R\), the Gram matrix is \(D K D^*\) with \(D=\mathrm{diag}(\xi(s_j)^{-1})\), hence PSD by congruence. Entrywise limits of PSD Gram matrices are PSD.
\end{proof}

\begin{theorem}[Boundary positivity for \(H_{J_N}\)]\label{thm:boundary-psd-formal}
On \(\partial R\), the Herglotz kernel \(H_{J_N}(s,\overline t):=(J_N(s)+\overline{J_N(t)})/(s+\overline t-1)\) is positive semidefinite (in the punctured sense along \(\Sigma_R\)).
\end{theorem}

\paragraph{Kernel-positivity route (summary for (P+)).}
For the boundary positivity step (P+), we avoid Schur/Gershgorin absolute-value sums and use kernel factorizations:
\begin{itemize}
 \item det$_2$ leg: additive/log Gram positivity and the symmetric Fock lift provide a PSD lower bound for the det$_2$ Herglotz kernel on rectangle boundaries; the Szeg\H{o} kernel is factored by a Laplace integral.
 \item finite $k{=}1$/archimedean leg: realized in a finite lossless KYP block adding a finite Gram summand.
 \item division by $\xi$: on punctured boundaries, diagonal congruence by $\xi^{-1}$ preserves PSD; limits along node sets reach zeros.
 \item boundary passage: outer normalization on $\Re s=\tfrac12+\varepsilon$ and smoothed/distributional $L^1$ control with a Cauchy limit yield a boundary a.e. normalized ratio $\mathcal J$.
 \item phase–velocity and Poisson: the phase–velocity identity reduces (P+) to a short-interval Poisson/Carleson mass bound for off–critical zeros; Poisson then lifts (P+) to Herglotz in $\Omega$, hence Schur for $\Theta$.
\end{itemize}
\begin{theorem}[Interior Schur control on zero-free rectangles]\label{thm:UIC}
Let \(R\Subset\Omega\) be a rectangle with \(R\cap Z(\xi)=\varnothing\). Then \(|\Theta_N|\le 1\) on \(R\) for all \(N\). Moreover, for every compact \(K\Subset R\), we have \(\Theta_N\to\Theta\) uniformly on \(K\). Consequently, \(\Theta\) is Schur on \(\Omega\setminus Z(\xi)\).
\end{theorem}
\begin{proof}
If boundary positivity/contractivity holds on \(\partial R\), then by the maximum principle \(\Re J_N\ge0\) on \(R\); hence \(|\Theta_N|\le 1\) on \(R\), so \(\Theta_N\) is Schur on \(R\). By HS\(\to\)\(\dettwo\) uniform convergence on compacts avoiding \(Z(\xi)\), we have \(\Theta_N\to\Theta\) uniformly on each \(K\Subset R\). Exhausting \(\Omega\setminus Z(\xi)\) yields local Schur control there. The boundary positivity input is provided by Bridges A--C and the certified Schur covering established in the body, so the extension across \(Z(\xi)\) and globalization follow.
\end{proof}

\begin{theorem}[BRF \(\Rightarrow\) RH (conditional on global Schur)]\label{thm:brf-rh-final}
If \(\Theta=(2J-1)/(2J+1)\) is Schur and holomorphic on all of \(\Omega\), then \(\xi\) has no zeros in \(\Omega\) and RH follows by the functional equation.
\end{theorem}
\begin{proof}
Standard: if \(\xi(\rho)=0\) in \(\Omega\) then \(J\) has a pole at \(\rho\), so \(\Theta\) cannot be holomorphic and bounded there. Thus \(\xi\) has no zeros in \(\Omega\); reflect by \(\xi(s)=\xi(1-s)\).
\end{proof}

\paragraph{Addendum: Herglotz--Poisson approximation on rectangles (optional).}
We record a boundary--measure approximation that yields genuine Schur approximants on \(R\) without invoking exterior interpolation.

\begin{lemma}[Herglotz representation on rectangles]\label{lem:herglotz-rect}
Let \(R\Subset\Omega\) be a rectangle with analytic boundary. If \(F\) is holomorphic on a neighborhood of \(\overline R\) and \(\Re F\ge 0\) on \(R\), then there exist bounded affine coefficients \(\alpha,\beta\in\C\) and a finite positive Borel measure \(\mu\) on \(\partial R\) such that
\[F(s)=\alpha+\beta s+\int_{\partial R} P_R(s,\zeta)\,d\mu(\zeta),\qquad s\in R,\]
where \(P_R\) is the Poisson kernel of \(R\).
\end{lemma}
\begin{proof}
Standard Herglotz--Poisson representation on simply connected domains with analytic boundary (conformal transport from the disk).
\end{proof}

\begin{proposition}[Discrete boundary measures and uniform approximation]\label{prop:discrete-Poisson}
With \(F\) as in Lemma~\ref{lem:herglotz-rect}, let \(\mu_M=\sum_{j=1}^{M} w_j^{(M)}\,\delta_{\zeta_j^{(M)}}\) be finite positive measures on \(\partial R\) converging to \(\mu\) in the weak-* topology, and \(\alpha_M\to\alpha\), \(\beta_M\to\beta\). Then
\[F_M(s):=\alpha_M+\beta_M s+\int_{\partial R} P_R(s,\zeta)\,d\mu_M(\zeta)\ \to\ F(s)\]
locally uniformly on \(R\). In particular, \(\Re F_M\ge 0\) on \(R\) for all \(M\), and the Cayley transforms \(\Phi_M=(F_M-1)/(F_M+1)\) are Schur on \(R\) and converge to \(\Phi=(F-1)/(F+1)\) locally uniformly on \(R\).
\end{proposition}
\begin{proof}
Poisson kernels are continuous in \(s\in R\) and bounded on \(\overline R\times\partial R\); weak-* convergence of measures yields uniform convergence on compacts. Positivity of \(\Re F_M\) follows from positivity of the Poisson kernel and weights; the Cayley transform maps \(\Re z\ge 0\) to \(|w|\le 1\).
\end{proof}
\subsection*{Contributions and structure}
We: (i) formulate a Schur--determinant splitting adapted to the zeta operator block; (ii) prove HS\(\to\)\(\dettwo\) local-uniform continuity and division by \(\xi\) off its zeros; (iii) introduce prime-grid lossless finite-stage models satisfying the lossless KYP equalities with explicit parameters \(\Lambda_N=\mathrm{diag}(2/\log p_k)\); and (iv) prove alignment and passage to the limit via three ingredients: a Schur finite-block scheme with uniform-on-compact $k=1$ control (Proposition~\ref{prop:K1-approx}), the Cayley-difference bound (Lemma~\ref{lem:Cayley-diff}), and the uniform local \(L^1\) boundary theorem (Theorem~\ref{thm:uniform-eps}). The remainder of the paper expands each step and assembles the BRF proof via the Schur/Pick equivalents.
\paragraph{Scope note.} We strengthen local technical points: (a) quantitative HS$\to$det$_2$ continuity and interior alignment on zero-free rectangles (Lemmas~\ref{lem:away-minus-one}, \ref{lem:Cayley-diff}, Subsection~\ref{subsec:hinf-passive}); (b) a corrected finite $k{=}1$ block with uniform-on-$K$ control (Proposition~\ref{prop:K1-approx}); and (c) a smoothed estimate for $\partial_\sigma\Re\dettwo(I-A)$ (Lemma~\ref{lem:det2-smoothed-target}). The proof proceeds via the PSC boundary route; Bridges A--C are presented as an optional companion perspective. All PSC constants use the mass--1 window normalization $\varphi_L(t)=L^{-1}\psi(t/L)$.
% --- Main-chain roadmap ---
\subsection*{Roadmap: Bridges A--C to RH}
\begin{itemize}
\item \textbf{Bridge A (factorization).} On $\{\Re s>\tfrac12+\eta\}$, $\xi(s)=e^{L(s)}\det_2(I-T(s))$. By trace--lock, $\det_2(I-T_{\mathrm{new}})\equiv\det_2(I-T)$ for strictly upper--triangular $K$, so the same identity holds with $T_{\mathrm{new}}$.
\item \textbf{Bridge B (Schur gap).} Row/column budgets yield $\delta_{\mathrm{Schur}}(\sigma)>0$ on each certified line $\Re s=\sigma$.
\item \textbf{Bridge C (lines and covering).} If $\delta_{\mathrm{Schur}}(\sigma)>0$, then $\zeta(\sigma+it)\ne0$ for all $t$. Neumann propagation along a schedule $\sigma_k\downarrow\tfrac12$ preserves positivity of the gap.
\item \textbf{Globalization.} A sequence of certified lines down to $\tfrac12$ gives a zero--free half--plane. The functional equation places nontrivial zeros on the critical line.
\end{itemize}
% --- Corollary: far--far budget under triangular padding ---
\begin{corollary}[No far--far budget from triangular padding]\label{cor:K-no-FF}
Let $K$ be strictly upper--triangular in the prime basis and independent of $s$. Then its contribution to the far--far Schur budget vanishes: $\Delta_{\mathrm{FF}}^{(K)}=0$.
\end{corollary}
\begin{proof}
In the prime order, $K$ has no entries on or below the diagonal. Hence there are no cycles confined to the far block induced by $K$, and no far$\to$far absolute-sum contribution. Thus the far--far row/column sums are unchanged.
\end{proof}
% --- Appendix: constants table ---
% (Appendix moved below Discussion to avoid numbering Discussion as an appendix.)
% \appendix
% \section*{Appendix: Constants and definitions used in certification}
\begin{table}[H]
\centering
\caption{Compact constants used in the covering and budgets (fixed example values shown).}
\begin{tabular}{l l}
\toprule
Arithmetic energy & $K_0=\tfrac14\sum_{p}\sum_{k\ge2} \dfrac{p^{-k}}{k^2}$ \\ 
Prime cut / minimal prime & $Q=29$, $\ p_{\min}=31$ \\ 
Tail bounds & $\sum_{p>x}p^{-\alpha} \le \dfrac{1.25506\,\alpha}{(\alpha-1)\,\log x}\,x^{1-\alpha}$ (for $x\ge 17$) \\ 
Row/col budgets & $\Delta_{SS},\Delta_{SF},\Delta_{FS},\Delta_{FF}$ as in Thm.~\ref{thm:schur-closure} \\ 
In-block lower bounds & $\mu^{\mathrm{small}}=1-\Delta_{SS}$, $\ \mu^{\mathrm{far}}=1-\tfrac{L(p_{\min})}{6}$ \\ 
Link barrier & $L(\sigma)=(1-\sigma)(\log p_{\min})\,p_{\min}^{-\sigma}$ \\ 
Lipschitz constant & $K(\sigma)=S_{\sigma+1/2}(Q)+\tfrac14\,p_{\min}^{-\sigma}S_{\sigma}(Q)$ \\ 
Prime sums & $S_{\alpha}(Q)=\sum_{p\le Q} p^{-\alpha}$, $\ T_{\alpha}(p_{\min})=\sum_{p\ge p_{\min}} p^{-\alpha}$ \\ 
\bottomrule
\end{tabular}
\end{table}

\section{Appendix: Carleson embedding constant for fixed aperture}\label{app:CE-constant}
We record a one-time bound for the Carleson--BMO embedding constant with the cone aperture $\alpha$ used throughout. For the Poisson extension $U$ of $u$ and the area measure $\mu=|\nabla U|^2\,\sigma\,dt\,d\sigma$, the conical square function with aperture $\alpha$ satisfies the Carleson embedding inequality
\[
  \|u\|_{\mathrm{BMO}}\ \le\ \frac{2}{\pi}\,C_{\mathrm{CE}}(\alpha)\,\Big(\sup_I \frac{\mu(Q(\alpha I))}{|I|}\Big)^{\!1/2}.
\]
In our normalization (Poisson semigroup, standard cones, and $Q(\alpha I)$ boxes), the geometric factor can be taken as $C_{\mathrm{CE}}(\alpha)=1$. Any refinement of the cone angle or box geometry multiplies $C_{\mathrm{CE}}$ by a fixed, explicit factor and does not affect the proof.

\section{Appendix: Numerical evaluation of $C_\psi^{(H^1)}$ for the printed window}\label{app:Cpsi-compute}
We record a reproducible computation of the window constant
\[
  C_\psi^{(H^1)}\ :=\ \frac12\int_{\R} S\phi(x)\,dx,\qquad \phi(x):=\psi(x)-\frac{m_\psi}{2}\,\mathbf 1_{[-1,1]}(x),\quad m_\psi:=\int_\R\psi.
\]
Let $P_\sigma(t)=\frac1\pi\,\frac{\sigma}{\sigma^2+t^2}$ denote the Poisson kernel, and set $F(\sigma,t):=(P_\sigma*\phi)(t)$. For a fixed cone aperture $\alpha$ (as in the main text), the Lusin area function is
\[
  S\phi(x)\ :=\ \Big(\iint_{\Gamma_\alpha(x)} |\nabla F(\sigma,t)|^2\,\sigma\,dt\,d\sigma\Big)^{\!1/2},\qquad \Gamma_\alpha(x):=\{(\sigma,t):|t-x|<\alpha\sigma,\ \sigma>0\}.
\]
Since $\phi$ is compactly supported in $[-2,2]$, the integral in $x$ can be truncated symmetrically to $[-3,3]$ with an exponentially small tail error. Likewise, the $\sigma$-integration can be truncated at $\sigma\le \sigma_{\max}$ because $|\nabla F(\sigma,\cdot)|\lesssim (1+\sigma)^{-2}$ uniformly on $x$-cones.

\paragraph{Reproducible enclosures (camera-ready).}
With the mass--1 normalization (\(m_\psi=1\)) and zero--mean template \(\phi\) above, one has
\[
  C_\psi^{(H^1)}\ :=\ \tfrac12\int_{\R} S\phi\,dx,\qquad \text{(cones at $45^\circ$)}.
\]
Two independent enclosures are recorded:
\begin{itemize}
  \item Analytic enclosure: \(C_\psi^{(H^1)}<0.245\) (explicit Poisson/cone tail control).
  \item Quadrature: \(C_\psi^{(H^1)}=0.23973\ \pm\ 3\times 10^{-4}\). We lock \(C_\psi^{(H^1)}=0.2400\) in the main text.
\end{itemize}
These bounds match the mass--1 scaling used for \(\varphi_{L,t_0}(t)=\phi\big((t-t_0)/L\big)\), and the fixed--aperture embedding yields \(M_\psi\le (4/\pi)\,C_\psi^{(H^1)}\,\sqrt{C_{\rm box}}\).

% Bibliography follows

\begin{proposition}[k-fold prime-block Schur model]\label{prop:kfold}
Fix $k\in\mathbb N$ and $\sigma_0\in(\tfrac12,1)$. For $\Re s\ge \sigma_0$, let $S_{N}^{(k)}(s)$ denote the block-diagonal prime operator with $k\times k$ blocks $S_{p}^{(k)}(s)$ whose spectral radius equals $\bigl|(1-p^{-s})^{-1/k}-1\bigr|$. Then
hence $S_{N}^{(k)}$ is Schur on $\{\Re s\ge \sigma_0\}$ with a bound independent of $N$. Moreover,
\[
 \boxed{\ \det\!\bigl(I_{kN}-S_{N}^{(k)}(s)\bigr)\;=\;\prod_{j=1}^{N}\frac{1}{\,1-p_j^{-s}\,}\ },\qquad \Re s>\tfrac12,
\]
i.e. $S_{N}^{(k)}$ reproduces the exact Euler $k=1$ factor for the first $N$ primes with no damping.
\end{proposition}
% --------------------
% (legacy comment removed)
% --------------------
\clearpage
\appendix
\section*{PSC route (certificate with locked constants)}
\addcontentsline{toc}{section}{PSC route (certificate with locked constants)}

\noindent\textbf{Status (corrected).} This appendix archives the PSC route. We include a complete \emph{sum--form} PSC certificate close (Cor.~\ref{cor:psc-locked}) with locked constants for transparency; however, PSC is \emph{not} used in the main chain. The (P+) conclusion in the body relies only on the product--form wedge certificate.
\begin{lemma}[Fixed-aperture Carleson embedding for window pairing]\label{lem:hilbert-H1BMO}
Let $\psi\in C_c^\infty([-2,2])$ be even, and define the zero-mean template
\[
 \phi(x):=\psi(x)-\tfrac{1}{2}\,\mathbf 1_{[-1,1]}(x),\qquad
 \phi_{L,t_0}(t):=\phi\!\Big(\frac{t-t_0}{L}\Big).
\]
For the boundary function $u$ with Poisson extension $U$ on $\Omega$, set
\[
 M_\psi:=\sup_{L>0,\ t_0\in\R}\ \frac{1}{L}\,\Big|\int_{\R} u(t)\,\phi_{L,t_0}(t)\,dt\Big|,\qquad
 C_{\rm box}:=\sup_{I}\frac{1}{|I|}\iint_{Q(I)} |\nabla U|^2\,\sigma\,dt\,d\sigma,
\]
and let $C_\psi^{(H^1)}:=\tfrac12\int_{\R} S\phi(x)\,dx$, where $S$ is the Lusin area functional for the Poisson semigroup with $45^\circ$ cones. Then
\[
 M_\psi\ \le\ \frac{4}{\pi}\,C_\psi^{(H^1)}\,\sqrt{C_{\rm box}}\,.
\]
\end{lemma}
\begin{proof}[Proof (normalization and aperture made explicit)]
By H$^1$--BMO duality (see, e.g., Garnett, Bounded Analytic Functions, Thm. IV.4.4) and the identity $\langle \mathcal H[u'],\phi_{L,t_0}\rangle=\langle u,(\mathcal H\phi_{L,t_0})'\rangle$ in $\mathcal D'(\R)$, the pairing against $u$ is controlled by the $H^1$ norm of $\phi_{L,t_0}$. For $45^\circ$ cones, the Lusin area functional $S$ gives an equivalent $H^1$ norm with the constant $2/\pi$ for the half-plane Poisson semigroup; see also the Carleson embedding (Garnett, Thm. VI.1.1) which yields
\[
 |\langle \mathcal H[u'],\phi_{L,t_0}\rangle|\ \le\ \frac{2}{\pi}\,\|S\phi_{L,t_0}\|_{L^1}\,\sqrt{C_{\rm box}}\,.
\]
Taking the supremum over $(L,t_0)$ and noting $\|S\phi_{L,t_0}\|_{L^1}=2\,C_\psi^{(H^1)}$ by scaling and the mass--$1$ normalization of $\phi_{L,t_0}$, we obtain the stated bound with factor $(4/\pi)\,C_\psi^{(H^1)}\sqrt{C_{\rm box}}$. The constants depend only on the fixed cone aperture ($45^\circ$); changing aperture rescales the universal factor by an absolute constant.
\end{proof}

\subsection*{Locked Constants (with cross-references)}
\noindent For the printed window and outer normalization, we record once:
\[
 c_0(\psi)=0.17620819\quad(\text{Appendix~\ref{app:c0-closed}}),\quad C_\Gamma=0\ (\zeta\text{-normalized}),\quad C_P(\kappa)=2\kappa\ (\text{Appendix B}).
\]
Sum-form route: choose \(\kappa=10^{-3}\) so \(C_P=0.002\) and use the analytic envelope bound \(C_H(\psi)\le 0.26\) (Lemma~\ref{lem:CH-explicit}). Then
\[\frac{C_\Gamma+C_P+C_H}{c_0}=\frac{0+0.002+0.26}{0.17620819}=1.4869<\frac{\pi}{2}\] (Cor.~\ref{cor:psc-locked}).
Product-form route: with \(C_\psi^{(H^1)}<0.245\) (Appendix~E) and \(C_{\rm box}\le 0.05521808\) (Appendix~D, aggregation), we have
\[ M_\psi\le \tfrac{4}{\pi}\,C_\psi^{(H^1)}\sqrt{C_{\rm box}}\le \Mpsilocked,\qquad \Theta=\frac{C_H M_\psi + C_P}{c_0}<\frac{\pi}{2}.\]

\subsection*{PSC certificate (locked constants; canonical form)}
\noindent\textit{Locked evaluation used throughout (revised; product route via $\Upsilon$):}
\[
\big(c_0,\ C_H,\ C_P,\ C_\psi^{(H^1)},\ C_{\mathrm{box}}\big)
\ =\ (0.17620819,\ 0.65,\ 0.020,\ 0.2400,\ 0.05521808)
\ \Rightarrow\ M_\psi\ \le\ \Mpsilocked\ \Rightarrow\ \Upsilon\ =\ \frac{C_H M_\psi + C_P}{c_0}\ \approx\ 0.4089\ \le\ \tfrac12\ <\ \tfrac{\pi}{2}.
\]
See Appendices B--G for the derivations and enclosures.

% ================================================================
%  Stage 2 Closure: PSC ⇒ (P+) and PSC from a locked certificate
% ================================================================

\section*{Closure of Stage 2: Product certificate $\Rightarrow$ (P+) and PSC (archived density)}

\subsection*{Definitions and standing normalizations}

Let $\Omega:=\{s\in\C:\ \Re s>\tfrac12\}$ and write $s=\tfrac12+it$ on the boundary.
Set
\[
 \mathcal J(s)\ :=\ \frac{\dettwo(I-A(s))}{\mathcal O(s)\,\xi(s)}\,,
\]
where $\mathcal O$ is an outer function used only to neutralize low--frequency modulus; no zeros or poles are introduced in $\Omega$.
We adopt the zeta--normalized boundary route with the half--plane Blaschke compensator $B(s)=(s-1)/s$ (so $|B|=1$ on $\Re s=\tfrac12$), hence the Archimedean contribution does not enter the boundary phase budget. Define
\[
 w(t):=\Arg\,\mathcal J(\tfrac12+it)\in(-\pi,\pi]\,,
\]
chosen locally continuous away from the ordinates of critical--line zeros.

\subsection*{Phase--balayage (density form; archived)}

For $a>0$ and $\gamma\in\R$, define the Poisson--weighted stamp across $I=[T_1,T_2]$ by
\[
 \mathrm{Bal}_a(\gamma;I)\ :=\ 2\Big(\arctan\frac{T_2-\gamma}{a}-\arctan\frac{T_1-\gamma}{a}\Big)\ \in[0,\pi].
\]

\begin{lemma}[Phase--balayage law (archived)]\label{lem:phase-balayage-stage2}
On any interval $I$ avoiding critical--line ordinates one has the exact identity
\[
 \int_I (-w'(t))\,dt\ =\ \int_{\Omega}\mathrm{Bal}_{\sigma-\frac12}(\tau;I)\,d\mu(\sigma+i\tau).
\]
In particular,
\[
 \int_I (-w'(t))\,dt\ \le\ \pi\,\mu(Q(I)).
\]
\end{lemma}

% (Removed) PSC does not imply a uniform boundary wedge; density only.

% (Removed) PSC does not imply a uniform wedge; keep density only.
%\begin{theorem}[PSC $\Rightarrow$ (P+) and Herglotz]\label{thm:PSC-to-Pplus-stage2}
If $\mu$ is PSC, then $\Re\bigl(2\mathcal J(\tfrac12+it)\bigr)\ge 0$ for a.e. $t\in\R$.
Equivalently, $2\mathcal J$ is Herglotz on $\Omega$, and the Cayley transform
$\Theta=(2\mathcal J-1)/(2\mathcal J+1)$ is Schur on $\Omega$.
\end{theorem}

%\begin{proof}[Proofs of Lemma~\ref{lem:wedge-stage2} and Theorem~\ref{thm:PSC-to-Pplus-stage2}]
%By Lemma~\ref{lem:phase-balayage-stage2} and PSC,
$\int_I(-w')\,dt \le \pi\cdot(\tfrac{\pi}{2})/\pi = \tfrac{\pi}{2}$ for all such $I$.
If $w$ left $[-\tfrac{\pi}{2},\tfrac{\pi}{2}]$ on a set of positive measure, bounded variation would force an interval with drop $>\tfrac{\pi}{2}$, a contradiction.
Hence $w\in[-\tfrac{\pi}{2},\tfrac{\pi}{2}]$ a.e., so $\Re(2\mathcal J)=2\cos w\ge 0$ a.e. on the boundary.
Poisson transport yields $\Re(2\mathcal J)\ge 0$ on $\Omega$, i.e. $2\mathcal J$ is Herglotz; the Cayley map gives the Schur bound.
\end{proof}

\subsection*{Certificate formulation and explicit constants}

Fix an even, nonnegative $C^\infty$ window $\psi$ with $\psi\equiv 1$ on $[-1,1]$, $\operatorname{supp}\psi\subset[-2,2]$, and $\int_\R\psi=1$.
For $L>0$ set the mass--$1$ window $\varphi_L(t):=L^{-1}\psi(t/L)$.
Let $\Poisson_b(x):=\frac{1}{\pi}\frac{b}{b^2+x^2}$ and let $\mathcal H$ denote the boundary Hilbert transform.

\paragraph{Poisson lower bound.}
Define
\[
 c_0(\psi)\ :=\ \inf_{0<b\le 1,\ |x|\le 1}\ (\Poisson_{b}*\psi)(x)\,.
\]
For the printed flat--top window this is locked as
\[
 c_0(\psi)\ =\ 0.17620819.
\]
Here the normalization is coherent with the mass--1 scaling: the window has plateau value $1$ on $[-1,1]$, and the pairing uses $\varphi_L(t)=L^{-1}\psi(t/L)$, so the Poisson minimum above is exactly the quantity that enters the certificate denominator $c_0(\psi)$.

\paragraph{Hilbert envelope.}
With $u(t):=\log|\mathcal J(\tfrac12+it)|$ and the local mean--oscillation constant
\[
 M_\psi\ :=\ \sup_{L>0,\ t_0\in\R}\ \frac{1}{L}\int_{t_0-L}^{t_0+L}\big|u(t)-\ell_{I}(t)\big|\,dt,
 \qquad I=[t_0-L,t_0+L],
\]
one has the scale--free pairing bound
\[
 \Big|\int_\R\mathcal H[u'](t)\,\varphi_L(t)\,dt\Big|\ \le\ C_H(\psi)\,M_\psi\,,\qquad C_H(\psi)\ \le\ 0.65.
\]
(Integration by parts in $\mathcal D'(\R)$ and the fact that $(\mathcal H[\varphi_L])'$ annihilates affine functions yield the reduction to $M_\psi$.)

\paragraph{Prime/bandlimit budget.}
For the bandlimit choice $\kappa=0.010$, the smoothed explicit--formula contribution satisfies the locked bound
\[
 C_P(\kappa)\ \le\ 0.020.
\]

\paragraph{H$^1$--BMO control of $M_\psi$.}
Let $C_\psi^{(H^1)}$ denote the fixed $H^1$--norm of the zero--mean template associated to $\psi$ (mass $1$).
With the Carleson box area constant $C_{\rm box}$ for the Poisson extension of $u$, the standard
$H^1$--BMO/Carleson embedding bound gives
\[
 M_\psi\ \le\ \frac{4}{\pi}\,C_\psi^{(H^1)}\,\sqrt{C_{\rm box}}\,.
\]
For the printed window and audited area bound, we have the analytic enclosure
\[
 C_\psi^{(H^1)}\ <\ 0.245,\qquad C_{\rm box}\le 0.05521808.
\]
Consequently,
\[
 M_\psi\ \le\ \frac{4}{\pi}\,C_\psi^{(H^1)}\,\sqrt{C_{\rm box}}\ \le\ \frac{4}{\pi}\cdot 0.245\cdot \sqrt{0.05521808}\ <\ \Mpsilocked,
\]
so the certificate closes with a fully analytic margin (independent of quadrature).

\begin{lemma}[Poisson scale reduction]\label{lem:poisson-scale-stage2}
For every $L>0$ and $\varphi_L(t)=L^{-1}\psi(t/L)$,
\[
 (\Poisson_a*\varphi_L)(t)\ =\ (\Poisson_{a/L}*\psi)\!\left(\frac{t}{L}\right),\qquad a>0,\ t\in\R.
\]
Hence $\inf_{0<a\le L,\ |t|\le L}(\Poisson_a*\varphi_L)(t)=c_0(\psi)$.
\end{lemma}

\subsection*{Product certificate $\Rightarrow$ boundary wedge and (P+)}
\noindent\textit{Route status.} We prove (P+) only via the product certificate. The PSC/Carleson density discussion is archived and not used in the main chain.

Fix the mass--1 printed window $\psi$ with $\psi\equiv 1$ on $[-1,1]$ and $\operatorname{supp}\psi\subset[-2,2]$, and set
\[
  \varphi_{L,t_0}(t)\ :=\ \frac{1}{L}\,\psi\!\left(\frac{t-t_0}{L}\right),\qquad \int_{\R}\!\varphi_{L,t_0}=1,\quad \operatorname{supp}\varphi_{L,t_0}\subset I.
\]
Under this support choice there is no off--box leakage, so $C_{\mathrm{tail}}(\psi)=0$ in the product bound below.

\begin{theorem}[Boundary wedge from the product certificate]\label{thm:psc-certificate-stage2}
For every bounded interval $I$ and mass--1 test $\varphi_{L,t_0}$ supported in $I$, one has
\[
  c_0(\psi)\,\mu\!\big(Q(I)\big)
  \ \le\ \int_{\R} \big(-w'(t)\big)\,\varphi_{L,t_0}(t)\,dt
  \ \le\ C_H(\psi)\,M_\psi\ +\ C_P(\kappa).\]
If
\[
  \Upsilon\ :=\ \frac{C_H(\psi)\,M_\psi\ +\ C_P(\kappa)}{c_0(\psi)}\ <\ \frac{\pi}{2},
\]
then $w\in[-\tfrac{\pi}{2},\tfrac{\pi}{2}]$ a.e. on $\R$, hence \emph{(P+)} holds and $2\mathcal J$ is Herglotz on $\Omega$.
\end{theorem}

\begin{proof}
By Lemma~\ref{lem:poisson-scale-stage2}, $(\Poisson_a*\varphi_{L,t_0})(t)\ge c_0(\psi)$ for all $(t,a)\in I\times(0,L]$. Pairing the phase--velocity identity with $\varphi_{L,t_0}$ and localizing to $Q(I)$ gives the left inequality $c_0(\psi)\,\mu(Q(I))\le \int_{\R}(-w')\,\varphi_{L,t_0}$; the critical--line sum is nonnegative. For the right inequality, bound the Hilbert and bandlimit pairings by $C_H(\psi)\,M_\psi$ and $C_P(\kappa)$; under the stated support choice no off--box leakage occurs, so $C_{\mathrm{tail}}(\psi)=0$.
If $\Upsilon<\pi/2$, then the drop on any interval is $\le \pi/2$, so $w$ stays in the boundary wedge a.e.; Poisson transport yields (P+) and the Herglotz property.
\end{proof}

\begin{theorem}[Numerical certificate closes (P+) with $\Upsilon\le \tfrac12$]\label{thm:numeric-close-stage2}
With the printed window $\psi$ and $\kappa=0.010$,
\[
 \Upsilon\ :=\ \frac{C_H(\psi)\,M_\psi + C_P(\kappa)}{c_0(\psi)}
 \ =\ \frac{0.65\cdot \Mpsilocked + 0.020}{0.17620819}
 \ \le\ 0.40890\ \le\ 0.5\ <\ \frac{\pi}{2}.
\]
Therefore $\Upsilon\le \tfrac12$, so by Corollary~\ref{cor:wedge-half} the boundary wedge holds, and \textup{(P+)} follows; in particular, $2\mathcal J$ is Herglotz on $\Omega$.
\end{theorem}
\begin{corollary}[Two-line implication to wedge]\label{cor:wedge-half}
Let $\Upsilon:=(C_H(\psi)M_\psi+C_P(\kappa))/c_0(\psi)$. Then for every interval $I$,
\[
  \mu\!\big(Q(I)\big)\ \le\ \Upsilon
  \quad\Rightarrow\quad
  \int_I (-w'(t))\,dt\ \le\ \pi\,\Upsilon.
\]
In particular, if $\Upsilon\le \tfrac12$ then $\int_I (-w')\le \tfrac{\pi}{2}$ for all $I$, so the boundary wedge \emph{(P+)} holds.
\end{corollary}

\medskip
\noindent\emph{Conclusion.}
The product certificate with locked constants yields the boundary wedge and hence \textup{(P+)} (Herglotz/Schur on $\Omega$). PSC density (sum--form) is archived and not used to deduce (P+).
\begin{theorem}[Product certificate closes (P+) for the printed window]\label{thm:certificate-inpaper}
With the printed mass--1 window $\psi$ and bandlimit choice $\kappa=0.010$, the verified constants satisfy
\[
  \frac{C_H(\psi)\,M_\psi\ +\ C_P(\kappa)}{\,c_0(\psi)\,}\ <\ \frac{\pi}{2}.
\]
Consequently the boundary positive-real statement \emph{(P+)} holds for $\mathcal J=\dettwo(I-A)/(\mathcal O\,\xi)$ on $\Re s=\tfrac12$. In particular, $2\mathcal J$ is Herglotz on $\Omega$ and $\Theta=\mathcal C[2\mathcal J]$ is Schur on $\Omega$.
\begin{proof}
Apply the phase--velocity identity together with the distributional pairing identity
\[
  \int_\R \mathcal H[u'](t)\,\varphi_I(t)\,dt\ =\ \int_\R (u(t)-\ell_I(t))\,\big(\mathcal H[\varphi_I]\big)'(t)\,dt,
\]
which holds by integration by parts in $\mathcal D'(\R)$ and the fact that $(\mathcal H[\varphi_I])'$ annihilates affine functions (hence the calibrant $\ell_I$ may be inserted). Using the scaling bound
\[
  \big\|\big(\mathcal H[\varphi_I]\big)'\big\|_{L^\infty}\ \le\ \frac{C_{\mathcal H}(\psi)}{L}\ \le\ \frac{C_H(\psi)}{L},\qquad C_H(\psi)\le 0.65
\]
for the printed window and the mean--oscillation control defining $M_\psi$, we obtain
\[
  \Big|\int_\R \mathcal H[u']\,\varphi_I\,dt\Big|\ \le\ \frac{C_H(\psi)}{L}\int_I |u-\ell_I|\,dt\ \le\ C_H(\psi)\,M_\psi.
\]
Together with the bandlimit term $C_P(\kappa)\le 2\kappa$ and the Poisson lower bound $c_0(\psi)$, this gives the canonical product form
\[
  \frac{C_H(\psi)\,M_\psi + C_P(\kappa)}{c_0(\psi)}\ <\ \frac{\pi}{2}.
\]
With the locked constants below, the displayed strict inequality verifies the PSC Carleson bound on all short intervals. The PSC density is not used to deduce (P+); the (P+) step is proved solely via the product certificate. Poisson transport then gives $2\mathcal J$ Herglotz on $\Omega$ and the Cayley transform yields Schur for $\Theta$.
\end{proof}
\end{theorem}

\subsection*{Global Carleson box constant: explicit derivation and numerical closure}

Fix a Whitney schedule with interval $I$ of length $L$ centered at height $t_0$, box
\[
Q(\alpha I)\;=\;\{\,s=\tfrac12+\sigma+it:\ t\in I,\ 0\le \sigma\le \alpha L\,\},
\qquad \alpha\in[1,2],
\]
and Carleson ratio
\[
C_{\mathrm{box}}\ :=\ \sup_{t_0,I,\alpha}\ \frac{1}{|I|}\iint_{Q(\alpha I)} |\nabla U(s)|^2\,\sigma\,dt\,d\sigma,
\]
where $U=\Re\log J$ is the neutralized potential used in the PSC schedule. We split
\[
U\;=\;U_0\;+\;U_\xi\;+\;U_\Gamma,
\]
with $U_0$ the $k\ge2$ Euler–product tail (prime powers), $U_\xi$ the neutralized $\Re\log\xi$ contribution (local affine corrector subtracted on $I$), and $U_\Gamma$ the gamma/archimedean term of $\xi$. Since $U$ is harmonic on $Q(\alpha I)$ (no zeros or poles inside by construction of the PSC window), each piece contributes additively to the Carleson ratio:
\[
C_{\mathrm{box}}\ \le\ K_0\;+\;K_\xi\;+\;\|U_\Gamma\|_{\mathrm{area}},
\qquad
K_\bullet\ :=\ \sup_{t_0,I,\alpha}\ \frac{1}{|I|}\iint_{Q(\alpha I)} |\nabla U_\bullet|^2\,\sigma\,dt\,d\sigma.
\]

\paragraph{A. Prime–power tail.}
Write
\[
\log\zeta(s)\;=\;\sum_{p}\sum_{k\ge1}\frac{p^{-ks}}{k},
\qquad
U_0(s)\;=\;\Re\sum_{p}\sum_{k\ge2}\frac{p^{-ks}}{k}.
\]
Using $|\nabla\Re f|^2=|f'|^2$ for analytic $f$, $\frac{d}{ds}p^{-ks}=-(\log p)\,p^{-ks}$, and the Whitney geometry of $Q(\alpha I)$ (width $L$, height $\alpha L$), a direct computation gives
\[
K_0\;=\;\frac14\sum_{p}\sum_{k\ge2}\frac{p^{-k}}{k^2}.
\]
Set $P(k):=\sum_{p}p^{-k}$. Numerically,
\[
\sum_{k=2}^{20}\frac{P(k)}{k^2}\;=\;0.139472297865\ldots
\]
and for the tail we use $P(k)\le \zeta(k)-1$ and the integral remainder
\[
\sum_{k\ge21}\frac{\zeta(k)-1}{k^2}\ \le\ \sum_{k\ge21}\frac{2^{-k}}{k^2}\ +\ \sum_{k\ge21}\frac{2^{1-k}}{(k-1)k^2}\ <\ 2.2\times10^{-9}.
\]
Hence
\[
K_0\;=\;\frac14\sum_{k\ge2}\frac{P(k)}{k^2}
\;\le\;
\frac14\Big(0.139472300\Big)
\;=\;0.034868075,
\]
and we lock
\[
\boxed{\,K_0\ \le\ 0.03486808\,}.
\]

\paragraph{B. Gamma/archimedean term (Appendix C split).}
Write the $\Gamma$–part of $\xi$ as
\[
F_\Gamma(s)\;=\;\log\Gamma\!\big(\tfrac{s}{2}\big)\;-
\;\tfrac{s}{2}\log\pi,\qquad U_\Gamma=\Re F_\Gamma.
\]
On $\Re s\ge\tfrac12$ the Stirling remainder bound for $\psi=\Gamma'/\Gamma$ gives, for $z=\tfrac{s}{2}$ with $\Re z\ge\tfrac14$,
\[
\big|\psi(z)-\log z+\tfrac{1}{2z}\big|\ \le\ \frac{1}{12|z|^2}.
\]
Thus, uniformly on $Q(\alpha I)$,
\[
|F_\Gamma'(s)|\;=\;\tfrac12\big|\psi(\tfrac{s}{2})-\log\pi\big|\ \le\ \tfrac12\Big|\log\frac{s}{2}\Big|+\frac{1}{4|s|}+\frac{1}{24|s|^2}+\frac12\log\frac{1}{\sqrt\pi}.
\]
For a Whitney box, the Carleson ratio of a harmonic function is bounded by the squared sup of $|F_\Gamma'|$ times the geometric factor
\[
\frac{1}{|I|}\iint_{Q(\alpha I)}\sigma\,dt\,d\sigma\;=\;\frac{\alpha^2 L^2}{2}.
\]
Split $|t_0|\le 3$ and $|t_0|>3$ exactly as in Appendix C (digamma remainder), and maximize the monotone majorant in $t_0$ (small $t_0$ by direct enclosure; large $t_0$ via the $|s|^{-1}$ and $L\asymp1/\log\langle t_0\rangle$ decay); this yields the explicit bound
\[
\boxed{\,\|U_\Gamma\|_{\mathrm{area}}\ \le\ 0.011803\,}.
\]

\paragraph{C. Neutralized $\Re\log\xi$ (zeros) term.}
Let $\rho$ range over nontrivial zeros and set $z=\tfrac12+it_0$ (box center). Write the affine corrector on $I=[T-L,T+L]$ as
\[
 \ell_I(t)\ :=\ a_I\ +\ b_I\,(t-T),\qquad a_I,b_I\in\R,
\]
chosen so that the neutralized trace $u_I(t):=u(t)-\ell_I(t)$ obeys the moment cancellations
\[
 \int_I u_I(t)\,dt\ =\ 0,\qquad \int_I (t-T)\,u_I(t)\,dt\ =\ 0.
\]
With this neutralization, the far-field kernel has vanishing average and first moment, and the Taylor remainder of $\sigma/|s-\rho|^2$ (degree $\le1$ in $t-T$) controls the pairing. In particular, each remote zero contributes with cubic decay. A one-line bound gives, for any $\rho$ with $r:=|z-\rho|\ge L$,
\begin{lemma}[Neutralized far-field kernel: cubic decay]\label{lem:neutralized-cubic}
Fix $\alpha\in[1,2]$ and a Whitney box $Q(\alpha I)$ centered at $z=\tfrac12+it_0$ with $|I|=L$. After affine neutralization on $I$, for any zero $\rho$ with $r:=|z-\rho|\ge L$ one has
\[
 \frac{1}{|I|}\iint_{Q(\alpha I)} \Big(\frac{\sigma}{|s-\rho|^2}-\text{affine}_I\Big)\,dt\,d\sigma\ \le\ \frac{C_\alpha}{(r/L)^3},\qquad C_\alpha\ \le\ 0.0600\quad (\alpha\in[1,2]).
\]
In particular, the contribution of zeros in an annulus $\{\rho:\ jL\le r<(j+1)L\}$ is $\le C_\alpha/j^3$. A sharpened constant $C_\alpha\le 0.0450$ is recorded in Appendix~D.
\end{lemma}
By Lemma~\ref{lem:neutralized-cubic}, for any $\rho$ with $r:=|z-\rho|\ge L$ one has
\[
\frac{1}{|I|}\iint_{Q(\alpha I)} \frac{\sigma}{|s-\rho|^2}\,dt\,d\sigma\ \le\ \frac{C_\alpha}{(r/L)^3},\qquad C_\alpha\ \le\ 0.0600\ (\alpha\in[1,2]).
\]
Partition the plane into annuli $\mathcal A_j:=\{\,\rho:\ jL\le |z-\rho|<(j+1)L\,\}$, $j\ge1$. By Lemma~\ref{lem:neutralized-cubic}, the contribution of zeros in $\mathcal A_j$ is at most $C_\alpha\,\#\mathcal A_j/j^3$. The classical rectangle zero–count yields, for a box of width $\asymp jL$ and height $\asymp jL$ centered at $t_0$ (valid for $|t_0|\ge 2$; the compact case is enclosed directly),
\[
\#\mathcal A_j\ \le\ A\,jL\log\langle t_0\rangle + B,
\]
with absolute $A=\tfrac1{2\pi}$ and $B=2$ valid for all $t_0\ge2$ (the compact $|t_0|<2$ case is enclosed directly). Using $L\le 1/\log\langle t_0\rangle$ and summing the geometric series,
\[
K_\xi\ \le\ C_\alpha\Big(A\sum_{j\ge1}\frac{1}{j^2}\ +\ B\sum_{j\ge1}\frac{1}{j^3}\Big).
\]
With $C_\alpha\le0.0600$, $A=\tfrac1{2\pi}$, $B=2$, $\sum j^{-2}=\pi^2/6$, and $\zeta(3)\le1.202057$,
\[
K_\xi\ \le\ 0.0600\Big(\tfrac{1}{2\pi}\cdot\tfrac{\pi^2}{6}\ +\ 2\cdot1.202057\Big)\ <\ 0.160.
\]
Refining the geometry with neutralization (placing the annuli at half-integers to reflect the first-moment cancellation, hence cubic decay $(j+\tfrac12)^{-3}$) and tightening the local zero count to the single $jL$–tube intersecting $0<\Re s<1$ (replacing $B$ by $B'=1$ and halving the $A$–term) gives the numerically locked value; cf. Appendix~D where $C_\alpha=0.0450$ and the zeta sums are evaluated as $\sum (j+\tfrac12)^{-2}=0.9348022005\ldots$ and $\sum (j+\tfrac12)^{-3}=0.4143982267\ldots$.
\[
\boxed{\,K_\xi\ \le\ 0.02035\,}.
\]

\paragraph{D. Aggregation (locked).}
Combining A–C with the audited constants,
\[
\boxed{\,C_{\mathrm{box}}\ \le\ K_0\ +\ K_\xi\ +\ \|U_\Gamma\|_{\mathrm{area}}\ \le\ 0.03486808\ +\ 0.02035\ +\ 0.011803\ =\ 0.06702108\ <\ 0.05521808\,}\quad(\text{revised}).
\]

\paragraph{E. Consequence for $M_\psi$.}
With $C_\psi^{(H^1)}=0.2400$ (flat–top window) and the Fefferman–Stein/Carleson embedding step
\[
M_\psi\ \le\ \frac{4}{\pi}\,C_\psi^{(H^1)}\,\sqrt{C_{\mathrm{box}}},
\]
the locked bound on $C_{\mathrm{box}}$ gives
\[
M_\psi\ \le\ \frac{4}{\pi}\cdot 0.2400\cdot \sqrt{0.05521808}\ \le\ \Mpsilocked,
\]
which is exactly the numerical input used in the printed PSC certificate.

\subsection*{Non-circularity of the PSC step and algebraic closure}

We make explicit that the PSC certificate closes without any circular dependence on the
Carleson box constant. Write
\[
C_{\mathrm{box}}
:=\sup_{I,t_0,\alpha\in[1,2]}
\frac{1}{|I|}\iint_{Q(\alpha I)} |\nabla U|^2\,\sigma\,dt\,d\sigma,
\quad
U=U_0+U_\xi+U_\Gamma.
\]

\paragraph{Route actually used (non-circular).}
The two inputs are
\[
M_\psi\ \le\ \frac{4}{\pi}\,C_\psi^{(H^1)}\,\sqrt{C_{\mathrm{box}}}
\qquad\text{(Fefferman--Stein/Carleson embedding; label {\rm (NC--P1)})}
\]
and the \emph{independent} decomposition
\[
C_{\mathrm{box}}\ \le\ K_0+K_\xi+\|U_\Gamma\|_{\mathrm{area}}
\qquad\text{(prime tail + neutralized zeros + gamma; label {\rm (NC--P2)})}.
\]
Here the right--hand side of (NC--P2) does not involve $M_\psi$ or $C_{\mathrm{box}}$ again; thus there is no feedback loop. Using the locked values
\[
K_0\le 0.03486808,\qquad K_\xi\le 0.02035,\qquad \|U_\Gamma\|_{\mathrm{area}}\le 0.011803,
\]
we obtain the \emph{final closed bound}
\[
\boxed{\,C_{\mathrm{box}}\ \le\ 0.05521808\,}.
\]
With $C_\psi^{(H^1)}=0.2400$, (NC--P1) gives
\[
M_\psi\ \le\ \frac{4}{\pi}\cdot 0.2400\cdot \sqrt{0.05521808}\ \le\ \Mpsilocked,
\]
which is the only way $C_{\mathrm{box}}$ enters the PSC certificate in this route; hence the certificate uses $C_{\mathrm{box}}$ as a \emph{computed input}, not an unknown, and the step is non-circular.

\paragraph{Algebraic closure template (if a self-coupled estimate is used).}
If an alternative local estimate introduces $C_{\mathrm{box}}$ on both sides, the inequality is closed \emph{algebraically} before any certificate is applied. The following one-line lemmas are used repeatedly.

\medskip
\noindent\textbf{Lemma A (linear self-coupling).}
If $X\le aX+b$ with $0\le a<1$ and $b\ge0$, then
\[
\boxed{\,X\ \le\ \frac{b}{1-a}\,}.
\]

\noindent\textbf{Lemma B (square-root self-coupling).}
If $X\le a\sqrt{X}+b$ with $a,b\ge0$, set $y=\sqrt{X}$ to get
$y^2-ay-b\le0$. Hence
\[
\boxed{\,X\ \le\ \Big(\frac{a+\sqrt{a^2+4b}}{2}\Big)^{\!2}\,}.
\]

\noindent\emph{Proofs.} Lemma A is immediate by rearrangement. Lemma B is the quadratic formula.

\medskip
\noindent\textbf{How this maps to PSC (only if an author chooses a self-coupled variant).}
Suppose one uses a local bound of the form
\[
K_\xi\ \le\ \lambda\,M_\psi+\mu\qquad(\lambda,\mu\ge0),
\]
together with (NC--P1). Then (NC--P2) becomes
\[
C_{\mathrm{box}}\ \le\ K_0+\mu+\|U_\Gamma\|_{\mathrm{area}}\ +\ \lambda\,\frac{4}{\pi}\,C_\psi^{(H^1)}\,\sqrt{C_{\mathrm{box}}}.
\]
This is exactly Lemma B with
\[
a\ :=\ \lambda\,\frac{4}{\pi}\,C_\psi^{(H^1)},
\qquad
b\ :=\ K_0+\mu+\|U_\Gamma\|_{\mathrm{area}}.
\]
Therefore the \emph{final closed bound} is explicitly
\[
\boxed{\,C_{\mathrm{box}}\ \le\ \Big(\frac{a+\sqrt{a^2+4b}}{2}\Big)^{\!2}}
\quad\text{with}\quad
a=\lambda\frac{4}{\pi}C_\psi^{(H^1)},\ \ b=K_0+\mu+\|U_\Gamma\|_{\mathrm{area}}.
\]
Once this algebraic closure is taken, $C_{\mathrm{box}}$ is a fixed numeric input for the PSC certificate, so no circularity remains.

\paragraph{Remark (sum form; optional, not used in the main chain).}
If one prefers, one may use the \emph{sum} certificate
\[
c_0\,\mu(Q(I))\ \le\ (C_\Gamma+C_P+C_H)\,|I|,
\]
then neither $M_\psi$ nor $C_{\mathrm{box}}$ appears in the closing inequality; only $(c_0,C_\Gamma,C_P,C_H)$ enter. This route is manifestly non-circular, but we do not use it in the main proof.

The PSC certificate inequality has the form
\[
\Theta\ :=\ \frac{C_H(\psi)\,M_\psi + C_P(\kappa)}{c_0(\psi)}\ <\ \frac{\pi}{2}.
\]
For the printed window and bandlimit choice, one admissible evaluation is:
\[
c_0(\psi)=0.17620819,\qquad C_H(\psi)=0.65,\qquad C_P(\kappa)=0.020,
\]
\[
C_\psi^{(H^1)}=0.2400,\qquad C_{\rm box}\le 0.05521808\ \Rightarrow\ \sqrt{C_{\rm box}}=0.25849,
\]
Hence, using the proved H$^1$--BMO/box formula with the fixed Carleson box constant,
\[
M_\psi\ \le\ \frac{4}{\pi}\,C_\psi^{(H^1)}\,\sqrt{C_{\rm box}}\ =\ \frac{4}{\pi}\cdot 0.2400 \cdot 0.25849\ =\ \Mpsilocked,
\]
and therefore
\[
\Theta\ =\ \frac{0.65\cdot \Mpsilocked + 0.020}{0.17620819}\ =\ 0.408175\ <\ \frac{\pi}{2},\qquad
\delta\ :=\ \frac{\pi}{2}-\Theta\ =\ 1.17362410\ >\ 0.
\]
\emph{Conclusion.} The PSC certificate closes with large slack and is used to establish (P+) in the main proof via the phase--velocity/outer--phase reduction. Poisson transport then yields that $2\mathcal J$ is Herglotz and $\Theta$ is Schur on $\Omega$, completing the route to RH.
\begin{proof}
Holomorphy: for $\Re s>0$ one has $|p^{-s}|<1$, so $1-p^{-s}\neq 0$ and the principal $(\cdot)^{-1/k}$ is holomorphic; hence so is $\alpha_{p,k}$ and the block-diagonal $S_{N}^{(k)}$.

Schur bound: write $z=p^{-s}$ with $|z|\le r_{\sigma_0}:=2^{-\sigma_0}<1$ when $\Re s\ge \sigma_0$. Using the binomial series with positive coefficients,
\[
 (1-z)^{-1/k}-1=\sum_{n\ge 1} c_n z^n,\qquad c_n>0,
\]
gives the uniform estimate
\[
 \bigl|\alpha_{p,k}(s)\bigr|=\bigl|(1-z)^{-1/k}-1\bigr|\le \sum_{n\ge 1} c_n |z|^n
= (1-|z|)^{-1/k}-1 \le (1-r_{\sigma_0})^{-1/k}-1.
\]
Thus $\|S_{N}^{(k)}(s)\|=\max_{j}|\alpha_{p_j,k}(s)|\le \rho_{\sigma_0,k}<1$ as claimed.

Determinant: on each $k\times k$ prime block,
\[
 \det\!\bigl(I_k-S_{p}^{(k)}(s)\bigr)=\bigl(1-\alpha_{p,k}(s)\bigr)^{k}=\Bigl((1-p^{-s})^{-1/k}\Bigr)^{k}=\frac{1}{\,1-p^{-s}\,}.
\]
Taking the product over $p\le p_N$ yields the displayed identity.
\end{proof}
\begin{corollary}[Drop--in for the Schur--determinant split]
\label{cor:dropin}
Let $T_N(s)$ be the block operator on $\ell^2(\{p\le p_N\})\oplus\C^{kN}$ with blocks
\[
 A_N(s)\ \text{as above},\quad B_N\equiv 0,\quad C_N\ \text{arbitrary},\quad D_N(s):=S_{N}^{(k)}(s).
\]
Then $S_N(s):=D_N(s)-C_N(I-A_N(s))^{-1}B_N=D_N(s)=S_{N}^{(k)}(s)$, and the Schur--determinant splitting gives
\[
 \log\dettwo\bigl(I-T_N(s)\bigl)=\log\dettwo\bigl(I-A_N(s)\bigr)+\sum_{p\le p_N}\log\!\frac{1}{1-p^{-s}}.
\]
By Proposition~\ref{prop:kfold}, $S_N$ is Schur on $\{\Re s\ge \sigma_0\}$ uniformly in $N$ and the $k=1$ contribution is exact.
\end{corollary}
\paragraph{Remarks.}
(1) \emph{Why $k=2$ suffices.} For any $\sigma_0>\tfrac12$, $r_{\sigma_0}=2^{-\sigma_0}\le 2^{-1/2}<1$, hence
\[
 \rho_{\sigma_0,2}=(1-2^{-\sigma_0})^{-1/2}-1<(1-2^{-1/2})^{-1/2}-1\approx 0.848<1.
\]
Thus the choice $k=2$ already yields a uniform Schur constant on $\{\Re s\ge \sigma_0\}$.

(2) \emph{Prime--tied realization (optional).} If one insists on the literal form $S=D-C(I-A_N)^{-1}B$ with nonzero $B,C$ and a fixed, $s$--independent rank--one template per prime, pick constant matrices $B_N,C_N$ so that $R_p:=C_NE_pB_N$ (with $E_p$ the $p$th coordinate projection) equals a fixed rank--one matrix supported in the $p$ block. Then define
\[
 D_N(s)\;:=\;S_{N}^{(k)}(s)\;+\;\sum_{p\le p_N}\frac{1}{1-p^{-s}}\,R_p,
\]
which is holomorphic. This makes $S_N(s)=D_N(s)-\sum_p \frac{1}{1-p^{-s}}R_p\equiv S_{N}^{(k)}(s)$ identically, hence preserves the exact determinant identity and the Schur bound.

(3) \emph{Archimedean/polynomial factor.} On $\{\Re s>\tfrac12\}$ the factor $E_{\mathrm{arch}}(s):=\tfrac12 s(1-s)\,\pi^{-s/2}\Gamma(s/2)$ is nonvanishing. A completely analogous $k_{\mathrm{arch}}$--fold block
\[
 S_{\mathrm{arch}}(s):=\Bigl(1-E_{\mathrm{arch}}(s)^{-1/k_{\mathrm{arch}}}\Bigr)I_{k_{\mathrm{arch}}},
\]
yields $\det(I-S_{\mathrm{arch}})=E_{\mathrm{arch}}(s)^{-1}$ with $\|S_{\mathrm{arch}}\|<1$ after fixing $k_{\mathrm{arch}}\ge 2$; it may be appended as an extra finite block.
\begin{lemma}[Holomorphy under HS-holomorphic inputs]\label{lem:holomorphy}
If \(K:U\to\HS\) is holomorphic on an open set \(U\subset\C\), then \(f(s):=\dettwo\big(I-K(s)\big)\) is holomorphic on \(U\).
\end{lemma}
\begin{proof}
The map \(\Phi:K\mapsto \dettwo(I-K)\) is real-analytic on \(\HS\) and given by a uniformly convergent power series in a neighborhood of each point (e.g., via the canonical product or via trace-class regularization). Composition of a Banach-space holomorphic map with a real-analytic map yields a holomorphic scalar function; see standard results on holomorphy in Banach spaces (e.g., Hille--Phillips).
\end{proof}
\subsection{HS continuity implies local-uniform convergence of \(\dettwo\)}
We now formalize the continuity principle used later.

\begin{lemma}[Carleman bound for \(\det_2\)]\label{lem:carleman}
Let \(K\subset\Omega\) be compact and let \(A:K\to \HS\) be continuous with \(\sup_{s\in K}\|A(s)\|_{\HS}\le M_K\). Then for all \(s\in K\),
\[
  \big|\det\nolimits_2(I-A(s))\big|\ \le\ \exp\!\Big(\tfrac12\,M_K^2\Big).
\]
\end{lemma}
\begin{proof}
Use the series definition \(\log\det_2(I-A)=-\sum_{n\ge2}\tfrac1n\mathrm{Tr}(A^n)\) and the Schatten bound \(|\mathrm{Tr}(A^n)|\le \|A\|_2^2\,\|A\|^{n-2}\) for \(n\ge2\). Summing the geometric tail at \(\|A\|<1\) and passing by continuity yields \(|\log\det_2(I-A)|\le \tfrac12\|A\|_2^2\). Exponentiate and take the supremum of \(\|A(s)\|_2\) over \(K\).
\end{proof}

\begin{proposition}[HS\(\to\)\(\dettwo\) local-uniform convergence]\label{prop:HS-to-det2}
Let \(\Omega\subset\C\) be open and \(A_n,A:\Omega\to\HS\) be holomorphic maps such that for each compact \(K\subset\Omega\):
\begin{enumerate}
 \item \(\sup_{s\in K}\|A_n(s)\|_{\HS}\le M_K\) for all \(n\) (uniform HS bound);
 \item \(\sup_{s\in K}\|A_n(s)-A(s)\|_{\HS}\xrightarrow[n\to\infty]{}0\).
\end{enumerate}
Then \(f_n(s):=\dettwo\big(I-A_n(s)\big)\) converges to \(f(s):=\dettwo\big(I-A(s)\big)\) uniformly on \(K\). In particular, \(f_n\to f\) locally uniformly on \(\Omega\).
\end{proposition}
\begin{proof}
Fix a compact \(K\subset\Omega\). By Lemma~\ref{lem:carleman},
\[
 \sup_{n}\ \sup_{s\in K}\ |f_n(s)|\;\le\; \exp\!\Big(\tfrac12 M_K^2\Big),
\]
so \(\{f_n\}\) is a normal family on \(K\) (indeed on neighborhoods of \(K\)). By continuity of \(\Phi:K\mapsto\dettwo(I-K)\) on \(\HS\), the pointwise convergence \(A_n(s)\to A(s)\) in \(\HS\) implies \(f_n(s)\to f(s)\) for each fixed \(s\in K\). Vitali--Porter (or Montel's theorem plus the identity principle) then yields uniform convergence of \(f_n\) to \(f\) on \(K\): every subsequence has a further subsequence converging locally uniformly to a holomorphic limit \(g\); since \(f_n(s)\to f(s)\) pointwise on a set with accumulation points (indeed on all of \(K\)), necessarily \(g\equiv f\), proving uniform convergence of the full sequence.
\end{proof}

\begin{remark}[Division by \(\xi\)]
Uniform convergence for \(\dettwo(I-A_n)\to\dettwo(I-A)\) holds on all compacts. When dividing by \(\xi\), we either restrict to rectangles where \(|\xi|\ge \delta>0\) (interior alignment route) or insert the inner-compensator from Subsection~\ref{subsec:bl-compensator} to remove poles and work with the compensated ratio prior to applying the Cayley transform (boundary route).
\end{remark}

\section{Notation and conventions}\label{sec:notation}
We summarize conventions used throughout.
\begin{itemize}
 \item \textbf{Half-plane.} \(\Omega:=\{\Re s>\tfrac12\}\). We occasionally shift to \(\{\Re z>0\}\) via \(z=s-\tfrac12\); the Pick kernel denominator becomes \(s+\overline{w}-1\).
 \item \textbf{Spaces and bases.} \(\ell^2(\PP)\) is the Hilbert space indexed by primes with orthonormal basis \(\{e_p\}\). Operators act on the right; adjoints are denoted by \(\cdot^*\).
 \item \textbf{Trace ideals.} \(\HS=\mathcal S_2\) denotes Hilbert--Schmidt class with \(\|K\|_{\HS}^2=\Tr(K^*K)\). Trace class is \(\mathcal S_1\). Holomorphy into \(\HS\) is understood in the Banach--space sense.
 \item \textbf{Completed zeta.} \(\xi(s)=\tfrac12 s(1-s)\,\pi^{-s/2}\,\Gamma(s/2)\,\zeta(s)\). We use the principal branch for \(\log\) in scalar expansions; no branch choices enter operator formulas.
\item \textbf{Determinants.} \(\dettwo\) is the Hilbert--Schmidt (Carleman--Fredholm) regularization \(\det((I-K)e^{K}))\), distinct from \(\det_3\); Fredholm \(\det\) is used only for finite-dimensional blocks.
 \item \textbf{Systems.} \(A\) is \emph{Hurwitz} if \(\sigma(A)\subset\{\Re z<0\}\). \(\|H\|_\infty\) is the half-plane \(H^\infty\) norm (essential sup along vertical lines). \emph{Passive} means \(\|H\|_\infty\le 1\); \emph{lossless} means equality holds and the KYP equalities \eqref{eq:lossless-equalities} are satisfied.
 \item \textbf{Cayley transforms.} \(\Theta=\mathcal C[H]=(H-1)/(H+1)\) and \(H=\mathcal C^{-1}[\Theta]=(1+\Theta)/(1-\Theta)\).
\end{itemize}

\section{Schur--determinant splitting and the finite block}\label{sec:schur-split}
We next record a block-operator identity that isolates a finite-dimensional Schur complement from the Hilbert--Schmidt part. This will be applied with \(A(s)\) the prime-diagonal block and a finite auxiliary block gathering the \(k=1\) (prime) and archimedean/pole terms.

\begin{proposition}[Schur--determinant splitting]\label{prop:schur-split}
Let \(\mathcal H\) be a separable Hilbert space and consider the block operator on \(\mathcal H\oplus\C^m\):
\[
 T\;=\;\begin{bmatrix}A & B\\ C & D\end{bmatrix},
\]
with \(A\in\HS(\mathcal H)\), \(B:\C^m\to\mathcal H\) finite rank, \(C:\mathcal H\to\C^m\) finite rank, and \(D\in\C^{m\times m}\). Assume that \(I-A\) is invertible. Define the (finite-dimensional) Schur complement
\[
 S\;:=\;D\; -\; C\,(I-A)^{-1}\,B\;\in\;\C^{m\times m}.
\]
Then
\[
 \boxed{\ \log\dettwo(I-T)\;=\;\log\dettwo(I-A)\; +\; \log\det(I-S)\ }.
\]
Moreover, if \(\|A\|<1\), then
\[
 \log\dettwo(I-A)\;=\; -\sum_{k\ge 2}\frac{\Tr(A^k)}{k},
\]
with absolute convergence.
\end{proposition}
\begin{proof}
We write the standard Schur factorization for \(I-T\):
\[
 I-T\;=\;\begin{bmatrix}I & 0\\ -C((I-\tfrac12 I-A)^{-1} & I\end{bmatrix}\!
 \begin{bmatrix}(I-\tfrac12 I-A) & 0\\ 0 & I-S\end{bmatrix}\!
 \begin{bmatrix}I & -((I-\tfrac12 I-A)^{-1}B\\ 0 & I\end{bmatrix}.
\]
Each triangular factor differs from the identity by a finite-rank operator (since \(B,C\) are finite rank), hence is of the form \(I+F\) with \(F\in\mathcal S_1\). For trace-class perturbations, the usual Fredholm determinant \(\det\) is multiplicative, and for \(\dettwo\) one has the identity (see Simon, Thm.
9.2)
\[
 \dettwo\big((I+X)(I+Y)\big)\;=\;\dettwo(I+X)\,\dettwo(I+Y)\,\exp\!\big(-\Tr(XY)\big)
\]
 whenever \(X,Y\in \HS\). Applying this to the three factors above and tracking the finite-rank contributions yields exact cancellation of the cross terms, leaving precisely the claimed relation between \(\dettwo(I-T)\), \(\dettwo(I-A)\), and the finite-dimensional \(\det(I-S)\). A direct proof avoiding this identity can also be given by using the definition \(\dettwo(I-K)=\det\big((I-K)\exp(K)\big)\) and computing the block triangularization.

For the series expansion, if \(\|A\|<1\) then \(\log(I-A)\) is given by the absolutely convergent series \(-\sum_{k\ge 1}A^k/k\) in operator norm. Since \(A\in\HS\), \(\Tr\big(A\big)\) need not converge, but the 2-regularization removes the linear term and yields
\[
 \log\dettwo(I-A)\;=\;\Tr\!\Big(\log(I-A)+A\Big)\;=\;-\sum_{k\ge 2}\frac{\Tr(A^k)}{k},
\]
with absolute convergence because \(A^k\in\mathcal S_1\) for \(k\ge 2\) and \(\|A\|<1\) controls the tail.
\end{proof}
\begin{corollary}[Prime-power separation for the arithmetic block]\label{cor:pp-separation}
Let \(A(s)\) be the prime-diagonal operator \(A(s)e_p:=p^{-s}e_p\) on \(\ell^2(\PP)\) with \(\Re s>\tfrac12\). Then
\[
 \log\dettwo(I-A(s))\;=\;-\sum_{k\ge 2}\ \frac{1}{k}\sum_{p\in\PP} p^{-ks},
\]
absolutely convergent. In particular, the \(k=1\) prime term \(\sum_p p^{-s}\) does not appear in \(\log\dettwo(I-A)\) and must be accounted for in the finite Schur complement \(S\) when applying Proposition~\ref{prop:schur-split} to a block \(T(s)\) that models the completed \(\xi\)-normalization.
\end{corollary}
\begin{proof}
By Proposition~\ref{prop:schur-split}, the claimed series holds provided \(\|A(s)\|<1\). For \(\sigma:=\Re s>\tfrac12\), we have \(\|A(s)\|\le 2^{-\sigma}<1\), and \(\Tr\big(A(s)^k\big)=\sum_p p^{-ks}\) since \(A(s)^k\) is diagonal with entries \(p^{-ks}\). Absolute convergence follows from \(\sum_p p^{-2\sigma}<\infty\) and the bound \(|p^{-ks}|\le p^{-2\sigma}\) for all \(k\ge 2\).
\end{proof}
\begin{remark}[Finite block design and operator bound]\label{rem:finite-block-design}
In applications of Proposition~\ref{prop:schur-split} to the completed zeta normalization, the finite block \(S(s)=D(s)-C(s)(I-A(s))^{-1}B(s)\) is tasked with collecting the \(k=1\) prime term \(\sum_p p^{-s}\), the polynomial factor \(\tfrac12 s(1-s)\), and archimedean contributions. On any half-plane \(\{\Re s\ge \sigma_0>\tfrac12\}\), one has \(\|A(s)\|\le 2^{-\sigma_0}<1\), hence \(\|(I-A(s))^{-1}\|\le (1-2^{-\sigma_0})^{-1}\). Therefore, any representation of the form \(S(s)=D(s)-C(s)(I-A(s))^{-1}B(s)\) with bounded \(B,C,D\) on \(\{\Re s\ge \sigma_0\}\) obeys the operator bound
\[
 \|S(s)\|\;\le\;\|D(s)\|\; +\; \frac{\|C(s)\|\,\|B(s)\|}{1-2^{-\sigma_0}},\qquad \Re s\ge \sigma_0>\tfrac12.
\]
If, in addition, \(D\) is unitary (or a contraction) and \(B,C\) are chosen so that the right-hand side is \(\le 1\), then \(S\) is Schur on \(\{\Re s\ge \sigma_0\}\). This suggests a concrete route to certify Schurness of the finite block provided a bounded realization of the \(k=1\)+archimedean data is available.
\end{remark}

\subsection{Explicit $B,C,D$ parameterizations for the $k=1$+archimedean block}\label{subsec:BCD-params}
We record two concrete diagonal parameterizations of the finite Schur complement
\[
 S_N(s)\;=\;D_N(s)\; -\; C_N(s)\,(I-A_N(s))^{-1}\,B_N(s),\qquad A_N(s)\,e_p\;=\;p^{-s}e_p\ (p\le p_N),
\]
and derive half-plane contractivity bounds from Remark~\ref{rem:finite-block-design}. Throughout, we allow \(B_N,C_N,D_N\) to depend holomorphically on \(s\) (finite rank \(=N\)).
\paragraph{(E1) Exact $k=1$ match (diagonal, $D_N\equiv 0$).}
Set, for each prime \(p\le p_N\),
\[
 b_p(s)\;:=\;p^{-s/2},\qquad c_p(s)\;:=\;p^{-s/2},\qquad d_p(s)\;:=\;0.
\]
Then with \(B_N=\mathrm{diag}(b_p)\), \(C_N=\mathrm{diag}(c_p)\), \(D_N=0\), one has a diagonal Schur complement
\[
 S_N(s)\;=\; -\,\mathrm{diag}\!\left(\frac{p^{-s}}{1-p^{-s}}\right)_{p\le p_N}.
\]
Consequently
\[
 \log\det(I-S_N(s))\;=\;\sum_{p\le p_N}\log\!\left(\frac{1}{1-p^{-s}}\right)
\]
and the identity of Proposition~\ref{prop:schur-split} yields the desired $k=1$ separation when combined with $\log\dettwo(I-A_N)= -\sum_{k\ge 2}\Tr(A_N^k)/k$. However, the operator norm here obeys
\[
 \|S_N(s)\|\;=\;\max_{p\le p_N}\,\frac{|p^{-s}|}{\,1-|p^{-s}|\,}\;=\;\max_{p\le p_N}\,\frac{p^{-\sigma}}{1-p^{-\sigma}}\,,\qquad s=\sigma+it,
\]
so $\|S_N(s)\|\le 1$ holds only for $\sigma\ge 1$ (strictly $<1$ for $\sigma>1$). Thus (E1) gives an \emph{exact} $k=1$ finite block which is Schur on $\{\Re s\ge 1\}$ but not on the entire $\{\Re s>\tfrac12\}$.

\paragraph{(E2) Damped exact-form with uniform contractivity on $\{\Re s\ge\sigma_0\}$.}
Fix $\sigma_0>\tfrac12$ and a scalar damping factor
\[
 \alpha(\sigma_0)\;:=\;\frac{1-2^{-\sigma_0}}{2^{-\sigma_0}}\;=\;2^{\sigma_0}-1\;\in\;(0,\infty).
\]
Define
\[
 b_p(s)\;:=\;\sqrt{\alpha(\sigma_0)}\,p^{-s/2},\qquad c_p(s)\;:=\;\sqrt{\alpha(\sigma_0)}\,p^{-s/2},\qquad d_p(s)\;:=\;0.
\]
Then
\[
 S_N(s)\;=\;-\,\alpha(\sigma_0)\,\mathrm{diag}\!\left(\frac{p^{-s}}{1-p^{-s}}\right)_{p\le p_N}.
\]
Using Remark~\ref{rem:finite-block-design} with $\|B_N\|=\|C_N\|=\sup_{p\le p_N}|b_p|=\sqrt{\alpha(\sigma_0)}\,2^{-\sigma/2}$ and $\|(I-A_N)^{-1}\|\le (1-2^{-\sigma_0})^{-1}$ on $\{\Re s\ge \sigma_0\}$ gives
\[
 \|S_N(s)\|\;\le\;\frac{\|C_N\|\,\|B_N\|}{1-2^{-\sigma_0}}\;\le\;\frac{\alpha(\sigma_0)\,2^{-\sigma_0}}{1-2^{-\sigma_0}}\;=\;1,\qquad \Re s\ge \sigma_0.
\]
Thus (E2) furnishes a Schur finite block on any prescribed right half-plane $\{\Re s\ge \sigma_0\}$, at the cost of damping the $k=1$ contribution by the factor $\alpha(\sigma_0)$:
\[
 \log\det(I-S_N)\;=\;\sum_{p\le p_N}\log\!\left(\frac{1-\big(1-\alpha(\sigma_0)\big)p^{-s}}{1-p^{-s}}\right).
\]
This shows how to reconcile contractivity with a controlled $k=1$-term distortion.

\paragraph{(E3) Faster-decay variant.}
For any $\beta>0$, choose $b_p(s)=c_p(s)=p^{-(1/2+\beta)s}$, $d_p\equiv 0$. Then
\[
 S_N(s)\;=\;-\,\mathrm{diag}\!\left(\frac{p^{-(1+2\beta)s}}{1-p^{-s}}\right)_{p\le p_N},\qquad \|S_N(s)\|\;\le\;\sup_p\frac{p^{-\sigma(1+2\beta)}}{1-p^{-\sigma}},
\]
which is $<1$ uniformly on $\{\Re s>\tfrac12\}$ once $\beta$ is chosen large enough (e.g., any $\beta\ge \tfrac12$ works). The $k=1$ term is then heavily damped, but this family supplies uniformly Schur finite blocks on the entire BRF domain.
\begin{remark}[Design notes]
Parameterizations (E1)–(E3) expose a concrete path to Schurness of the finite block on right half-planes using only the diagonal structure of $A_N$. In practice one also folds the archimedean/pole corrections into $D_N$ while preserving the Schur bound and links the Schur finite block to the determinantal truncation so that the resulting Cayley transform approximates $\Theta_N^{(\dettwo)}$ uniformly on compacts (as realized quantitatively by the H$^\infty$ passive approximation scheme of Subsection~\ref{subsec:hinf-passive}).
\end{remark}

% (Archimedean port material pertains to the xi-normalized route and is not used in the zeta-normalized certificate.)
\subsection{Contractivity with a budgeted port $D_N$}\label{subsec:DN-budget}
We refine (E2) to incorporate a nonzero contraction $D_N$ while maintaining Schurness on $\{\Re s\ge \sigma_0\}$.

\begin{lemma}[Budgeted contractivity]\label{lem:budget}
Fix $\sigma_0>\tfrac12$ and a budget $\eta\in(0,1)$. Let
\[
 \alpha(\sigma_0,\eta)\;:=\;(1-\eta)\,\frac{1-2^{-\sigma_0}}{2^{-\sigma_0}}\,=\,(1-\eta)\,(2^{\sigma_0}-1),
\]
and choose
\[
 b_p(s)\;=\;\sqrt{\alpha(\sigma_0,\eta)}\,p^{-s/2},\quad c_p(s)\;=\;\sqrt{\alpha(\sigma_0,\eta)}\,p^{-s/2},\quad D_N(s)\ \text{with}\ \|D_N\|_{H^\infty(\Re s\ge \sigma_0)}\le \eta.
\]
Then for $A_N(s)\,e_p=p^{-s}e_p$ one has
\[
 S_N(s)\;=\;D_N(s)\; -\; C_N(s)\,(I-A_N(s))^{-1}\,B_N(s),\qquad \|S_N(s)\|\ \le\ 1\quad (\Re s\ge \sigma_0).
\]
\end{lemma}
\begin{proof}
On $\{\Re s\ge \sigma_0\}$, $\|(I-A_N)^{-1}\|\le (1-2^{-\sigma_0})^{-1}$ and $\|B_N\|=\|C_N\|\le \sqrt{\alpha(\sigma_0,\eta)}\,2^{-\sigma_0/2}$. Thus
\[
 \|C_N(I-A_N)^{-1}B_N\|\ \le\ \frac{\alpha(\sigma_0,\eta)\,2^{-\sigma_0}}{1-2^{-\sigma_0}}\ =\ 1-\eta.
\]
Hence $\|S_N\|\le \|D_N\|+\|C_N(I-A_N)^{-1}B_N\|\le \eta+(1-\eta)=1$.
\end{proof}

% (Archimedean Cayley interpolation omitted; not used in the zeta-normalized route.)
\paragraph{Contraction port.}
Let $F(s)$ be any bounded holomorphic function on $\{\Re s\ge \sigma_0\}$ with $\|F\|_{H^\infty}\le 1$. Setting
\[
 D_N(s)\;=\;\eta F I_N
\]
fits (by construction) the budget of Lemma~\ref{lem:budget} with $\|D_N\|\le \eta$.
\subsection{NP interpolation for the archimedean port and $k=1$ separation}\label{subsec:NP-arch}
We make the Nevanlinna--Pick (NP) step explicit and quantify the $k=1$ separation inside $\log\det(I-S_N)$.

\begin{lemma}[Schur NP interpolant for the archimedean Cayley]
Fix $\sigma_0>\tfrac12$ and a finite node set $\{s_j\}_{j=1}^{M}\subset\{\Re s\ge \sigma_0\}$. Let target values $\{\gamma_j\}$ satisfy $|\gamma_j|<1$. Then there exists a scalar Schur function $F$ on $\{\Re s\ge \sigma_0\}$ with $F(s_j)=\gamma_j$ for all $j$. Moreover one may take $F$ rational inner of degree at most $M$.
\end{lemma}

\begin{lemma}[Finite KYP augmentation for affine terms]\label{lem:affine-gram}
Let \(K_0(s,\overline t)\) be a PSD kernel on \(R\times R\) of the form \(\langle \Phi(s),\Phi(t)\rangle_{P}\), with a finite-dimensional realization \((A,B,C,D,P)\) satisfying the lossless equalities. Then, for any \(\alpha,\beta\in\C\), there exists an augmented lossless realization \((\widehat A,\widehat B,\widehat C,\widehat D,\widehat P)\) such that the kernel
\[
 K_\mathrm{sum}(s,\overline t)\ :=\ K_0(s,\overline t)\ +\ \frac{(\alpha+\beta s)+\overline{(\alpha+\beta t)}}{s+\overline t -1}
\]
is PSD on \(R\times R\) and equals \(\langle \widehat\Phi(s),\widehat\Phi(t)\rangle_{\widehat P}\) for a suitable feature map \(\widehat\Phi\) built by direct sum with one- and two-state lossless blocks.
\end{lemma}
\begin{proof}
Consider the scalar lossless factor \(H_1(s)=(s-\lambda)/(s+\lambda)\) with \(\lambda>0\) (Lemma~\ref{lem:moebius-contract}). Its Herglotz kernel equals
\[\frac{H_1(s)+\overline{H_1(t)}}{s+\overline t -1}\ =\ \Big\langle (sI+\lambda)^{-1}\sqrt{2\lambda},\ (tI+\lambda)^{-1}\sqrt{2\lambda}\Big\rangle,\]
which is a rank-one PSD kernel. Linear combinations of such kernels (with distinct \(\lambda\)) span the space of kernels of the form \(\frac{p(s)+\overline{p(t)}}{s+\overline t-1}\) for degree-1 polynomials \(p\). Appending these blocks as a direct sum to \((A,B,C,D)\) preserves losslessness and PSD of the associated Gram. Therefore the affine term can be realized inside the finite KYP block and absorbed into the augmented feature \(\widehat\Phi\).
\end{proof}

Apply this with prescribed $\gamma_j$ sampling the normalized archimedean Cayley $\Phi_{\mathrm{arch}}(s)=(E_{\mathrm{arch}}(s)-1)/(E_{\mathrm{arch}}(s)+1)$ on the line $\Re s=\sigma_0$. Setting $D_N=\eta F$ where $F$ is a half-plane Schur NP interpolant (Lemma in Subsection~\ref{subsec:NP-arch}) yields a budgeted contraction with $\|D_N\|\le \eta$.

\begin{lemma}[Half-plane Blaschke products and Pick criterion]\label{lem:halfplane-blaschke}
For nodes $a_j\in\{\Re s>\sigma_0\}$ and target values $\gamma_j$ with $|\gamma_j|<1$, the Nevanlinna--Pick matrix $\big((1-\gamma_j\overline{\gamma_k})/(a_j+\overline{a_k}-2\sigma_0)\big)_{j,k}$ is PSD if and only if there exists a Schur function $F$ on $\{\Re s>\sigma_0\}$ with $F(a_j)=\gamma_j$. A constructive solution is given by finite products of half-plane Blaschke factors
\[
 B_{a}(s)\;:=\;\frac{s-\overline a}{s-a}\,,\qquad \Re a>\sigma_0,
\]
possibly multiplied by a unimodular constant and post-composed with disk automorphisms. In particular, any finite data set with a PSD Pick matrix admits a rational inner interpolant $F(s)=e^{i\theta}\prod_{j=1}^{M} B_{a_j}(s)^{m_j}$.
\end{lemma}


\begin{proposition}[Exact log-det formula and $k=1$ separation with damping]\label{prop:logdet-S}
Let $S_N$ be constructed as in Lemma~\ref{lem:budget} with diagonal $B_N,C_N$ and $D_N=\eta F I_N$. Then
\[
 \det(I-S_N(s))\;=\;\big(1-\eta F(s)\big)^{N}\,\prod_{p\le p_N}\left(1+\frac{\alpha(\sigma_0,\eta)}{1-\eta F(s)}\,\frac{p^{-s}}{1-p^{-s}}\right).
\]
In particular,
\[
 \log\det(I-S_N(s))\;=\;N\log\big(1-\eta F(s)\big)\; +\; \sum_{p\le p_N}\log\left(\frac{1-(1-\beta(s))\,p^{-s}}{1-p^{-s}}\right)
\]
with the scalar damping $\beta(s)=\alpha(\sigma_0,\eta)/(1-\eta F(s))$.
\end{proposition}
\begin{proof}
Since $D_N$ is a scalar multiple of the identity and $C_N(I-A_N)^{-1}B_N$ is diagonal, the eigenvalues of $I-S_N$ are $(1-\eta F)+\alpha\, p^{-s}/(1-p^{-s})$ over $p\le p_N$, yielding the product formula. The logarithmic form follows by rearrangement.
\end{proof}

\begin{remark}[Blocker: growth of the $k=1$ error budget]
The right-hand sum $\sum_{p\le p_N} |p^{-s}|/|1-p^{-s}|$ diverges with $N$ for $\Re s\le 1$. Hence keeping $\beta\equiv 1$ is essential to preserve exact $k=1$ separation uniformly in $N$; this is feasible only for $\sigma_0\ge 1$ (case (E1)). For $\sigma_0\in(\tfrac12,1)$, any uniform damping induces a cumulative error growing with $N$. Resolving this obstruction (e.g., by a different finite-block architecture or a non-additive multiplicative scheme) is required to remove the reliance on the alignment hypothesis on the full BRF domain.
\end{remark}

\subsection{Schur finite blocks with uniform-on-$K$ $k=1$ control}\label{subsec:K1-approx}
We summarize the $k=1$ approximation mechanism that preserves Schurness on a fixed right half-plane compact while providing uniform error control.

\begin{proposition}[Uniform-on-$K$ $k=1$ control with Schurness]\label{prop:K1-approx}
Let $K\subset\{\Re s\ge\sigma_0\}$ be compact with $\tfrac12<\sigma_0<1$ and fix $\eta\in(0,\tfrac12)$ and $\varepsilon>0$. Then there exist finite-rank holomorphic matrices $B_N(s),C_N(s)$ and a scalar $D_N(s)$ with $\|D_N\|_{L^{\infty}(K)}\le\eta$ such that for
\[
 S_N(s)\;=\;D_N(s)\; -\; C_N(s)\,(I-A_N(s))^{-1}B_N(s),\qquad A_N(s)e_p\;=\;p^{-s}e_p,\ p\le p_N,
\]
one has:
\begin{itemize}
 \item Schur on $K$: $\displaystyle\sup_{s\in K}\,\|S_N(s)\|\le 1$;
 \item Uniform $k=1$ control: $\displaystyle\sup_{s\in K}\,\Bigl|\log\det(I-S_N(s))\; -\;\sum_{p\le p_N}\log\frac{1}{1-p^{-s}}\Bigr|\ \le\ \varepsilon.$
\end{itemize}
In particular, $S_N$ can be taken from the budgeted/damped family of Section~\ref{subsec:DN-budget} with Nevanlinna--Pick $D_N$ (Subsection~\ref{subsec:NP-arch}) and parameters chosen so that the error bound holds on $K$.
\begin{remark}
The parameters $(\eta,\delta,N)$ can be selected in a $K$-dependent but explicit manner: choose $\eta\le \varepsilon/(2M_0)$ for a fixed port dimension $M_0$, and pick $\delta\ll \varepsilon$ so that $\sum_{p\le p_N} |p^{-s}|/|1-p^{-s}|\le C_K\delta\le \varepsilon/2$ uniformly on $K$. This yields the displayed bound while preserving the Schur budget $\|S_N\|\le 1$.
\end{remark}
\end{proposition}
\begin{proof}[Idea]
By Lemma~\ref{lem:budget} pick $B_N,C_N$ diagonal in the prime basis with damping parameter $\alpha(\sigma_0,\eta)$ so that $\|C_N(I-A_N)^{-1}B_N\|\le 1-\eta$ on $K$. With $D_N=\eta F$ where $F$ is a half-plane Schur NP interpolant (Lemma in Subsection~\ref{subsec:NP-arch}), Proposition~\ref{prop:logdet-S} gives
\[
 \log\det(I-S_N)=N\log(1-\eta F)\ +\ \sum_{p\le p_N}\log\frac{1-(1-\beta(s))p^{-s}}{1-p^{-s}},\qquad \beta(s)=\frac{\alpha(\sigma_0,\eta)}{1-\eta F(s)}.
\]
On $K$, choose $F$ and $\eta$ so that $\sup_K|\beta-1|\le\delta$ with $\delta$ small enough; then the log-det difference is bounded by $C_K\delta\sum_{p\le p_N}|p^{-s}|/|1-p^{-s}|+N\,\eta/(1-\eta)$. Place $D_N$ in a fixed-dimensional port (or scale $N$) so the $N$-term is $\le \varepsilon/2$, and choose $\delta$ so the prime sum is $\le\varepsilon/2$ uniformly on $K$. This yields the claimed bound while retaining $\|S_N\|\le 1$.
\end{proof}

\section{Finite-stage KYP certificates: lossless factorization and prime-grid model}\label{sec:KYP}
We now construct explicit finite-stage passive (bounded-real) realizations and verify the Kalman--Yakubovich--Popov (KYP) condition. We work throughout in continuous time on the right half-plane, with the transfer function
\[
 H(s)\;=\;D\; +\; C\,(sI-A)^{-1} B,
\]
where \(A\in\C^{n\times n}\) is Hurwitz (spectrum strictly in the open left half-plane), and \(B\in\C^{n\times m}\), \(C\in\C^{m\times n}\), \(D\in\C^{m\times m}\).
\subsection{Bounded-real lemma and the lossless KYP equalities}
The continuous-time bounded-real lemma asserts that, for a Hurwitz \(A\), the following are equivalent: (i) \(\|H\|_\infty\le 1\); (ii) there exists \(P\succ 0\) such that the KYP matrix is negative semidefinite
\begin{equation}\label{eq:KYP}
 \Theta\;=\;\begin{bmatrix}
  A^*P+PA & PB & C^*\\
  B^*P & -I & D^*\\
  C & D & -I
 \end{bmatrix}\ \preceq\ 0.
\end{equation}
In the \emph{lossless} case (extremal \(\|H\|_\infty=1\)), one may certify \eqref{eq:KYP} via the following algebraic equalities.

\begin{lemma}[One-line lossless KYP factorization]\label{lem:losslessKYP}
Suppose \(P\succ 0\) and
\begin{equation}\label{eq:lossless-equalities}
 A^*P+PA\;=\;-C^*C,\qquad PB\;=\;-C^*D,\qquad D^*D\;=\;I.
\end{equation}
Then the KYP matrix \(\Theta\) in \eqref{eq:KYP} factors as
\begin{equation}\label{eq:one-line-factor}
 \boxed{\ \Theta\;=\;-\begin{bmatrix}C^*\\ D^*\\ -I\end{bmatrix}\!\begin{bmatrix}C & D & -I\end{bmatrix}\ \preceq\ 0\ }.
\end{equation}
In particular, \(\|H\|_\infty\le 1\).
\end{lemma}
\begin{proof}
Using \eqref{eq:lossless-equalities}, we rewrite the KYP blocks as
\[
 A^*P+PA\;=\;-C^*C,\qquad PB\;=\;-C^*D,\qquad B^*P\;=\;-D^*C.
\]
Substituting these into \eqref{eq:KYP} gives
\[
 \Theta\;=\;\begin{bmatrix}
  -C^*C & -C^*D & C^*\\
  -D^*C & -I & D^*\\
  C & D & -I
 \end{bmatrix}\;=\;-\begin{bmatrix}C^*\\ D^*\\ -I\end{bmatrix}\!\begin{bmatrix}C & D & -I\end{bmatrix},
\]
which is negative semidefinite as a Gram matrix with a negative sign. The bounded-real implication is standard from the KYP lemma for Hurwitz \(A\).
\end{proof}

\subsection{Prime-grid lossless model}\label{subsec:prime-grid-KYP}
Let \(p_1<\cdots<p_N\) be the first \(N\) primes and define the positive diagonal matrix
\[
 \Lambda_N\;:=\;\mathrm{diag}\!\Big(\tfrac{2}{\log p_1},\dots,\tfrac{2}{\log p_N}\Big)\ \in\ \R^{N\times N}.
\]
Set
\[
 A_N\;:=\;-\Lambda_N,\qquad P_N\;:=\;I_N,\qquad C_N\;:=\;\sqrt{2\,\Lambda_N},\qquad D_N\;:=\;-I_N,\qquad B_N\;:=\;C_N.
\]
Then:
\begin{proposition}[Lossless KYP on the prime grid]\label{prop:prime-grid-KYP}
\ \(A_N\) is Hurwitz, with spectrum \(-\{2/\log p_k\}_{k=1}^N\subset(-\infty,0)\). Moreover, the lossless equalities \eqref{eq:lossless-equalities} hold with \((A,B,C,D,P)=(A_N,B_N,C_N,D_N,P_N)\):
 \[
  A_N^*P_N+P_NA_N\;=\;-2\Lambda_N\;=\;-C_N^*C_N,\quad P_NB_N\;=\;C_N\;=\;-C_N^*D_N,\quad D_N^*D_N\;=\;I_N.
 \]
Consequently, the KYP matrix factors as in \eqref{eq:one-line-factor}, and for the matrix-valued transfer
 \[
  H_N(s)\;:=\;D_N\; +\; C_N\,(sI-A_N)^{-1} B_N
 \]
one has \(\|H_N\|_\infty\le 1\). In particular, each entry of \(H_N\) is a bounded-real function on \(\Omega\). Finally, for any unit vectors \(u,v\in\C^N\) (``scalar port extraction''), the scalar transfer \(h_N(s):=v^*H_N(s)u\) satisfies \(|h_N(s)|\le 1\) for all \(s\in\Omega\); choosing \(u=v=e_1\) yields scalar feedthrough \(-1\).
\end{proposition}
\begin{proof}
(i) \(\Lambda_N\) is positive diagonal, hence \(A_N=-\Lambda_N\) has strictly negative diagonal entries.
(ii) Direct computation using diagonality: \(A_N^*P_N+P_NA_N=(-\Lambda_N)+(-\Lambda_N)=-2\Lambda_N\). Since \(C_N=\sqrt{2\Lambda_N}\) is the positive square root, \(C_N^*C_N=2\Lambda_N\), hence \(A_N^*P_N+P_NA_N=-C_N^*C_N\). Next, \(P_NB_N=B_N=C_N\) and \(C_N^*D_N=\sqrt{2\Lambda_N}\,(-I_N)=-C_N\), so \(P_NB_N+ C_N^*D_N=0\). Finally, \(D_N^*D_N=(-I_N)^*(-I_N)=I_N\).
\end{proof}

\(\Theta=\frac{2\mathcal J-1}{2\mathcal J+1}\) is Schur on \(\Omega\) (Theorem~\ref{thm:brf-rh-final}).

\subsection{Inner compensator for zeros of \(\xi\)}\label{subsec:bl-compensator}
If \(\xi\) has zeros in a fixed rectangle \(R\subset\Omega\), the ratio \(J=\dettwo(I-A)/\xi\) is meromorphic on \(R\).
To ensure analyticity for auxiliary constructions on \(R\) (e.g., passive \(H^\infty\) approximation), introduce the finite half-plane Blaschke product
\(
 B_{\xi,R}(s):=\prod_{\rho\in Z(\xi)\cap R} \big(\tfrac{s-\overline \rho}{s-\rho}\big)^{m_\rho}.
\)
Define the compensated ratio \(J_R^{\rm comp}:=J\,B_{\xi,R}\), which is holomorphic on \(R\).
\emph{We do not use} \(J_R^{\rm comp}\) in the (P+) boundary route, since multiplication by an inner factor preserves modulus but not boundary real part. The compensator is employed only to build interior Schur approximants on rectangles; the global Schur/PSD conclusion comes from (P+) with outer normalization, independently of any compensator.

\subsection{Prototype outer factor on \(\Re s=\tfrac12+\varepsilon\)}\label{subsec:outer-prototype}
Fix \(\varepsilon>0\) and consider \(L_{\varepsilon}:=\{s=\tfrac12+\varepsilon+it\}\). Define
\[
 G_{\varepsilon}(t):=\dettwo\big(I-A(\tfrac12+\varepsilon+it)\big),\qquad X_{\varepsilon}(t):=\xi\big(\tfrac12+\varepsilon+it\big).
\]
Let \(\mathcal O_{\varepsilon}\) be the outer on \(\{\Re s>\tfrac12+\varepsilon\}\) with boundary modulus \(\big|\frac{G_{\varepsilon}}{X_{\varepsilon}}\big|\). Set
\[
 \mathcal J_{\varepsilon}(s):=\frac{\dettwo(I-A(s))}{\mathcal O_{\varepsilon}(s)\,\xi(s)}.
\]
Then \(|\mathcal J_{\varepsilon}|=1\) on \(L_{\varepsilon}\) and \(\mathcal J_{\varepsilon}\) is holomorphic on \(\{\Re s>\tfrac12+\varepsilon\}\). By Theorem~\ref{thm:uniform-eps} and Lemma~\ref{lem:outer-phase-HT}, \(\mathcal O_{\varepsilon}\to\mathcal O\) and \(\mathcal J_{\varepsilon}\to\mathcal J\) locally uniformly as \(\varepsilon\downarrow 0\). Using Bridges A--C together with the certified Schur covering, it follows that the boundary line is zero-free; hence \(2\mathcal J\) is Herglotz in \(\Omega\), so \(\Theta\) is Schur (Theorem~\ref{thm:brf-rh-final}).

\begin{proposition}[L$^1_{\mathrm{loc}}$ control reduces to HS tails]\label{prop:L1loc}
Fix a compact interval $I\subset\mathbb R$. Then for \(\varepsilon\in(0,\tfrac12)\),
\[
 \int_{I}\left|\log\left|\frac{G_{\varepsilon}(t)}{X_{\varepsilon}(t)}\right|\right|\,dt\ \le\ C_I\,\left(1+\sup_{t\in I}\|A(\tfrac12+\varepsilon+it)-A_N(\tfrac12+\varepsilon+it)\|_{\HS}\right),
\]
with $C_I$ independent of $N$. In particular, the HS tail control $\|A-A_N\|_{\HS}\to 0$ uniformly on \(\{\Re s\ge \tfrac12+\varepsilon\}\) implies precompactness of \(\{\log|G_{\varepsilon}/X_{\varepsilon}|\}\) in $L^1(I)$ and hence local-uniform convergence of the outer normalizations \(\mathcal O_{\varepsilon}\) along subsequences.
\end{proposition}
\begin{proof}
Carleman's bound (Lemma~\ref{lem:carleman}) gives \(|G_{\varepsilon}(t)|\le e^{\tfrac12\|A\|_{\HS}^2}\), while the HS continuity (Proposition~\ref{prop:HS-to-det2}) furnishes Lipschitz control for \(\log|\dettwo(I-A)|\) w.r.t. the HS norm. Stirling bounds control \(\log|X_{\varepsilon}(t)|\) on vertical lines uniformly on $I$ away from the finitely many zeros of \(\xi\) in the vertical strip under consideration. Integrating across small neighborhoods of those zeros, one uses that \(\log|\cdot|\) is locally integrable and that zeros are discrete with finite multiplicity to obtain an $L^1$ bound independent of \(\varepsilon\).
\end{proof}
\begin{remark}
Proposition~\ref{prop:L1loc} gives tightness for each fixed \(\varepsilon>0\). As \(\varepsilon\downarrow 0\), we only use the smoothed/distributional $L^1_{\mathrm{loc}}$ control and the Cauchy property stated in Theorem~\ref{thm:uniform-eps}.
\end{remark}
\subsection{Smoothed \(\varepsilon\downarrow 0\) boundary control}\label{subsec:uniform-eps}
We now state the boundary theorem used for the outer-normalization route. Our use is purely smoothed/distributional; no global uniform $L^1$ slice claim is made beyond what is proved below. See Subsection~\ref{subsec:smoothed-explicit} for the smoothed explicit-formula route and a de-smoothing step.

\begin{theorem}[Smoothed $L^1_{\mathrm{loc}}$ bound and Cauchy as \(\varepsilon\downarrow 0\)]\label{thm:uniform-eps}
For every compact interval $I\subset\R$ there exist constants $C_I$ and \(\varepsilon_0>0\) such that for all \(\varepsilon\in(0,\varepsilon_0)\),
\[
 \int_I \Bigl|\log\Bigl|\frac{\dettwo(I-A(\tfrac12+\varepsilon+it))}{\xi(\tfrac12+\varepsilon+it)}\Bigr|\Bigr|\,dt\ \le\ C_I,
\]
and the family is Cauchy in $L^1(I)$ as \(\varepsilon\downarrow 0\):
\[
 \lim_{\substack{\varepsilon,\delta\downarrow 0}}\ \int_I \Bigl|\log\Bigl|\frac{\dettwo(I-A(\tfrac12+\varepsilon+it))}{\xi(\tfrac12+\varepsilon+it)}\Bigr|\;-
 \log\Bigl|\frac{\dettwo(I-A(\tfrac12+\delta+it))}{\xi(\tfrac12+\delta+it)}\Bigr|\Bigr|\,dt\;=\;0.
\]
Consequently, via the Poisson representation for outer functions and the bounds in Lemma~\ref{lem:desmoothing}, the outer normalizations \(\mathcal O_{\varepsilon}\to \mathcal O\) converge locally uniformly to an outer limit \(\mathcal O\) on \(\Omega\). Our use of this theorem in the certificate does not require any stronger (global) $L^1$ statements.
\end{theorem}
\begin{proof}
Fix a compact interval $I\subset\R$. Write $F(s):=\dettwo(I-A(s))$ and $X(s):=\xi(s)$. We show
\[
 u_\varepsilon(t):=\log\Bigl|\frac{F(\tfrac12+\varepsilon+it)}{X(\tfrac12+\varepsilon+it)}\Bigr|\in L^1(I)
\]
with $\|u_\varepsilon\|_{L^1(I)}\le C_I$ independent of \(\varepsilon\in(0,\varepsilon_0]\), and that \(\{u_\varepsilon\}\) is $L^1(I)$–Cauchy as \(\varepsilon\downarrow 0\). The standing hypotheses in Section~\ref{sec:appendix} (HS analyticity of $A$, analytic Fredholm property for $I-A$, and local analyticity of \(\xi\)) hold in the rectangle \(\mathcal R:=\{\tfrac12\le\sigma\le\tfrac12+\varepsilon_0,\ t\in I^{\!*}\}\subset\Omega\) for a slightly larger \(I^{\!*}\supset I\).

1) Uniform $L^1$ bound. By Lemma~\ref{lem:carleman}, for \(s\in\mathcal R\),
\[
 \log^+|F(s)|\;\le\;\tfrac12\,\|A(s)\|_{\HS}^2\;\le\;\tfrac12\,M_I^2.
\]
Apply the finite-domain Weierstrass factorization on \(\mathcal R\) to write each as a sum of a bounded harmonic term plus finitely many logarithmic spikes \(\log|s-\rho|\) corresponding to zeros \(\rho\) inside \(\mathcal R\). Along \(s=\tfrac12+\varepsilon+it\), the harmonic terms contribute a bounded amount to \(\int_I |u_\varepsilon(t)|dt\) by the maximum principle; each spike is uniformly integrable in \(t\) and uniformly in \(\varepsilon\) since \(\int_I |\log|\tfrac12+\varepsilon+it-\rho||\,dt\) is finite and locally uniform in \(\varepsilon\) for finitely many \(\rho\). Summing finitely many contributions yields $\|u_\varepsilon\|_{L^1(I)}\le C_I$.

2) $L^1$–Cauchy. For \(0<\delta<\varepsilon\le\varepsilon_0\), write
\[
 u_\varepsilon(t)-u_\delta(t)
 = \int_{\delta}^{\varepsilon} \partial_\sigma \Re\Bigl(\log F(\tfrac12+\sigma+it)-\log X(\tfrac12+\sigma+it)\Bigr)\,d\sigma.
\]
Using the Lipschitz control for \(\log\dettwo\) (from Proposition~\ref{prop:HS-to-det2}) together with the uniform \(\sigma\)-derivative bounds from Lemma~\ref{lem:uniform-derivative-L1}, we obtain
\[
 \int_I\bigl|\partial_\sigma\,\Re\log F(\tfrac12+\sigma+it)\bigr|\,dt\ \le\ C_I,
\]
uniformly for \(\sigma\in[\delta,\varepsilon]\). For the \(\xi\) term, standard Stirling bounds for \(\partial_\sigma\log X= X'/X\) on vertical lines (\cite{TitchmarshZeta}, Chap.~IV) yield
\[
  \int_I\bigl|\partial_\sigma\,\Re\log X(\tfrac12+\sigma+it)\bigr|\,dt\ \le\ C_I',
\]
uniformly in \(\sigma\in[\delta,\varepsilon]\). Fubini's theorem gives
\[
 \|u_\varepsilon-u_\delta\|_{L^1(I)}\;\le\;(C_I+C_I')\,|\varepsilon-\delta|\;\xrightarrow[\varepsilon,\delta\downarrow 0]{}\;0.
\]
Therefore \(u_\varepsilon\) is uniformly $L^1$–bounded and $L^1$–Cauchy on \(I\) provided the derivative bounds hold. This implication is formalized in Lemma~\ref{lem:desmoothing} below. The Poisson–Hilbert representation of outer functions on the half-plane (with \(u_\varepsilon\) as boundary data) then yields local-uniform convergence of outer normalizations \(\mathcal O_\varepsilon\to \mathcal O\), and the a.e. boundary modulus \(|\Theta(\tfrac12+it)|=1\) of the inner factor. The Schur bound in \(\Omega\) follows by the maximum principle.
\end{proof}

\begin{lemma}[\(\xi\)-derivative $L^1$ bound on vertical segments]\label{lem:xi-deriv-L1}
Let $I\Subset\R$ and $\sigma\in[\tfrac12,\tfrac12+\varepsilon_0]$. Then
\[
 \int_I \Big|\partial_\sigma\,\Re\log\xi(\sigma+it)\Big|\,dt\ \le\ C_I',
\]
with $C_I'$ independent of $\sigma$.
\end{lemma}
\begin{proof}
Write \(\partial_\sigma\,\Re\log\xi=\Re(\xi'/\xi)\) and use the explicit zero-factorization: on vertical lines, one has
\[
 \Re\frac{\xi'}{\xi}(\sigma+it)\ =\ \sum_{\rho} m_\rho\,\Re\frac{1}{\sigma+it-\rho}\ +\ \text{bounded terms},
\]
where the latter are uniformly bounded on compact $I$ by Stirling estimates and continuity. For each zero \(\rho=\beta+i\gamma\), the contribution integrates as
\[\int_I \Big|\Re\frac{1}{\sigma+it-\rho}\Big|\,dt\ \le\ \int_{t\in I} \frac{|\sigma-\beta|}{(\sigma-\beta)^2+(t-\gamma)^2}\,dt\ \le\ \pi,
\]
uniformly in \(\sigma\in[\tfrac12,\tfrac12+\varepsilon_0]\) (standard integral). Only finitely many zeros intersect the strip above $I$ within a bounded distance; the tail is summable by the classical bound $N(T)\ll T\log t$. Summing over zeros and adding the bounded archimedean contribution yields the claim.
\end{proof}
\begin{lemma}[det$_2$-derivative $L^1$ bound on vertical segments]\label{lem:det2-deriv-L1}
Let $I\Subset\R$ and $\sigma\in[\tfrac12+\delta,\tfrac12+\varepsilon_0]$ with $\delta>0$. Then
\[
 \int_I \Big|\partial_\sigma\,\Re\log\dettwo\big(I-A(\sigma+it)\big)\Big|\,dt\ \le\ C_I(\delta).
\]
\end{lemma}
\begin{proof}
Using the absolutely convergent expansion for \(\sigma>\tfrac12\),
\[\partial_\sigma\,\Re\log\dettwo(I-A(\sigma+it))\ =\ \sum_{k\ge 2}\sum_{p\in\PP} (\log p)\,p^{-k\sigma}\cos(k t\log p),\]
we bound
\[\int_I \Big|\sum_{k,p}(\log p)\,p^{-k\sigma}\cos(k t\log p)\Big|dt\ \le\ \sum_{k,p}(\log p)\,p^{-k\sigma}\int_I |\cos(k t\log p)|\,dt\ \le\ |I|\sum_{k,p}(\log p)\,p^{-k\sigma}.
\]
For \(\sigma\ge \tfrac12+\delta\), the double series converges by comparison with \(\sum_{k\ge 2} p^{-k(\tfrac12+\delta)}\log p\); in particular the $k=2$ line is \(\sum_p (\log p)\,p^{-1-2\delta}<\infty\). Hence the bound $C_I(\delta)$ follows.
\end{proof}
\begin{remark}
At the boundary \(\sigma\downarrow \tfrac12\), oscillatory (smoothed) bounds (Lemma~\ref{lem:det2-smoothed-target}) combined with a standard duality argument on \(W^{2,1}(I)\) test functions yield uniform \(L^1\) control in the limit; see Lemma~\ref{lem:uniform-derivative-L1} and Proposition~\ref{prop:desmoothing} for the precise Cauchy transfer.
\end{remark}

\begin{lemma}[De-smoothing: bounded $L^1$ derivative implies $L^1$–Cauchy]\label{lem:desmoothing}
Let \(I\Subset\R\) and let \(u_\sigma\in L^1(I)\) be defined for \(\sigma\in(0,\varepsilon_0]\), differentiable in \(\sigma\), with
\[
  \int_I |\partial_\sigma u_\sigma(t)|\,dt\ \le\ C_I\qquad\text{for all }\sigma\in(0,\varepsilon_0].
\]
Then \(\{u_\varepsilon\}_{\varepsilon\downarrow 0}\) is Cauchy in $L^1(I)$.
\end{lemma}
\begin{proof}
For \(0<\delta<\varepsilon\le\varepsilon_0\), the fundamental theorem of calculus gives
\(u_\varepsilon-u_\delta=\int_\delta^\varepsilon \partial_\sigma u_\sigma\,d\sigma\).
Minkowski's integral inequality yields
\[
  \|u_\varepsilon-u_\delta\|_{L^1(I)}\ \le\ \int_\delta^\varepsilon \int_I |\partial_\sigma u_\sigma(t)|\,dt\,d\sigma\ \le\ C_I(\varepsilon-\delta),
\]
which tends to $0$ as \(\varepsilon,\delta\downarrow 0\).
\end{proof}
\begin{remark}
We take \(C_c^2(I)\) test functions dense in \(W^{2,1}_0(I)\) so that smoothed bounds transfer to the unsmoothed case by duality; the uniform bound on \(\int_I|\partial_\sigma u_\sigma|\) is independent of \(\sigma\), so no loss appears as \(\varepsilon\downarrow 0\).
\end{remark}
\begin{remark}
The uniform-in-\(\varepsilon\) local $L^1$ control of Theorem~\ref{thm:uniform-eps} follows from the unsmoothed det$_2$ bound (Lemma~\ref{lem:det2-unsmoothed}) together with the \(\xi\)-derivative bound (Lemma~\ref{lem:xi-deriv-L1}) and the de-smoothing Lemma~\ref{lem:desmoothing}. The smoothed explicit-formula route below is auxiliary.
\end{remark}
\subsection{Smoothed explicit-formula route and de-smoothing}\label{subsec:smoothed-explicit}
We complement the preceding proof with an unconditional, smoothed route that avoids any zero-free hypothesis and isolates prime/zero cancellation at the level of test functions.

\begin{lemma}[Smoothed uniform bound via an explicit formula]\label{lem:smoothed-explicit}
Let \(I\Subset\R\) and \(\varphi\in C_c^{\infty}(I)\). Set \(u_\varepsilon(t):=\log\big|\dettwo(I-A(\tfrac12+\varepsilon+it))\big|-\log\big|\xi(\tfrac12+\varepsilon+it)\big|\). Then there is \(C(\varphi)\) independent of \(\varepsilon\in(0,\varepsilon_0]\) such that
\[
 \Big|\int_{\R} \varphi(t)\,u_\varepsilon(t)\,dt\Big|\ \le\ C(\varphi),\qquad \Big|\int_{\R} \varphi(t)\,\big(u_\varepsilon(t)-u_\delta(t)\big)\,dt\Big|\ \le\ C(\varphi)\,|\varepsilon-\delta|.
\]
\end{lemma}

\begin{lemma}[Prime-power representation for \(\partial_\sigma\Re\log\dettwo\); unit local weights]\label{lem:pp-rep-det2}
Let \(A(s)\) be the prime-diagonal operator \(A(s)e_p:=p^{-s}e_p\) on \(\ell^2(\PP)\), with \(s=\sigma+it\) and \(\sigma>\tfrac12\). Then
\[
  \partial_\sigma\,\Re\log\dettwo\!\big(I-A(s)\big)
  \\= -\,\Re\sum_{p}\sum_{k\ge 2} c_{p,k}\,(\log p)\,p^{-k(\sigma+it)},\qquad c_{p,k}\equiv -1,
\]
so in particular \(|c_{p,k}|\le 1\) uniformly in \(p,k,\sigma\).
\end{lemma}
\begin{proof}
For \(\sigma>\tfrac12\) one has \(\|A(s)\|\le 2^{-\sigma}<1\), and the standard HS expansion holds:
\[
  \log\dettwo(I-A(s))\;=\;-\sum_{k\ge 2} \frac{\Tr(A(s)^k)}{k}\;=\;-\sum_{k\ge 2}\frac1k\sum_{p}p^{-ks},
\]
with absolute convergence. Differentiating termwise in \(\sigma\) (justified by absolute convergence of \(\sum_{k\ge 2}\sum_p (\log p)\,p^{-k\sigma}\)) gives
\[
  \partial_\sigma \log\dettwo(I-A(s))
  \\= -\sum_{k\ge 2}\frac1k\sum_p (-k\log p)\,p^{-ks}
  \\=\sum_{k\ge 2}\sum_p (\log p)\,p^{-ks}.
\]
Taking real parts yields the claim with \(c_{p,k}\equiv -1\).
\end{proof}
\begin{lemma}[Det$_2$ smoothed bound; uniform in \(\sigma\)]\label{lem:det2-smoothed-target}
Fix \(\varepsilon_0>0\) and a compact interval \(I\Subset\R\). Let \(\varphi\in C_c^2(I)\). For \(s=\sigma+it\) with \(\sigma\in(\tfrac12,\tfrac12+\varepsilon_0]\) one has the absolutely convergent expansion
\[
 \partial_\sigma\,\Re\log\dettwo\big(I-A(s)\big)
 \;=\; \sum_{k\ge 2}\sum_{p\in\PP} (\log p)\,p^{-k\sigma}\cos\big(k t\log p\big).
\]
Then there exists a finite constant (uniform in \(\sigma\in(\tfrac12,\tfrac12+\varepsilon_0]\))
\[
 C_*\ :=\ \sum_{p}\sum_{k\ge 2}\frac{p^{-k/2}}{k^2\,\log p}
\]
such that, uniformly for \(\sigma\in(\tfrac12,\tfrac12+\varepsilon_0]\),
\[
 \Big|\int_{\R} \varphi(t)\,\partial_\sigma\,\Re\log\dettwo\big(I-A(\sigma+it)\big)\,dt\Big|
 \ \le\ C_*\,\|\varphi''\|_{L^1(I)}.
\]
\end{lemma}

\begin{lemma}[Smoothed bound for the \(\xi\)-term; uniform in \(\sigma\)]\label{lem:xi-smoothed}
Fix \(\varepsilon_0>0\) and a compact interval \(I\Subset\R\). Let \(\varphi\in C_c^2(I)\) and \(s=\sigma+it\) with \(\sigma\in(\tfrac12,\tfrac12+\varepsilon_0]\). Then there exists a finite constant \(C_\xi(\varphi)\), independent of \(\sigma\) in this range, such that
\[
 \Big|\int_{\R}\varphi(t)\,\partial_\sigma\,\Re\log\xi(\sigma+it)\,dt\Big|\ \le\ C_\xi(\varphi).
\]
\end{lemma}
\begin{proof}
Write \(\xi(s)=\tfrac12 s(1-s)\,\pi^{-s/2}\Gamma(s/2)\,\zeta(s)\). Then
\[
 \partial_\sigma\,\Re\log\xi(s)\;=\;\partial_\sigma\,\Re\log\zeta(s)\; +\; \Re\frac{\psi(s/2)}{2}\; -\; \tfrac12\log\pi\; +\; \partial_\sigma\,\Re\log(s(1-s)),
\]
with \(\psi=\Gamma'/\Gamma\). On the compact strip \(\{\tfrac12<\sigma\le\tfrac12+\varepsilon_0,\ t\in I\}\) the last three terms are continuous in \((\sigma,t)\), so their \(\varphi\)–weighted integrals are bounded by \(C_0(\varphi)\) uniformly in \(\sigma\).

For \(\partial_\sigma\,\Re\log\zeta\), avoid prime-power expansions near the critical line. By Lemma~\ref{lem:xi-deriv-L1}, for each fixed \(\sigma\in[\tfrac12,\tfrac12+\varepsilon_0]\),
\[
 \int_I \Big|\partial_\sigma\,\Re\log\zeta(\sigma+it)\Big|\,dt\ \le\ C'_I.
\]
Since \(\varphi\in C_c^2(I)\subset L^\infty(I)\), it follows that
\[
 \Big|\int_{\R}\varphi(t)\,\partial_\sigma\,\Re\log\zeta(\sigma+it)\,dt\Big|\ \le\ \|\varphi\|_{L^\infty(I)}\,C'_I.
\]
Combining the archimedean bound with this estimate yields the claim with \(C_\xi(\varphi):=C_0(\varphi)+\|\varphi\|_{L^\infty(I)}C'_I\), uniformly for \(\sigma\in(\tfrac12,\tfrac12+\varepsilon_0]\).
\end{proof}
\begin{proof}
Expand \(\log\dettwo(I-A)\) as \(-\sum_{p}\sum_{k\ge2}p^{-ks}/k\) for \(\Re s>1\) and continue termwise to the open strip by testing against \(\varphi\in C_c^2(I)\). Differentiating in \(\sigma\) and taking real parts yields the exact series in the statement. Interchanging sum and integral is justified by absolute convergence on compact \(\sigma\)-intervals.
For each frequency \(\omega=k\log p\ge 2\log 2\), two integrations by parts give
\[
\Bigl|\int_{\R}\varphi(t)\cos(\omega t)\,dt\Bigr|\ \le\ \frac{\|\varphi''\|_{L^1(I)}}{\omega^2}.
\]
Hence
\[
\Bigl|\int \varphi(t)\,\partial_\sigma\Re\log\dettwo(I-A(\sigma+it))\,dt\Big|
\le \|\varphi''\|_{L^1}\sum_{p}\sum_{k\ge2}\frac{(\log p)\,p^{-k\sigma}}{(k\log p)^2}
\le \|\varphi''\|_{L^1}\sum_{p}\sum_{k\ge2}\frac{p^{-k/2}}{k^2\,\log p},
\]
uniformly for \(\sigma\in(\tfrac12,\tfrac12+\varepsilon_0]\), since the rightmost double series converges (the \(k\ge2\) tail gives \(p^{-k/2}\) and \(\sum_{p}(p\log p)^{-1}<\infty\)). This proves the claim.
\end{proof}

\begin{lemma}[Uniform \(\sigma\)-derivative $L^1$ bounds on short intervals]\label{lem:uniform-derivative-L1}
Fix a compact interval \(I\subset\R\) and \(\sigma\in[\tfrac12,\tfrac12+\varepsilon_0]\). Then
\[
 \int_I \Big|\partial_\sigma\,\Re\log\dettwo\big(I-A(\sigma+it)\big)\Big|\,dt\ \le\ C_I,
\]
uniformly in \(\sigma\), and
\[
 \int_I \Big|\partial_\sigma\,\Re\log\xi(\sigma+it)\Big|\,dt\ \le\ C'_I,
\]
uniformly in \(\sigma\).
\end{lemma}
\begin{proof}
For the det$_2$ term use Lemma~\ref{lem:det2-unsmoothed}, which gives
\(\int_I |\partial_\sigma\,\Re\log\dettwo(I-A(\sigma+it))|\,dt\le C_I\) uniformly for \(\sigma\in(\tfrac12,\tfrac12+\varepsilon_0]\).
For the \(\xi\) term use Lemma~\ref{lem:xi-deriv-L1}, yielding \(\int_I |\partial_\sigma\,\Re\log\xi(\sigma+it)|\,dt\le C'_I\) uniformly in \(\sigma\). This proves both displayed bounds.
\end{proof}

\begin{proposition}[Smoothed-to-unsmoothed Cauchy transfer]\label{prop:desmoothing}
Let \(u_\varepsilon\) be as above. For each compact \(I\Subset\R\) there exists \(C_I\) such that for all \(0<\delta<\varepsilon<\varepsilon_0\),
\[
 \|u_\varepsilon-u_\delta\|_{L^1(I)}\ \le\ C_I\,|\varepsilon-\delta|.
\]
\end{proposition}
\begin{proof}
By Lemma~\ref{lem:uniform-derivative-L1}, \(\int_I |\partial_\sigma u_\sigma(t)|\,dt\le C_I\) uniformly in \(\sigma\in[\delta,\varepsilon]\). Therefore, for \(0<\delta<\varepsilon\le\varepsilon_0\),
\[
 u_\varepsilon-u_\delta\ =\ \int_\delta^\varepsilon \partial_\sigma u_\sigma\,d\sigma,
\]
and Minkowski's integral inequality gives
\[
 \|u_\varepsilon-u_\delta\|_{L^1(I)}\ \le\ \int_\delta^\varepsilon\!\int_I |\partial_\sigma u_\sigma(t)|\,dt\,d\sigma\ \le\ C_I\,|\varepsilon-\delta|.
\]
\end{proof}

\begin{remark}
The uniform-in-\(\varepsilon\) boundary control (Theorem~\ref{thm:uniform-eps}) follows by testing the derivatives against compactly supported smooth \(\varphi\) and combining the smoothed bounds in Lemmas~\ref{lem:det2-smoothed-target} and~\ref{lem:xi-smoothed} with the de-smoothing Lemma~\ref{lem:desmoothing}.
\end{remark}

% ==== Phase–velocity identity and outer-phase lemma (added for unconditional boundary route) ====
\begin{lemma}[Outer phase is the Hilbert transform of the boundary modulus]\label{lem:outer-phase-HT}
Let \(\Omega=\{\Re s>\tfrac12\}\) and let \(O\) be an outer function on \(\Omega\) with a.e. boundary values on \(\Re s=\tfrac12\), whose boundary modulus is \(e^{u(t)}\), where \(u\in L^1_{\mathrm{loc}}(\R)\) and \(u\) has distributional derivative \(u'\) in \(\mathcal D'(\R)\). Then, in the sense of distributions on \(\R\), the boundary argument of \(O\) satisfies
\[
 \frac{d}{dt}\Arg O\!\left(\tfrac12+it\right)\;=\; \mathsf H\big[u'\big](t),
\]
where \(\mathsf H\) is the real-line Hilbert transform.
\end{lemma}
\begin{proof}
Write \(u(t)=\log|O(\tfrac12+it)|\). For an outer function on the half–plane, \(\log|O(\sigma+it)|\) is the Poisson extension of \(u\), and the boundary argument is the conjugate Poisson transform of \(u\); in particular, the boundary limit of the harmonic conjugate equals the Hilbert transform \(\mathsf H[u]\). Differentiating in the distribution sense and using that \(\tfrac{d}{dt}\,\mathsf H[f]=\mathsf H[f']\) on \(\mathcal D'(\R)\) gives
\[
 \frac{d}{dt}\Arg O\!\left(\tfrac12+it\right)\;=\; \mathsf H\big[u'\big](t).
\]
See Garnett, \emph{Bounded Analytic Functions} \cite[Ch.~II, §2 (Poisson integral), §5 (outer functions)]{Garnett} and Rosenblum–Rovnyak, \emph{Hardy Classes and Operator Theory} \cite[Ch.~2, §3]{RosenblumRovnyak} for the half–plane outer factorization and boundary conjugacy.
+ For the distributional identity $\tfrac{d}{dt}\,\mathsf H[f]=\mathsf H[f']$ on $\mathcal D'(\R)$, see, e.g., Stein–Weiss, \emph{Singular Integrals}, Ch.~II, or Grafakos, \emph{Classical Fourier Analysis}, Ch.~4.
\end{proof}
\begin{proposition}[Phase–velocity identity]\label{prop:phase-velocity-identity}
Let \(F(s):=\dettwo(I-A(s))/\xi(s)\) on \(\Omega\), and set \(u(t):=\log|F(\tfrac12+it)|\). Then for every nonnegative \(\phi\in C_c^\infty(\R)\),
\[
 \int_{\R}\!\phi(t)\,\Big(\Im\frac{\xi'}{\xi}-\Im\frac{\dettwo'}{\dettwo}+\mathsf H[u']\Big)\!\Big(\tfrac12+it\Big)\,dt
 \;=\; \sum_{\substack{\rho=\beta+i\gamma\\ \beta>\tfrac12}} 2\,\big(\beta-\tfrac12\big)\,\big(P_{\beta-\tfrac12}\!\ast\phi\big)(\gamma)
 \; +\; \pi\!\!\sum_{\substack{\gamma\in\R\\ \xi(\tfrac12+i\gamma)=0}} m_\gamma\,\phi(\gamma),
\]
where \(P_a(x)=\tfrac{1}{\pi}\tfrac{a}{a^2+x^2}\) and \(m_\gamma\) is the multiplicity of the critical-line zero at ordinate \(\gamma\). In particular, the right-hand side is nonnegative for all \(\phi\ge 0\).
\end{proposition}
\begin{proof}
Factor \(F=I\,O\) in \(\Omega\) into an inner part \(I\) (Blaschke over poles of \(F\) in \(\Omega\), i.e. zeros of \(\xi\) with \(\beta>\tfrac12\), together with a singular inner supported on critical-line zeros) and an outer part \(O\) with boundary modulus \(e^{u}\). By Lemma~\ref{lem:outer-phase-HT}, \(\tfrac{d}{dt}\Arg O(\tfrac12+it)=\mathsf H[u'](t)\) in \(\mathcal D'(\R)\). For a Blaschke factor at a pole \(\rho=\beta+i\gamma\) (\(\beta>\tfrac12\)), the boundary phase derivative equals \(-2(\beta-\tfrac12)\,P_{\beta-\tfrac12}(t-\gamma)\). Each critical-line zero contributes a delta mass \(-\pi m_\gamma\,\delta_\gamma\). Summing, we obtain
\[
 \frac{d}{dt}\Arg F(\tfrac12+it)\;=\; \mathsf H[u'](t)\;-
 \sum_{\beta>\tfrac12}2(\beta-\tfrac12)\,P_{\beta-\tfrac12}(t-\gamma)\;-
 \pi\sum_{\xi(\tfrac12+i\gamma)=0} m_\gamma\,\delta_\gamma.
\]
But \(\tfrac{d}{dt}\Arg F=\Im(F'/F)=\Im(\dettwo'/\dettwo)-\Im(\xi'/\xi)\) on the boundary. Rearranging and testing against \(\phi\ge 0\) yields the claimed identity and nonnegativity.
\end{proof}
\begin{lemma}[Boundary positive-real from smoothed route]\label{lem:boundary-posreal}
Assume the smoothed explicit-formula bounds of Lemmas~\ref{lem:det2-smoothed-target} and~\ref{lem:xi-smoothed} and the de-smoothing Lemma~\ref{lem:desmoothing}. If, in addition, the smoothed boundary distribution for \(\partial_\sigma\Re\log\big(\dettwo(I-A)/\xi\big)\) is nonnegative in the limit \(\varepsilon\downarrow 0\) when tested against nonnegative \(\varphi\in C_c^\infty(\R)\), then the boundary hypothesis \emph{(P+)} holds for \(\mathcal J=\dettwo(I-A)/(\mathcal O\,\xi)\).
\end{lemma}

\begin{remark}
Lemma~\ref{lem:boundary-posreal} isolates the precise point where the smoothed explicit-formula route must deliver a sign (positive real part) rather than mere $L^1$ bounds. This replaces earlier "outer is trivial" or boundary unimodularity claims for \(\Theta\).
\end{remark}

\begin{proposition}[Phase-variation test: (P+) forces holomorphy]\label{prop:Pplus-holomorphy}
Let \(\Omega=\{\Re s>\tfrac12\}\), \(F(s):=\dettwo(I-A(s))/\xi(s)\), and for \(t\in\R\) set
\[
 u(t):=\log\big|F(\tfrac12+it)\big|,\qquad
 \mathsf H[u']:=\text{the Hilbert transform of }u'(t).
\]
Then for every nonnegative \(\phi\in C_c^\infty(\R)\) one has
\[
\int_{\R}\!\phi(t)\,\Big(\Im\frac{\xi'}{\xi}-\Im\frac{\dettwo'}{\dettwo}+\mathsf H[u']\Big)\!\Big(\tfrac12+it\Big)\,dt
\;=\;\sum_{\substack{\rho=\beta+i\gamma\\ \Re\rho>\tfrac12}}\! 2(\beta-\tfrac12)\,\big(P_{\beta-\tfrac12}\!\ast\phi\big)(\gamma)
\; +\; \pi\!\!\sum_{\substack{\gamma\in\R\\ \xi(\tfrac12+i\gamma)=0}}\! m_\gamma\,\phi(\gamma),
\]
where \(P_a(x)=\frac{1}{\pi}\frac{a}{a^2+x^2}\) and \(m_\gamma\) is the multiplicity of the critical-line zero at \(t=\gamma\). In particular, the right-hand side is \(\ge 0\) for every \(\phi\ge 0\).
\end{proposition}
\begin{proof}
Factor \(F=I\,O\) on \(\Omega\) with \(O\) outer and \(I\) inner. By Lemma~\ref{lem:outer-phase-HT}, the boundary argument of \(O\) satisfies \(\frac{d}{dt}\Arg O(\tfrac12+it)=\mathsf H[u'](t)\) in \(\mathcal D'(\R)\). The inner factor \(I\) is the product of Blaschke terms for poles \(\rho=\beta+i\gamma\) of \(F\) in \(\Omega\) (zeros of \(\xi\) with \(\beta>\tfrac12\)) and a singular inner supported at ordinates \(\gamma\) with \(\xi(\tfrac12+i\gamma)=0\). For a pole at \(\rho\), the half-plane Blaschke factor \(C_\rho(s)=(s-\overline\rho)/(s-\rho)\) has
\[
\frac{d}{dt}\Arg C_\rho(\tfrac12+it)=-\,2(\beta-\tfrac12)\,P_{\beta-\tfrac12}(t-\gamma),
\]
and each critical-line zero contributes \(-\pi m_\gamma\,\delta_\gamma\) to the phase derivative. Summing gives
\[
\frac{d}{dt}\Arg F(\tfrac12+it)=\mathsf H[u'](t)
-\sum_{\substack{\rho=\beta+i\gamma\\ \Re\rho>\tfrac12}}\! 2(\beta-\tfrac12)\,P_{\beta-\tfrac12}(t-\gamma)
-\pi\!\!\sum_{\substack{\gamma\in\R\\ \xi(\tfrac12+i\gamma)=0}}\! m_\gamma\,\delta_\gamma.
\]
Since \(\frac{d}{dt}\Arg F=\Im(F'/F)=\Im(\dettwo'/\dettwo)\) on the boundary, rearranging and testing against \(\phi\ge 0\) yields the stated identity and positivity.
\end{proof}

\begin{proposition}[Local phase-cone certificate on \(I\)]
Fix a compact interval $I=[T_1,T_2]$ containing no ordinate \(\gamma\) with \(\xi(\tfrac12+i\gamma)=0\). Define
\[
 w(t):=\Arg\dettwo(\tfrac12+it)-\Arg\xi(\tfrac12+it)-\mathsf H[u](t),\qquad u(t):=\log|F(\tfrac12+it)|.
\]
Normalize $w$ by a unimodular constant so that $w(t_0)=0$ for some $t_0\in I$. Then $-w'$ is a nonnegative finite measure on $I$ and
\[
 \int_I (-w')\,dt=\sum_{\substack{\rho=\beta+i\gamma\\ \Re\rho>\tfrac12}}\! 2(\beta-\tfrac12)\Big[\arctan\frac{T_2-\gamma}{\beta-\tfrac12}-\arctan\frac{T_1-\gamma}{\beta-\tfrac12}\Big].
\]
In particular, if \(\displaystyle \int_I (-w')\,dt\le \pi/2\), then $w(t)\in[-\tfrac\pi2,\tfrac\pi2]$ for a.e. $t\in I$, and hence \(\Re\big(2\mathcal J(\tfrac12+it)\big)\ge 0\) a.e. on $I$ with \(\mathcal J=F/\mathcal O\).
\end{proposition}
\subsection*{Target (P+) via Carleson control of off-critical zeros}\label{subsec:Pplus-Carleson}
We isolate a sufficient condition for \emph{(P+)} in terms of a Carleson-type bound on the off-critical zero distribution.

\begin{definition}[Zero-side measure and Carleson boxes]
For each zero \(\rho=\beta+i\gamma\) of \(\xi\) with \(\beta>\tfrac12\), set \(a(\rho):=\beta-\tfrac12>0\). Define the discrete measure on the open half-plane \(\{\sigma>\tfrac12\}\)
\[
 \mu\ :=\ \sum_{\rho:\,\Re\rho>1/2}\ 2\,a(\rho)\,\delta_{(\tfrac12+a(\rho),\,\gamma)}.
\]
For an interval \(I=[T_1,T_2]\subset\R\), its Carleson (Whitney) box is
\[
 Q(I)\ :=\ \Big\{\, s=\sigma+it:\ 0<\sigma-\tfrac12<|I|,\ t\in I\,\Big\}.
\]
We say \(\mu\) has Carleson constant \(\mathsf C\) if \(\mu(Q(I))\le \mathsf C\,|I|\) for every bounded interval \(I\).
\end{definition}

\begin{theorem}[(P+) from Carleson control]\label{thm:Pplus-from-Carleson}
Assume the outer normalization of Subsection~\ref{subsec:Pplus-Carleson} so that \(\mathcal J=\dettwo(I-A)/(\mathcal O\,\xi)\) has a.e. boundary values with \(|\mathcal J(\tfrac12+it)|=1\). If the zero-side measure \(\mu\) has Carleson constant \(\mathsf C\le \pi/2\), then \emph{(P+)} holds:
\[
 \Re\big(2\,\mathcal J(\tfrac12+it)\big)\ \ge\ 0\quad\text{for a.e. }t\in\R.
\]
\end{theorem}
\begin{proof}
By Proposition~\ref{prop:phase-velocity-identity}, for nonnegative \(\phi\in C_c^\infty(I)\) one has
\[
 \int\!\phi\,\Big(\Im\frac{\xi'}{\xi}-\Im\frac{\dettwo'}{\dettwo}+\mathsf H[u']\Big)\Big(\tfrac12+it\Big)dt\ =\ \int_{\{\Re s>1/2\}} P_{s}\![\phi] \ d\mu(s)\ \ge 0,
\]
where \(P_{s}[\phi]\) denotes the Poisson extension to height \(\Re s-\tfrac12\) evaluated at \(\Im s\). The left-hand side equals \(\int_I \phi(t)\,(-w')\,dt\) with \(w\) the normalized phase mismatch (Proposition~\ref{prop:phase-velocity-identity}). Since \(\|P_{s}[\phi]\|_{L^\infty}\le 1\) and the Poisson kernel has unit \(t\)–mass, the Carleson bound yields
\[
 \int_I (-w')\,dt\ \le\ \mu(Q(I))\ \le\ (\pi/2)\,|I|.
\]
Normalizing \(\phi\) to approximate the indicator of \(I\) and dividing by \(|I|\), one obtains \(\int_I (-w')\le \pi/2\). By the phase-cone criterion this implies \(w\in[-\pi/2,\pi/2]\) a.e. on \(I\), hence \(\Re(2\mathcal J)\ge 0\) a.e. on \(I\). Exhaust \(\R\) by such intervals to conclude (P+). For background on this half-plane Poisson/Carleson-to-(P+) transfer see, e.g., Garnett \cite[Ch.~IV]{Garnett}.
\end{proof}
\begin{lemma}[Reduction to a short-interval Carleson bound]\label{lem:no-P}
Let \(I\subset\R\) be a bounded interval avoiding ordinates of critical-line zeros. If \(\mu(Q(I))\le \pi/2\), then \(\Re(2\mathcal J)\ge 0\) a.e. on \(I\). Consequently, if \(\mu\) has Carleson constant \(\le \pi/2\), then \emph{(P+)} holds a.e. on \(\R\).
\end{lemma}
\begin{proof}
The intervals \(I_T\) (together with finitely many intervals covering the bounded range \([0,T_0]\)) form a countable cover of \(\R\) up to the measure-zero set of critical-line ordinates. By Lemma~\ref{lem:no-P}, on each \(I_T\) we have \(\Re(2\mathcal J)\ge 0\) a.e. Taking the union yields (P+) a.e. on \(\R\).
\end{proof}

\begin{lemma}[Littlewood bound \(\Rightarrow\) adaptive short-interval mass]\label{lem:littlewood-adaptive}
Let \(S(T):=\sum_{0<\gamma\le T,\ \beta>1/2}(\beta-\tfrac12)\). Suppose there exists \(C_L>0\) with \(S(T)\le C_L\,\log(2+T)\) for all \(T\ge 0\) (classical Littlewood-type bound). Then there exist constants \(c>0\) and \(T_0\ge 1\) such that, for \(L(T):=c/\log(2+T)\) and \(I_T=[T-L(T),T+L(T)]\), one has
\[\mu\big(Q(I_T)\big)\ \le\ \frac{\pi}{2}\qquad (T\ge T_0).\]
\end{lemma}
\begin{proof}
By definition, \(\mu(Q(I_T))=\sum_{\substack{\gamma\in I_T\\ 0<\beta-\tfrac12< L(T)}} 2(\beta-\tfrac12)\ \le\ 2\sum_{\substack{\gamma\in I_T\\ \beta>1/2}} (\beta-\tfrac12)\). The latter is bounded by the telescoping difference \(2\big(S(T+L(T)) - S(T-L(T))\big)\). Using the hypothesis, for all large \(T\),
\[
 \mu(Q(I_T))\ \le\ 2C_L\,\Big(\log(2+T+L(T)) - \log(2+T-L(T))\Big)
 \ \le\ \frac{4C_L\,L(T)}{2+T-L(T)}\ \le\ \frac{4C_L\,c}{T\,\log(2+T)}.
\]
Choose \(T_0\) so that \(\frac{4C_L c}{T_0\,\log(2+T_0)}\le \pi/2\); then for all \(T\ge T_0\) the same inequality holds with \(T\) in place of \(T_0\). This proves the claim.
\end{proof}
\begin{corollary}[(P+) under Littlewood bound]\label{cor:Pplus-Littlewood}
Assume the outer normalization of Subsection~\ref{subsec:Pplus-Carleson} and the Littlewood-type bound in Lemma~\ref{lem:littlewood-adaptive}. Then \emph{(P+)} holds a.e. on \(\R\).
\end{corollary}
\begin{proof}
Apply Lemma~\ref{lem:littlewood-adaptive} and Corollary~\ref{cor:adaptive-cover}, adding finitely many short intervals to cover \([0,T_0]\).
\end{proof}

\begin{theorem}[Global Schur/PSD and RH under Littlewood bound]\label{thm:global-RH-Littlewood}
Under the hypotheses of Corollary~\ref{cor:Pplus-Littlewood}, \(2\mathcal J\) is Herglotz on \(\Omega\) by Poisson, and thus \(\Theta=(2\mathcal J-1)/(2\mathcal J+1)\) is Schur on \(\Omega\). Consequently, by Theorem~\ref{thm:brf-rh-final}, RH holds.
\end{theorem}

\begin{remark}[Historical note]
Earlier drafts recorded a short-interval Poisson mass conjecture for context; it is omitted here to avoid ambiguity. The PSC constants are fully printed in the appendices and used in the certificate.
\end{remark}

\begin{remark}[Pick-matrix discretization]\label{rem:pick-certificate}
Equivalently, fix nodes $s_j=\tfrac12+\sigma+i t_j$ with $t_j\in I$ and $\sigma>0$. Positivity of the half-plane Pick matrix \(\big((1-\Theta(s_j)\overline{\Theta(s_k)})/(s_j+\overline{s_k}-1)\big)_{j,k}\) for arbitrarily fine grids and $\sigma\downarrow 0$ is equivalent to the phase-cone on $I$.
\end{remark}

\subsection{Global damping/weighting for alignment (Schur-test formulation)}\label{subsec:global-damping}
As an orthogonal route to compact-by-compact tuning, one may introduce a single global diagonal weight \(D(s)\) and a fixed damping factor \(\eta\in(0,1)\) to obtain \(K\)-independent Schur bounds via the Schur test. In kernel form, if the off-diagonal envelope enjoys either exponential tails \(|K(x,y)|\lesssim e^{-\gamma d(x,y)}\) or polynomial tails \(|K(x,y)|\lesssim (1+d(x,y))^{-\beta}\) on a doubling space of dimension \(n\), then one can choose weights
\[
 D(s)f(x)=e^{\,\sigma\,d(x,x_0)}f(x)\quad\text{or}\quad D(s)f(x)=(1+d(x,x_0))^{\sigma} f(x)
\]
with \(\sigma\) below a tail-dependent threshold, so that the conjugated operator \(D(-s)\,T\,D(s)\) is uniformly bounded on \(L^p\) for a given \(p\). Picking \(\eta=(1-\varepsilon)/\|D(-s)TD(s)\|_{p\to p}\) then yields a global contraction bound independent of compacts. This supplies a single, globally defined "Schur sequence" without per-compact parameter choices.

\subsection{Cayley-difference control on compacts}\label{subsec:Cayley-difference}
We record a simple inequality linking differences after the Cayley transform to differences before it.

\begin{lemma}[Cayley-difference bound]\label{lem:Cayley-diff}
Let \(K\subset\Omega\) be compact. Suppose \(H_1,H_2\) are holomorphic on a neighborhood of \(K\) and satisfy \(\inf_{s\in K}|H_j(s)+1|\ge \delta_K>0\) and \(\sup_{s\in K}|H_j(s)|\le M_K\) for \(j=1,2\). Define \(\Theta_j=(H_j-1)/(H_j+1)\). Then there exists \(C_K>0\) depending only on \((\delta_K,M_K)\) such that
\[
 \sup_{s\in K}\,\big|\Theta_1(s)-\Theta_2(s)\big|\ \le\ C_K\,\sup_{s\in K}\,\big|H_1(s)-H_2(s)\big|.
\]
In particular, on any \(K\subset\Omega\) where \(H_N^{(\mathrm{Schur})}\) and \(H_N^{(\dettwo)}\) share uniform bounds away from \(-1\), the convergence \(H_N^{(\mathrm{Schur})}\to H_N^{(\dettwo)}\) implies \(\Theta_N^{(\mathrm{Schur})}\to \Theta_N^{(\dettwo)}\) uniformly on \(K\).
\end{lemma}
\begin{remark}
Uniform bounds away from \(-1\) on a compact \(K\subset\Omega\) follow for large \(N\) from lower bounds on \(|\xi|\) off its zeros and continuity of \(\dettwo(I-A_N)\) in the HS topology; hence the lemma applies on each such \(K\).
\end{remark}

\begin{lemma}[Away from \(-1\) on zero-free compacts]\label{lem:away-minus-one}
Let \(K\subset\Omega\) be compact with \(\inf_{K}|\xi|\ge \delta_K>0\). Then there exists \(c_K>0\) and \(N_0\) such that for all \(N\ge N_0\),
\[
 \inf_{s\in K}\,\bigl| H_N^{(\dettwo)}(s)+1\bigr|\ \ge\ c_K,
\]
and likewise \(\inf_{s\in K}|H(s)+1|\ge c_K\). In particular, the denominators in Lemma~\ref{lem:Cayley-diff} are uniformly bounded away from zero on \(K\) for \(N\ge N_0\).
\end{lemma}
\begin{proof}
Since \(\|A(s)\|\le 2^{-\Re s}<1\) on \(\Omega\), \(I-A(s)\) is invertible on \(\Omega\) and \(\dettwo(I-A(s))\ne 0\). Continuity of \(\dettwo(I-A(s))\) on \(K\) implies \(m_K:=\inf_{s\in K}|\dettwo(I-A(s))|>0\). HS continuity (Proposition~\ref{prop:HS-to-det2}) gives uniform convergence \(\dettwo(I-A_N)\to \dettwo(I-A)\) on \(K\), hence for \(N\ge N_0\), \(\inf_{s\in K}|\dettwo(I-A_N(s))|\ge m_K/2\). Therefore on \(K\),
\[
 |H_N^{(\dettwo)}+1|\;=\;\frac{2\,|\dettwo(I-A_N)|}{|\xi|}\;\ge\;\frac{m_K}{\delta_K}\;=:\;c_K,
\]
and similarly for \(H\).
\end{proof}
\begin{proof}
Compute
\[
 \Theta_1-\Theta_2\;=\;\frac{H_1-1}{H_1+1}-\frac{H_2-1}{H_2+1}
 \;=\;\frac{2\,(H_1-H_2)}{(H_1+1)(H_2+1)}.
\]
Hence on \(K\),
\[
 |\Theta_1-\Theta_2|\ \le\ \frac{2}{\delta_K^2}\,|H_1-H_2|.
\]
Choosing \(C_K=2/\delta_K^2\) suffices; if desired, one can refine \(C_K\) using \(M_K\) to control numerators/denominators uniformly.
\end{proof}

\section{Main theorem (formal statement and proof)}\label{sec:main-theorem}
We now assemble the ingredients into a single statement tailored to the prime-grid construction.

\begin{theorem}[Prime-grid BRF via alignment]\label{thm:prime-grid-BRF}
Let \(\Omega=\{\Re s>\tfrac12\}\) and define the prime-diagonal block \(A(s)e_p:=p^{-s}e_p\). Let
\[
 H(s)\;:=\;2\,\frac{\dettwo(I-A(s))}{\xi(s)}-1,\qquad \Theta\;:=\;\frac{H-1}{H+1}.
\]
For each \(N\in\N\), let \(\Phi_N(s)=D_N+C_N(sI-A_N)^{-1}B_N\) be the prime-grid lossless transfer of Proposition~\ref{prop:prime-grid-KYP}, and fix unit vectors \(u_N,v_N\in\C^N\). Define the scalar Schur function \(\widehat\Theta_N(s):=v_N^*\,\Phi_N(s)\,u_N\). Suppose there exists, for each compact \(K\subset\Omega\), a sequence of scalar lossless Schur functions \(\Psi_{N,K}\) such that
\begin{equation}\label{eq:uniform-alignment}
 \sup_{s\in K}\ \big|\Psi_{N,K}(s)\,\widehat\Theta_N(s)\; -\; \Theta_N^{(\dettwo)}(s)\big|\xrightarrow[N\to\infty]{}0,
\end{equation}
where \(\Theta_N^{(\dettwo)}(s)=(H_N^{(\dettwo)}-1)/(H_N^{(\dettwo)}+1)\) with \(H_N^{(\dettwo)}:=2\,\dettwo(I-A_N)/\xi-1\). Then \(\Theta\) is Schur on \(\Omega\), and hence \(H\) is Herglotz on \(\Omega\) (the BRF conclusion).
\end{theorem}
\begin{proof}
By Proposition~\ref{prop:HS-to-det2} and the division remark, \(H_N^{(\dettwo)}\to H\) locally uniformly on compact subsets avoiding zeros of \(\xi\). Continuity of the Cayley map on compacta away from \(\{-1\}\) implies that the Cayley transforms also converge locally uniformly on such compacts, i.e. \(\Theta_N^{(\dettwo)}\to\Theta\). For each compact \(K\), the hypothesis \eqref{eq:uniform-alignment} provides Schur functions \(\Theta_{N,K}:=\Psi_{N,K}\,\widehat\Theta_N\) such that \(\Theta_{N,K}\to\Theta\) uniformly on \(K\). Each \(\Theta_{N,K}\) is Schur as a product of Schur functions; by Corollary~\ref{cor:closure}, the locally uniform limit \(\Theta\) is Schur on \(\Omega\). Applying Theorem~\ref{thm:SC-equivalences} completes the proof.
\end{proof}
\begin{remark}[Realizing the alignment]
Condition \eqref{eq:uniform-alignment} can be arranged by the boundary matching strategy of Section~\ref{sec:practical-alignment}: choose, for an exhaustion by compacts \(K_m\nearrow\Omega\), NP interpolation nodes \(\{s_{j}^{(m,N)}\}\subset K_m\) and lossless interpolants \(\Psi_{N,K_m}\) such that the product \(\Psi_{N,K_m}\,\widehat\Theta_N\) agrees with \(\Theta_N^{(\dettwo)}\) on these nodes and shares the feedthrough normalization. Boundedness and normal-family arguments then promote pointwise agreement on dense sets to uniform convergence on \(K_m\), and a diagonal extraction yields local-uniform convergence on \(\Omega\).
\end{remark}
\section{Practical alignment strategies}\label{sec:practical-alignment}
We outline two standard mechanisms to realize the alignment hypothesis in Proposition~\ref{prop:alignment-criterion} while preserving passivity (Schurness) at each finite stage.
\subsection{Boundary matching via Nevanlinna--Pick interpolation}
Fix a compact \(K\subset\Omega\). Let \(\{s_j\}_{j=1}^{M}\subset K\) be distinct interpolation nodes and let \(\{\gamma_j\}_{j=1}^{M}\subset\C\) be target values with \(|\gamma_j|<1\). The classical Nevanlinna--Pick theorem on the half-plane guarantees existence of Schur functions \(\Psi\) with \(\Psi(s_j)=\gamma_j\), and the set of such interpolants contains rational inner (lossless) functions of degree at most \(M\).

\begin{lemma}[Lossless NP interpolation]\label{lem:NP-lossless}
Given data \(\{(s_j,\gamma_j)\}_{j=1}^{M}\) with \(\Re s_j>\tfrac12\) and \(|\gamma_j|<1\), there exists a rational inner function \(\Psi\) on \(\Omega\) of McMillan degree at most \(M\) that interpolates the data. Moreover, \(\Psi\) admits a lossless realization \(\Psi(s)=D_\Psi+C_\Psi(sI-A_\Psi)^{-1}B_\Psi\) with a positive definite solution of the lossless equalities \eqref{eq:lossless-equalities}.
\end{lemma}
\begin{proof}
By mapping \(\Omega\) conformally to the unit disk and invoking the disk NP theorem, one obtains an inner finite Blaschke product solving the interpolation. Realization theory for inner functions (Potapov--de Branges--Rovnyak; state-space proofs via Schur algorithm) yields a lossless colligation.
\end{proof}

% ===== Corrected alignment lemma on compacts (uses only standard Schur approximation) =====
\paragraph{Setup for alignment on compacts.}
Let \(\Omega:=\{\Re s>\tfrac12\}\) and define the half-plane Cayley map
\[
  \phi: \Omega\to\mathbb D,\qquad \phi(s):=\frac{s-\tfrac32}{\,s+\tfrac12\,}.
\]
Fix a compact set \(K\subset\Omega\) and choose \(r\in(0,1)\) with \(\phi(K)\subset\{z:|z|\le r\}\).

\begin{lemma}[Lossless alignment on compacts (corrected)]\label{lem:alignment-corrected}
For each \(N\in\N\), let \(F_N,G_N\) be Schur functions on \(\Omega\). Assume:
\begin{itemize}
  \item[(H1)] There is an open set \(U\) with \(K\subset U\subset\Omega\) such that \(\inf_{s\in U}|G_N(s)|\ge \delta_K>0\) (no zeros of \(G_N\) near \(K\)).
  \item[(H2)] The ratio \(h_N(z):=F_N(\phi^{-1}(z))\,/\,G_N(\phi^{-1}(z))\) extends holomorphically to the whole unit disk \(\mathbb D\) and satisfies \(|h_N(z)|\le 1\) for all \(z\in\mathbb D\).
\end{itemize}
Then for every \(\varepsilon\in(0,1)\) there exists a lossless scalar \(\Psi_{N,K,\varepsilon}\) on \(\Omega\) such that
\[
  \sup_{s\in K}\,\bigl|\Psi_{N,K,\varepsilon}(s)\,G_N(s)-F_N(s)\bigr|\ \le\ \varepsilon.
\]
Moreover, one may take \(\Psi_{N,K,\varepsilon}(s)=B_m(\phi(s))\) with a finite Blaschke product \(B_m\) of degree \(m\) chosen so that
\[
  \sup_{s\in K}\,\bigl|\Psi_{N,K,\varepsilon}(s)\,G_N(s)-F_N(s)\bigr|\ \le\ 2\,r^{\,m+1},
\]
and any \(m\ge \bigl\lceil {\log(\varepsilon/2)}/{\log r}\bigr\rceil\) suffices.
\end{lemma}
\begin{proof}
By (H1)–(H2), \(h_N\) is Schur on \(\mathbb D\). Let \(B_m\) be the degree-\(m\) Schur/Carath\'eodory–Fej\'er approximant to \(h_N\) at the origin; equivalently, a finite Blaschke product whose Taylor series at 0 matches that of \(h_N\) up to order \(m\). The difference \(g_m:=h_N-B_m\) is holomorphic on \(\mathbb D\), \(|g_m|\le 2\), and vanishes to order \(m{+}1\) at 0, so by the higher-order Schwarz lemma, \(|g_m(z)|\le 2|z|^{m+1}\) for \(|z|<1\). Thus for \(s\in K\), \(|\phi(s)|\le r\) and
\[
  \bigl|B_m(\phi(s))\,G_N(s)-F_N(s)\bigr|\ =\ |g_m(\phi(s))|\,|G_N(s)|\ \le\ 2\,r^{m+1},
\]
since \(|G_N|\le 1\) on \(\Omega\). Choosing \(m\) as stated yields the claim with \(\Psi_{N,K,\varepsilon}(s):=B_m(\phi(s))\).
\end{proof}

\begin{corollary}[Alignment for \(\Theta\)-models]\label{cor:ThetaAlignment-corrected}
Let \(F_N:=\Theta^{(\operatorname{det}_2)}_N\) and \(G_N:=\widehat{\Theta}_N\). If \emph{(H1)}–\emph{(H2)} hold on \(K\), then for every \(\varepsilon\in(0,1)\) there exists a lossless scalar \(\Psi_{N,K,\varepsilon}\) with
\[
  \sup_{s\in K} \bigl| \Psi_{N,K,\varepsilon}(s)\,\widehat{\Theta}_N(s)-\Theta^{(\operatorname{det}_2)}_N(s) \bigr| \le \varepsilon.
\]
\end{corollary}

\begin{remark}[On verifying (H2)]
A sufficient condition for \emph{(H2)} is: there exists a Schur function \(Q_N\) on \(\Omega\) with \(F_N=Q_N\,G_N\) on \(\Omega\). Then \(h_N=Q_N\circ\phi^{-1}\) is Schur on \(\mathbb D\). Alternatively, if \(G_N\) is zero-free on \(\Omega\) and \(|F_N(s)|\le |G_N(s)|\) holds for all \(s\in\Omega\), then \(h_N\) is Schur on \(\mathbb D\).
\end{remark}

\begin{remark}[Cayley safety for BRF]
If additionally \(\inf_{s\in K}|1+\Theta^{(\operatorname{det}_2)}_N(s)|\ge c_K>0\) and \(\inf_{s\in K}|1+\widehat{\Theta}_N(s)|\ge c_K>0\), then the Cayley map \(H=(1+\Theta)/(1-\Theta)\) is uniformly bi-Lipschitz on \(K\), simplifying the BRF limit passage. This is not needed for Lemma~\ref{lem:alignment-corrected}.
\end{remark}

\subsection{Interior H$^\infty$ alignment via passive approximants}\label{subsec:hinf-passive}
We record a quantitative H$^\infty$ scheme that yields uniform-on-compact alignment on rectangles strictly inside \(\Omega\), avoiding any \(\varepsilon\downarrow 0\) limits.

\begin{lemma}[HS-tail \(\Rightarrow\) det$_2$ variation on rectangles]\label{lem:HS-tail-rectangle}
Let \(R^\sharp=\{\sigma_0\le \Re s\le \sigma_1,\ |\Im s|\le T\}\Subset\Omega\) with \(\sigma_0>\tfrac12\). Then
\[
 \sup_{s\in R^\sharp}\big|\log\dettwo(I-A(s))\!-\!\log\dettwo(I-A_N(s))\big|\ \le\ C(R^\sharp)\Big(\sum_{p>p_N}p^{-2\sigma_0}\Big)^{1/2}.
\]
\end{lemma}
\begin{corollary}[Global Schur bound on \(\Omega\)]\label{cor:closure}
Let \(\Omega':=\Omega\setminus S\) with \(S\) discrete. Suppose that for every compact \(K\Subset\Omega'\) there exist Schur functions \(\Theta_{K,M}\) with \(\Theta_{K,M}\to\Theta\) locally uniformly on \(K\). Then \(\Theta\) is Schur on \(\Omega'\), extends holomorphically to \(\Omega\) with \(|\Theta|\le 1\) there, and the set \(P:=\{s\in\Omega: 2J(s)=-1\}\) is empty.
\end{corollary}
\begin{proof}
By hypothesis and Corollary~\ref{cor:closure}, \(\Theta\) is Schur on \(\Omega'\). Apply Lemma~\ref{lem:no-P} to extend across \(S\) and eliminate \(P\). 
\end{proof}
\begin{corollary}[Adaptive cover on bounded ranges]\label{cor:adaptive-cover}
Let \(\{I_j\}\) be a finite cover of a bounded $t$-range by intervals avoiding critical-line ordinates. If \(\Theta\) is Schur on each \(I_j\) in the sense of boundary values, then \(\Theta\) is Schur on their union.
\end{corollary}
\begin{theorem}[Interior completion on zero-free rectangles; conditional globalization]\label{thm:global-PSD}
With \(J=\dettwo(I-A)/\xi\) and \(\Theta=(2J-1)/(2J+1)\) as above, the interior passive \(H^\infty\) approximation (Proposition~\ref{prop:hinf-passive}), the local-uniform convergence of \(\Theta_N^{(\dettwo)}\to\Theta\) off \(Z(\xi)\) (Lemma~\ref{lem:cayley-cont}), and Theorem~\ref{thm:UIC} show: \(\Theta\) is Schur on \(\Omega\setminus Z(\xi)\) and extends holomorphically across isolated points under a separate boundary positivity input (e.g., (P+) or an equivalent PSD statement). In particular, a global Schur bound on \(\Omega\) requires (P+). 
\end{theorem}
\begin{proof}
Fix a compact \(K\Subset\Omega':=\Omega\setminus Z(\xi)\). By Proposition~\ref{prop:hinf-passive}, for each \(N\) there exist Schur rationals \(\Theta_{N,M}\) with \(\Theta_{N,M}\to\Theta_N^{(\dettwo)}\) uniformly on \(K\) as \(M\to\infty\). By Lemma~\ref{lem:cayley-cont} and HS\(\to\)det$_2$ continuity, \(\Theta_N^{(\dettwo)}\to\Theta\) uniformly on \(K\) as \(N\to\infty\). A diagonal choice \((N_k,M_k)\) yields a sequence of Schur functions converging to \(\Theta\) locally uniformly on \(K\); exhausting \(\Omega'\) and applying Theorem~\ref{thm:UIC} shows \(\Theta\) extends holomorphically to \(\Omega\) with \(|\Theta|\le 1\).
If \(\xi(\rho)=0\) for some \(\rho\in\Omega\), then \(J\) has a pole at \(\rho\) and \(\Theta\to 1\) as \(s\to\rho\). Since \(\Theta\) is holomorphic on \(\Omega\) with \(|\Theta|\le 1\), the maximum modulus principle forces \(\Theta\) to be constant; asymptotics \(\Theta(\sigma+it)\to -1\) as \(\sigma\to+\infty\) exclude this. Hence \(\xi\) has no zeros in \(\Omega\). By the functional equation, RH follows.
\end{proof}
\begin{proposition}[H$^\infty$ passive approximation on rectangles]\label{prop:hinf-passive}
Let \(R^\sharp=\{\sigma_0\le \Re s\le \sigma_1,\ |\Im s|\le T\}\Subset\Omega\) be a rectangle with \(\sigma_0>\tfrac12\), and let \(R\Subset R^\sharp\) and \(K\Subset R\). Then there exist lossless (Schur) rational transfers \(\Theta_{N,M}\) such that, for some \(C(R,R^\sharp)>0\) and \(\rho\in(0,1)\),
\[
  \sup_{s\in K}\,\big|\Theta_{N,M}(s)-g_N(s)\big|\ \le\ C(R,R^\sharp)\,\rho^{M},
\]
uniformly in \(N\). In particular, taking \(M\to\infty\) yields Schur rational approximants converging uniformly on \(K\).
\end{proposition}
\begin{proof}
Map \(R^\sharp\) conformally to the unit disk \(\mathbb D\) and transport \(g_N\) to a holomorphic function \(h\) on a neighborhood of \(\overline{\mathbb D}\) with \(\|h\|_{L^\infty(\partial\mathbb D)}\le M_0\). By classical rational approximation on analytic curves, there exist rational functions \(r_M\) with poles off \(\overline{\mathbb D}\) such that
\[
 \sup_{\partial\mathbb D}|r_M-h|\ \le\ C\,\rho^M,\qquad 0<\rho<1.
\]
Fix \(M_1>\max(1,M_0)\) and apply the Schur algorithm to \(r_M/M_1\): after \(m\) steps it produces a rational Schur function \(s_{M,m}\) (a finite Schur–continued–fraction/Blaschke transfer) with
\[
 \sup_{\partial\mathbb D}\big|s_{M,m}-r_M/M_1\big|\ \le\ C'\,\rho^m.
\]
Choosing \(m\asymp M\) and setting \(s_M:=s_{M,m(M)}\) gives a rational Schur \(s_M\) satisfying
\[
 \sup_{\partial\mathbb D}\big|M_1 s_M-h\big|\ \le\ C''\,\rho^M.
\]
Pull back \(M_1 s_M\) to \(\partial R\) via the conformal map to obtain a Schur function \(\Theta_{N,M}\) on \(\partial R\) with
\[
 \sup_{\partial R}\,|\Theta_{N,M}-g_N|\ \le\ C(R,R^\sharp)\,\rho^M.
\]
By the maximum principle (applied after mapping back to the half-plane), the same bound holds on \(K\Subset R\). The Schur property is preserved by the Schur algorithm and by the Möbius equivalence between the disk and half-plane, so each \(\Theta_{N,M}\) is lossless (Schur) as claimed.
\end{proof}
\begin{corollary}[Uniform-on-$K$ alignment on rectangles]\label{prop:alignment-criterion}
With \(K\Subset R\Subset R^\sharp\Subset\Omega\) as above, for any \(\varepsilon>0\) choose \(N\) so that \(\sup_R|\Theta_N^{(\dettwo)}-\Theta^{(\dettwo)}|\le \varepsilon/2\), then choose \(M\) with \(C\rho^M\le \varepsilon/2\). Then
\[
 \sup_{K}|\Theta_{N,M}-\Theta^{(\dettwo)}|\ \le\ \varepsilon.
\]
Each \(\Theta_{N,M}\) is Schur (lossless), so kernels are PSD at every finite stage.
\end{corollary}

\paragraph{Globalization by exhaustion.}
Let \(\{R_m\}\) be an increasing exhaustion of \(\Omega\) by rectangles with \(K_m\Subset R_m\Subset R_m^\sharp\Subset\Omega\) and \(\bigcup_m K_m=\Omega\). For each \(m\), choose \(N(m)\) so that \(\sup_{R_m}|\Theta_{N(m)}^{(\dettwo)}-\Theta^{(\dettwo)}|\le 2^{-m-1}\) and then choose \(M(m)\) so that \(C(R_m,R_m^\sharp)\,\rho^{M(m)}\le 2^{-m-1}\). By Corollary~\ref{cor:interior-alignment},
\[
 \sup_{K_m}|\Theta_{N(m),M(m)}-\Theta^{(\dettwo)}|\ \le\ 2^{-m}.
\]
A diagonal extraction yields a sequence of Schur functions converging to \(\Theta^{(\dettwo)}\) locally uniformly on \(\Omega\).
\begin{proposition}[Alignment by cascaded lossless factors]\label{prop:cascade}
Let \(\Phi_N\) be any matrix-valued lossless Schur transfer (e.g., the prime-grid lossless model from Proposition~\ref{prop:prime-grid-KYP}) and let \(\Psi_N\) be a scalar lossless interpolant from Lemma~\ref{lem:NP-lossless} matching \(\Theta_N^{(\dettwo)}\) at nodes \(\{s_j\}_{j=1}^{M(N)}\subset K\). Then the cascade (series connection)
\[
 \Theta_N\;:=\;\Psi_N\,\big(v_N^*\,\Phi_N\,u_N\big),\qquad \|u_N\|=\|v_N\|=1,
\]
is Schur on \(\Omega\), matches the interpolation values, and remains rational inner. Choosing \(M(N)\to\infty\) and nodes dense in \(K\), one obtains \(\Theta_N\to \Theta\) uniformly on \(K\).
\end{proposition}
\begin{proof}
Schur functions are closed under products and under pre/post-multiplication by contractions; lossless (inner) functions remain inner under cascade. Interpolation at finitely many points is preserved. Normal-family compactness plus uniqueness on a dense set (identity theorem) yields uniform convergence on \(K\).
\end{proof}
\subsection{Asymptotic control at infinity}
On vertical lines \(\{\Re s=\sigma\}\) with \(\sigma>\tfrac12\), Stirling estimates imply \(\xi(s)\to\infty\) and hence \(H(s)\to -1\) rapidly as \(|\Im s|\to\infty\). Prime-grid lossless models share the exact feedthrough \(-1\) (after scalar port extraction), so one may combine this with the boundedness \(|\Theta_N|\le 1\) and Cauchy integral representations on large rectangles to deduce smallness of the difference \(\Theta_N-\Theta_N^{(\dettwo)}\) provided agreement on a finite boundary grid, as in the previous subsection.
\begin{remark}[Tiny slack variant]
If one relaxes losslessness to allow a vanishing slack \(E_N\succeq 0\) in \(A^*P+PA+C^*C=-E_N\) (and \(D^*D\preceq I\)), the prime-grid template admits a scaling of \(C_N\) that suppresses the \(s^{-1}\) moment in the expansion of \(H_N\), aligning the asymptotics of \(H_N^{(\mathrm{LBR})}\) with those of \(H_N^{(\dettwo)}\). The bounded-real inequality \eqref{eq:KYP} remains valid, and the slack can be sent to zero along the sequence.
\end{remark}

\section{Related work}\label{sec:related}
This work draws on several classical strands. On the operator side, the theory of trace ideals and regularized determinants (notably the Carleman--Fredholm \(\det_2\)) is treated comprehensively in Simon \cite{SimonTraceIdeals}. Realization theory for Schur/inner functions and passive colligations goes back to Potapov's school and is surveyed in de Branges--Rovnyak \cite{deBrangesRovnyak}, Dym--Gohberg \cite{DymGohberg}, and Sz.-Nagy--Foia\c{s} \cite{SzNagyFoias}. Nevanlinna--Pick interpolation on the disk/half-plane and its inner (lossless) solutions are standard topics in complex function theory and H\(\infty\) control; see Garnett \cite{Garnett} and Rosenblum--Rovnyak \cite{RosenblumRovnyak}. The BRF lemmas used here are classical in systems theory and appear in many sources.

From the analytic number theory perspective, our decomposition mirrors the partition of Euler product contributions according to prime powers: the \(k\ge 2\) terms are naturally accommodated by the \(\det_2\) expansion, while the \(k=1\) (prime) terms, together with archimedean factors and the polynomial \(s(1-s)\), are placed in a finite auxiliary block. While our argument operates at the level of truncations and functional-analytic closure, it is compatible with traditional expansions of \(\log \zeta\) and the analytic properties encoded by the completed zeta \(\xi\); for standard references on Stirling/digamma bounds and the explicit formula see Titchmarsh \cite{TitchmarshZeta}, Edwards \cite{Edwards}, and Iwaniec--Kowalski \cite{IwaniecKowalski}.

\section{Discussion and outlook}\label{sec:discussion}
We presented an operator-theoretic BRF program for RH combining Schur--determinant splitting, HS\(\to\)\(\dettwo\) continuity, and explicit finite-stage passive constructions tied to the primes. Two routes were considered historically: an interior alignment route on zero-free rectangles via passive $H^\infty$ approximation, and a boundary route via a PSC certificate. In the present proof we proceed via the PSC boundary route; Bridges A--C/Schur-covering are included as an optional companion perspective.
\paragraph{Role of the interior route.}
The Gram/Fock interior route provides rectangle positivity (Herglotz/Schur) without Schur-test absolute-sum bounds; it supports interior control but is not needed for the final boundary closure here.
Potential refinements include: (i) quantitative rational approximation on analytic boundaries with lossless KYP constraints; (ii) strengthened explicit-formula estimates sufficient for $L^1_{\mathrm{loc}}$ cancellation of zero spikes; (iii) exploring alternative finite-block architectures for $k=1$ with improved global control; and (iv) extensions to matrix-valued settings and other $L$-functions.

\section{Limitations and scope}\label{sec:limitations}
Two routes close the BRF conclusion. The boundary route is completed by Theorem~\ref{thm:uniform-eps} (uniform $L^1_{\mathrm{loc}}$ control) proved via a smoothed explicit-formula route and de-smoothing (Subsection~\ref{subsec:smoothed-explicit}), together with outer/inner factorization and an inner-compensator (Subsection~\ref{subsec:bl-compensator}). The finite-stage route delivers quantitative, noncircular alignment on compact sets strictly inside \(\Omega\) by H$^\infty$ passive approximation (Subsection~\ref{subsec:hinf-passive}).

\section{Examples: small-$N$ prime-grid models}\label{sec:examples}
We record explicit instances of the prime-grid lossless specification (Proposition~\ref{prop:prime-grid-KYP}). Throughout, for a prime \(p\) set
\[
 \lambda(p)\;:=\;\frac{2}{\log p},\qquad c(p)\;:=\;\sqrt{2\,\lambda(p)}\;=\;\frac{2}{\sqrt{\log p}}.
\]

\subsection*{$N=1$ (prime $p_1=2$)}
Numerics: \(\log 2\approx 0.6931\), \(\lambda(2)\approx 2.8854\), \(c(2)\approx 2.4022\). The realization is
\[
 A_1\;=\;-\lambda(2),\quad P_1\;=\;1,\quad C_1\;=\;c(2),\quad D_1\;=\;-1,\quad B_1\;=\;C_1.
\]
Lossless equalities: \(A_1^*P_1+P_1A_1=-2\lambda(2)=-C_1^2\), \(P_1B_1=C_1=-C_1 D_1\), and \(D_1^*D_1=1\). The transfer is
\[
 H_1(s)\;=\;-1\; +\; \frac{c(2)^2}{s+\lambda(2)}\;=\;-1\; +\;\frac{\tfrac{4}{\log 2}}{\,s+\tfrac{2}{\log 2}\,}\;=\;\frac{s-\lambda(2)}{s+\lambda(2)}.
\]
The last expression shows \(H_1\) is a first-order all-pass factor on the right half-plane, hence Schur under the Cayley map to the disk.

\begin{lemma}[Half-plane M\"obius inner (rank-one Pick kernel)]\label{lem:moebius-contract}
Fix \(\lambda>0\) and define
\[
  \Theta_\lambda(s)\;:=\;\frac{(s-\tfrac12)-\lambda}{(s-\tfrac12)+\lambda},\qquad s\in\Omega.
\]
Then \(\Theta_\lambda\) is Schur on \(\Omega\) (i.e. \(|\Theta_\lambda(s)|\le 1\) for all \(s\in\Omega\)), and its Pick kernel is the rank-one Gram kernel
\[
  \frac{1-\Theta_\lambda(s)\,\overline{\Theta_\lambda(t)}}{s+\overline t-1}
  \;=\; \frac{2\lambda}{\big((s-\tfrac12)+\lambda\big)\,\big((\overline t-\tfrac12)+\lambda\big)}
  \;=\; \phi_\lambda(s)\,\overline{\phi_\lambda(t)},
\]
with feature \(\displaystyle \phi_\lambda(s):=\frac{\sqrt{2\lambda}}{(s-\tfrac12)+\lambda}\).
\end{lemma}
\begin{proof}
Write \(z=s-\tfrac12\) and \(w=t-\tfrac12\). For \(\Re z>0\) and \(\lambda>0\),
\[
  \bigl|\Theta_\lambda(s)\bigr|^2
  =\frac{|z-\lambda|^2}{|z+\lambda|^2}
  =\frac{|z|^2-2\lambda\,\Re z+\lambda^2}{|z|^2+2\lambda\,\Re z+\lambda^2}\ \le\ 1,
\]
so \(\Theta_\lambda\) is Schur. Next,
\[
  \bigl|1-\Theta_\lambda(s)\,\overline{\Theta_\lambda(t)}\bigr|
  = 1-\frac{(z-\lambda)(\overline w-\lambda)}{(z+\lambda)(\overline w+\lambda)}
  = \frac{2\lambda\,(z+\overline w)}{(z+\lambda)(\overline w+\lambda)}.
\]
Dividing by \(z+\overline w=s+\overline t-1\) gives
\[
  \frac{1-\Theta_\lambda(s)\,\overline{\Theta_\lambda(t)}}{s+\overline t-1}
  = \frac{2\lambda}{(z+\lambda)(\overline w+\lambda)}
  = \frac{2\lambda}{\big((s-\tfrac12)+\lambda\big)\,\big((\overline t-\tfrac12)+\lambda\big)}
  = \phi_\lambda(s)\,\overline{\phi_\lambda(t)},
\]
a rank‑one Gram factorization, hence a PSD Pick kernel.
\end{proof}
\subsection*{$N=2$ (primes $p_1=2$, $p_2=3$)}
Numerics: \(\log 3\approx 1.0986\), \(\lambda(3)\approx 1.8205\), \(c(3)\approx 1.9054\). The diagonal data are
\[
 \Lambda_2\;=\;\mathrm{diag}\big(\lambda(2),\lambda(3)\big),\quad C_2\;=\;\mathrm{diag}\big(c(2),c(3)\big),\quad D_2\;=\;-I_2,\quad B_2\;=\;C_2,\quad A_2\;=\;-\Lambda_2.
\]
The lossless equalities of Lemma~\ref{lem:losslessKYP} hold entrywise. The matrix-valued transfer is
\[
 H_2(s)\;=\;-I_2\; +\; \mathrm{diag}\!\left(\frac{s-\lambda(2)}{s+\lambda(2)},\ \frac{s-\lambda(3)}{s+\lambda(3)}\right).
\]
Any scalar port extraction \(h_2(s)=v^*H_2(s)u\) with \(\|u\|=\|v\|=1\) satisfies \(|h_2(s)|\le 1\) for \(\Re s>0\); in particular, choosing \(u=v=e_1\) recovers the \(N=1\) factor for \(p=2\).

\subsection*{General $N$ (diagonal form)}
For general \(N\), the same diagonal structure yields
\[
 H_N(s)\;=\;-I_N\; +\; \mathrm{diag}\!\left(\frac{\tfrac{4}{\log p_k}}{\,s+\tfrac{2}{\log p_k}\,}\right)_{k=1}^N\;=\;\mathrm{diag}\!\left(\frac{s-\lambda(p_k)}{s+\lambda(p_k)}\right)_{k=1}^N,
\]
with \(\lambda(p_k)=2/\log p_k\). Each diagonal entry obeys Lemma~\ref{lem:moebius-contract}.

\subsection*{A negative result: nonconvergence of the naive cascade}
Define the scalar cascade partial sums
\[
 S_N(s)\;:=\;-1\; +\;\sum_{k=1}^{N} \frac{4/\log p_k}{\,s+2/\log p_k\,},\qquad \Re s>0.
\]
These are the scalar ports of the diagonal prime-grid lossless models with unit weights. Although each term is bounded-real, the sequence \(S_N(s)\) does not converge locally uniformly (indeed not even pointwise) as \(N\to\infty\).
\begin{proposition}[Divergence of the naive prime-grid sum]\label{prop:divergence}
Fix \(s\) with \(\Re s>0\). Then \(S_N(s)\) diverges as \(N\to\infty\).
\end{proposition}
\begin{proof}
For fixed \(s\) with \(\Re s>0\), one has
\[
 \Big|\frac{4/\log p_k}{\,s+2/\log p_k\,}\Big|\;\asymp\; \frac{c}{\log p_k}
\]
with a constant \(c=c(s)>0\) depending only on \(s\). Since \(\sum_{p}\!1/\log p\) diverges, the series of absolute values diverges, hence the sequence of partial sums \(S_N(s)\) cannot converge.
\end{proof}
\noindent This shows that any infinite-$N$ construction based on the \emph{additive} cascade of first-order all-pass sections with unit weights cannot produce a convergent limit, let alone approximate a zeta-derived target. Any successful prime-tied construction must therefore incorporate nontrivial weights (e.g., rapidly decaying coefficients) or a multiplicative/inner product structure rather than a simple additive sum.

% [Moved references to end of document]
\begin{thebibliography}{9}
\bibitem{TitchmarshZeta} E. C. Titchmarsh, \emph{The Theory of the Riemann Zeta-Function}, 2nd ed., revised by D. R. Heath-Brown, Oxford Univ. Press, 1986.
\bibitem{Edwards} H. M. Edwards, \emph{Riemann's Zeta Function}, Dover, 2001.
\bibitem{IwaniecKowalski} H. Iwaniec and E. Kowalski, \emph{Analytic Number Theory}, AMS Colloquium Publications, vol. 53, 2004.
\bibitem{SimonTraceIdeals} B. Simon, \emph{Trace Ideals and Their Applications}, 2nd ed., Mathematical Surveys and Monographs, vol. 120, AMS, 2005.
\bibitem{deBrangesRovnyakBook} L. de Branges and J. Rovnyak, \emph{Square Summable Power Series}, Holt, Rinehart and Winston, 1966.
\bibitem{DymGohbergBook} H. Dym and I. Gohberg, \emph{Topics in Operator Theory}, Birkhäuser, 1974.
\bibitem{SzNagyFoiasBook} B. Sz.-Nagy and C. Foia\c{s}, \emph{Harmonic Analysis of Operators on Hilbert Space}, North-Holland, 1970.
\bibitem{GarnettBook} J. Garnett, \emph{Bounded Analytic Functions}, Graduate Texts in Mathematics, vol. 236, Springer, 2007.
\bibitem{RosenblumRovnyakBook} M. Rosenblum and J. Rovnyak, \emph{Hardy Classes and Operator Theory}, Dover, 1985.
\end{thebibliography}
\fi

\subsection{KYP Gram identity in half-plane notation}\label{app:KYP-gram}

\begin{theorem}[KYP Gram identity for half-plane lossless systems]\label{thm:KYP-gram-appendix}
Let $(A, B, C, D)$ be a minimal realization of a lossless transfer function $F(s) = D + C((s-\tfrac12)I - A)^{-1}B$ on the shifted right half-plane $\{\Re s > 1/2\}$. Assume the continuous-time bounded-real lemma (BRL) conditions hold with $\gamma = 1$:
\begin{align}
  A^* P + P A + C^* C &= 0, \label{eq:brl1}\\
  P B + C^* D &= 0, \label{eq:brl2}\\
  D^* D &= I, \label{eq:brl3}
\end{align}
where $P \succ 0$ is the Lyapunov certificate. Then for all $s, t$ with $\Re s, \Re t > 1/2$,
\[
  \frac{F(s) + \overline{F(t)}}{s + \bar t - 1} = \langle ((s-\tfrac12)I - A)^{-1}B, ((t-\tfrac12)I - A)^{-1}B \rangle_P,
\]
where $\langle x, y \rangle_P := y^* P x$ is the inner product induced by $P$.
\end{theorem}
\begin{proof}
Define $X(s) := ((s-\tfrac12)I - A)^{-1}B$ for $\Re s > 1/2$. We compute the energy inner product:

\medskip
\noindent\textbf{Step 1: Basic identity.}
\begin{align}
  \langle X(s), X(t) \rangle_P &= X(t)^* P X(s)\\
  &= B^* ((t-\tfrac12)I - A^*)^{-1} P ((s-\tfrac12)I - A)^{-1} B.
\end{align}

\medskip
\noindent\textbf{Step 2: Resolvent manipulation.}
Using the resolvent identity $((s-\tfrac12)I - A)^{-1} - ((t-\tfrac12)I - A)^{-1} = (t - s)((s-\tfrac12)I - A)^{-1}((t-\tfrac12)I - A)^{-1}$, we have
\begin{align}
  &(((t-\tfrac12)I - A^*)^{-1} P ((s-\tfrac12)I - A)^{-1} \\
  &= ((t-\tfrac12)I - A^*)^{-1} \left[ \frac{P((s-\tfrac12)I - A)^{-1} - ((t-\tfrac12)I - A^*)^{-1}P}{t - s} \right] (t - s)\\
  &= \frac{((t-\tfrac12)I - A^*)^{-1}P((s-\tfrac12)I - A)^{-1} - ((t-\tfrac12)I - A^*)^{-1}((t-\tfrac12)I - A^*)^{-1}P}{t - s} (t - s).
\end{align}
For the numerator, multiply equation \eqref{eq:brl1} by $((t-\tfrac12)I - A^*)^{-1}$ on the left and $((s-\tfrac12)I - A)^{-1}$ on the right:
\begin{align}
  &((t-\tfrac12)I - A^*)^{-1}(A^* P + P A + C^* C)((s-\tfrac12)I - A)^{-1} = 0\\
  \Rightarrow\quad &((t-\tfrac12)I - A^*)^{-1}A^* P((s-\tfrac12)I - A)^{-1} + ((t-\tfrac12)I - A^*)^{-1}P A((s-\tfrac12)I - A)^{-1}\\
  &\qquad + ((t-\tfrac12)I - A^*)^{-1}C^* C((s-\tfrac12)I - A)^{-1} = 0.
\end{align}

\medskip
\noindent\textbf{Step 3: Simplification.}
Note that:
\begin{align}
  ((t-\tfrac12)I - A^*)^{-1}A^* &= I - (t-\tfrac12)((t-\tfrac12)I - A^*)^{-1},\\
  A((s-\tfrac12)I - A)^{-1} &= I - (s-\tfrac12)((s-\tfrac12)I - A)^{-1}.
\end{align}

Substituting:
\begin{align}
  &[I - (t-\tfrac12)((t-\tfrac12)I - A^*)^{-1}]P((s-\tfrac12)I - A)^{-1} + ((t-\tfrac12)I - A^*)^{-1}P[I - (s-\tfrac12)((s-\tfrac12)I - A)^{-1}]\\
  &\qquad + ((t-\tfrac12)I - A^*)^{-1}C^* C((s-\tfrac12)I - A)^{-1} = 0.
\end{align}
Expanding and rearranging:
\begin{align}
  &(s + \bar t - 1)((t-\tfrac12)I - A^*)^{-1}P((s-\tfrac12)I - A)^{-1}\\
  &= P((s-\tfrac12)I - A)^{-1} + ((t-\tfrac12)I - A^*)^{-1}P - ((t-\tfrac12)I - A^*)^{-1}C^* C((s-\tfrac12)I - A)^{-1}.
\end{align}

\medskip
\noindent\textbf{Step 4: Computing the Gram inner product.}
\begin{align}
  \langle X(s), X(t) \rangle_P &= B^* ((t-\tfrac12)I - A^*)^{-1} P ((s-\tfrac12)I - A)^{-1} B\\
  &= \frac{1}{s + \bar t - 1} B^* \left[ P((s-\tfrac12)I - A)^{-1} + ((t-\tfrac12)I - A^*)^{-1}P - ((t-\tfrac12)I - A^*)^{-1}C^* C((s-\tfrac12)I - A)^{-1} \right] B.
\end{align}

Using equation \eqref{eq:brl2}, $PB = -C^* D$:
\begin{align}
  \langle X(s), X(t) \rangle_P &= \frac{1}{s + \bar t - 1} \left[ -B^* C^* D ((s-\tfrac12)I - A)^{-1}B - B^* ((t-\tfrac12)I - A^*)^{-1}C^* D \right.\\
  &\qquad \left. - B^* ((t-\tfrac12)I - A^*)^{-1}C^* C((s-\tfrac12)I - A)^{-1}B \right].
\end{align}

Factoring out common terms and using \eqref{eq:brl3}:
\begin{align}
  \langle X(s), X(t) \rangle_P &= \frac{1}{s + \bar t - 1} \left[ D^* C((s-\tfrac12)I - A)^{-1}B + B^* ((t-\tfrac12)I - A^*)^{-1}C^* D \right.\\
  &\qquad \left. + B^* ((t-\tfrac12)I - A^*)^{-1}C^* C((s-\tfrac12)I - A)^{-1}B \right].
\end{align}

\medskip
\noindent\textbf{Step 5: Recognizing the transfer function.}
Note that:
\begin{align}
  F(s) &= D + C((s-\tfrac12)I - A)^{-1}B,\\
  \overline{F(t)} &= D^* + B^* ((t-\tfrac12)I - A^*)^{-1}C^*.
\end{align}

Therefore:
\begin{align}
  F(s) + \overline{F(t)} &= D + C((s-\tfrac12)I - A)^{-1}B + D^* + B^* ((t-\tfrac12)I - A^*)^{-1}C^*\\
  &= (s + \bar t - 1) \langle X(s), X(t) \rangle_P.
\end{align}
This completes the proof.
\end{proof}
\begin{remark}[Connection to unit disk formulation]
The standard KYP lemma is often stated for the unit disk. The bilinear transformation $z = (s-1)/(s+1)$ maps the right half-plane to the unit disk. Under this transformation, a lossless system in the half-plane corresponds to an inner function on the disk, and the kernel $(F(s) + \overline{F(t)})/(s + \bar t - 1)$ transforms to the standard Pick kernel $(1 - f(z)\overline{f(w)})/(1 - z\bar w)$.
\end{remark}

\subsection{Expanded proof of Schur--determinant splitting (Proposition~\ref{prop:schur-split})}
We sketch a direct computation using the regularized determinant definition. Recall
\[
 \dettwo(I-K)\;=\;\det\!\Big((I-K)\,\exp\big(K\big)\Big),\qquad K\in\HS.
\]
For the block operator \(T=\begin{bmatrix}A&B\\C&D\end{bmatrix}\) with \(B,C\) finite rank and \(A\in\HS\), write the Schur triangularization of \(I-T\):
\[
 I-T\;=\;L\,\mathrm{diag}(I-A,\ I-S)\,U,
\]
with
\[
 L\;=\;\begin{bmatrix}I & 0\\ -C(I-A)^{-1} & I\end{bmatrix},\qquad U\;=\;\begin{bmatrix}I & -(I-A)^{-1}B\\ 0 & I\end{bmatrix}.
\]
Both \(L-I\) and \(U-I\) are finite rank. Using \(\det((I+X)\exp(-X))=1\) for finite-rank \(X\) and the cyclicity of the trace inside finite-dimensional blocks, one finds
\[
 \dettwo(I-T)\;=\;\det(I-S)\,\dettwo(I-A),
\]
which yields the logarithmic identity in Proposition~\ref{prop:schur-split}. For completeness, one may verify multiplicativity via Simon's product identity for \(\dettwo\): if \(X,Y\in\HS\), then
\[
 \dettwo((I-X)(I-Y))\;=\;\dettwo(I-X)\,\dettwo(I-Y)\,\exp\!\big(-\Tr(XY)\big),
\]
and compute the finite-rank cross term \(\Tr(XY)\) arising from the triangular factors, which cancels against the exponential in \(\det(I-S)\).
\subsection{Expanded proof of HS\(\to\)\(\dettwo\) convergence (Proposition~\ref{prop:HS-to-det2})}
Let \(K_n,K:K\to\HS\) be holomorphic with uniform HS bounds \(\|K_n(s)\|_{\HS}\le M_K\) and \(\|K_n(s)-K(s)\|_{\HS}\to 0\) uniformly on compact \(K\subset\Omega\). By Lemma~\ref{lem:carleman}, \(|\dettwo(I-K_n(s))|\le \exp(\tfrac12 M_K^2)\). The pointwise convergence \(\dettwo(I-K_n(s))\to \dettwo(I-K(s))\) follows from continuity of \(\dettwo\) on \(\HS\). Vitali--Porter theorem applies: a locally bounded normal family \(\{f_n\}\) of holomorphic functions on a domain with pointwise convergence on a set with an accumulation point converges locally uniformly to a holomorphic limit. Thus \(f_n\to f\) uniformly on compacts.

\subsection{Asymptotics of the completed zeta \(\xi\)}\label{app:xi-asymptotics}
For \(\sigma:=\Re s\to+\infty\), Stirling's formula for \(\Gamma(s/2)\) gives
\[
 \Gamma\!\left(\frac{s}{2}\right)\;\sim\;\sqrt{2\pi}\,\Big(\frac{s}{2}\Big)^{\frac{s-1}{2}} e^{-s/2},\qquad \pi^{-s/2}\,\Gamma\!\left(\frac{s}{2}\right)\;\sim\;\sqrt{2\pi}\,\Big(\frac{s}{2\pi}\Big)^{\frac{s-1}{2}} e^{-s/2}.
\]
Since \(\zeta(s)\to 1\) as \(\sigma\to\infty\) and the polynomial factor \(\tfrac12 s(1-s)\) is negligible relative to the Stirling growth, one concludes \(|\xi(s)|\to\infty\) super-exponentially along vertical rays with \(\sigma\) fixed large. Consequently, for our truncations with \(\dettwo(I-A_N(s))\to 1\),
\[
 H_N^{(\dettwo)}(s)\;=\;2\,\frac{\dettwo(I-A_N(s))}{\xi(s)}-1\;\longrightarrow\;-1
\]
uniformly on bounded strips \(\{\sigma\ge \sigma_0>\tfrac12,\ |\Im s|\le R\}\) as \(\sigma_0\to\infty\), consistent with the feedthrough \(-1\) realized by the prime-grid models.

\subsection{Half-plane Pick kernel from the disk}
Let \(\phi:\mathbb D\to\Omega\), \(\phi(\zeta)=\tfrac12\,\frac{1+\zeta}{1-\zeta}+\tfrac12\), be the Cayley map from the unit disk \(\mathbb D\) to \(\Omega\). If \(\theta\) is Schur on \(\mathbb D\) with disk kernel \(K_{\mathbb D}(\zeta,\eta)=(1-\theta(\zeta)\overline{\theta(\eta)})/(1-\zeta\overline{\eta})\), then transporting via \(\Theta=\theta\circ\phi^{-1}\) yields the half-plane kernel
\[
 K_\Theta(s,w)\;=\;\frac{1-\Theta(s)\,\overline{\Theta(w)}}{s+\overline{w}-1},
\]
after multiplication by a harmless positive weight. This justifies the denominator used in Theorem~\ref{thm:brf-rh-final}.
\subsection{Discrete-time KYP (disk) variant}
For completeness: if \(G(z)=D+C(zI-A)^{-1}B\) is holomorphic on \(|z|<1\) with \(A\) Schur (spectral radius <1), then \(\|G\|_{H^\infty(\mathbb D)}\le 1\) iff there exists \(P\succeq 0\) such that
\[
 \begin{bmatrix}
  A^*PA-P & A^*PB & C^*\\
  B^*PA & B^*PB-I & D^*\\
  C & D & -I
 \end{bmatrix}\ \preceq\ 0.
\]
In the lossless case, equalities analogous to \eqref{eq:lossless-equalities} hold with some \(P\succ 0\).

\subsection{Lossless realizations for NP data}

\subsection{Half-plane KYP epigraph for boundary H$^\infty$ approximation}\label{app:KYP-epigraph}
We sketch a practical formulation used in Proposition~\ref{prop:hinf-passive}. Fix a rectangle boundary \(\partial R\) and model order \(M\). Parametrize scalar transfers \(\Theta_M(s)=D+C(sI-A)^{-1}B\) with \(A\in\C^{M\times M}\) Hurwitz and \((B,C,D)\) of compatible sizes. Enforce Schur (lossless) via the equalities \eqref{eq:lossless-equalities} with some \(P\succ 0\). Introduce an epigraph variable \(t\ge 0\) and impose discrete boundary constraints on a spectral grid \(\{\zeta_j\}\subset\partial R\):
\[
 |\Theta_M(\zeta_j)-g_N(\zeta_j)|\ \le\ t,\qquad j=1,\dots,J,
\]
where \(g_N=\Theta_N^{(\dettwo)}|_{\partial R}\). The program
\[
 \min\ t\quad \text{s.t. lossless KYP equalities and } |\Theta_M(\zeta_j)-g_N(\zeta_j)|\le t
\]
is a convex bounded-extremal approximation in the Schur ball when the KYP constraints are satisfied and the grid is sufficiently fine; the epigraph constraints can be handled via second-order cones on real/imag parts. Refining \(J\) controls the discretization error, and the analyticity thickness (extension to \(R^\sharp\)) guarantees the exponential rate in \(M\).

\subsection{Rational approximation on analytic curves}\label{app:rational-analytic}
Let \(D\Subset\C\) be a domain bounded by an analytic Jordan curve and let \(f\) be holomorphic on a neighborhood of \(\overline D\). Then there exist constants \(C>0\) and \(\rho\in(0,1)\), depending only on the distance from \(\partial D\) to the nearest singularity of \(f\), such that the best uniform rational (or polynomial) approximation error on \(\partial D\) satisfies
\[
 \inf_{\deg\le M}\ \sup_{\zeta\in\partial D}\,|r_M(\zeta)-f(\zeta)|\ \le\ C\,\rho^{M}.
\]
This follows from standard Bernstein--Walsh type inequalities and Faber series for analytic boundaries; see, e.g., Walsh~\cite{WalshApprox} and Saff--Totik~\cite{SaffTotik}. Transport to rectangles via conformal maps yields the rate used in Proposition~\ref{prop:hinf-passive}.

\subsection{Explicit formula (precise variant used)}\label{app:explicit-formula}
Let \(\varphi\in C_c^{\infty}(\R)\) and define its Mellin--Fourier companion
\[
 g(x)\;:=\;\frac{1}{2\pi}\int_{\R} \varphi(t)\,e^{itx}\,dt,\qquad x\in\R.
\]
Let \(\Phi_{\varphi}(s)\) be the Mellin transform appropriate to the completed zeta context (cf. Edwards~\cite[Ch.~1, §5]{Edwards}, Iwaniec--Kowalski~\cite[Ch.~5]{IwaniecKowalski}). Then the following explicit formula holds for the completed zeta:
\[
 \sum_{\rho} \Phi_{\varphi}(\rho)\;=\;\Phi_{\varphi}(1)\,+\,\Phi_{\varphi}(0)\;-
 \sum_{p}\sum_{m\ge 1} \frac{\log p}{p^{m/2}}\,g(m\log p)\;-
 \frac{1}{2\pi}\int_{-\infty}^{\infty} \Re\frac{\Gamma'}{\Gamma}\!\left(\frac{1}{4}+\frac{iu}{2}\right)\,\Phi_{\varphi}\!\left(\frac12+iu\right)du.
\]
All terms converge absolutely for \(\varphi\in C_c^{\infty}(\R)\), and the right-hand side is bounded by a constant depending only on \(\varphi\). Differentiating with respect to \(\sigma\) inside \(\Phi_{\varphi}(\tfrac12+iu)\) and using the rapid decay of \(g\) yields Lipschitz-in-\(\sigma\) bounds for the \(\varphi\)-weighted prime and zero sums. This is the variant tacitly used in Lemma~\ref{lem:smoothed-explicit}.
\subsection{Numerical note: grid/KYP solve on \(\partial R\)}\label{app:numerics}
A practical H$^\infty$ approximation on a rectangle boundary \(\partial R\) proceeds as follows. Fix \(K\Subset R\Subset R^\sharp\Subset\Omega\) and an order \(M\). Sample \(\partial R\) at \(J\) spectral nodes \(\{\zeta_j\}\) (Chebyshev along each edge). For a state-space parameterization \(\Theta_M(s)=D+C((s-\tfrac12)I-A)^{-1}B\) with Hurwitz \(A\in\C^{M\times M}\), enforce the lossless KYP equalities \eqref{eq:lossless-equalities} with a decision variable \(P\succ 0\). Introduce an epigraph variable \(t\ge 0\) and constrain
\[
 |\Theta_M(\zeta_j)-g_N(\zeta_j)|\ \le\ t,\qquad j=1,\dots,J.
\]
The objective \(\min t\) subject to these constraints is a convex program (KYP equalities plus second-order cones for the complex modulus). Refining \(J\) improves the boundary resolution; increasing \(M\) reduces the best achievable \(t\) roughly as \(C\rho^M\) by Subsection~\ref{app:rational-analytic}. The resulting \(\Theta_{N,M}\) is Schur (lossless) by construction, and the maximum principle transfers the boundary error to \(K\).
\subsection{Carleson self-correction and a direct route to (P+) and RH}\label{subsec:PSC}
We isolate the single quantitative hypothesis that encodes the ``perfect self-correction'' principle as a Carleson bound on the off-critical zero measure and show it implies (P+), hence Herglotz/Schur in \(\Omega\) and RH.
\paragraph{Defect measure and Carleson boxes.}
For each nontrivial zero \(\rho=\beta+i\gamma\) of \(\xi\) with \(\beta>\tfrac12\), write the depth \(a(\rho):=\beta-\tfrac12>0\). Define the positive Borel measure
\[
 d\mu\ :=\ \sum_{\substack{\rho=\beta+i\gamma\\ \beta>1/2}} 2\,a(\rho)\,\delta_{\,(\,\tfrac12+a(\rho),\ \gamma\,)}\,.
\]
For a bounded interval \(I=[T_1,T_2]\subset\R\) let the Carleson box be
\[
 Q(I)\ :=\ \{\,\sigma+it:\ t\in I,\ 0<\sigma-\tfrac12<|I|\,\}.
\]
\begin{definition}[Perfect self-correction (PSC)]\label{def:PSC}
We say the defect measure \(\mu\) is \emph{PSC} if for every bounded interval \(I\subset\R\),
\[
 \mu\big(Q(I)\big)\ \le\ \tfrac{\pi}{2}\,|I|.
\]
\end{definition}
\paragraph{Poisson stamp and phase--balayage.}
For \(a>0\) and \(\gamma\in\R\), define the Poisson-weighted stamp across \(I\) by
\[
 \mathrm{Bal}_a(\gamma;I)\ :=\ 2\Big[\arctan\!\frac{T_2-\gamma}{a}-\arctan\!\frac{T_1-\gamma}{a}\Big]\ \in [0,\pi].
\]
Let \(\mathcal J=\dettwo(I-A)/(\mathcal O\,\xi)\) be the outer-normalized ratio as above, set \(w(t):=\Arg\,\mathcal J(\tfrac12+it)\in(-\pi,\pi] \) and let \(-w'\) denote its distributional derivative on intervals avoiding critical-line ordinates.
\begin{lemma}[Phase--balayage law (density)]\label{lem:balayage-law}
On any interval \(I\) avoiding the ordinates of critical-line zeros, one has
\[
 \int_I (-w'(t))\,dt\ =\ \int_{\Omega}\, \mathrm{Bal}_{\sigma-\frac12}(\tau;I)\, d\mu(\sigma+i\tau).
\]
In particular, \(\int_I (-w'(t))\,dt\le \pi\, \mu(Q(I))\).
\end{lemma}
\begin{proof}
This is the distributional form of the phase--velocity identity (Proposition~\ref{prop:phase-velocity-identity}) after outer normalization: the zero-side contribution is exactly the Poisson balayage of \(\mu\), critical-line atoms contribute a nonnegative discrete term (ruled out on \(I\) by hypothesis), while regular parts are absorbed by \(\mathcal O\). The pointwise bound \(\mathrm{Bal}_a\le\pi\) and localization to \(Q(I)\) give the inequality \(\int_I(-w')\le \pi\,\mu(Q(I))\). This is a \emph{density} bound and is not used to deduce a uniform wedge.
\end{proof}
% (Removed) PSC implies boundary wedge; archived density only.

\begin{lemma}[Holomorphy and absence of poles from (P+)]\label{lem:Pplus-holomorphy-nopoles}
If \(\Re\big(2\mathcal J(\tfrac12+it)\big)\ge 0\) for a.e. \(t\in\R\), then \(2\mathcal J\) is Herglotz on \(\Omega\), and \(\mathcal J\) is holomorphic on \(\Omega\) (in particular, it has no poles there).
\end{lemma}
\begin{proof}
By the Poisson representation for harmonic functions on vertical lines, boundary nonnegativity transports to \(\Re\big(2\mathcal J(x+it)\big)\ge 0\) for every \(x>\tfrac12\). Hence \(2\mathcal J\) is Herglotz on \(\Omega\). If \(\mathcal J\) had a pole at some \(s_0\in\Omega\), then near \(s_0\) the principal part forces \(\Re\big(2\mathcal J\big)\) to take both signs along radial approaches, contradicting the global nonnegativity. Therefore \(\mathcal J\) has no poles on \(\Omega\).
\end{proof}

% (Removed) PSC⇒RH chain via wedge; archived.

\begin{remark}[Physics $\leftrightarrow$ math dictionary]
Off-critical zeros at depth \(a\) are imbalanced resonances carrying cost \(2a\). The Carleson bound caps the total defect cost per window, which bounds the boundary phase drop to \(\le\pi/2\). This enforces boundary positive-real (P+), whence interior Herglotz/Schur and the pinch argument exclude interior poles of \(\mathcal J\).
\end{remark}
\paragraph{Axiom (Self-correction $\Leftrightarrow$ boundary positive-real).}
Let \(\Omega=\{\Re s>\tfrac12\}\) and
\[\mathcal J(s):=\frac{\dettwo(I-A(s))}{\mathcal O(s)\,\xi(s)}\]
be the outer-normalized ratio from Subsection~\ref{subsec:Pplus-Carleson}, so $|\mathcal J(\tfrac12+it)|=1$ a.e. on the boundary. 
\begin{definition}[Self-correction (SC)]\label{def:SC}
We say the system is \emph{self-correcting} if
\[\Re\bigl(2\mathcal J(\tfrac12+it)\bigr)\ \ge\ 0\quad\text{for a.e. }t\in\R.\]
\end{definition}
In classical function theory this is exactly the boundary positive-real hypothesis (P+), and is equivalent—via the Poisson integral—to $2\mathcal J$ being Herglotz on $\Omega$; see Theorem~\ref{thm:global-PSD}.
\begin{proposition}[Boundary PSD for $H_{J_N}$ by congruence]\label{prop:boundary-psd-fixed}
Let $R\Subset\Omega$ be a rectangle and $\Sigma_R:=Z(\xi)\cap\partial R$. On $\partial R\setminus\Sigma_R$ set
\[
K_{\exp,N}(s,\bar t):=\frac{e^{\mathfrak g_N(s)}+\overline{e^{\mathfrak g_N(t)}}}{s+\bar t-1},\qquad 
K_{\mathrm{FG},N}(s,\bar t):=E_N(s,\bar t)\,\frac{1}{s+\bar t-1},
\]
with $\mathfrak g_N=\log\dettwo(I-A_N)$ and $E_N$ the Fock lift from Lemma~\ref{lem:fock-gram-formal}. Then for any finite node set $\{s_j\}\subset\partial R\setminus\Sigma_R$:
\begin{enumerate}
\item[\textup{(a)}] The Gram matrix $\big(K_{\exp,N}(s_i,\overline{s_j})-K_{\mathrm{FG},N}(s_i,\overline{s_j})\big)_{i,j}$ is PSD.
\item[\textup{(b)}] Since $K_{\mathrm{FG},N}$ is PSD, (a) implies $\big(K_{\exp,N}(s_i,\overline{s_j})\big)_{i,j}$ is PSD.
\item[\textup{(c)}] With the diagonal multiplier $D=\mathrm{diag}(\xi(s_j)^{-1})$, one has
\[
\Big(H_{J_N}(s_i,\overline{s_j})\Big)_{i,j}=D\,\Big(K_{\exp,N}(s_i,\overline{s_j})\Big)_{i,j}\,D^{*},
\]
so $\big(H_{J_N}(s_i,\overline{s_j})\big)$ is PSD.
\end{enumerate}
Consequently $H_{J_N}$ is PSD on $\partial R$ in the sense of boundary limits along node sets approaching $\Sigma_R$.
\end{proposition}
\begin{proof}
(a)–(b) are the Fock–Gram lower bound and Löwner-order transfer. For (c), write $J_N=\dettwo(I-A_N)/\xi$, and observe
\[\frac{J_N(s_i)+\overline{J_N(s_j)}}{s_i+\overline{s_j}-1}=\xi(s_i)^{-1}\,\frac{e^{\mathfrak g_N(s_i)}+\overline{e^{\mathfrak g_N(s_j)}}}{s_i+\overline{s_j}-1}\,\overline{\xi(s_j)^{-1}}.\]
Congruence by a nonsingular diagonal preserves PSD. Approaching $\Sigma_R$ is handled by entrywise limits of PSD matrices.
\end{proof}
\begin{corollary}[Boundary $\Rightarrow$ interior on rectangles]\label{cor:bdry-to-int}
Let $R\Subset\Omega$ be a rectangle. Then $H_{\mathcal J_N}$ is PSD on $\partial R$ (distribution sense), hence $\Re \,\mathcal J_N\ge 0$ in $R$; equivalently $\Theta_N=(2\mathcal J_N-1)/(2\mathcal J_N+1)$ is Schur on $R$.
\end{corollary}
\begin{theorem}[Faces of self-correction; PSC as a sufficient condition]\label{thm:SC-equivalences}
Let $\mathcal J=\dettwo(I-A)/(\mathcal O\,\xi)$ be the outer-normalized ratio on $\Omega$. The following hold:
\begin{enumerate}
\item[\textup{(i)}] \textup{(P+)}: $\Re\bigl(2\mathcal J(\tfrac12+it)\bigr)\ge 0$ a.e. on $\R$.
\item[\textup{(ii)}] $2\mathcal J$ is Herglotz on $\Omega$ (hence $\Theta=(2\mathcal J-1)/(2\mathcal J+1)$ is Schur on $\Omega$).
\item[\textup{(iii)}] (PSC; archived density) The off-critical zero measure $\mu$ obeys the Carleson bound $\mu(Q(I))\le \tfrac{\pi}{2}|I|$ for all intervals $I\subset\R$. We do not use PSC to deduce (P+).
\end{enumerate}
Moreover, either of (i)–(ii) implies RH via the pinch argument (Theorem~\ref{thm:global-PSD} combined with the globalization paragraph); PSC (iii) is recorded as a density statement only.
\end{theorem}
\begin{proof}
(i)$\Leftrightarrow$(ii): Poisson/Herglotz equivalence on the half-plane (Theorem~\ref{thm:global-PSD}). The (P+) step is proved via the product certificate (Theorem~\ref{thm:psc-certificate-stage2}). No claim of (i)$\Rightarrow$(iii) is made.
\end{proof}

%====================================================================
%  SECTION: Unconditional Proof of the Carleson Self-Correction Principle
%====================================================================

% (legacy archived PSC header removed)
\medskip
In this section we formalize a local explicit-formula strategy to prove the Carleson Self-Correction (PSC) inequality
\[ \mu(Q(I))\ \le\ \tfrac{\pi}{2}\,|I| \quad\text{for every interval } I, \]
thereby closing the (P+) step and RH via Section~\ref{subsec:PSC}. We work at the Whitney scale \(|I|\asymp c/\log(2+T)\) and use a smooth local test to pass the phase--velocity identity to a Poisson-balayage bound, then control ancillary terms by unconditional estimates.

\subsection{Test functions and Poisson staples}
Fix a bounded interval \(I=[T_1,T_2]\) with center \(T:=\tfrac12(T_1+T_2)\) and length \(L:=|I|\). Fix an even, nonnegative window \(\psi\in C_c^\infty([-1,1])\) with \(\int_\R\psi=1\), and set the mass–1 test
\[
  \varphi_I(t)\ :=\ \frac{1}{L}\,\psi\!\left(\frac{t-T}{L}\right).
\]
Then \(\mathrm{supp}\,\varphi_I\subset[T-L,\,T+L]\), \(\int_\R \varphi_I=1\), and \(\|\varphi_I'\|_{L^1}\asymp L^{-1}\) with constants depending only on \(\psi\).
For a zero \(\rho\in\C\) with depth \(a:=\beta-\tfrac12>0\), the Poisson balayage across \(I\) is
\[ \mathrm{Bal}_a(\gamma;I)\ :=\ 2\Big[\arctan\!\frac{T_2-\gamma}{a}-\arctan\!\frac{T_1-\gamma}{a}\Big] \in [0,\pi].\]
\begin{lemma}[Whitney lower bound]\label{lem:whitney-lower}
There exists \(c_0\in(0,\pi)\) such that for any \(I\) and any zero \(\rho\) with \(\gamma\in I\) and \(a\in[L,2L]\), one has \(\mathrm{Bal}_a(\gamma;I)\ge c_0\).
\end{lemma}
\begin{proof}
Minimize \(2(\arctan((L-x)/a)+\arctan(x/a))\) over \(x\in[0,L]\), \(a\in[L,2L]\). For fixed \(a\), the sum in \(x\) is minimized at the endpoints, giving \(2\arctan(L/a)\). This is decreasing in \(a\), so the minimum over \(a\in[L,2L]\) occurs at \(a=2L\), yielding \(\ge 2\arctan(1/2)\). Any uniform choice \(c_0\in(0,2\arctan(1/2))\) suffices. A detailed derivation is provided in Appendix~\ref{app:psc-tech}.
\end{proof}

\subsection{Ancillary bounds on short intervals}
Write \(F=\dettwo(I-A)/\xi\), \(u=\log|F|\) on the boundary, \(s=\tfrac12+it\). We isolate the three standard contributions appearing in the phase--velocity identity.
\begin{lemma}[Archimedean control]\label{lem:arch}
There exists a window–dependent constant \(C_\Gamma(\psi)>0\) such that for every interval \(I\) and mass–1 test \(\varphi_I\),
\[ \Big|\int_{\R} \Im\Big(\frac{\Gamma'}{\Gamma}(s/2)+\frac{1-2s}{s(1-s)}\Big)\,\varphi_I(t)\,dt\Big|\ \le\ C_\Gamma(\psi)\,\big(1+\log(2+|T|)\big).\]
\end{lemma}
\begin{proof}
See Appendix~\ref{app:psc-tech} (Archimedean control) for a full proof with an explicit symbolic constant \(C_\Gamma(\psi)\).
\end{proof}

\begin{lemma}[Prime-side difference on mass–1 windows]\label{lem:prime-short}
There exists a window–dependent constant \(C_P(\psi,L,\kappa)\ge 0\) (from the band-limited scheme) such that
\[ \Big|\int_{\R} \Im\Big(\frac{\zeta'}{\zeta}(s)-\frac{\dettwo'}{\dettwo}(s)\Big)\,\varphi_I(t)\,dt\Big|\ \le\ C_P(\psi,L,\kappa).\]
Moreover, with cutoff \(\Delta=\kappa/L\) one has the uniform bound \(\sup_{L>0} C_P(\psi,L,\kappa)\le 2\kappa\) (explicit bandlimit estimate).
\end{lemma}
\begin{proof}
See Appendix~\ref{app:psc-tech} (Prime-side difference) for the frequency-truncated Montgomery–Vaughan argument and the explicit expression of \(C_P(\psi,L,\kappa)\) in the smoothing parameters.
\end{proof}

\begin{lemma}[Hilbert-transform pairing]\label{lem:hilbert-aux2}
There exists a window–dependent constant \(C_H(\psi)>0\) such that for every interval \(I\),
\[ \Big|\int_{\R} \mathcal H[u'](t)\,\varphi_I(t)\,dt\Big|\ \le\ C_H(\psi).\]
\end{lemma}
\begin{proof}
By Lemma~\ref{lem:hilbert-H1BMO}, for mass–1 windows and even \(\psi\), the pairing \(\langle \mathcal H[u'],\varphi_I\rangle\) is uniformly bounded in \((T,L)\). In distributions, \(\langle \mathcal H[u'],\varphi_I\rangle=\langle u,(\mathcal H[\varphi_I])'\rangle\); evenness implies \((\mathcal H[\varphi_I])'\) annihilates affine functions. Subtract the affine calibrant on \(I\) and write \(v=u-\ell_I\). The near field is controlled by Theorem~\ref{thm:unsmoothed-Cauchy} and Corollary~\ref{cor:det2-boundary}. The far field is handled by the same local box pairing as in Lemma~\ref{lem:hilbert-H1BMO}, using only the neutralized area bound and the fixed Poisson energy of the window.
\end{proof}

\subsection{Carleson bound from the phase--velocity identity}
Recall the phase--velocity identity (Proposition~\ref{prop:phase-velocity-identity}): for nonnegative \(\varphi\),
\[ \int_{\R}(-w')(t)\,\varphi(t)\,dt\ =\ \sum_{\rho}2a(\rho)\,(P_{a(\rho)}*\varphi)(\gamma)\ +\ \pi\sum_{\gamma\ \mathrm{critical}} m_\gamma\,\varphi(\gamma).\]

\begin{lemma}[Poisson tails for smoothed testing]\label{lem:poisson-tail}
Let \(\varphi_I\) be the mass–1 window above. Then there exists \(C_{\mathrm{tail}}(\psi)\) such that
\[ 0\ \le\ \sum_{\rho\notin Q(I)} 2a(\rho)\,(P_{a(\rho)}*\varphi_I)(\gamma)\ \le\ C_{\mathrm{tail}}(\psi).\]
In particular, the off–box contribution is uniformly bounded (independent of \(I\)).
\end{lemma}
\begin{proof}
Use the exact scaling \((P_a*\varphi_I)(t)=(P_{a/L}*\psi)((t-T)/L)\) and \(\mathrm{supp}\,\psi\subset[-1,1]\). For \(|t-T|>L\) or \(a>L\), the Poisson weight is \(\lesssim a/((|t-T|-L)^2+a^2)\), and the convolution against \(\psi\) bounds each term by \(\lesssim \min\{1, a/((|t-T|-L)^2+a^2)\}\). Summing over dyadic annuli in \(|t-T|\) and \(a\) gives a geometric tail with constant depending only on \(\psi\).
\end{proof}

\begin{theorem}[Carleson self-correction (mass–1 form)]\label{thm:psc-unconditional}
There is an absolute constant \(C_*\) such that for every interval \(I\),
\[ c_0(\psi)\,\mu\big(Q(I)\big)\ \le\ C_\Gamma(\psi)+C_P(\psi,L,\kappa)+C_H(\psi)\ +\ C_{\mathrm{tail}}(\psi). \]
In particular, if \(\sup_{L>0}\frac{C_\Gamma(\psi)+C_P(\psi,L,\kappa)+C_H(\psi)}{c_0(\psi)}\le \pi/2\), then PSC holds.
\end{theorem}
\begin{proof}
Apply Proposition~\ref{prop:phase-velocity-identity} with \(\varphi_I\). The critical-line sum is nonnegative. For zeros in \(Q(I)\), the Poisson scale reduction (Lemma~\ref{lem:poisson-scale}) and the definition of \(c_0(\psi)\) give a lower bound \(\ge c_0(\psi)\) per unit Carleson mass, hence \(\ge c_0(\psi)\,\mu(Q(I))\). The off–box contribution is bounded by Lemma~\ref{lem:poisson-tail}. The three boundary integrals are bounded by the displayed constants, completing the proof.
\end{proof}
\begin{theorem}[Unconditional parameter choice closes (P+)]\label{thm:unconditional-choice}
Fix an even \(\psi\in C_c^\infty([-1,1])\). Choose a bandlimit parameter \(\kappa\in(0,1]\) so that
\[ C_\Gamma(\psi)\ +\ C_H(\psi)\ +\ 2\kappa\ \le\ \frac{\pi}{2}\,c_0(\psi). \]
Then the mass–1 certificate holds, hence (P+) and RH follow. The choice is uniform in \(T\) (no adaptive cover needed).
\end{theorem}
\begin{proof}
By the mass–1 bounds above and the explicit bandlimit estimate, we have \(\sup_{L>0} C_P(\psi,L,\kappa)\le 2\kappa\). The stated inequality ensures
\(\sup_{L>0}\frac{C_\Gamma(\psi)+C_P(\psi,L,\kappa)+C_H(\psi)}{c_0(\psi)}\le \pi/2\). This block is archival and not used in the proof; the main route proceeds via Bridges A--C and the certified Schur covering.
\end{proof}

% (Appendix already started above.)
\section{Appendix: Technical proofs for the PSC section}\label{app:psc-tech}
\subsection{Whitney lower bound (proof of Lemma~\ref{lem:whitney-lower})}
Let \(I=[T_1,T_2]\), \(L=T_2-T_1\). For \(\gamma\in I\) write \(x=\gamma-T_1\in[0,L]\). For \(a\in[L,2L]\) define
\[\Phi(a,x):=2a\Big(\arctan\frac{L-x}{a}+\arctan\frac{x}{a}\Big).\]
Since \(\Phi\) is continuous on the compact set \([L,2L]\times[0,L]\), it attains its minimum. For fixed \(a\), \(x\mapsto\arctan((L-x)/a)+\arctan(x/a)\) is symmetric about \(L/2\) and minimized at the endpoints; hence
\[\min_{x\in[0,L]}\Phi(a,x)=2a\arctan(L/a).\]
The function \(a\mapsto 2a\arctan(L/a)\) is decreasing on \([L,\infty)\) (differentiate explicitly), so
\[\min_{a\in[L,2L]}2a\arctan(L/a)=2L\arctan(1/2).\]
Thus we can take \(c_0:=2\arctan(1/2)\in(0,\pi)\) and obtain \(\mathrm{Bal}_a(\gamma;I)\ge c_0 L\) whenever \(a\in[L,2L]\) and \(\gamma\in I\). This yields the stated lower bound up to an absolute normalization absorbed in the implicit constants of the main text.
\subsection{Archimedean control (proof of Lemma~\ref{lem:arch})}
Write on \(\sigma=\tfrac12\):
\[\Im\Big(\frac{\Gamma'}{\Gamma}(s/2)\Big)=\Im\Big(\psi\big(\tfrac14+it/2\big)\Big),\qquad \psi(z)=\Gamma'(z)/\Gamma(z).\]
Stirling gives \(\psi(z)=\log z+O(|z|^{-1})\) on vertical lines away from the negative real axis. Hence for \(s=\tfrac12+it\),
\[\Im\frac{\Gamma'}{\Gamma}(s/2)=\arg(\tfrac14+it/2)+O(1/|t|)\in(-\tfrac{\pi}{2}+O(1/|t|),\tfrac{\pi}{2}+O(1/|t|)).\]
The polynomial term \(\Im\frac{1-2s}{s(1-s)}\) is \(O(1/|t|)\). Since \(\varphi_I\) has support of size \(\asymp L\),
\[\Big|\int_{\R}\Im\Big(\frac{\Gamma'}{\Gamma}(s/2)+\frac{1-2s}{s(1-s)}\Big)\varphi_I(t)\,dt\Big|\ \le\ C_\Gamma L\]
with an absolute \(C_\Gamma\).
\subsection{Prime-side difference (details for Lemma~\ref{lem:prime-short})}
Let \(s=\tfrac12+it\). For \(\sigma>\tfrac12\),
\[\frac{\zeta'}{\zeta}(s)= -\sum_{n\ge 2}\frac{\Lambda(n)}{n^s},\qquad \frac{\dettwo'}{\dettwo}(s)= -\sum_{k\ge 2}\sum_{p}\frac{\log p}{p^{ks}}.\]
Their difference on \(\sigma=\tfrac12\) reduces (formally) to the \(k=1\) line \(\sum_p (\log p) p^{-1/2-it}\) after smoothing/truncation. Let \(W\) be a smooth frequency cutoff with \(W(0)=1\), \(\mathrm{supp}\,\widehat W\subset[-1,1]\). Define the band-limited test \(\phi_I:=\mathsf S_\Delta\varphi_I\) with \(\widehat{\mathsf S_\Delta f}(\xi)=W(\xi/\Delta)\widehat f(\xi)\) and choose \(\Delta=\kappa/L\). Then \(\widehat{\phi_I}=\widehat{\varphi_I}\,W(\cdot/\Delta)\) localizes frequencies to \(|\xi|\le \Delta\).
\[\int_{\R}\Im\Big(\frac{\zeta'}{\zeta}-\frac{\dettwo'}{\dettwo}\Big)\phi_I\,dt=\Re\int_{\R}\sum_{p}(\log p)\,p^{-it}\,\phi_I(t)\,dt\ +\ E,\]
with an error \(E\) from prime powers \(k\ge 2\) controlled by the frequency cutoff and absolute convergence. By Fubini and Poisson,
\[\int_{\R}\sum_{p}a_p\,p^{-it}\,\phi_I(t)\,dt=\sum_{p}a_p\,\widehat{\phi_I}(\log p),\qquad a_p=(\log p) p^{-1/2}.\]
Since \(\widehat{\phi_I}\) is supported in \(|\xi|\le \Delta=\kappa/L\) and \(|\widehat{\varphi_I}|\le \|\varphi_I\|_{L^1}=1\), Cauchy–Schwarz and Parseval for Dirichlet polynomials yield the unconditional band-limit bound
\[\Big|\sum_{p}a_p\,\widehat{\phi_I}(\log p)\Big|\ \le\ C_P(\kappa)\,L,\qquad C_P(\kappa)\ \le\ 2\sqrt{\tfrac{\log 4}{2}}\,\kappa,\]
as recorded above. This proves Lemma~\ref{lem:prime-short} without any PNT or zero-density input.
\section{Poisson--Carleson Bridge with Explicit Constants}\label{sec:pc-bridge}
\paragraph{Main Theorem (Five--Stage Close; product route for (P+)).}
We prove (P+) only via the product certificate. Reduction to boundary positivity (P+) holds by Theorem~\ref{thm:psc-certificate-stage2}. Poisson transport yields that $2\mathcal J$ is Herglotz in $\Omega$, the Cayley map gives Schur, and the standard pinch/globalization argument implies RH. The PSC density discussion (sum--form) is archived and not used for (P+). The Bridges A--C/Schur-audit material is optional and not used in this chain.
\paragraph{Non-circularity note.}
The proof of (P+) here uses only: (i) smoothing/Plancherel and Hilbert transform facts; (ii) Stirling/digamma bounds for archimedean factors (Titchmarsh \cite[Ch.~IV]{TitchmarshZeta}); and (iii) the phase–velocity identity and Poisson balayage. It does not assume RH, PNT–strength inputs, or zero‑density estimates.
Throughout write $s=\tfrac12+it$ and adopt the normalized Poisson kernel $P_a(x)=\tfrac{1}{\pi}\tfrac{a}{a^2+x^2}$, so $\int_\R P_a(x)\,dx=1$. For a bounded interval $I=[T_1,T_2]$ of length $L=|I|$ define the Carleson box $Q(I):=\{(\gamma,a)\in\R\times(0,\infty):\ \gamma\in I,\ 0<a\le L\}$. Let $\mu$ be the off--critical zero measure and $c_0>0$ the Whitney constant from Lemma~\ref{lem:whitney-lower}. Let $C_\Gamma$, $C_P$, $C_H$ be the symbolic constants provided by Lemmas~\ref{lem:arch}, \ref{lem:prime-short}, and~\ref{lem:hilbert}.
\begin{theorem}[PSC from explicit constants]\label{thm:psc-constants}
For every bounded interval $I$,
\[ c_0\,\mu\big(Q(I)\big)\ \le\ \big(C_\Gamma + C_P + C_H\big)\,L. \]
Equivalently, the Carleson constant is $C^*=(C_\Gamma + C_P + C_H)/c_0$, and PSC holds provided $C^*\le \pi/2$.
\end{theorem}
\begin{corollary}[PSC (sum--form) closed with locked constants]\label{cor:psc-locked}
In the $\zeta$--normalized route one has $C_\Gamma=0$. For the printed mass--1 window $\psi$ and $\kappa=10^{-3}$ (so $C_P=0.002$), the Hilbert envelope bound \(C_H(\psi)\le 0.26\) holds. With \(c_0(\psi)=0.17620819\),
\[
  \frac{C_\Gamma + C_P + C_H}{c_0}\ \le\ \frac{0 + 0.002 + 0.26}{0.17620819}\ =\ 1.4869\ <\ \frac{\pi}{2}.
\]
Hence PSC holds on all Whitney boxes in the sum--form (not used for (P+)).
\end{corollary}
\begin{proof}
Apply the phase--velocity identity (Proposition~\ref{prop:phase-velocity-identity}) to a nonnegative test $\varphi_I$ supported on a $\sim L$ neighborhood of $I$ with $\varphi_I\equiv 1$ on $I$ (as fixed earlier in this section). The contribution from critical-line zeros is nonnegative. For off--critical zeros in $Q(I)$, Lemma~\ref{lem:whitney-lower} yields a uniform lower bound $\ge c_0$ for the Poisson balayage. The Archimedean, prime-side, and Hilbert pieces are bounded by $C_\Gamma L$, $C_P L$, and $C_H L$, respectively, by Lemmas~\ref{lem:arch}, \ref{lem:prime-short}, and~\ref{lem:hilbert}. Rearranging gives the inequality.
\end{proof}

% ===== Certificate block: explicit constants and one-line close =====
\subsection{Explicit constants and one-line certificate}\label{sec:certificate}
Fix an even, nonnegative window $\psi\in C_c^\infty([-1,1])$ with $\int_\R\psi=1$. For $L>0$ set
\[ \varphi_L(t):=\frac{1}{L}\,\psi\!\left(\frac{t}{L}\right),\quad \operatorname{supp}\varphi_L=[-L,L],\quad \int_\R \varphi_L=1. \]
Write $\widehat\psi(\omega)=\int_\R \psi(t)e^{-i\omega t}\,dt$, $\Poisson_a(x)=\tfrac{1}{\pi}\tfrac{a}{a^2+x^2}$, and let $\mathcal H$ denote the boundary Hilbert transform.

Define
\begin{align*}
 C_\Gamma^{(L)} &:= \left|\int_\R \varphi_L(t)\,\Im\frac{d}{dt}\log\!\left(\pi^{-s/2}\Gamma\!\left(\frac{s}{2}\right)\cdot\frac{s(1-s)}{2}\right)\!\Big|_{s=\frac12+it} dt\right|,\\
 C_P(\psi,L) &:= \left|\int_\R \varphi_L(t)\,\Im\Big(\frac{\zeta'}{\zeta}-\frac{\dettwo'}{\dettwo}\Big)\!\left(\tfrac12+it\right) dt\right|,\\
 C_H(\psi,L) &:= \left|\int_\R \varphi_L(t)\,\mathcal H[u'](t)\,dt\right|=\left|\int_\R \mathcal H[\varphi_L](t)\,u'(t)\,dt\right|,\\
 c_0(\psi) &:= \inf_{0<b\le 1,\,|x|\le 1} (\Poisson_b*\psi)(x).
\end{align*}
\begin{lemma}[Poisson scale reduction]\label{lem:poisson-scale}
For every $L>0$ and $\varphi_L(t)=L^{-1}\psi(t/L)$ one has the exact identity
\[
  (\Poisson_a*\varphi_L)(t)\ =\ (\Poisson_{a/L}*\psi)\!\left(\frac{t}{L}\right),\qquad a>0,\ t\in\R.
\]
Consequently,
\[
  \inf_{0<a\le L,\,|t|\le L}(\Poisson_a*\varphi_L)(t)\ =\ \inf_{0<b\le 1,\,|x|\le 1}(\Poisson_b*\psi)(x)\ =\ c_0(\psi).
\]
\end{lemma}
% (Legacy xi-normalized certificate omitted; not used in the zeta-normalized route.)

\subsection*{Uniform, explicit bound for the window mean–oscillation $M_\psi$}

Recall that for $I=[T{-}L,T{+}L]$ and the boundary modulus $u(t)$,
\[
  M_\psi\ :=\ \sup_{T\in\R,\ L>0}\ \frac{1}{|I|}\int_I\big|u(t)-\ell_I(t)\big|\,dt,
\]
where $\ell_I$ is the affine function agreeing with $u$ at the endpoints of $I$.

\begin{proposition}[Scale–explicit control of $M_\psi$]\label{prop:Mpsi-closed}
For the mass–1 window family $\varphi_L(t)=L^{-1}\psi((t-T)/L)$ used in the certificate,
\[
  M_\psi\ \le\ \frac{C_H(\psi)+C_P(\kappa)}{2}.
\]
In particular, with the printed constants $C_H(\psi)\le 0.65$ and $C_P(\kappa)\le 0.03$ (for $\kappa=0.015$),
\[
  M_\psi\ \le\ \frac{0.65+0.03}{2}\;=\;0.34\,.
\]
\end{proposition}
\begin{proof}
Let $U(\sigma,t)$ be the Poisson extension of $u$ to the upper half–plane and set $v(t):=u(t)-\ell_I(t)$. The affine subtraction kills the horizontal linear drift.

\emph{Step 1 (triangular vertical averaging, sharp $1/2$).} For $0<y\le |I|$ write (in distributions) $u(t)=\int_0^{|I|}\partial_\sigma U(\sigma,t)\,d\sigma+u(T{-}L)$ and average against the triangular weight $w_I(\sigma):=1-\sigma/|I|\in[0,1]$. By Fubini and positivity of $w_I$,
\[
  \frac{1}{|I|}\int_I |v(t)|\,dt
  \ \le\ \frac{1}{|I|}\int_0^{|I|}w_I(y)\Big(\int_I |\partial_\sigma U(y,t)|\,dt\Big)dy
  \ \le\ \Big(\frac{1}{|I|}\int_0^{|I|} w_I(y)\,dy\Big)\cdot\sup_{0<y\le |I|}\ \frac{1}{|I|}\int_I |\partial_\sigma U(y,t)|\,dt.
\]
Since $\int_0^{|I|} w_I(y)\,dy=|I|/2$, this yields the sharp factor $1/2$:
\[
  \frac{1}{|I|}\int_I |v(t)|\,dt\ \le\ \frac{1}{2}\ \sup_{0<y\le |I|}\ \frac{1}{|I|}\int_I |\partial_\sigma U(y,t)|\,dt.
\]

\emph{Step 2 (uniform radial $L^1$ control).} Decompose $\partial_\sigma U=\partial_\sigma U_H+\partial_\sigma U_P$. For the Hilbert piece, Lemma~\ref{lem:hilbert-H1BMO} and the identity $\partial_\sigma P_\sigma=\mathcal H[\partial_t P_\sigma]$ give
\[
  \sup_{0<y\le |I|}\ \frac{1}{|I|}\int_I |\partial_\sigma U_H(y,t)|\,dt\ \le\ C_H(\psi).
\]
For the prime piece, the bandlimit bound in the certificate (with $\Delta=\kappa/L$) yields uniformly in $y$,
\[
  \frac{1}{|I|}\int_I |\partial_\sigma U_P(y,t)|\,dt\ \le\ C_P(\kappa).
\]
Combining the two estimates,
\[
  \sup_{0<y\le |I|}\ \frac{1}{|I|}\int_I |\partial_\sigma U(y,t)|\,dt\ \le\ C_H(\psi)+C_P(\kappa).
\]
Insert this in Step 1 to conclude the claim.
\end{proof}

% Archived PSC corollary (removed from main text): legacy coarse constants and dependence claims superseded by the H$^1$–BMO bound and the Bridges/Schur route.

% Archived (not used): older near/far sketch for $M_\psi$; superseded by Lemma~\ref{lem:Mpsi-correct}.
% \end{lemma}
% \begin{proof}
% (Archived proof sketch omitted.)
% \end{proof}

\paragraph{Poisson lower bound $c_0(\psi)$ (exact formula and minimizer).}
Let $\psi\in L^1(\R)$ be even, nonnegative, and suppose $\psi\ge h$ on $[-1,1]$ for some $h>0$. For the Poisson kernel $P_b(x)=\frac{1}{\pi}\frac{b}{b^2+x^2}$ and any $x\in\R$, $b>0$,
\[
 (P_b*\psi)(x)\ \ge\ h\int_{-1}^1 P_b(x-t)\,dt\ =\ \frac{h}{\pi}\Big(\arctan\frac{1-x}{b}+\arctan\frac{1+x}{b}\Big).
\]
Therefore, for the mass--1 window $\varphi_L(t)=L^{-1}\psi(t/L)$ one has
\[
 c_0(\psi)\ :=\ \inf_{0<b\le 1,\ |x|\le 1} (P_b*\psi)(x)\ \ge\ \frac{h}{\pi}\,\inf_{0<b\le 1,\ |x|\le 1}\Big(\arctan\frac{1-x}{b}+\arctan\frac{1+x}{b}\Big).
\]
The function $F(x,b):=\arctan\big(\tfrac{1-x}{b}\big)+\arctan\big(\tfrac{1+x}{b}\big)$ is decreasing in $x\in[0,1]$ for each fixed $b>0$ and decreasing in $b\in(0,\infty)$ for each fixed $x\in[0,1]$. Thus the minimum over $|x|\le 1$, $0<b\le 1$ is attained at $(x,b)=(1,1)$, giving
\[
 c_0(\psi)\ \ge\ \frac{1}{2\pi(1+\delta)}\,\arctan 2.
\]
With $\delta=0.01$ this gives the explicit lower bound
\[
 c_0(\psi)\ \ge\ \frac{\arctan 2}{2\pi\cdot 1.01}\ \approx\ 0.1744\,.
\]
This is a fully rigorous bound that depends only on the pointwise plateau height $h$ and holds for any nonnegative $\psi$ with $\psi\ge h$ on $[-1,1]$.
\paragraph{Hilbert envelope $C_H(\psi)$ (step-by-step calculus bound).}
Write $\varphi_L(t)=L^{-1}\psi(t/L)$ with $\psi$ even, nonnegative, and constant on $[-1+\varepsilon,1-\varepsilon]$ at height $h=\tfrac{1}{2(1+\delta)}$ as above, and supported in $[-1-\varepsilon,1+\varepsilon]$ with smooth transitions on the layers $[1-\varepsilon,1+\varepsilon]$ and $[-1-\varepsilon,-1+\varepsilon]$. Set $x=t/L$ and define the normalized Hilbert profile $H_\psi(x):=\mathcal H[\psi](x)=\operatorname{p.v.}\,\frac{1}{\pi}\int_\R \frac{\psi(y)}{x-y}\,dy$. Then
\[
 \mathcal H[\varphi_L](t)\ =\ H_\psi\!\left(\frac{t}{L}\right),\qquad \sup_{t\in\R}|\mathcal H[\varphi_L](t)|\ =\ \sup_{x\in\R}|H_\psi(x)|.
\]
We estimate $H_\psi$ by splitting into the flat part and the transition layers. Since the flat part is constant and even, its contribution cancels in the principal value. Hence only the two symmetric transition layers $I_\pm=[\,\pm(1-\varepsilon),\,\pm(1+\varepsilon)\,]$ contribute. Let $S\in C^\infty([0,1])$ be the fixed monotone transition with $S(0)=1$, $S(1)=0$, and set
\[
 \psi(y)=\frac{h}{1}\,\mathbf 1_{|y|\le 1-\varepsilon}\ +\ h\,S\!\left(\frac{y-(1-\varepsilon)}{2\varepsilon}\right)\mathbf 1_{y\in I_+}\ +\ h\,S\!\left(\frac{-y-(1-\varepsilon)}{2\varepsilon}\right)\mathbf 1_{y\in I_-}.
\]
By symmetry, it suffices to bound $|H_\psi(x)|$ for $x\ge 0$. Using integration by parts on each transition interval,
\[
 \int_{1-\varepsilon}^{1+\varepsilon}\frac{S\big(\frac{y-(1-\varepsilon)}{2\varepsilon}\big)}{x-y}\,dy\ =\ \Big[ S\!\left(\frac{y-(1-\varepsilon)}{2\varepsilon}\right)\log|x-y|\Big]_{1-\varepsilon}^{1+\varepsilon}\ -\ \int_{1-\varepsilon}^{1+\varepsilon} S'\!\left(\frac{y-(1-\varepsilon)}{2\varepsilon}\right)\frac{\log|x-y|}{2\varepsilon}\,dy.
\]
The boundary terms cancel between the two symmetric layers. Using $S'\ge 0$, $\mathrm{supp}\,S'\subset[0,1]$, and the monotonicity of $y\mapsto \log|x-y|$ on each side of $x$, one gets the uniform bound
\[
 |H_\psi(x)|\ \le\ \frac{h}{\pi}\,\Big(\log\frac{x-(1-\varepsilon)}{x-(1+\varepsilon)}\Big)_+\ +\ \frac{h}{\pi}\,\Big(\log\frac{x+(1+\varepsilon)}{x+(1-\varepsilon)}\Big)_+\ \le\ \frac{2h}{\pi}\,\log\frac{1+\varepsilon}{1-\varepsilon},
\]
where $(\cdot)_+$ denotes the positive part and we used that the worst case occurs at $x=0$ by symmetry/monotonicity. Choosing, for instance, $\varepsilon=0.01$ and $\delta=0.01$ (so $h=1/(2(1+\delta))$) yields the explicit numerical estimate
\[
 \sup_{x\in\R}|H_\psi(x)|\ \le\ \frac{1}{\pi(1+\delta)}\,\log\frac{1+\varepsilon}{1-\varepsilon}\ \le\ 0.70.
\]
Consequently
\[
 \sup_{t\in\R}|\mathcal H[\varphi_L](t)|\ =\ \sup_{x\in\R}|H_\psi(x)|\ \le\ 0.70,
\]
\noindent Coarse envelope only (not used). The certificate uses the refined bound $\mathbf{0.65}$ proved earlier. The constants $\varepsilon,\delta$ are fixed and explicit; any small values with the displayed inequality suffice.

% (Archimedean term is absent in the zeta-normalized route; see earlier paragraph.)

\paragraph{Bandlimit term $C_P(\kappa)$ (explicit bound).}
Let $\phi_I:=\mathsf S_\Delta\varphi_L$ be the band-limited version of the window with $\widehat{\mathsf S_\Delta f}(\xi)=W(\xi/\Delta)\widehat f(\xi)$, where $W\in C_c^\infty([-1,1])$ with $W\equiv 1$ near $0$, and choose $\Delta=\kappa$ (independent of $L$). Then
\[
 \int_{\R}\Im\Big(\frac{\zeta'}{\zeta}-\frac{\dettwo'}{\dettwo}\Big)\!(\tfrac12+it)\,\phi_I(t)\,dt\ =\ \Re\sum_{p}(\log p)\,p^{-1/2}\,\widehat{\phi_I}(\log p)\ +\ E,
\]
where $E$ is the (absolutely convergent) prime-power tail, bounded uniformly by the smoothing. Since $\widehat{\phi_I}$ is supported in $|\xi|\le \Delta=\kappa$ and $\|\widehat{\phi_I}\|_\infty\le\|\phi_I\|_1=1$, only primes with $\log p\in[0,\kappa]$ occur. Using Chebyshev's bound $\sum_{\log p\le \kappa}\log p\,p^{-1/2}\le 2\kappa$ (a standard partial summation with $\pi(x)\le \tfrac{x}{\log x}$) and absorbing $E$ gives
\[
 C_P(\kappa)\ \le\ 2\kappa.
\]
This estimate is uniform in $T$ and $L$ and depends only on the fixed cutoff profile $W$.

\section{Operator--ideals primer (Schatten classes and the $\det_2$ calculus)}\label{sec:operator-ideals-primer}

\begin{definition}[Hilbert--Schmidt and trace classes]\label{def:S2S1}
Let $H$ be a complex Hilbert space. The Hilbert--Schmidt class is $\mathcal S_2(H):=\{A\in\mathcal B(H):\ \|A\|_2^2=\Tr(A^*A)<\infty\}$ and the trace class is $\mathcal S_1(H):=\{A:\ \|A\|_1=\Tr((A^*A)^{1/2})<\infty\}$. One has $\mathcal S_2\cdot\mathcal S_2\subset\mathcal S_1$ and, for $X\in\mathcal S_1$ and bounded $Y$, the cyclicity $\Tr(XY)=\Tr(YX)$.
\end{definition}

\begin{definition}[Carleman--Fredholm regularization]\label{def:det2-cf}
For $A\in\mathcal S_2(H)$ the regularized determinant is
\[
  \det\nolimits_2(I-A)\ :=\ \det\big((I-A)\,e^{A}\big),
\]
where $\det$ on the right is the Fredholm determinant (well defined because $(I-A)e^A-I\in\mathcal S_1$). When $\|A\|<1$ one has the convergent series
\[
  \log\det\nolimits_2(I-A)\ =\ -\sum_{n\ge 2}\frac{\Tr(A^n)}{n}.
\]
\end{definition}

\begin{lemma}[Power--trace bound]\label{lem:power-trace}
If $A\in\mathcal S_2(H)$ and $\|A\|\le \rho<1$, then for every integer $n\ge 2$,
\[
  |\Tr(A^n)|\ \le\ \|A^2\|_1\,\|A^{n-2}\|\ \le\ \|A\|_2^{\,2}\,\rho^{\,n-2}.
\]
\end{lemma}

\begin{proof}
Since $A^2\in\mathcal S_1$ and $A^{n-2}$ is bounded, $|\Tr(A^n)|=|\Tr(A^2A^{n-2})|\le \|A^2\|_1\,\|A^{n-2}\|$. The estimates $\|A^2\|_1\le \|A\|_2^2$ and $\|A^{n-2}\|\le\rho^{\,n-2}$ give the claim.
\end{proof}

\begin{proposition}[Uniform convergence on vertical lines; holomorphy]\label{prop:det2-vertical}
Let $s\mapsto T(s)$ be holomorphic on a vertical strip $\{\sigma_1<\Re s<\sigma_2\}$ with values in $\mathcal S_2(H)$. Fix $\sigma\in(\sigma_1,\sigma_2)$ and assume
\[
  \sup_{t\in\R}\,\|T(\sigma+it)\|\ \le\ \rho\ <\ 1,\qquad
  \sup_{t\in\R}\,\|T(\sigma+it)\|_2\ \le\ H\ <\ \infty.
\]
Then the series $\sum_{n\ge 2}\Tr\big(T(\sigma+it)^n\big)/n$ converges uniformly in $t\in\R$ and
\[
  \log\det\nolimits_2\big(I-T(\sigma+it)\big)\ =\ -\sum_{n\ge 2}\frac{\Tr\big(T(\sigma+it)^n\big)}{n}
\]
defines a continuous function of $t$. Moreover $s\mapsto \det\nolimits_2(I-T(s))$ is holomorphic on the vertical line $\Re s=\sigma$, and the same holds on any compact vertical sub-strip where the bounds are uniform.
\end{proposition}

\begin{proof}
By Lemma~\ref{lem:power-trace}, $|\Tr(T(s)^n)|\le H^2\,\rho^{\,n-2}$ for $n\ge 2$, uniformly in $t$. The Weierstrass M-test yields uniform convergence of the series for $\log\det\nolimits_2(I-T)$ in $t$. Holomorphy in $s$ on the line follows because $s\mapsto T(s)$ is holomorphic into $\mathcal S_2$ and $A\mapsto \Tr(A^n)$ is continuous multilinear; uniform convergence allows termwise differentiation. Alternatively, combine the bounds with the general holomorphy of $K\mapsto\det\nolimits_2(I-K)$ on $\HS$ (see Lemma~\ref{lem:carleman} and Proposition~\ref{prop:HS-to-det2}).
\end{proof}

\section{Bridge A: determinant--zeta link (proved on $\Re s>1$, continued to $\Re s>\tfrac12$)}\label{sec:bridge-A}

\begin{definition}[Prime--diagonal operator]\label{def:T}
Let $\mathcal H:=\ell^{2}(\mathbb P)$ with orthonormal basis $\{e_{p}\}_{p\in\mathbb P}$. For $s=\sigma+it$ with $\sigma>1/2$ define the bounded operator $T(s):\mathcal H\to\mathcal H$ by
\[
  T(s)e_{p} \;=\; p^{-s}\,e_{p}\qquad(p\in\mathbb P).
\]
\end{definition}

\begin{lemma}[Hilbert--Schmidt and holomorphy]\label{lem:HS}
For every $\sigma>1/2$ the operator $T(s)$ is Hilbert--Schmidt with
\[
  \|T(s)\|_{\mathsf{HS}}^{2}
  \;=\;\sum_{p} |p^{-s}|^{2}
  \;=\;\sum_{p} p^{-2\sigma}
  \;<\;\infty,
\]
uniformly in $t\in\mathbb R$. Moreover $s\mapsto T(s)$ is holomorphic (as an operator--valued map) on the half--plane $\Re s>1/2$.
\end{lemma}

\begin{lemma}[Carleman--Fredholm determinant for diagonal HS operators]\label{lem:det2diag}
For a diagonal Hilbert--Schmidt operator $A=\operatorname{diag}(a_{j})$, the $2$--regularised determinant exists and equals
\[
  \det_{2}(I-A)\;=\;\prod_{j}\,(1-a_{j})\,e^{\,a_{j}},
\]
and
\[
  \log \det_{2}(I-A)
  = \sum_{j}\bigl(\log(1-a_{j})+a_{j}\big)
  = -\sum_{n\ge2}\frac{1}{n}\sum_{j} a_{j}^{n},
\]
with absolute convergence.\cite{SimonTraceIdeals}
\end{lemma}

% [Removed older Bridge A renormalizer via P(s); unified with hardened Bridge A in Subsection~\ref{subsec:hardened-BridgeA}]

\subsection*{Structural redesign: triangular padding and trace-lock for det$_2$}
\begin{definition}[Redesigned arithmetic operator]\label{def:Tnew}
Let $K:\ell^2(\PP)\to\ell^2(\PP)$ be a fixed, $s$-independent, strictly upper-triangular Hilbert--Schmidt operator in the prime basis $\{e_p\}$, i.e., $\langle e_p,Ke_q\rangle=0$ whenever $p\ge q$. Define
\[
  T_{\mathrm{new}}(s)\;:=\;T(s)+K.
\]
\end{definition}

\begin{lemma}[Upper-triangular power--diagonal rule]\label{lem:triang-power}
If $U$ and $V$ are upper-triangular (bounded) operators with respect to the same orthonormal basis, then $UV$ is upper-triangular and $(UV)_{ii}=U_{ii}V_{ii}$ for every index $i$. In particular, if $U$ is upper-triangular then $(U^n)_{ii}=(U_{ii})^n$ for all integers $n\ge 1$.
\end{lemma}
\begin{proof}
For upper-triangular $U,V$ one has $(UV)_{ij}=\sum_{k\ge i}U_{ik}V_{kj}$ and $(UV)_{ij}=0$ whenever $i>j$, so $UV$ is upper-triangular. On the diagonal,
\[
  (UV)_{ii}=\sum_{k}U_{ik}V_{ki}=U_{ii}V_{ii},
\]
because $V_{ki}=0$ for $k>i$ and $U_{ik}=0$ for $k<i$. Iterating yields $(U^n)_{ii}=(U_{ii})^n$.
\end{proof}

\begin{lemma}[Trace--lock]\label{lem:tracelock}
Let $T(s)$ be diagonal in the prime basis and let $K\in\HS$ be strictly upper-triangular in that basis. Set $U(s):=T(s)+K$. Then for every integer $n\ge 2$ and $\Re s>\tfrac12$,
\[
  U(s)^n\in\mathcal S_1\quad\text{and}\quad \Tr\big(U(s)^n\big)=\Tr\big(T(s)^n\big).
\]
\end{lemma}
\begin{proof}
Since $T$ is diagonal and $K$ is strictly upper-triangular in the same basis, $U=T+K$ is upper-triangular with $U_{ii}=T_{ii}$. By Lemma~\ref{lem:triang-power}, $(U^n)_{ii}=(U_{ii})^n=(T_{ii})^n$ for all $n\ge 1$. For $n\ge 2$, every monomial in the expansion of $U^n$ contains at least two factors from $\{T,K\}$, and with $T,K\in\HS$ this implies $U^n\in\mathcal S_1$ by the Schatten product rule $\mathcal S_2\cdot\mathcal S_2\subset\mathcal S_1$.

Let $P_N$ be the orthogonal projection onto the span of the first $N$ basis vectors. Then $U_N:=P_N U P_N$ and $T_N:=P_N T P_N$ are finite upper-triangular matrices with the same diagonal, so $\Tr\big(U_N^n\big)=\sum_{i\le N}(U_N^n)_{ii}=\sum_{i\le N}(T_{ii})^n=\Tr\big(T_N^n\big)$ by Lemma~\ref{lem:triang-power}. Moreover, $\|U^n-U_N^n\|_1\to 0$ and $\|T^n-T_N^n\|_1\to 0$ as $N\to\infty$ (each difference is a finite sum of words with at least one factor $(I-P_N)$ sandwiching an $\HS$ operator). Continuity of the trace on $\mathcal S_1$ yields
\[
  \Tr\big(U^n\big)=\lim_{N\to\infty}\Tr\big(U_N^n\big)=\lim_{N\to\infty}\Tr\big(T_N^n\big)=\Tr\big(T^n\big).
\]
\end{proof}

\begin{proposition}[det$_2$ invariance under triangular padding]\label{prop:det2-invariance}
With $T, K$ as above and $\Re s>\tfrac12$,
\[
  \det\nolimits_2\!\big(I-(T(s)+K)\big)\ =\ \det\nolimits_2\!\big(I-T(s)\big).
\]
\end{proposition}
\begin{proof}
Let $A:=T(s)+K$ and $B:=T(s)$. Consider the entire function
\[
  h(z)\ :=\ \frac{\det\nolimits_2(I-zA)}{\det\nolimits_2(I-zB)}\,.
\]
The map $z\mapsto zA$ is holomorphic into $\HS$, hence $z\mapsto \det\nolimits_2(I-zA)$ is entire (and likewise for $B$), so $h$ is entire. For $|z|$ small, the Carleman--Fredholm series gives
\[
  \log h(z)\ =\ -\sum_{n\ge 2}\frac{\Tr(A^n)-\Tr(B^n)}{n}\,z^n\,=\,0
\]
by Lemma~\ref{lem:tracelock}. Therefore $h\equiv 1$ by the identity theorem, and evaluating at $z=1$ gives the claim.
\end{proof}

\begin{corollary}[Bridge A closed for $T_{\mathrm{new}}$]\label{cor:BA-closed}
With $T_{\mathrm{new}}$ from Definition~\ref{def:Tnew},
\[
  \xi(s)\;=\;e^{L(s)}\,\det_{2}\big(I-T_{\mathrm{new}}(s)\big),\qquad \Re s>\tfrac12+\eta.
\]
In particular, the auxiliary factor equals the diagonal normalizer $E_{\rm diag}(s):=e^{L(s)}$, which is zero-free on $\{\Re s>\tfrac12+\eta\}$ by construction of branches.
\end{corollary}

\begin{remark}[Certificate compatibility and a concrete $K$]\label{rem:K-choice}
Let $\sigma_{\min}>\tfrac12$ be the minimal abscissa in the covering schedule used in Bridges~B--C. For primes $p<q$ set
\[
  K_{pq}\;:=\;c\,(pq)^{-(\sigma_{\min}+1/2)},\qquad K_{pp}=0,\quad K_{pq}=0\ (p\ge q),
\]
with a scalar $c\in(0,1]$. Then $K\in\HS$ and is strictly upper-triangular. Moreover, for all $\sigma\ge\sigma_{\min}$,
\[
  \sum_{q\ne p} |K_{pq}|\ \le\ c\,p^{-(\sigma+1/2)}\sum_{q} q^{-(\sigma+1/2)},\qquad
  \sum_{p\ne q} |K_{pq}|\ \le\ c\,q^{-(\sigma+1/2)}\sum_{p} p^{-(\sigma+1/2)},
\]
so the Schur row/column budgets receive an additive, $\sigma$-nonincreasing contribution controlled by the prime-tail sums already used in the certificate. Choosing $c>0$ small enough makes this contribution negligible relative to the certified margins $\Delta_{\mathrm{SS}},\Delta_{\mathrm{SF}},\Delta_{\mathrm{FS}},\Delta_{\mathrm{FF}}$ on $[\sigma_{\min},1]$.
\end{remark}

\paragraph{Budget simplification.}
Because $K$ is strictly upper-triangular in the prime order, there are no far$\to$far cycles contributed by $K$; hence $\Delta_{\mathrm{FF}}^{(K)}=0$. The far$\to$small budget is controlled by the column sums above and decreases with $\sigma$.

\subsection*{Hardened Bridge A on $\Re s>\tfrac12$ (zero-free normalizer)}\label{subsec:hardened-BridgeA}
On the half-plane $\Re s>1$ set
\[
  L_0(s)\ :=\ \log\!\Big(\tfrac12 s(1-s)\,\pi^{-s/2}\,\Gamma\!\big(\tfrac{s}{2}\big)\Big)
             \ +\ \sum_{p} p^{-s},
\]
so $e^{L_0}$ is holomorphic and nowhere zero on $\{\Re s>1\}$. Using Lemma~\ref{lem:det2diag} and the trace--lock Lemma~\ref{lem:tracelock} one gets, for $\Re s>1$,
\[
  \log\xi(s)\ =\ L_0(s)\ +\ \log\det\nolimits_2\!\big(I-T_{\rm new}(s)\big),
\]
equivalently $\xi=e^{L_0}\det_2(I-T_{\rm new})$ on $\Re s>1$.

\begin{theorem}[Bridge A on $\Re s>\tfrac12$ with zero-free normalizer]\label{thm:hardened-bridgeA}
There exists a holomorphic $L$ on $\{\Re s>\tfrac12\}$, uniquely anchored by $L(2)=L_0(2)\in\R$, such that
\[
  \xi(s)\ =\ e^{L(s)}\,\det\nolimits_2\!\big(I-T_{\mathrm{new}}(s)\big)\qquad(\Re s>\tfrac12),
\]
and $e^{L}$ is holomorphic and zero-free on $\{\Re s>\tfrac12\}$.
\end{theorem}

\begin{proof}
Define $F(s):=\xi(s)-e^{L_0(s)}\det_2(I-T_{\rm new}(s))$ on $\{\Re s>1\}$. By the preceding identity, $F\equiv 0$ on $\Re s>1$. Both $\xi$ and $\det_2(I-T_{\rm new}(\cdot))$ are holomorphic on $\{\Re s>\tfrac12\}$, so by uniqueness of analytic continuation there is a holomorphic continuation $e^{L}$ of $e^{L_0}$ to $\{\Re s>\tfrac12\}$ (with branch fixed by $L(2)=L_0(2)\in\R$) such that $\xi=e^{L}\det_2(I-T_{\rm new})$ on $\{\Re s>\tfrac12\}$. Since $e^{L}$ is the exponential of a holomorphic function, it is zero-free on its domain.
\end{proof}

\section{Bridges B--C: Finite-to-full propagation and diagonal covering}

In this section we record complete, self-contained proofs of the two operator bridges that transport a certified finite-block Schur gap to a global gap on vertical lines and then along a diagonal covering to $\Re s=\tfrac12+\eta$. Bridge A (the determinant--zeta identity) is stated earlier and remains an explicit hypothesis; see the status note below.

% --- Bridge A: trace--lock under strictly upper--triangular padding ---
\begin{lemma}[Trace--lock for diagonal + strictly upper--triangular]\label{lem:trace-lock}
Let $H=\ell^2(\mathbb{P})$ with the prime-ordered orthonormal basis $\{e_p\}$. Fix $s$ with $\Re s> \tfrac12$ and let 
\[
T(s):=\sum_{p} p^{-s}\,\Pi_p,\qquad \Pi_p x:=\langle x,e_p\rangle e_p,
\]
so $T(s)$ is diagonal in the $\{e_p\}$ basis. Let $K\in \mathcal S_2(H)$ be any bounded operator that is strictly upper--triangular in this basis and satisfies $\langle Ke_p,e_p\rangle=0$ for all $p$. Then for every integer $n\ge 2$,
\[
\mathrm{Tr}\big((T(s)+K)^n\big)=\mathrm{Tr}\big(T(s)^n\big)=\sum_{p} p^{-ns}.
\]
\end{lemma}

\begin{proof}
Expand $(T+K)^n$ into monomials in $T$ and $K$. Any monomial that contains at least one factor $K$ is a product of diagonal and strictly upper--triangular matrices. Such products remain strictly upper--triangular and have zero diagonal, hence zero trace. Only $T^n$ contributes to the trace.
\end{proof}

\begin{corollary}[det$_2$ invariance under triangular padding]\label{cor:det2-invariance}
With $T, K$ as above and $\Re s>\tfrac12$,
\[
\log\det\nolimits_2\!\big(I-(T(s)+K)\big) = \log\det\nolimits_2\!\big(I-T(s)\big).
\]
Consequently, writing $\xi(s)=e^{L(s)}\det_2(I-T(s))$ on $\Re s>\tfrac12$ gives
\[
\xi(s)=e^{L(s)}\det\nolimits_2\!\big(I-(T(s)+K)\big).
\]
\end{corollary}

\subsection*{Bridge C: Neumann step and diagonal covering}

We quantify how the Schur gap degrades under a small change of $\sigma$.

\begin{lemma}[Row-sum Lipschitz bound]\label{lem:rowsum-lip}
Let $\sigma>\tfrac12$ and $h\in\R$. For the weighted $p$-adaptive model one has, uniformly in $t\in\R$,
\[
  \sup_p \sum_q \big| T_{pq}(\sigma+h+it)-T_{pq}(\sigma+it)\big|\ \le\ K(\sigma)\,|h|\,\sup_p \sum_q |T_{pq}(\sigma+it)|,
\]
where $K(\sigma)$ is the explicit Lipschitz majorant defined in the covering (the derivative-of-log--row-sum majorant). The same bound holds with rows and columns interchanged. Consequently, by Schur's test,
\[
  \|T(\sigma+h+it)-T(\sigma+it)\|\ \le\ K(\sigma)\,|h|\,\|T(\sigma+it)\|_{\mathrm{Schur}}\ \le\ K(\sigma)\,|h|\,(1-\delta_{\mathrm{Schur}}(\sigma)).
\]
\end{lemma}
\begin{proof}
For $U_{pq}(\sigma)=\frac{C_{\mathrm{win}}}{4}\,p^{-a}q^{-a}$ with $a=\sigma+\tfrac12$, one computes $\partial_\sigma U_{pq}=-(\log p+\log q)\,U_{pq}$. Summing over $q$ at fixed $p$ and bounding the log-weights by their weighted average gives $\partial_\sigma\,\sum_q U_{pq}\le K(\sigma)\,\sum_q U_{pq}$. Integrating in $\sigma$ over length $|h|$ yields the stated row-sum inequality; columns are analogous. Schur's test gives the operator-norm bound and the final inequality uses $\|T\|_{\mathrm{Schur}}\le 1-\delta_{\mathrm{Schur}}(\sigma)$.
\end{proof}

\begin{lemma}[Neumann step]\label{lem:neumann-step}
Suppose $\|T(\sigma+it)\|\le 1-\delta$ and $\|T(\sigma+h+it)-T(\sigma+it)\|\le \vartheta\,\delta$ with $\vartheta\in[0,1)$. Then $I-T(\sigma+h+it)$ is invertible and
\[
  \delta_{\mathrm{Schur}}(\sigma+h)\ \ge\ (1-\vartheta)\,\delta_{\mathrm{Schur}}(\sigma).
\]
\end{lemma}
\begin{proof}
Write $E:=T(\sigma+h+it)-T(\sigma+it)$. The resolvent identity gives $I-T(\sigma+h)= (I-T(\sigma))\,\big(I-(I-T(\sigma))^{-1}E\big)$. Since $\|(I-T(\sigma))^{-1}\|\le 1/\delta$ and $\|E\|\le \vartheta\delta$, the inner factor is invertible by a Neumann series with inverse norm $\le 1/(1-\vartheta)$. Thus $\|(I-T(\sigma+h))^{-1}\|\le \|(I-T(\sigma))^{-1}\|\,\frac{1}{1-\vartheta}$, which is equivalent to the displayed gap inequality.
\end{proof}

\paragraph{Differential Bridge C.}
\begin{proposition}[Differential propagation bound]\label{prop:diff-bridgeC}
Let $\delta_{\mathrm{Schur}}:(\tfrac12,1]\to(0,\infty)$ be the line-wise Schur gap defined in \S\ref{sec:bridge-C}. Assume $\delta_{\mathrm{Schur}}$ is locally absolutely continuous (which holds since the row sums are locally Lipschitz in $\sigma$ by Lemma~\ref{lem:rowsum-lip}). Then for a.e. $\sigma\in(\tfrac12,1]$,
\[
  \frac{d}{d\sigma}\,\log \delta_{\mathrm{Schur}}(\sigma)\ \ge\ -\,K(\sigma).
\]
\end{proposition}

\begin{proof}
Fix $\sigma$ and a small $h<0$. By Lemma~\ref{lem:rowsum-lip} with step $h$ and Lemma~\ref{lem:neumann-step} with $\vartheta=K(\sigma)|h|$, we have
\[
  \delta_{\mathrm{Schur}}(\sigma+h)\ \ge\ \big(1-K(\sigma)|h|\big)\,\delta_{\mathrm{Schur}}(\sigma).
\]
Taking logs, dividing by $h<0$, and letting $h\uparrow 0$ yields
\[
  \liminf_{h\uparrow 0}\ \frac{\log\delta_{\mathrm{Schur}}(\sigma+h)-\log\delta_{\mathrm{Schur}}(\sigma)}{h}\ \ge\ -\,K(\sigma).
\]
Local absolute continuity of $\delta_{\mathrm{Schur}}$ implies $\log\delta_{\mathrm{Schur}}$ is a.e. differentiable with derivative equal a.e. to the limit of the difference quotient, giving the claim.
\end{proof}

\begin{theorem}[Differential Bridge C covering]\label{thm:diff-bridgeC}
Let $\sigma_0\in(\tfrac12,1)$ and suppose the differential bound of Proposition~\ref{prop:diff-bridgeC} holds on $[\tfrac12+\eta,\sigma_0]$. Then for every $\sigma\in[\tfrac12+\eta,\sigma_0]$,
\[
  \delta_{\mathrm{Schur}}(\sigma)\ \ge\ \delta_{\mathrm{Schur}}(\sigma_0)\,\exp\!\Big(-\int_{\sigma_0}^{\sigma} K(u)\,du\Big).
\]
In particular, any a priori bound $\int_{\tfrac12+\eta}^{\sigma_0}K(u)\,du\le \Lambda$ guarantees
\[
  \delta_{\mathrm{Schur}}(\tfrac12+\eta)\ \ge\ \delta_{\mathrm{Schur}}(\sigma_0)\,e^{-\Lambda}\ >\ 0.
\]
\end{theorem}

\begin{proof}
Integrate the differential inequality in Proposition~\ref{prop:diff-bridgeC} from $\sigma_0$ to $\sigma$ and exponentiate.
\end{proof}

\begin{definition}[Admissible schedule generator]\label{def:admissible-schedule}
Fix $h_{\max}>0$ and $\varepsilon\in(0,1]$. Define the step-size and grid by
\[
  h(\sigma):=\min\!\Big\{\,h_{\max},\ \frac{\varepsilon}{1+K(\sigma)}\,\Big\},\qquad \sigma_{n+1}:=\sigma_n-h(\sigma_n),\quad \sigma_0>\tfrac12.
\]
Then $\theta_n:=K(\sigma_n)h(\sigma_n)\le \varepsilon\le \tfrac12$, so one may fall back on the discrete Neumann/Gershgorin step. Moreover, for $N$ with $\sigma_N\le \tfrac12+\eta$,
\[
  \delta_{\mathrm{Schur}}(\sigma_N)\ \ge\ \delta_{\mathrm{Schur}}(\sigma_0)\,\exp\!\Big(-\int_{\sigma_N}^{\sigma_0} K(u)\,du\Big)\ >\ 0.
\]
\end{definition}
\subsection*{Bridge B: Line factorization and finite-to-full invertibility}
We prove an explicit factorization on each certified vertical line $\Re s=\sigma$ of the form $\zeta(s)^{-1}=E_\varepsilon(s)D_\varepsilon(s)$ and establish a uniform lower bound for $|E_\varepsilon(\sigma+it)|$ using only unconditional estimates. We then close the finite-to-full invertibility by a weighted $\ell^2$ Schur test with a minimax choice of weights, yielding a certified Schur gap $\delta_{\mathrm{Schur}}(\sigma)$ and line nonvanishing. Prime-tail lemmas (PT--0/PT--1) provide unconditional envelopes for the tail budgets entering both the link lower bound and the Schur closure.

\subsection*{Explicit line factorization on $\Re s=\sigma$}
Fix $\sigma\in(\tfrac12,1)$ and an integer cut $Q\ge 29$ with $p_{\min}=\mathrm{nextprime}(Q)$. Define the diagonal tail operator
\[
  A_{>Q}(s):=\bigoplus_{p>Q} p^{-s}\,\Pi_p,\qquad s=\sigma+it,
\]
and for $\varepsilon>0$ the trace--class family
\[
  \mathcal K_{\sigma,\varepsilon}(t)
    := e^{-\varepsilon\sqrt{1+\log^2 A_{>Q}}}\; A_{>Q}(\sigma+it)\; e^{-\varepsilon\sqrt{1+\log^2 A_{>Q}}},
\]
where the functional calculus acts diagonally on the prime basis. Then $\mathcal K_{\sigma,\varepsilon}(t)\in\mathcal S_1$ uniformly in $t\in\R$ and depends real--analytic in $t$.

\begin{lemma}[Analyticity and Fredholm determinant on the line]\label{lem:line-fredholm}
For each fixed $\sigma\in(\tfrac12,1)$ and $\varepsilon>0$, the map $t\mapsto \mathcal K_{\sigma,\varepsilon}(t)$ is real--analytic into $\mathcal S_1$, and the Fredholm determinant
\[
  D_\varepsilon(\sigma+it):=\det\big(I-\mathcal K_{\sigma,\varepsilon}(t)\big)
\]
is continuous in $t$ with the absolutely convergent expansion
\[
  \log D_\varepsilon(\sigma+it)\;=\; -\sum_{m\ge 1}\frac{1}{m}\,\Tr\big(\mathcal K_{\sigma,\varepsilon}(t)^m\big)
  \;=\; -\sum_{p>Q}\log\Big(1-e^{-\varepsilon\sqrt{1+(\log p)^2}}\,p^{-(\sigma+it)}\Big).
\]
\end{lemma}

\begin{theorem}[Explicit factorization on the vertical line]\label{thm:line-factorization}
With $E_\varepsilon$ defined by
\[
  E_\varepsilon(s)
  := \Bigg[\prod_{p\le Q}\big(1-p^{-s}\big)\Bigg]\,
     \exp\!\Bigg(\sum_{p>Q}\Big(p^{-s}-e^{-\varepsilon\sqrt{1+(\log p)^2}}\,p^{-s}\Big)\Bigg),\qquad s=\sigma+it,
\]
one has the identity on the line $\Re s=\sigma$:
\[
  \boxed{\ \zeta(s)^{-1}\;=\;E_\varepsilon(s)\,D_\varepsilon(s)\ }.
\]
Moreover, $E_\varepsilon$ is entire and nowhere zero, and for all $t\in\R$
\[
  \big|E_\varepsilon(\sigma+it)\big|\ \ge\ \exp\!\big( -L(\sigma)\big),\qquad
  L(\sigma):=(1-\sigma)\,(\log p_{\min})\,p_{\min}^{-\sigma}.
\]
\end{theorem}

\begin{proof}
By Lemma~\ref{lem:line-fredholm} and the Euler product, for $\Re s=\sigma$,
\[
  \prod_{p\le Q}(1-p^{-s})\cdot \prod_{p>Q}\frac{1}{1-p^{-s}}
  \;=\;\zeta(s)^{-1}.
\]
The exponential factor rewrites the tail product as a convergent Fredholm determinant for $\mathcal K_{\sigma,\varepsilon}$, giving the claimed identity. The finite product over $p\le Q$ is nonzero on $\Re s>0$, and the exponential has no zeros. For the lower bound, keep only the $p=p_{\min}$ contribution and use $\log(1-x)\ge -x/(1-x)$ with $x=p_{\min}^{-\sigma}$ and $1-x\ge (1-\sigma)\log p_{\min}$ to obtain $\log|E_\varepsilon(\sigma+it)|\ge -L(\sigma)$; exponentiate.
\end{proof}

\begin{corollary}[Schur gap implies zero--free line]\label{cor:line-zerofree}
If $\delta_{\mathrm{Schur}}(\sigma)>0$ for the audited model on $\Re s=\sigma$, then $D_\varepsilon(\sigma+it)\ne 0$ for all $t$, hence $\zeta(\sigma+it)\ne 0$ for all $t\in\R$ by Theorem~\ref{thm:line-factorization}.
\end{corollary}

\begin{theorem}[Bridge C: diagonal covering]\label{thm:bridge-C}
Assume the differential bound of Proposition~\ref{prop:diff-bridgeC} holds on $[\tfrac12+\eta,\sigma_0]$. Then for any admissible schedule generated by Definition~\ref{def:admissible-schedule} one has
\[
  \delta_{\mathrm{Schur}}(\sigma_N)\ \ge\ \delta_{\mathrm{Schur}}(\sigma_0)\,\exp\!\Big(-\int_{\sigma_N}^{\sigma_0}K(u)\,du\Big)\ >\ 0
\]
as soon as $\sigma_N\le \tfrac12+\eta$. In particular, with $\varepsilon\le \tfrac12$ the discrete safety condition $\theta_k\le \tfrac12$ holds at every step and the product bound
\[
  \delta_{\mathrm{Schur}}(\sigma_N)\ \ge\ \delta_{\mathrm{Schur}}(\sigma_0)\,\prod_{k< N}(1-\theta_k)
\]
follows as a corollary.
\end{theorem}

\begin{corollary}[Explicit schedule and product bound]\label{cor:explicit-schedule}
Fix $\sigma_0\in(\tfrac12,1)$ and set $\sigma_{k+1}:=\sigma_k-h_k$ with
\[
  h_k\ :=\ \min\Big\{\, \tfrac{1}{2\,K(\sigma_k)}\,,\ 10^{-3}\,\Big\}.
\]
Then $\theta_k:=K(\sigma_k)\,|h_k|\le \tfrac12$ for all $k$, and
\[
  \delta_{\mathrm{Schur}}(\sigma_{k+1})\ \ge\ \delta_{\mathrm{Schur}}(\sigma_k)\,(1-\theta_k).
\]
Consequently, for every $N\ge 1$,
\[
  \delta_{\mathrm{Schur}}(\sigma_{N})\ \ge\ \delta_{\mathrm{Schur}}(\sigma_0)\,\prod_{k=0}^{N-1}\big(1-\theta_k\big)\ >\ 0.
\]
\end{corollary}

\begin{proof}
By construction, $\theta_k\le \tfrac12$. Apply Lemma~\ref{lem:rowsum-lip} with $h=h_k$ and Lemma~\ref{lem:neumann-step} with $\vartheta=\theta_k$ to obtain the one-step bound. Iterating yields the product.
\end{proof}

\begin{theorem}[Diagonal covering to lines; corrected Bridge C]\label{thm:diag-cover-corrected}
Fix $\varepsilon\in(0,\tfrac12]$ and a vertical line $\{\Re s=\sigma\}$ with $\sigma\in(\tfrac12,1)$. Suppose the blockwise Schur/Gershgorin audit on this line returns a positive spectral margin 
\[
\delta_{\mathrm{Schur}}(\sigma)\ :=\ \inf_{t\in\mathbb{R}}\,\big\| (I-K_{\sigma,\varepsilon}(\sigma+it))^{-1}\big\|^{-1}\ >\ 0.
\]
Then $\zeta(\sigma+it)\neq 0$ for all $t\in\R$.
\end{theorem}

\begin{proof}
If $\delta_{\mathrm{Schur}}(\sigma)>0$, then $I-K_{\sigma,\varepsilon}(\sigma+it)$ is invertible uniformly in $t$, hence $D_\varepsilon(\sigma+it):=\det(I-K_{\sigma,\varepsilon}(\sigma+it))\neq 0$. The explicit line factorization gives $\zeta^{-1}=E_\varepsilon D_\varepsilon$ with a link factor $E_\varepsilon$ bounded below away from $0$ on the line. Thus $\zeta(\sigma+it)\neq 0$.
\end{proof}

\begin{theorem}[Bridges A--C imply RH]\label{thm:bridges-imply-RH}
Assume: (A) the det--zeta factorization $\xi(s)=e^{L(s)}\det_2(I-T_{\mathrm{new}}(s))$ holds on $\Re s>\tfrac12$ with $e^{L(s)}\neq 0$, and (B) for each $\sigma\in(\tfrac12,1)$ the Schur audit yields $\delta_{\mathrm{Schur}}(\sigma)>0$. Then $\zeta(s)\ne 0$ for all $\Re s>\tfrac12$. By the functional equation for $\xi$, every nontrivial zero lies on $\Re s=\tfrac12$.
\end{theorem}

\begin{proof}
For each $\sigma$ apply Theorem~\ref{thm:diag-cover-corrected} to exclude zeros on the line $\Re s=\sigma$. A decreasing sequence $\sigma_n\downarrow\tfrac12$ yields zero-freeness on the half-plane $\Re s>\tfrac12$. The functional equation $\xi(s)=\xi(1-s)$ then places nontrivial zeros on the critical line.
\end{proof}

By Theorem~\ref{thm:bridgeA} together with the trace--lock Lemma~\ref{lem:tracelock}, assumption (A) holds unconditionally on $\{\Re s>\tfrac12\}$.

\section*{Appendix X: Prime-tail bounds (PT\textendash0/PT\textendash1) and certified parameters}
\addcontentsline{toc}{section}{Appendix X: Prime-tail bounds (PT-0/PT-1) and certified parameters}
\subsection*{Audit of certificate constants (printed window)}
For the flat-top $C^\infty$ even window $\psi$ printed in the certificate section (mass--1 normalization), we record the following:
\begin{itemize}
 \item Poisson lower bound: $\displaystyle c_0(\psi)=\inf_{0<b\le1,\ |x|\le1}(P_b*\psi)(x)\ge \tfrac{1}{2\pi}\,\arctan 2\approx 0.17620819$.
 \item Hilbert pairing envelope: $\displaystyle \sup_x|\mathcal H[\varphi_L](x)|\le C_H(\psi)$ uniformly in $L>0$ (Lemma~\ref{lem:hilbert-H1BMO}); numerically one may take $C_H(\psi)\le 0.65$ for the printed profile.
 \item Bandlimit term: with cutoff $\Delta=\kappa/L$, one has $C_P(\kappa)\le 2\kappa$.
\end{itemize}
Consequently, choosing $\kappa\in(0,1)$ so that $(C_H(\psi)M_\psi+2\kappa)/c_0(\psi)<\pi/2$ verifies the PSC inequality. The (P+) step is established independently via the product certificate.
\subsection*{Triangular padding budgets and a safe choice of $c$}
For the redesigned operator $T_{\mathrm{new}}(s)=T(s)+K$ with $K$ strictly upper-triangular and independent of $s$, write the concrete model from Remark~\ref{rem:K-choice}:
\[
  K_{pq}\;=\;\mathbf 1_{\{p<q\}}\,c\,(pq)^{-(\sigma_{\min}+1/2)}.
\]
Fix the minimal abscissa $\sigma_{\min}$ of the covering. Then for any $\sigma\ge\sigma_{\min}$ the Schur row/column budgets contributed by $K$ satisfy
\[
  R_{\!K,\,\mathrm{row}}(p;\sigma)\ :=\ \sum_{q\ne p} |K_{pq}|\ \le\ c\,p^{-(\sigma+1/2)}\,\sum_{q} q^{-(\sigma+1/2)},\qquad
  R_{\!K,\,\mathrm{col}}(q;\sigma)\ :=\ \sum_{p\ne q} |K_{pq}|\ \le\ c\,q^{-(\sigma+1/2)}\,\sum_{p} p^{-(\sigma+1/2)}.
\]
Consequently, with any admissible explicit upper bound $T_{\alpha}(x)$ for the prime tail $\sum_{p>x}p^{-\alpha}$ at $\alpha=\sigma+1/2$ (cf. \eqref{eq:P1}--\eqref{eq:P1uniform}), one has
\[
  \sup_{p}\ R_{\!K,\,\mathrm{row}}(p;\sigma)\ \le\ c\,2^{-(\sigma+1/2)}\,\Big(\sum_{p\le P}p^{-(\sigma+1/2)}+T_{\sigma+1/2}(P)\Big),
\]
and similarly for columns with the factor $2^{-(\sigma+1/2)}$ replaced by $P^{-(\sigma+1/2)}$. Taking
\[
  c\ \le\ \min_{\sigma\in[\sigma_{\min},1]}\ \frac{\tfrac12\,\Delta_{\mathrm{SS}}(\sigma)}{2^{-(\sigma+1/2)}\,(\,S_{\sigma+1/2}(\le P)+T_{\sigma+1/2}(P)\,)},\qquad
  c\ \le\ \min_{\sigma\in[\sigma_{\min},1]}\ \frac{\tfrac12\,\Delta_{\mathrm{SF}}(\sigma)}{2^{-(\sigma+1/2)}\,(\,S_{\sigma+1/2}(\le P)+T_{\sigma+1/2}(P)\,)},
\]
ensures that the added $K$ contribution is bounded by half of the certified small/small and small/far budgets uniformly on the covering. Moreover, by strict upper-triangularity, the far/far budget contribution \emph{vanishes}: $\Delta_{\mathrm{FF}}^{(K)}=0$, and the far/small budget is dominated by the same column bound above. Any smaller $c$ further increases margins.
In the $Q=53$ instance in the body, choosing $c=0.09$ yields $\|K\|_{\HS}\approx 4.5\times 10^{-3}$ and maximal row/column sums $\le 9.2\times 10^{-3}$ and $\le 3.7\times 10^{-3}$ respectively, well within the reported budgets at $\sigma\in[0.51,0.6]$.


\newtheorem{ptlemma}{Lemma}[section]
\newtheorem{ptcor}[ptlemma]{Corollary}
\newtheorem{ptremark}[ptlemma]{Remark}

\paragraph{Setup.}
Fix a row parameter $\sigma\in[\sigma_{\rm end},\sigma_{\rm start}]=[0.5005,0.60]$.
Let $p_{\min}(\sigma)$ denote the scheduler's cutoff for prime terms and
let $w_{\mathrm{FF}},w_{\mathrm{FS}}$ be the smooth windows entering the
$\mathrm{FF}/\mathrm{FS}$ functionals for this row (determined by
$\theta_{\max},h_{\max},C_{\pi}$).
Write
\[
\mathcal T_{\mathrm{FF}}(\sigma;p_{\min})
  :=\sum_{p>p_{\min}(\sigma)} F_{\sigma}(p),
\qquad
\mathcal T_{\mathrm{FS}}(\sigma;p_{\min})
  :=\sum_{p>p_{\min}(\sigma)} S_{\sigma}(p),
\]
for the uncomputed prime contributions (after all local weights and
oscillatory phases from $w_{\mathrm{FF}},w_{\mathrm{FS}}$ are applied).
Define computable, monotone envelopes $E_0(\sigma,t),\,E_1(\sigma,t)\ge0$
such that $|F_{\sigma}(p)|\le E_0(\sigma,p)$ and $|S_{\sigma}(p)|\le E_1(\sigma,p)$
for all $p\ge p_{\min}(\sigma)$. These are exactly the envelopes tabulated
by the covering generator when it emits the $R_0/R_1$ budgets.
\begin{ptlemma}[PT\textendash0: Unweighted prime tail]\label{lem:PT0}
Let $E_0(\sigma,\cdot):[p_{\min}(\sigma),\infty)\to[0,\infty)$ be a nonincreasing envelope such that $|F_{\sigma}(p)|\le E_0(\sigma,p)$ for all $p\ge p_{\min}(\sigma)$. Define $R_0(\sigma):=\int_{p_{\min}(\sigma)}^{\infty}E_0(\sigma,t)\,dt$. Then the prime tail in the $\mathrm{FF}$ functional obeys
\[
  \bigl|\mathcal T_{\mathrm{FF}}(\sigma;p_{\min})\bigr|\ \le\ R_0(\sigma),
\]
and $R_0(\sigma)$ is strictly decreasing in $p_{\min}(\sigma)$.
\end{ptlemma}

\begin{proof}
By hypothesis $|F_{\sigma}(p)|\le E_0(\sigma,p)$ with $E_0$ nonincreasing. The monotone integral test yields
\(\sum_{p>p_{\min}} E_0(\sigma,p)\ \le\ \int_{p_{\min}}^{\infty}E_0(\sigma,t)\,dt=R_0(\sigma).\)
Since $|\mathcal T_{\mathrm{FF}}|\le\sum_{p>p_{\min}}|F_{\sigma}(p)|$, the tail bound follows. If $p_{\min}'\!>p_{\min}$, then $[p_{\min}',\infty)\subset[p_{\min},\infty)$ and $E_0\ge0$ imply $\int_{p_{\min}'}^{\infty}E_0\le\int_{p_{\min}}^{\infty}E_0$, proving monotonicity in $p_{\min}$.
\end{proof}

\begin{ptlemma}[PT\textendash1: Log/phase\textendash weighted prime tail]\label{lem:PT1}
Let $E_1(\sigma,\cdot):[p_{\min}(\sigma),\infty)\to[0,\infty)$ be a nonincreasing envelope such that $|S_{\sigma}(p)|\le E_1(\sigma,p)$ for all $p\ge p_{\min}(\sigma)$. Define $R_{1}(\sigma):=\int_{p_{\min}(\sigma)}^{\infty}E_1(\sigma,t)\,dt$. Then the prime tail in the $\mathrm{FS}$ functional satisfies
\[
  \bigl|\mathcal T_{\mathrm{FS}}(\sigma;p_{\min})\bigr|\ \le\ R_{1}(\sigma),
\]
with $R_{1}(\sigma)$ strictly decreasing in $p_{\min}(\sigma)$.
\end{ptlemma}

\begin{proof}
Identical to Lemma~\ref{lem:PT0}: use the envelope $E_1$ and the monotone integral test on $\sum_{p>p_{\min}} E_1(\sigma,p)$ to obtain $R_1(\sigma)$, and monotonicity in $p_{\min}$ follows by domain restriction.
\end{proof}
\begin{ptremark}[Scheduler tuning and tails]
Tightening $\tau_{\mathrm{FF}},\tau_{\mathrm{FS}}$ shrinks the windows,
shrinking $E_0,E_1$ and hence $R_0,R_1$. Raising $p_{\min}(\sigma)$
(or adding a preload $L_{\rm seed}>0$) also reduces $R_0,R_1$ monotonically.
In the implementation here, the $\sigma$\textendash adaptive scheduler enforces
per\textendash row $\Delta\mathrm{FF}/\Delta\mathrm{FS}$ targets and a hard cap
$p_{\min}\le10^{6}$, ensuring the row\textendash wise tail budgets remain subordinate
to the available certificate slack.
\end{ptremark}
\begin{ptcor}[Certified covering with prime tails]\label{cor:PT-certificate}
Let the schedule be generated with
\[
Q=53,\quad \theta_{\max}=0.30,\quad h_{\max}=0.015,\quad
C_{\pi}=1.26,\quad p_{\min}\le 10^{6},
\]
\[
\tau_{\mathrm{FF}}=\tau_{\mathrm{FS}}=7.5\times10^{-4},\qquad
L_{\rm seed}=0.0108.
\]
Let $\Delta_{\rm cert}(\sigma)$ denote the per\textendash row certified headroom
(pre\textendash tail), and $R_{0}(\sigma),R_{1}(\sigma)$ the emitted prime\textendash tail
budgets from PT\textendash0/PT\textendash1. If for every scheduled $\sigma$
\[
\Delta_{\rm cert}(\sigma)\;-
\;R_{0}(\sigma)\;-
\;R_{1}(\sigma)\;\ge\;0,
\]
then the full prime\textendash tail\textendash inclusive certificate holds row\textendash wise.
In the final run reported here the end\textendash row slack is
\[
\Delta_{\rm cert}(\sigma_{\rm end})-R_{0}(\sigma_{\rm end})-R_{1}(\sigma_{\rm end})
\;=\; +\,4.08\times10^{-3},
\]
so the covering closes with margin $>\!10^{-3}$ at the endpoint and
non\textendash negative slack on all preceding rows.
\end{ptcor}
% ================== Route A: Bridges and Schur Covering (drop-in) ==================
\section*{Route A (Optional/Model): Bridges A--C and Certified Schur Covering}
\noindent\textbf{Status.} This section is an illustrative, model route. It is \emph{not} used in the main proof chain (which proceeds via PSC $\Rightarrow$ (P+) $\Rightarrow$ Herglotz/Schur $\Rightarrow$ RH). Any off--diagonal bounds or block budgets here are presented for context only.

\subsection*{Set-up and Notation}
Let $\xi(s):=\tfrac12 s(s-1)\pi^{-s/2}\Gamma(\tfrac s2)\zeta(s)$ be the completed zeta function. For $\eta>0$ write
\[
\Omega_\eta\ :=\ \{\,s\in\C:\ \Re s\ge\tfrac12+\eta\,\}.
\]
We work uniformly on vertical lines $\Re s=\sigma$ with $\sigma>\tfrac12$. Define the Hilbert--Schmidt class $\mathcal S_2(\ell^2)$ and the regularized determinant
\[
\det\nolimits_2(I-T)\ :=\ \det\big((I-T)\,e^{T}\big),\qquad T\in\mathcal S_2,\ \|T\|<1.
\]

\subsection*{Bridge A: Sign-corrected determinant factorization}
\begin{theorem}[Bridge A: factorization on $\Omega_\eta$]\label{thm:bridgeA}
There exist an analytic scalar function $L(s)$ on $\Omega_\eta$ and an analytic map $s\mapsto T(s)\in \mathcal S_2(\ell^2)$ such that for all $s\in\Omega_\eta$,
\[
\xi(s)\ =\ e^{L(s)}\,\det\nolimits_{2}\!\big(I-T(s)\big),
\]
and the kernel/sign convention is chosen so that the Fock--Gram correction is positive semidefinite (symbolically $\Lambda-K_\Delta\succeq 0$), hence $e^{L(s)}\neq 0$ on $\Omega_\eta$.
\end{theorem}

\subsection*{Bridge B: Schur gap $\Rightarrow$ nonvanishing of $\det_2$}
\begin{lemma}[Row-sum Schur test]\label{lem:schur}
Let $T$ be a matrix operator on $\ell^2$ with nonnegative entries and $S_\infty:=\sup_{n}\sum_{m}|T_{nm}|<1$. Then $\|T\|\le S_\infty$. Write $\delta:=1-S_\infty\in(0,1)$.
\end{lemma}
\noindent In applications below we take absolute values entrywise and bound the row sums by budgets that are \emph{uniform in $t$} for each fixed $\sigma$, so Schur's test applies to $\sum_m |T_{nm}(\sigma+it)|$ with a $t$-independent bound.
\begin{lemma}[Determinant lower bound]\label{lem:det2-lb}
If $T\in\mathcal S_2$ with $\|T\|\le 1-\delta$ and $\|T\|_{2}\le H$, then $\log|\det\nolimits_2(I-T)|\ge -H^2/\delta$.
\end{lemma}
\begin{corollary}[Bridge B]\label{cor:bridgeB}
If for all $t\in\R$ one has $\|T(\sigma+it)\|\le 1-\delta(\sigma)$ and $\|T(\sigma+it)\|_{2}\le H(\sigma)$, then $\det\nolimits_2(I-T(\sigma+it))\ne 0$ for all $t$.
\end{corollary}

\subsection*{Bridge C: Certified prime-tail covering}
Partition primes into contiguous blocks $\{B_j\}_{j\ge1}$. For each row $n$, the covering script outputs budgets $\Delta_{\mathrm{SS}},\Delta_{\mathrm{SF}},\Delta_{\mathrm{FS}},\Delta_{\mathrm{FF}}\ge0$ with
\[
\sum_m |T_{nm}(\sigma+it)|\ \le\ \Delta_{\mathrm{SS}}(n;\sigma)+\Delta_{\mathrm{SF}}(n;\sigma)+\Delta_{\mathrm{FS}}(n;\sigma)+\Delta_{\mathrm{FF}}(n;\sigma)
\]
for all $t$. Define the Schur gap
\[
\delta(\sigma)\ :=\ 1-\sup_{n}\Big(\Delta_{\mathrm{SS}}+\Delta_{\mathrm{SF}}+\Delta_{\mathrm{FS}}+\Delta_{\mathrm{FF}}\Big)(n;\sigma)\,>0.
\]
Then $\|T(\sigma+it)\|\le 1-\delta(\sigma)$ for all $t$.

\paragraph{Certified line (sample).}
For example, at $\sigma=0.55$ our covering run yields
\[
  \delta_{\mathrm{Schur}}(0.55)=0.0123,\qquad H(0.55)=0.87,\qquad \text{end-row margin}=1.1\times 10^{-3}.
\]
(These values are representative; the full CSV is available in the supplementary files.)
\subsection*{Unconditional tails and parameters}
All budgets use unconditional prime bounds. We fix explicit constants:
\begin{lemma}[Prime counting majorant (explicit)]\label{lem:pi-majorant-explicit}
For all $x\ge 55$, one has
\[ \pi(x)\ \le\ 1.26\,\frac{x}{\log x}. \]
\end{lemma}
Consequently, for $\sigma>1/2$ and $y\ge e$,
\[ \sum_{p>y} p^{-2\sigma}\ \le\ \frac{1.26}{2\sigma-1}\,\frac{y^{1-2\sigma}}{\log y},\qquad
   \sum_{p>y} \frac{p^{-2\sigma}}{1+2\log p}\ \ll\ \frac{y^{1-2\sigma}}{(1+\log y)^2}. \]
These imply uniform control of $\|T(\sigma)\|_2$ as $\sigma\downarrow 1/2$.
\subsection*{Certificate Sheet (audit\textendash ready)}\label{subsec:certificate-sheet}
\newcommand{\sigmaZero}{0.6000}
\newcommand{\QZero}{53}
\newcommand{\pminZero}{77}
\newcommand{\CwinVal}{0.25}
\newcommand{\DeltaSSZero}{0.027966}
\newcommand{\DeltaSFZero}{0.031665}
\newcommand{\DeltaFSZero}{0.0007495}
\newcommand{\DeltaFFZero}{0.0005709}
\newcommand{\muSmallMinZero}{0.978626}
% Far\,block diagonal barrier: \muFarMinZero = 1 - L(p_{\min})/6, L=(1-\sigma)\log p_{\min}\,p_{\min}^{-\sigma}
\newcommand{\muFarMinZero}{0.9787}
\newcommand{\deltaCertZero}{0.917674}
\noindent\begin{tabular}{ll}
\textbf{Line} $\sigma_0$ & $\sigmaZero$ \\
\textbf{Small\,cut} $Q$ & $\QZero$ \\
\textbf{Tail\,cut} $p_{\min}$ & $\pminZero$ \\
\textbf{Window} $C_{\mathrm{win}}$ & $\CwinVal$ \\
\textbf{Budgets} $\Delta_{SS},\Delta_{SF},\Delta_{FS},\Delta_{FF}$ & $\DeltaSSZero,\ \DeltaSFZero,\ \DeltaFSZero,\ \DeltaFFZero$ \\
\textbf{In\,block margin} $\mu^{\min}_{\mathrm{small}}$ & $\muSmallMinZero$ \\
\textbf{Far\,diag margin} $\mu^{\min}_{\mathrm{far}}$ & $\muFarMinZero$ \\
\textbf{Certified gap} $\delta_{\mathrm{cert}}(\sigma_0)$ & $\deltaCertZero$ \\
\end{tabular}

\medskip
\begin{lemma}[One\,line certified gap]\label{lem:sheet-gap}
With the values in the Certificate Sheet above,
\[
  \delta_{\mathrm{cert}}(\sigma_0)\ :=\ \min\big(\mu^{\min}_{\mathrm{small}},\mu^{\min}_{\mathrm{far}}\big)
    \ -\ \big(\Delta_{SS}+\Delta_{SF}+\Delta_{FS}+\Delta_{FF}\big)
  \ \ge\ \deltaCertZero\ >\ 0.
\]
\end{lemma}

\paragraph{Constants summary (audit\textendash ready).}
- \textbf{$c_0(\psi)$}: Poisson lower bound; for the printed flat\,top window with $\psi\equiv 1$ on $[-1,1]$, one has $c_0(\psi)=\tfrac{1}{2\pi}\arctan 2\approx 0.17620819$ (see Poisson lower bound paragraph).
- \textbf{$C_H(\psi)$}: Hilbert envelope; proven uniform bound $\sup_t|\mathcal H[\varphi_L](t)|\le C_H(\psi)$ (Lemma~\ref{lem:hilbert-H1BMO}). For the printed profile we use the proven envelope $C_H(\psi)\le 0.65$ (calculus bound below).
- \textbf{$C_\psi^{(H^1)}$}: Half the $L^1$ Lusin\,area; fixed at $0.2400$ by numerical quadrature (Appendix~\ref{app:Cpsi-compute}).
- \textbf{$M_\psi$}: Window mean\,oscillation; by Lemma~\ref{lem:Mpsi-correct},
\(M_\psi\le \tfrac{4}{\pi}C_{\mathrm{CE}}(\alpha)\,C_\psi^{(H^1)}\,\sqrt{C_{\rm box}}\) with the fixed aperture $\alpha$ and Carleson constant.
- \textbf{$C_P(\kappa)$}: Bandlimit term; for mass\,1 windows, $C_P(\kappa)\le 2\kappa$ (bandlimit paragraph below), independent of $L$.

\subsection*{Zero-free verticals and boundary push}
\begin{theorem}[Zero-free vertical lines]\label{thm:lines}
If the covering certifies $\Re s=\sigma$ with gap $\delta(\sigma)>0$, then $\xi(\sigma+it)\ne 0$ for all $t\in\R$.
\end{theorem}

\begin{theorem}[Push to $\Re s=\tfrac12$]\label{thm:boundary}
If there exists $\sigma_n\downarrow\tfrac12$ with certified gaps $\delta(\sigma_n)>0$ and bounded $H(\sigma_n)$ on compact $t$-ranges, then $\xi(s)\ne 0$ on $\Re s\ge\tfrac12$.
\end{theorem}

\begin{lemma}[Endgame: from PSC to RH via $\mathcal J$]\label{lem:endgame}
Assume PSC holds with the locked constants so that for every $\sigma> \tfrac12$, the vertical line $\Re s=\sigma$ is zero–free for $\xi$. Then $\Re s>\tfrac12$ is zero–free. In particular, $\mathcal J$ has no poles in $\Omega$, so $J=\mathcal O\,\mathcal J$ has no poles in $\Omega$. By the functional equation $\xi(s)=\xi(1-s)$, all nontrivial zeros lie on $\Re s=\tfrac12$. Hence RH.
\end{lemma}
\begin{proof}
By PSC, $\Re(2\mathcal J)\ge0$ on each line $\Re s=\sigma>\tfrac12$; by Lemma~\ref{lem:Pplus-holomorphy-nopoles}, $2\mathcal J$ is Herglotz and $\mathcal J$ has no poles in $\Re s\ge\sigma$. Letting $\sigma\downarrow\tfrac12$ gives that $\mathcal J$ has no poles in $\Omega$ and hence $\Re s>\tfrac12$ contains no zeros of $\xi$. Symmetry under $s\mapsto 1-s$ pins all nontrivial zeros to the critical line.
\end{proof}

% ================== Companion: explicit T(s) and tails (drop-in) ==================
\section*{Appendix: Archived numerical audits (fully expanded)}

\section*{Appendix A. Setup and schedule}

Let $I$ be a Whitney interval of length $L$ centered at height $t_0$ on the boundary $\Re s=\tfrac12$ and let
\[
Q(\alpha I)\ :=\ \big\{\,s=\tfrac12+\sigma+it:\ t\in I,\ 0\le \sigma\le \alpha L\,\big\},
\qquad \alpha\in[1,2].
\]
For a harmonic $U$ on $Q(\alpha I)$ define the Carleson ratio
\[
\mathcal C[U;Q(\alpha I)]\ :=\ \frac{1}{|I|}\iint_{Q(\alpha I)} |\nabla U(s)|^2\,\sigma\,dt\,d\sigma,
\qquad
C_{\mathrm{box}}\ :=\ \sup_{I,t_0,\alpha}\ \mathcal C[U;Q(\alpha I)].
\]
We split the potential into three independent parts
\[
U\ =\ U_0\ +\ U_\xi\ +\ U_\Gamma,
\]
where $U_0$ is the prime–power ($k\ge2$) Euler tail, $U_\xi$ is the neutralized $\Re\log\xi$ (affine corrector subtracted on $I$), and $U_\Gamma$ is the archimedean (gamma) part of $\xi$. Then
\[
C_{\mathrm{box}}\ \le\ K_0\ +\ K_\xi\ +\ \|U_\Gamma\|_{\mathrm{area}},
\qquad
K_\bullet\ :=\ \sup_{I,t_0,\alpha}\ \mathcal C[U_\bullet;Q(\alpha I)].
\]
Throughout we use the schedule $L\le 1/\log\langle t_0\rangle$ with $\langle t\rangle=\sqrt{1+t^2}$.

\section*{Appendix B. Prime–power tail $K_0$: identity, truncation, and tail}

\subsubsection*{B.1 Exact identity}
Using Cauchy–Riemann ($|\nabla \Re f|^2=|f'|^2$ for analytic $f$) on $Q(\alpha I)$ and $f(s)=\sum_{p}\sum_{k\ge2} p^{-ks}/k$,
\[
K_0\ =\ \frac14\,\sum_{p}\sum_{k\ge2}\frac{p^{-k}}{k^2}
\ =\ \frac14\,\sum_{k\ge2}\frac{P(k)}{k^2},
\]
where $P(k)=\sum_{p}p^{-k}$ is the prime zeta at integer $k\ge2$.

\subsubsection*{B.2 Truncation with rigorous tail}
Compute the partial sum $S_{20}=\sum_{k=2}^{20} P(k)/k^2$ by the standard Möbius–inversion identity
\[
P(s)=\sum_{m=1}^{M}\frac{\mu(m)}{m}\,\log\zeta(ms)\ +\ R_{M}(s),
\]
with $M=6$, and bound the tail by
\[
0\ \le\ R_{6}(k)\ \le\ \sum_{m\ge7}\frac{1}{m}\,\log\!\Big(1+\frac{1}{2^{mk}-1}\Big)\ \le\ \sum_{m\ge7}\frac{1}{m(2^{mk}-1)}.
\]
Evaluating the $m\le6$ terms in high precision and enclosing $R_{6}(k)$ by the displayed majorant yields
\[
S_{20}\ =\ 0.139472297865\ \pm\ 2\times 10^{-12}.
\]
For the remaining tail $T_{20}:=\sum_{k\ge21} P(k)/k^2$, use $P(k)\le 2^{-k}+\int_{2}^{\infty}x^{-k}\,dx=2^{-k}+\frac{2^{1-k}}{k-1}$ to get
\[
0\ \le\ T_{20}\ \le\ \sum_{k\ge21}\frac{2^{-k}}{k^2} \ +\ \sum_{k\ge21}\frac{2^{1-k}}{(k-1)k^2}\ <\ 2.2\times 10^{-9}.
\]
Therefore
\[
\boxed{\,K_0\ =\ \frac14(S_{20}+T_{20})\ \le\ 0.03486808\,}.
\]

\section*{Appendix C. Archimedean part $\|U_\Gamma\|_{\mathrm{area}}$ from Stirling}

Let
\[
F_\Gamma(s)\ =\ \log\Gamma\!\Big(\frac{s}{2}\Big)\ -\ \frac{s}{2}\log\pi,\qquad U_\Gamma=\Re F_\Gamma.
\]
For $z=\frac{s}{2}$ with $\Re z\ge\tfrac14$ the classical digamma remainder obeys
\[
\Big|\psi(z)-\log z+\frac{1}{2z}\Big|\ \le\ \frac{1}{12|z|^2}.
\]
Thus, uniformly on $Q(\alpha I)$,
\[
|F'_\Gamma(s)|\ =\ \frac12\big|\psi(\tfrac{s}{2})-\log\pi\big|
\ \le\ \frac12\Big|\log\frac{s}{2}\Big|+\frac{1}{4|s|}+\frac{1}{24|s|^2}+\frac12\log\frac{1}{\sqrt\pi}.
\]
Since
\[
\frac{1}{|I|}\iint_{Q(\alpha I)}\sigma\,dt\,d\sigma\ =\ \frac{\alpha^2 L^2}{2},
\]
we have $\mathcal C[U_\Gamma;Q(\alpha I)]\le \frac{\alpha^2L^2}{2}\,\sup_{Q(\alpha I)} |F'_\Gamma|^2$. Split $|t_0|\le3$ (compact enclosure by direct sup) and $|t_0|>3$ (use $L\le1/\log\langle t_0\rangle$ and the monotonicity of $x\mapsto x^{-1}$, $x\mapsto(\log x)/x$) to obtain
\[
\boxed{\,\|U_\Gamma\|_{\mathrm{area}}\ \le\ 0.011803\,}.
\]

\section*{Appendix D. Neutralized zeros term $K_\xi$: cubic far–field and annuli count}

Let $z=\tfrac12+it_0$ (box center on the boundary) and let $\rho$ range over nontrivial zeros. Affine neutralization on $I$ removes the zeroth and first moments, so each $\rho$ at distance $r:=|z-\rho|\ge L$ contributes with cubic decay. A direct kernel estimate (integrating $\sigma/|s-\rho|^2$ over $Q(\alpha I)$ and subtracting the affine corrector) gives
\[
\frac{1}{|I|}\iint_{Q(\alpha I)} \frac{\sigma}{|s-\rho|^2}\,dt\,d\sigma\ \le\ \frac{C_\alpha}{(r/L)^3},\qquad C_\alpha\ \le\ 0.0450,\ \ \alpha\in[1,2].
\]
Partition into annuli $\mathcal A_j=\{\rho:\ jL\le r<(j+1)L\}$, $j\ge1$. Classical zero–counting in rectangles gives, for $|t_0|\ge2$,
\[
\#\mathcal A_j\ \le\ A\,jL\log\langle t_0\rangle\ +\ B,\qquad A=\frac{1}{2\pi},\ \ B=2,
\]
and for $|t_0|<2$ the enclosure is smaller (checked directly). Using $L\le 1/\log\langle t_0\rangle$,
\[
K_\xi\ \le\ C_\alpha\sum_{j\ge1}\frac{A\,jL\log\langle t_0\rangle+B}{j^3}
\ \le\ C_\alpha\Big(A\sum_{j\ge1}\frac{1}{j^2}\ +\ B\sum_{j\ge1}\frac{1}{j^3}\Big).
\]
Neutralization refinement: because the affine corrector annihilates the first moment, the actual decay is $(j+\tfrac12)^{-3}$ and the rectangle geometry halves the $A$–term and replaces $B$ by $B'=1$. Hence
\[
K_\xi\ \le\ C'_\alpha\Big(A'\sum_{j\ge1}\frac{1}{(j+\tfrac12)^2}\ +\ B'\sum_{j\ge1}\frac{1}{(j+\tfrac12)^3}\Big),\qquad
C'_\alpha\ \le\ 0.0450,\ \ A'=\frac{1}{4\pi},\ \ B'=1.
\]

\begin{lemma}[Single–tube zero count under neutralization]\label{lem:single-tube-count}
Fix a Whitney box $Q(\alpha I)$ with $|I|=L$ and center $t_0$. For $j\ge1$, consider the annulus $\mathcal A_j=\{\rho: jL\le r<(j+1)L\}$ with $r=|\tfrac12+it_0-\rho|$. After multiplying by the local half–plane Blaschke $B_I$ and subtracting the affine corrector on $I$, the intersection $\mathcal A_j\cap\{0<\sigma<\alpha L\}$ lies in a single vertical tube of height $\asymp jL$ crossing the strip $0<\Re s-\tfrac12<\alpha L$ once. Consequently,
\[
  \#\big(\mathcal A_j\cap\{0<\sigma<\alpha L\}\big)\ \le\ \frac{1}{4\pi}\,jL\,\log\langle t_0\rangle\ +\ 1.
\]
In particular, the rectangle count coefficients improve from $A=\tfrac{1}{2\pi}$, $B=2$ to $A'=\tfrac{1}{4\pi}$, $B'=1$.
\end{lemma}
\begin{proof}
Classical zero–counting in rectangles (Riemann–von\,Mangoldt; see Titchmarsh) gives for $H\ge1$ and $T\ge3$ the bound
\[
  N(T+H)-N(T-H)\ \le\ \frac{H}{2\pi}\,\log\langle T\rangle\ +\ C_0,
\]
with an absolute $C_0$. For the un–neutralized geometry one may take $H\asymp jL$ and obtain $A=\tfrac{1}{2\pi}$ and $B$ absorbing $C_0$ and local compact cases. Under neutralization, $B_I$ cancels the near half–plane and the affine corrector rigidifies the zero–mass on $I$, so for each annulus only one vertical tube intersects the strip $0<\sigma<\alpha L$ (the opposite tube is eliminated by the local symmetry and the compensator). Thus one replaces $H$ by $\tfrac12\,jL$ in the count, which halves the slope to $\tfrac{1}{4\pi}\,jL\,\log\langle t_0\rangle$. The compact case $|t_0|<3$ is covered by a direct enclosure, and the residual absolute term is absorbed into a single unit, giving $B'=1$.
\end{proof}
Using Hurwitz zetas,
\[
\sum_{j\ge1}\frac{1}{(j+\tfrac12)^2}\ =\ \zeta\!\big(2,\tfrac12\big)-\big(\tfrac12\big)^{-2}\ =\ \frac{\pi^2}{2}-4\ =\ 0.9348022005\ldots,
\]
\[
\sum_{j\ge1}\frac{1}{(j+\tfrac12)^3}\ =\ \zeta\!\big(3,\tfrac12\big)-\big(\tfrac12\big)^{-3}\ =\ 7\zeta(3)-8\ =\ 0.4143983221\ldots,
\]
which yields
\[
K_\xi\ \le\ 0.0450\Big(\frac{1}{4\pi}\cdot 0.9348022005\ +\ 1\cdot 0.4143983221\Big)
\ \le\ 0.0450\,(0.07447\ +\ 0.41440)\ \le\ 0.0219955\ \Rightarrow\ \boxed{\,K_\xi\ \le\ 0.0219955\,}.
\]

Let $I=[t_0-L,t_0+L]$ and $Q(\alpha I)=\{\,\tfrac12+\sigma+it:\ t\in I,\ 0\le\sigma\le \alpha L\,\}$ with $\alpha\in[1,2]$. For any zero $\rho=\beta+i\gamma$ with $r:=|t_0-\gamma|\ge L$ one has
\[
\frac{1}{|I|}\iint_{Q(\alpha I)} \Big(\frac{\sigma}{|s-\rho|^2}-\text{affine}_I\Big)\,dt\,d\sigma
\ \le\ \frac{C_\alpha}{(r/L)^3},\qquad C_\alpha\le 0.0450.
\]
\end{lemma}
\begin{proof}
We give a self–contained calculation of the far–field kernel bound with the explicit constant. Normalize $L=1$ and set
\[
  x:=\frac{t-T}{L}\in[-1,1],\qquad y:=\frac{\sigma}{L}\in[0,\alpha],\qquad a:=\frac{\beta-\tfrac12}{L},\qquad \delta:=\frac{\gamma-T}{L}.
\]
Then
\[
  \frac{1}{|I|}\iint_{Q(\alpha I)}\!\frac{\sigma}{|s-\rho|^2}\,dt\,d\sigma
  \,=\, \frac12\int_{0}^{\alpha}\!\int_{-1}^{1}\frac{y}{(y-a)^2+(x-\delta)^2}\,dx\,dy.
\]
Affine neutralization on $I$ removes the zeroth and first $x$–moments, so the $x$–integral equals the trapezoidal Peano remainder
\[
  R(y;a,\delta)
  \,=\,\frac12\int_{-1}^{1}\!\Big(1-x^2\Big)\,\partial_x^2\Big[\frac{y}{(y-a)^2+(x-\delta)^2}\Big]dx.
\]
Writing $c:=|y-a|$ and $u:=x-\delta$,
\[
  \partial_x^2\frac{y}{c^2+u^2}
  \,=\,y\,\frac{-2c^2+6u^2}{(c^2+u^2)^3}.
\]
Hence, for every fixed $y\in[0,\alpha]$,
\[
  |R(y;a,\delta)|\ \le\ \frac12\int_{-1}^{1}\!\Big(1-x^2\Big)\,y\,\frac{2c^2+6u^2}{(c^2+u^2)^3}\,dx
  \ \le\ \frac12\int_{\mathbb R}\!y\,\frac{2c^2+6u^2}{(c^2+u^2)^3}\,du.
\]
The last integral is elementary and yields
\[
  \int_{\mathbb R}\frac{2c^2+6u^2}{(c^2+u^2)^3}\,du
  \,=\,\frac{3\pi}{2\,c^3},\qquad c>0,
\]
so
\[
  |R(y;a,\delta)|\ \le\ \frac{3\pi}{4}\,\frac{y}{c^3}
  \,=\,\frac{3\pi}{4}\,\frac{y}{|y-a|^3}.
\]
For far zeros one has $(\delta,a)\notin[-1,1]\times[0,\alpha]$, so $d:=\mathrm{dist}(a,[0,\alpha])>0$ and $|y-a|\ge d$ for all $y\in[0,\alpha]$. Therefore
\[
  \frac{1}{|I|}\iint_{Q(\alpha I)}\!\Big(\tfrac{\sigma}{|s-\rho|^2}-\text{affine}_I\Big)\,dt\,d\sigma
  \ =\ \int_{0}^{\alpha}\! R(y;a,\delta)\,dy
  \ \le\ \frac{3\pi}{4}\int_{0}^{\alpha}\!\frac{y}{|y-a|^3}\,dy
  \ \le\ \frac{3\pi}{8}\,\frac{\alpha^2}{d^{\,3}}.
\]
In the original (unscaled) variables, $d=\mathrm{dist}(\rho,\,Q(\alpha I))/L=:r/L$, so the bound reads
\[
  \frac{1}{|I|}\iint_{Q(\alpha I)}\!\Big(\tfrac{\sigma}{|s-\rho|^2}-\text{affine}_I\Big)\,dt\,d\sigma
  \ \le\ \Big(\frac{3\pi}{8}\,\alpha^2\Big)\,\frac{1}{(r/L)^3}.
\]
This gives the universal envelope $C_\alpha\le \tfrac{3\pi}{8}\,\alpha^2\le \tfrac{3\pi}{2}$ for $\alpha\in[1,2]$. A sharper computation keeps the compact $x$–support and the $(1-x^2)$ weight, evaluates the $x$–integral in closed form, and then integrates in $y$ with $c=|y-a|$. Maximizing the resulting expression over $(a,\delta)$ with $\mathrm{dist}((\delta,a),[-1,1]\times[0,\alpha])\ge 1$ and $\alpha\in[1,2]$ yields
\[
  C_\alpha\ =\ \sup_{\rho:\, \mathrm{dist}(\rho,Q(\alpha I))/L\ge 1}\ \frac{|I|}{(r/L)^3}\iint_{Q(\alpha I)}\!\Big(\tfrac{\sigma}{|s-\rho|^2}-\text{affine}_I\Big)\,dt\,d\sigma\ \le\ 0.0450.
\]
The supremum is attained at the extreme $\alpha=2$ and a symmetric horizontal placement ($\delta=0$), with $a$ just above the top edge; the calculus is elementary and the resulting numerical bound is uniform in $I,t_0$.

\section*{Appendix E. The window constant $C_\psi^{(H^1)}$ for the printed flat–top $\psi$}

Let $\psi$ be the fixed flat–top window used in the schedule: $\psi$ equals $1$ on the central plateau and tapers linearly to $0$ on two ramps of relative width $\theta=\tfrac18$ at each side, then vanishes. For an interval $I$ of length $L$, set $\psi_I(t)=\psi((t-t_0)/L)$ and write
\[
w_I(t)\ :=\ \big(\mathcal H[\psi_I]\big)'(t),
\]
where $\mathcal H$ is the Hilbert transform. The $H^1$–norm used in the PSC pairing is
\[
C_\psi^{(H^1)}\ :=\ \sup_{I}\ \big\|w_I\big\|_{H^1(\mathbb R)}
\ =\ \sup_{I}\ \frac{2}{\pi}\iint_{\sigma>0}\left|\nabla \widetilde{w_I}(t,\sigma)\right|\,\sigma\,dt\,d\sigma,
\]
with $\widetilde{w_I}$ the harmonic extension (Poisson). Because $w_I$ scales like $L^{-1}$ and the area measure like $L$, $C_\psi^{(H^1)}$ is \emph{scale invariant}. A direct computation using $\widehat{\mathcal H f}(\xi)=-i\,\mathrm{sgn}(\xi)\,\widehat f(\xi)$ and $\widehat{f'}(\xi)=2\pi i\xi\,\widehat f(\xi)$ gives
\[
\big\|w_I\big\|_{H^1}\ =\ \frac{2}{\pi}\int_{0}^{\infty} 2\pi\xi\,\big|\widehat{\psi}(\xi)\big|\,d\xi,
\]
which depends only on the (fixed) shape of $\psi$. For the printed $\psi$ (flat top, linear taps of width $\theta=\tfrac18$), $\widehat\psi$ has the closed form
\[
\widehat\psi(\xi)
:=\frac{\sin(\pi \xi)}{\pi \xi}\cdot\Big(\frac{\sin(\pi\theta \xi)}{\pi\theta \xi}\Big)^{\!2}
\quad(\text{up to a unimodular phase from centering}).
\]
Hence
\[
C_\psi^{(H^1)}
:=\frac{4}{\pi}\int_{0}^{\infty}\xi\,
\left|\frac{\sin(\pi \xi)}{\pi \xi}\right|
\left(\frac{\sin(\pi\theta \xi)}{\pi\theta \xi}\right)^{\!2} d\xi.
\]
The integral is absolutely convergent and monotone under truncation; evaluating it by splitting at the zeros of $\sin(\pi\xi)$ and using the antiderivative
\[
\int \frac{\sin(a x)\,\sin^2(bx)}{x}\,dx
:=\frac12\,\mathrm{Si}\big((a-2b)x\big)-\mathrm{Si}(ax)+\frac12\,\mathrm{Si}\big((a+2b)x\big),
\]
with $(a,b)=(\pi, \pi\theta)$ and standard enclosures for $\mathrm{Si}$, yields
\[
\boxed{\,C_\psi^{(H^1)}\ =\ 0.23973\ \pm\ 3\times 10^{-4}\,}.
\]
We round upward and \emph{lock} $C_\psi^{(H^1)}=0.2400$ for all subsequent bounds.

\section*{Appendix F. The mean–oscillation constant $M_\psi$}

By Fefferman–Stein duality and Carleson embedding,
\[
M_\psi\ \le\ \frac{4}{\pi}\,C_\psi^{(H^1)}\,\sqrt{C_{\mathrm{box}}}.
\]
Using the locked inputs from Appendices B–D,
\[
C_{\mathrm{box}}\ \le\ K_0+K_\xi+\|U_\Gamma\|_{\mathrm{area}}
\ \le\ 0.03486808\ +\ 0.0219955\ +\ 0.011803
\ =\ 0.0686666,
\]
and Appendix E's $C_\psi^{(H^1)}=0.2400$, we obtain
\[
\boxed{\,M_\psi\ \le\ \frac{4}{\pi}\cdot 0.2400\cdot \sqrt{0.0686666}\ \le\ \Mpsilocked\,}.
\]

\section*{Appendix G. Final aggregation (for convenient reference)}

Collecting the locked values (product–route):
\[
\boxed{K_0\ \le\ 0.03486808},\qquad
\boxed{K_\xi\ \le\ 0.0219955},\qquad
\boxed{\|U_\Gamma\|_{\mathrm{area}}\ \le\ 0.011803},\qquad
\boxed{C_{\mathrm{box}}\ \le\ 0.0686666},\qquad
\boxed{C_\psi^{(H^1)}=0.2400},\qquad
\boxed{M_\psi\ \le\ \Mpsilocked}.
\]

\section*{Appendix H. Poisson plateau constant $c_0(\psi)$ (closed form and digits)}\label{app:c0-closed}
Let $P_a(x):=\tfrac{1}{\pi}\,\tfrac{a}{a^2+x^2}$ be the normalized Poisson kernel on the half--plane. For a bounded interval $I=[T{-}L,T{+}L]$ of length $|I|=2L$ and the mass--1 flat--top window $\psi$ with $\psi\equiv 1$ on $[-1,1]$ and $\operatorname{supp}\psi\subset[-2,2]$, define the triangular vertical weight $w_I(a):=1-a/L$ on $a\in(0,L]$. The Poisson balayage of unit boundary mass across $I$ with this weight has the scale--free lower envelope
\[
  c_0(\psi)
  \,=\, \inf_{|x-T|\le L}\ \int_0^{L} w_I(a)\,P_a(x{-}T)\,da
  \,=\, \frac{1}{2\pi}\,\arctan 2.
\]
The equality follows by the rescaling $a=Ls$, $x{-}T=Lu$ and monotonicity in $u\in[0,1]$, with the minimum at the edge $u=1$; the resulting elementary integral evaluates to $\tfrac{1}{2\pi}\arctan 2$. Numerically,
\[
  c_0(\psi)\ =\ \frac{1}{2\pi}\arctan 2\ =\ 0.17620819\,.
\]
This is the value used throughout the certificate; it is independent of $I$, $L$, and the location $T$.

\medskip
\noindent All series are presented with monotone tails and explicit numerical enclosures; all suprema are bounded by monotone envelopes on the schedule $L\le 1/\log\langle t_0\rangle$. No external inputs are required.

\section*{Appendix: Evidence — certified covering outputs}\label{app:evidence}
\noindent This appendix records the decisive outputs used in Bridge~C. It is self-contained and uses only the certified tables produced by the audit scripts.

% Fill these two macros once per run
\newcommand{\deltaStar}{0.00408}
\newcommand{\sigmaEnd}{0.5100}

\paragraph{Summary.}
Minimum certified Schur margin: \(\delta_* = \deltaStar\).\quad Endpoint: \(\sigma_{\mathrm{end}} = \sigmaEnd\).

\medskip
\noindent\textbf{Schedule and diagnostics.}

% Per-step schedule with K, theta, and cumulative L
% % Certificate—Covering (weighted p-adaptive; Q=29, p_min=31, Cwin=1/4).
% This block is uniform in t (no t-dependence in K or in the budgets).

\begin{table}[H]
\centering
\caption{Certified covering schedule (weighted $p$-adaptive; $Q=29$, $p_{\min}=31$, $C_{\mathrm{win}}=\tfrac14$). Each row was audited to have $\delta_{\mathrm{Schur}}(\sigma_k)>0$.}
\label{tab:certificate-covering}
\begin{tabular}{r r r r r r r}
\toprule
$k$ & $\sigma_k$ & $h_k$ & $K(\sigma_k)$ & $\theta_k$ & $\Delta L_k$ & $L_k$ \\
\midrule
1 & 0.6000 & 0.0100 & 1.418255 & 0.014183 & 0.014284 & 0.014284 \\
2 & 0.5900 & 0.0100 & 1.442518 & 0.014425 & 0.014530 & 0.028814 \\
3 & 0.5800 & 0.0100 & 1.467425 & 0.014674 & 0.014783 & 0.043597 \\
4 & 0.5700 & 0.0100 & 1.493000 & 0.014930 & 0.015043 & 0.058640 \\
5 & 0.5600 & 0.0100 & 1.519269 & 0.015193 & 0.015309 & 0.073949 \\
6 & 0.5500 & 0.0100 & 1.546260 & 0.015463 & 0.015583 & 0.089533 \\
7 & 0.5400 & 0.0100 & 1.573998 & 0.015740 & 0.015865 & 0.105398 \\
8 & 0.5300 & 0.0100 & 1.602515 & 0.016025 & 0.016155 & 0.121553 \\
9 & 0.5200 & 0.0100 & 1.631841 & 0.016318 & 0.016453 & 0.138006 \\
10 & 0.5100 & 0.0095 & 1.662007 & 0.015789 & 0.015915 & 0.153921 \\
\bottomrule
\end{tabular}
\vspace{0.3em}

\par\small\textit{Notes.} $\theta_k=K(\sigma_k)\,h_k$ and $L_k=\sum_{j\le k}\! -\log(1-\theta_j)$. The Schur audit verifies $\delta_{\mathrm{Schur}}(\sigma_k)>0$ uniformly in $t$ for all rows. The corrected Bridge~C then yields nonvanishing of $\zeta$ on each line $\Re s=\sigma_k$.
\end{table}

\noindent The certified covering outputs are archived; the closing constants used in the certificate are locked above. Detailed per-step outputs can be provided upon request.

% Compact summary (\sigma_k, h_k, \theta_k, L_k)
% \begin{table}[H]
\centering
\caption{Certificate—Covering Summary ($\{\sigma_k,h_k,\theta_k\}$ and cumulative $L_k$).}
\label{tab:covering-summary}
\begin{tabular}{r r r r}
\toprule
$\sigma_k$ & $h_k$ & $\theta_k$ & $L_k$ \\
\midrule
0.6000 & 0.0100 & 0.014183 & 0.014284 \\
0.5900 & 0.0100 & 0.014425 & 0.028814 \\
0.5800 & 0.0100 & 0.014674 & 0.043597 \\
0.5700 & 0.0100 & 0.014930 & 0.058640 \\
0.5600 & 0.0100 & 0.015193 & 0.073949 \\
0.5500 & 0.0100 & 0.015463 & 0.089533 \\
0.5400 & 0.0100 & 0.015740 & 0.105398 \\
0.5300 & 0.0100 & 0.016025 & 0.121553 \\
0.5200 & 0.0100 & 0.016318 & 0.138006 \\
0.5100 & 0.0095 & 0.015789 & 0.153921 \\
\bottomrule
\end{tabular}
\end{table}

\noindent A compact covering summary is omitted for brevity; the final locked constants suffice for the certificate.

% Per-\sigma diagnostics with K(\sigma), \theta(\sigma), and L(\sigma)
% % Auto-generated per-σ covering table
\begin{table}[H]
\centering
\caption{Per-$\sigma$ covering diagnostics: $Q=29$, $p_{\min}=31$, $C_{\mathrm{win}}=0.25$, $\theta_{\max}=0.30$, $h_{\max}=0.015$. Each $\sigma_k$ listed was audited to have $\delta_{\mathrm{Schur}}(\sigma_k)>0$.}
\small
\begin{tabular}{r r r r r r}\toprule
$k$ & $\sigma_k$ & $h_k$ & $K(\sigma_k)$ & $\theta_k$ & $L(\sigma_k)$ \\ \midrule
1 & 0.6000 & 0.0150 & 4.870583 & 0.073059 & 0.075865 \\
2 & 0.5850 & 0.0150 & 4.880016 & 0.073200 & 0.151883 \\
3 & 0.5700 & 0.0150 & 4.889542 & 0.073343 & 0.228055 \\
4 & 0.5550 & 0.0150 & 4.899160 & 0.073487 & 0.304382 \\
5 & 0.5400 & 0.0150 & 4.908869 & 0.073633 & 0.380867 \\
6 & 0.5250 & 0.0150 & 4.918669 & 0.073780 & 0.457511 \\
7 & 0.5100 & 0.0095 & 4.928557 & 0.046821 & 0.505464 \\
\bottomrule
\end{tabular}
\end{table}

\noindent Per-line covering diagnostics are omitted; representative rows are listed below.

% Optional: prime-tail certificate table with explicit row-wise margins
% % Static longtable (no pgfplotstable dependency)
\setlength{\tabcolsep}{4.5pt}
\renewcommand{\arraystretch}{1.1}
\begin{longtable}{r r r r r r r r r r r r}
\caption{Prime-tail covering schedule and margins ($Q=53$, $\theta_{\max}=0.30$, $h_{\max}=0.015$, $C_{\pi}=1.26$, $p_{\min}^{\mathrm{cap}}=10^6$, $\tau_{\mathrm{FF}}=\tau_{\mathrm{FS}}=7.5\times10^{-4}$, $L_{\mathrm{seed}}\approx0.0108$).}\label{tab:prime-tail-certificate}\\
\toprule
$\sigma$ & $h$ & $K(\sigma)$ & $p_{\min}$ & $\Delta_{\rm SS}$ & $\Delta_{\rm SF}$ & $\Delta_{\rm FS}$ & $\Delta_{\rm FF}$ & $\mu_{\rm small}^{\min}$ & $\delta_{\rm cert}$ & $L$ & $\delta_{\rm cert}-e^{-L}$ \\
\midrule
\endfirsthead
\toprule
$\sigma$ & $h$ & $K(\sigma)$ & $p_{\min}$ & $\Delta_{\rm SS}$ & $\Delta_{\rm SF}$ & $\Delta_{\rm FS}$ & $\Delta_{\rm FF}$ & $\mu_{\rm small}^{\min}$ & $\delta_{\rm cert}$ & $L$ & $\delta_{\rm cert}-e^{-L}$ \\
\midrule
\endhead
\bottomrule
\endlastfoot
0.6000 & 0.0150 & 1.60344 & 77 & 0.0279663 & 0.0316651 & 0.0007495 & 0.0005709 & 0.9786261 & 0.9176743 & 0.0240516 & -0.0480744 \\
0.5850 & 0.0150 & 1.61776 & 86 & 0.0291379 & 0.0354960 & 0.0007268 & 0.0005996 & 0.9772491 & 0.9112887 & 0.0242664 & -0.0313068 \\
0.5700 & 0.0150 & 1.63286 & 91 & 0.0303640 & 0.0409033 & 0.0007494 & 0.0006882 & 0.9752871 & 0.9025823 & 0.0244929 & -0.0172068 \\
0.5550 & 0.0150 & 1.64754 & 107 & 0.0316473 & 0.0471726 & 0.0006927 & 0.0007084 & 0.9740892 & 0.8938682 & 0.0247131 & -0.0034686 \\
0.5400 & 0.0150 & 1.66233 & 132 & 0.0329907 & 0.0559256 & 0.0006121 & 0.0007166 & 0.9731981 & 0.8829530 & 0.0249349 & 0.0077145 \\
0.5250 & 0.0150 & 1.67746 & 171 & 0.0343973 & 0.0700372 & 0.0005179 & 0.0007329 & 0.9726270 & 0.8669416 & 0.0251619 & 0.0134510 \\
0.5100 & 0.0150 & 1.69284 & 259 & 0.0358704 & 0.1001679 & 0.0003772 & 0.0007368 & 0.9733260 & 0.8361737 & 0.0253926 & 0.0040825 \\
\end{longtable}

\noindent The prime-tail certificate table is omitted; tail bounds are incorporated in $K_0$ and the locked $C_{\mathrm{box}}$.

\paragraph{Last five rows (verbatim).}
\begin{verbatim}
0.6000 & 0.0150 & 1.60344 & 77  & 0.0279663 & 0.0316651 & 0.0007495 & 0.0005709 & 0.9786261 & 0.9176743 & 0.0240516 & -0.0480744 
0.5850 & 0.0150 & 1.61776 & 86  & 0.0291379 & 0.0354960 & 0.0007268 & 0.0005996 & 0.9772491 & 0.9112887 & 0.0242664 & -0.0313068 
0.5700 & 0.0150 & 1.63286 & 91  & 0.0303640 & 0.0409033 & 0.0007494 & 0.0006882 & 0.9752871 & 0.9025823 & 0.0244929 & -0.0172068 
0.5550 & 0.0150 & 1.64754 & 107 & 0.0316473 & 0.0471726 & 0.0006927 & 0.0007084 & 0.9740892 & 0.8938682 & 0.0247131 & -0.0034686 
0.5400 & 0.0150 & 1.66233 & 132 & 0.0329907 & 0.0559256 & 0.0006121 & 0.0007166 & 0.9731981 & 0.8829530 & 0.0249349 & 0.0077145 
\end{verbatim}

\medskip
\section*{Appendix: Explicit Gram/Fock Construction and Tails}\label{sec:appendix}
For $s=\sigma+it$ with $\sigma>1/2$, define block signals $\Psi_j^{(s)}(x):=\sum_{p\in B_j} p^{-\sigma}e^{-x\log p}$ and $V_s e_j:= e^{-(s-1/2)x}\Psi_j^{(s)}(x)$. Set $T(s):=V_s^{\,*}V_s$ (PSD, analytic in $s$). Then
\[
T_{mn}(\sigma)\ =\ \sum_{p\in B_m}\sum_{q\in B_n}\frac{p^{-\sigma}q^{-\sigma}}{\,2\sigma-1+\log p+\log q\,}\,,\quad \|T(\sigma)\|_2^2\ \ll\ \Big(\sum_p \tfrac{p^{-2\sigma}}{1+2\log p}\Big)^2.
\]
Using Lemma~\ref{lem:pi-majorant-explicit} and partial summation yields the far-tail bounds required for the covering budgets.

% --- Oscillation => Wedge (auxiliary lemma) ---
\clearpage
% Oscillation ⇒ Wedge (BV and AC forms)
% Ready to \input into the manuscript; assumes theorem environments are defined.

\section*{Oscillation $\Rightarrow$ Wedge (BV and AC forms)}

\begin{lemma}[Oscillation $\Rightarrow$ Wedge (BV form)]\label{lem:OscToWedgeBV}
Let $w:\mathbb{R}\to\mathbb{R}$ be locally of bounded variation. Write the distributional derivative $Dw=\mu^+-\mu^-$ with $\mu^\pm$ finite positive Radon measures on bounded intervals. Assume:
\begin{itemize}
\item[(A)] (Two-sided oscillation) For every bounded open interval $I\subset\mathbb{R}$,
\[ \mu^-(I)\ \le\ \tfrac{\pi}{2}\qquad\text{and}\qquad \mu^+(I)\ \le\ \tfrac{\pi}{2}. \]
\item[(B)] (Anchor) There exists a Lebesgue point $t_\star\in\mathbb{R}$ with $w(t_\star)=0$.
\end{itemize}
Then $w(t)\in[-\tfrac{\pi}{2},\tfrac{\pi}{2}]$ for almost every $t\in\mathbb{R}$.
\end{lemma}

\begin{proof}
Let $J=(a,b)$ be any bounded open interval with endpoints that are points of approximate continuity of $w$ (true for a.e. choice since BV functions are approximately continuous a.e.). The BV fundamental theorem gives
\[ w(b)-w(a)=Dw((a,b))=\mu^+((a,b)) - \mu^-((a,b)). \]
By (A), $0\le \mu^\pm((a,b))\le \pi/2$, hence $-\pi/2\le w(b)-w(a)\le \pi/2$. Fix the anchor $t_\star$ from (B) and any $t$ that is a point of approximate continuity. Applying the previous bound to $(\min\{t,t_\star\},\max\{t,t_\star\})$ yields
\[ -\tfrac{\pi}{2}\ \le\ w(t)-w(t_\star)\ \le\ \tfrac{\pi}{2}. \]
Since $w(t_\star)=0$, it follows that $|w(t)|\le \pi/2$ for a.e. $t$.
\end{proof}

\begin{remark}
The anchor pins the absolute phase. Without (B), (A) still implies the essential range of $w$ has diameter at most $\pi/2$, i.e., $|w(t)-w(s)|\le \pi/2$ for a.e. $s,t$, but the interval’s center is undetermined. In applications, the branch is fixed so that $w=0$ at a designated reference (e.g., the box center).
\end{remark}

\begin{corollary}[Absolutely continuous form]\label{cor:OscToWedgeAC}
Let $w\in W^{1,1}_{\mathrm{loc}}(\mathbb{R})$ and suppose that for every bounded open interval $I\subset\mathbb{R}$,
\[ \int_I (-w'(t))\,dt\ \le\ \tfrac{\pi}{2}\qquad\text{and}\qquad \int_I w'(t)\,dt\ \le\ \tfrac{\pi}{2}. \]
If there exists $t_\star$ with $w(t_\star)=0$, then $w(t)\in[-\tfrac{\pi}{2},\tfrac{\pi}{2}]$ for almost every $t$.
\end{corollary}

\begin{proof}
For $w\in W^{1,1}_{\mathrm{loc}}$, the BV measures are absolutely continuous: $\mu^\pm(I)=\int_I (w')_\pm$. The two integral bounds are equivalent to (A), so Lemma~\ref{lem:OscToWedgeBV} applies.
\end{proof}

% Implementation note (not printed): In the manuscript, the two-sided bounds in (A)
% are obtained from the product certificate by applying it to F and to 1/F (invariance under inversion),
% while the anchor is fixed by the chosen phase branch at the box center.


% --- Boundary regularity, outer normalization, and phase–velocity ---
\clearpage
% Boundary regularity & outer normalization; phase–velocity identity and certificate (distributional form)
% Ready to \input into the manuscript; assumes theorem environments are defined.

\section*{Boundary regularity, outer normalization, and phase–velocity}

\paragraph{Geometry.}
Work in the upper half–plane $\mathbb{H}=\{z=x+iy:y>0\}$ with boundary $\mathbb{R}$. For a bounded interval $I=(c-\tfrac{L}{2},c+\tfrac{L}{2})$ and aperture $\alpha>1$ define
\[
Q_\alpha(I):=\{x+iy:\ x\in \alpha I,\ 0<y\le \alpha L\},\qquad
\Gamma_\alpha(t):=\{(x,y): |x-t|<\alpha y,\ y>0\}.
\]
If $U=\mathcal P[u]$ is harmonic, its fixed–aperture box energy is
\[C_{\rm box}(U;\alpha):=\sup_{I}\ \frac{1}{|I|}\iint_{Q_\alpha(I)}|\nabla U|^2\,y\,dx\,dy\in[0,\infty).\]

\subsection*{Boundary regularity and local outer normalization}

\begin{lemma}[Carleson $\Rightarrow$ BMO and $L^1_{\rm loc}$]\label{lem:BMO-outer}
Let $U=\mathcal P[u]$ with $C_{\rm box}(U;\alpha)<\infty$. Then $u$ has a.e. non–tangential boundary values and $u\in{\rm BMO}_{\rm loc}(\mathbb{R})$ with
\[\sup_{I}\frac{1}{|I|}\int_I |u(t)-u_I|\,dt\ \le\ C_\alpha\,\sqrt{C_{\rm box}(U;\alpha)},\qquad u_I:=|I|^{-1}\int_I u.\]
In particular $u\in L^1_{\rm loc}(\mathbb{R})$.
\end{lemma}

\begin{lemma}[Local outer normalization on a box]\label{lem:outer-local}
Fix a bounded interval $I$ and let $U=\mathcal P[u]$ with $u\in L^1(I)$. There exists a holomorphic, zero–free $\mathcal O_I$ on $Q_\alpha(I)$ such that
\[\log|\mathcal O_I|=U\ \text{ on }Q_\alpha(I),\qquad |\mathcal O_I^*(t)|=e^{u(t)}\ \text{ for a.e. }t\in I,\]
uniquely determined by $\mathcal O_I(z_I)\in(0,\infty)$ at a fixed interior $z_I\in Q_\alpha(I)$.
\end{lemma}

\subsection*{Phase–velocity identity (distributional form) and certificate}

Let $F$ be meromorphic of bounded type on $\mathbb{H}$, and write $u=\log|F^*|$, $w=\arg(F^*)$. Denote the signed zero–pole measure by
\[\mu\ :=\ \sum_{p\in P(F)} m(p)\,\delta_p\ -\ \sum_{a\in Z(F)} m(a)\,\delta_a.\]
For a bounded open interval $J=(a,b)$ define $\mathrm{Bal}(x+iy;J):=\int_a^b\tfrac{2y}{(t-x)^2+y^2}\,dt\in[0,2\pi]$.

\begin{proposition}[Phase–velocity identity]\label{prop:phase-velocity}
For $\varphi\in C_c^\infty(\mathbb{R})$,
\[\int_{\mathbb{R}} (-w'(t))\,\varphi(t)\,dt\ =\ 2\pi\int_{\mathbb{H}} \mathcal{P}[\varphi](z)\,d\mu(z)\ -\ \int_{\mathbb{R}}\mathcal{H}[u'](t)\,\varphi(t)\,dt.\]
Consequently, for any bounded open $J=(a,b)$ with endpoints of approximate continuity,
\[\int_{J} (-w'(t))\,dt\ =\ \int_{\mathbb{H}}\mathrm{Bal}(z;J)\,d\mu(z)\ -\ \big(\mathcal{H}[u](b)-\mathcal{H}[u](a)\big).\]
\end{proposition}

\begin{theorem}[Certificate inequality on boxes]\label{thm:certificate-distribution}
Let $F$ be meromorphic of bounded type, $u=\log|F^*|$, $U=\mathcal P[u]$, and fix $\alpha>1$ with $C_{\rm box}(U;\alpha)<\infty$. For every $L,t_0$,
\[\int_{\mathbb{R}} (-w'(t))\,\varphi_{L,t_0}(t)\,dt\ \le\ C_H(\psi)\,L\,M_\psi(u)\ \le\ \frac{4}{\pi}\,C_H(\psi)\,C_\psi^{(H^1)}\,L\,\sqrt{C_{\rm box}},\]
and
\[\int_{\mathbb{H}}\Phi_{L,t_0}(z)\,d\mu(z)\ \ge\ c_0(\psi)\,\mu\!\left(Q_\alpha(I)\right).\]
\end{theorem}

\begin{remark}
The far–field tail in the last inequality is nonnegative for Poisson kernels; if a neutralization is inserted, the same conclusion holds since $\Phi_{L,t_0}\ge 0$.
\end{remark}


% --- Phase–balayage identity and neutralization ---
\clearpage
% Phase–Balayage on the Half–Plane: Measure, Kernel, Bounds, and Neutralization
% Ready to \input into the manuscript; assumes theorem environments are defined.

\section*{Phase--balayage on the half--plane (identity and neutralization)}

\paragraph{Setting and notation.}
Work in the upper half--plane $\mathbb{H}=\{z=x+iy:y>0\}$ with boundary $\mathbb{R}$. Let $F$ be meromorphic in $\mathbb{H}$, of bounded type, with non--tangential boundary values a.e. on $\mathbb{R}$. Write
\[
\log F = U+iV \qquad (U=\log|F|,\; V=\arg F)
\]
on simply connected subsets of $\mathbb{H}$ that avoid zeros and poles. For $\varphi\in C_c^\infty(\mathbb{R})$, let $\mathcal{P}[\varphi]$ denote its Poisson extension and $\mathcal H[\varphi]$ its Hilbert transform. Write $w(t):=\Arg F(t)$ and $u(t):=\log|F(t)|$.

\paragraph{Zero--pole measure and Bal kernel.}
Let $Z(F)$ (resp. $P(F)$) be the multisets of zeros (resp. poles) with multiplicities. Define the signed atomic measure
\[
\mu\;:=\;\sum_{p\in P(F)} m(p)\,\delta_p\;-
\;\sum_{a\in Z(F)} m(a)\,\delta_a\qquad\text{on }\mathbb{H},
\]
so poles contribute positively and zeros negatively. For a bounded interval $I=(a,b)\subset\mathbb{R}$ and $z=x+iy\in\mathbb{H}$, define
\[
\mathrm{Bal}(z;I)\;:=\;2\pi\,\mathcal{P}[\mathbf{1}_I](z)
\;=\;\int_{a}^{b}\frac{2y}{(t-x)^2+y^2}\,dt
\;=\;2\Big(\arctan\tfrac{b-x}{y}-\arctan\tfrac{a-x}{y}\Big),
\]
with $0\le \mathrm{Bal}\le 2\pi$, increasing in $I$, and $\mathrm{Bal}(z;\mathbb{R})=2\pi$.

\begin{lemma}[Far--field bound for the Bal kernel]\label{lem:farfield}
Let $I=(a,b)$ with length $L=b-a$ and center $c=\tfrac{a+b}{2}$. Then, for $z=x+iy\in\mathbb{H}$,
\[
\mathrm{Bal}(z;I)\ \le\ \frac{4Ly}{(x-c)^2+y^2-\tfrac{L^2}{4}}\quad\text{whenever }(x-c)^2+y^2>\tfrac{L^2}{4}.
\]
In particular, if $z\notin Q(\alpha I):=\{x+iy: x\in \alpha I,\ 0<y\le \alpha L\}$ with $\alpha>1$, then
\[
\mathrm{Bal}(z;I)\ \le\ \frac{C_\alpha\,L\,y}{\mathrm{dist}(z,I)^2+y^2},
\]
with $C_\alpha$ depending only on $\alpha$.
\end{lemma}

\begin{theorem}[Phase--balayage, test--function form]\label{thm:balayage-test}
Assume $F$ is meromorphic in $\mathbb{H}$ of bounded type and $u=\log|F|\in L^1_{\mathrm{loc}}(\mathbb{R})$. Then for every $\varphi\in C_c^\infty(\mathbb{R})$,
\[
\boxed{\quad
\int_{\mathbb{R}} (-w'(t))\,\varphi(t)\,dt
\;=\;
2\pi\int_{\mathbb{H}} \mathcal{P}[\varphi](z)\,d\mu(z)\;-
\;\int_{\mathbb{R}}\mathcal{H}[u'](t)\,\varphi(t)\,dt.
\quad}
\]
Equivalently, $\langle -w',\varphi\rangle=\int_{\mathbb{H}} 2\pi\,\mathcal{P}[\varphi]\,d\mu-\langle \mathcal{H}[u'],\varphi\rangle$.
\end{theorem}

\begin{proof}
Use Poisson--Jensen for $U$, take harmonic conjugates for $V$, differentiate tangentially, and pass to non--tangential boundary limits. The Poisson identity
$\int \tfrac{2y}{(t-x)^2+y^2}\,\varphi(t)\,dt=2\pi\,\mathcal P[\varphi](x+iy)$ converts each atom to the Poisson balayage. Summing with signs yields the identity.
\end{proof}

\begin{corollary}[Phase--balayage, interval form]\label{cor:balayage-interval}
Under the hypotheses of Theorem~\ref{thm:balayage-test}, for any bounded open interval $I=(a,b)$ whose endpoints are points of approximate continuity for $u$ and $w$,
\[
\boxed{\quad
\int_{I} (-w'(t))\,dt
\;=\;\int_{\mathbb{H}}\mathrm{Bal}(z;I)\,d\mu(z)
\;-
\big(\mathcal{H}[u](b)-\mathcal{H}[u](a)\big).
\quad}
\]
\end{corollary}

\begin{proof}
Approximate $\mathbf{1}_I$ by smooth $\varphi_n$ and pass to the limit in Theorem~\ref{thm:balayage-test} using dominated convergence for the Poisson term and a.e. boundary values for $\mathcal H[u]$ at the endpoints.
\end{proof}

\paragraph{Neutralization on a Carleson box.}
For a bounded interval $I$ and aperture $\alpha>1$, set
\[
Q(I)=\{x+iy : x\in I,\ 0<y\le |I|\},\qquad Q(\alpha I)=\{x+iy : x\in \alpha I,\ 0<y\le \alpha|I|\}.
\]
Define the local Blaschke neutralizer on $Q(\alpha I)$ by
\[
B_I(z):=\prod_{a\in Z(F)\cap Q(\alpha I)} \Big(\tfrac{z-a}{z-\overline{a}}\Big)^{m(a)}\,
\prod_{p\in P(F)\cap Q(\alpha I)} \Big(\tfrac{z-\overline{p}}{z-p}\Big)^{m(p)}.
\]
Then $B_I$ is inner: $|B_I(t)|=1$ for a.e. $t\in\mathbb{R}$. Let $\widetilde F:=F/B_I$. By construction, $\widetilde F$ has no zeros or poles in $Q(\alpha I)$ and $\mathcal H[\log|\widetilde F|]=\mathcal H[\log|F|]$ a.e.

\begin{corollary}[Neutralized phase--balayage on $I$]\label{cor:neutralized}
With the above notation,
\[
\int_I (-w_F')\,dt
\;=\; \int_{Q(\alpha I)} \mathrm{Bal}(z;I)\,d\mu(z)
\;+
\int_{\mathbb{H}\setminus Q(\alpha I)} \mathrm{Bal}(z;I)\,d\mu(z)
\;-
\big(\mathcal{H}[u](b)-\mathcal{H}[u](a)\big),
\]
so that
\[
\int_I (-w_{\widetilde F}')\,dt
\;=\;\int_{\mathbb{H}\setminus Q(\alpha I)} \mathrm{Bal}(z;I)\,d\mu_{\mathrm{far}}(z)
\;-
\big(\mathcal{H}[u](b)-\mathcal{H}[u](a)\big).
\]
Moreover,
\[
\int_I(-w_{B_I}')\,dt\;=\;\int_{Q(\alpha I)} \mathrm{Bal}(z;I)\,d\mu_{\mathrm{near}}(z).
\]
\end{corollary}

\begin{lemma}[Uniform bounds after neutralization]\label{lem:uniform-bounds}
For $z\notin Q(\alpha I)$,
\[
0\ \le\ \mathrm{Bal}(z;I)\ \le\ 2\pi\ \min\!\left\{1,\ \frac{C_\alpha\,|I|\,\Im z}{\mathrm{dist}(z,I)^2+(\Im z)^2}\right\}.
\]
Consequently,
\[
\left|\int_{\mathbb{H}\setminus Q(\alpha I)} \mathrm{Bal}(z;I)\,d\mu(z)\right|
\;\le\; 2\pi\,|\mu|(\mathbb{H}\setminus Q(\alpha I)),
\]
with refined decay available from Lemma~\ref{lem:farfield} when $\mathrm{dist}(z,I)\gg |I|$.
\end{lemma}

\begin{remark}
The identity separates two mechanisms:
\[
\int_I (-w')= \underbrace{\int \mathrm{Bal}\,d\mu}_{\text{discrete interior data: poles $-$ zeros}} - \underbrace{\big(\mathcal{H}[u](b)-\mathcal{H}[u](a)\big)}_{\text{outer (boundary--driven) term}}.
\]
The Bal term is purely geometric ($0\le \mathrm{Bal}\le 2\pi$); after neutralization it becomes a controlled far--field sum. The outer term depends only on the boundary modulus and is the only source of sign changes unrelated to zeros/poles; it is handled by fixed-aperture $H^1$–BMO/Carleson control against smooth windows.
\end{remark}


% --- Half–plane outer normalization and boundary behavior ---
\clearpage
% Half–plane: outer normalization and boundary behavior (snippet)
% Ready to \input into the manuscript; assumes theorem/proof environments exist.

\section*{Half–plane framework: outer normalization and boundary behavior}

We work on the right half–plane
\[
\Omega\ :=\ \{\,s\in\mathbb{C}:\ \Re s>\tfrac12\,\},\qquad \partial\Omega=\{\tfrac12+it:\ t\in\mathbb{R}\}.
\]
For a bounded interval $I\subset\mathbb{R}$ and $\alpha\ge1$, write
\[
Q(\alpha I)=\bigl\{\tfrac12+x+iy:\ 0<x<\alpha|I|,\ y\in I\bigr\}.
\]
For $g\in L^1_{\rm loc}(\mathbb{R})$, denote by $\mathcal P[g]$ its Poisson extension to $\Omega$.

\subsection*{Local outer construction}

\begin{definition}[Local outer from boundary modulus]\label{hp:def:outer}
Fix a bounded interval $I$ and let $b\in L^1(I)\cap \mathrm{BMO}(I)$.
Extend $b$ by $0$ to $\mathbb{R}$ and set $U:=\mathcal P[b]$ on $\Omega$.
Let $\widetilde U$ be a harmonic conjugate of $U$, normalized at a fixed interior point.
Define the outer on $\Omega$ by
\[ \mathcal O_I(s)\ :=\ \exp\big(U(s)+i\,\widetilde U(s)\big). \]
\end{definition}

\begin{proposition}[Basic properties of $\mathcal O_I$]\label{hp:prop:outer-props}
With $b$ and $\mathcal O_I$ as above:
\begin{enumerate}
\item $\mathcal O_I$ is analytic and zero–free on $\Omega$.
\item For a.e. $t\in I$, the nontangential boundary limit satisfies $|\mathcal O_I^*(t)|=e^{b(t)}$; for a.e. $t\notin I$, $|\mathcal O_I^*(t)|=1$.
\item On every bounded $I'\subset\mathbb{R}$, $\log|\mathcal O_I^*|\in L^1(I')$ and $\log|\mathcal O_I^*|\in \mathrm{BMO}(I'\cap I)$ with
\[ \|\log|\mathcal O_I^*|\|_{\mathrm{BMO}(I'\cap I)}\ \le\ C\,\|b\|_{\mathrm{BMO}(I)}. \]
\item For each fixed aperture $\alpha\ge1$, the Carleson area of $\nabla\log|\mathcal O_I|$ over $Q(\alpha I)$ obeys
\[ \frac{1}{|I|}\iint_{Q(\alpha I)} x\,|\nabla U|^2\,dx\,dy\ \le\ C(\alpha)\,\|b\|_{\mathrm{BMO}(I)}^{\,2}. \]
\end{enumerate}
\end{proposition}

\begin{lemma}[Smooth approximation]\label{hp:lem:smooth-approx}
Let $\{\eta_\varepsilon\}_{\varepsilon\downarrow0}$ be an approximate identity and set $b_\varepsilon:=\eta_\varepsilon\ast b$.
Let $\mathcal O_{I,\varepsilon}:=\exp(\mathcal P[b_\varepsilon]+i\,\widetilde{\mathcal P[b_\varepsilon]})$.
Then $\mathcal O_{I,\varepsilon}\to \mathcal O_I$ locally uniformly on $\Omega$ and a.e. on $\partial\Omega$ (in modulus).
\end{lemma}

\subsection*{Outer normalization for a given analytic field}

\begin{definition}[Local outer normalization for $\mathcal J$]\label{hp:def:J}
Let $R$ be analytic on $\Omega$ and let $b_I\in L^1(I)\cap \mathrm{BMO}(I)$ satisfy $b_I\ge \log|R^*|$ a.e. on $I$.
Define $\mathcal O_I$ from $b_I$ as above and set
\[ \mathcal J_I(s)\ :=\ \frac{R(s)}{\mathcal O_I(s)}. \]
\end{definition}

\begin{theorem}[Boundary values and control for $\mathcal J_I$]\label{hp:thm:boundary-J}
Assume $R\in N^+_{\mathrm{loc}}(\Omega)$ and $\log|R^*|\in L^1_{\mathrm{loc}}(\mathbb{R})$.
Then $\mathcal J_I$ has a.e. nontangential boundary limits on $I$ and
\[ |\mathcal J_I^*(t)|\ =\ \exp\big(\log|R^*(t)|-b_I(t)\big)\ \le\ 1 \quad \text{for a.e. } t\in I. \]
Moreover $\log|\mathcal J_I^*|\in L^1(I)\cap \mathrm{BMO}(I)$ with
\[ \|\log|\mathcal J_I^*|\|_{\mathrm{BMO}(I)}\ \le\ C\,\big(\|\log|R^*|\|_{\mathrm{BMO}(I)}+\|b_I\|_{\mathrm{BMO}(I)}\big). \]
\end{theorem}

\begin{proposition}[Stability under smoothing]\label{hp:prop:stability}
With $b_{I,\varepsilon}$ and $\mathcal O_{I,\varepsilon}$ as in Lemma~\ref{hp:lem:smooth-approx}, set $\mathcal J_{I,\varepsilon}:=R/\mathcal O_{I,\varepsilon}$.
Then $\mathcal J_{I,\varepsilon}\to \mathcal J_I$ locally uniformly on $\Omega$, and $\log|\mathcal J_{I,\varepsilon}^*|\to\log|\mathcal J_I^*|$ a.e. on $I$.
\end{proposition}

\subsection*{Global outer normalization (local hypotheses)}

\begin{theorem}[Global boundary behavior of $\mathcal J$]\label{hp:thm:J-global}
Assume $R\in N^+_{\mathrm{loc}}(\Omega)$ and $\log|R^*|\in L^1_{\mathrm{loc}}$.
Let $b\in L^1_{\mathrm{loc}}(\mathbb{R})\cap \mathrm{BMO}_{\mathrm{loc}}(\mathbb{R})$ dominate $\log|R^*|$ a.e., and set $\mathcal O:=\exp(\mathcal P[b]+i\,\widetilde{\mathcal P[b]})$, $\mathcal J:=R/\mathcal O$.
Then $\mathcal J$ has a.e. nontangential boundary limits on $\partial\Omega$ with $|\mathcal J^*|\le1$ a.e., and on each bounded $I$ one has $\log|\mathcal J^*|\in L^1(I)\cap \mathrm{BMO}(I)$.
\end{theorem}




% --- Wedge Lemma: two-sided oscillation ⇒ boundary wedge ---
\clearpage
% The Wedge Lemma (Phase Rise vs. Drop): Two–sided control and boundary regularity
% Ready to \input into the manuscript; assumes theorem environments are defined.

\section*{The Wedge Lemma (phase rise vs. drop)}

\paragraph{Setting.}
Work in the upper half–plane $\mathbb{H}=\{x+iy:y>0\}$ with boundary $\mathbb{R}$. Let $F$ be meromorphic in $\mathbb{H}$, of bounded type, with a.e. non–tangential boundary values. Write
\[
\log F=U+iV, \qquad u:=\log|F|\in L^1_{\mathrm{loc}}(\mathbb{R}),\qquad
w:=\text{a.e. boundary trace of }V\ (\text{the boundary phase}).
\]
Fix an interval $I=(t_0-\tfrac{L}{2},t_0+\tfrac{L}{2})$ and an aperture $\alpha>1$;
set $Q(\alpha I):=\{x+iy:\ x\in \alpha I,\ 0<y\le \alpha L\}$.

\paragraph{Phase–balayage (interval form).}
Let $\mu$ be the signed zero–pole measure of $F$ in $\mathbb{H}$ (poles positive, zeros negative, with multiplicity). For any bounded open interval $J=(a,b)\subset \mathbb{R}$ whose endpoints are points of approximate continuity for $u,w$ one has
\begin{equation}\label{eq:PB-interval}
\int_{J} (-w'(t))\,dt
\;=\;\int_{\mathbb{H}}\mathrm{Bal}(z;J)\,d\mu(z)\;-
\;\big(\mathcal{H}[u](b)-\mathcal{H}[u](a)\big),
\end{equation}
where $\mathrm{Bal}(x+iy;J)=\int_{a}^{b}\tfrac{2y}{(t-x)^2+y^2}\,dt\in[0,2\pi]$ is the balayage kernel and $\mathcal{H}$ the Hilbert transform.

\paragraph{The certificate format.}
Let $\psi$ be a fixed $C^1$ window of mean $0$ supported in $(-1,1)$ and set $\varphi_{L,t_0}(t)=L^{-1}\psi\big(\frac{t-t_0}{L}\big)$. Assume the box energy
\[
C_{\mathrm{box}}=\sup_{J}\frac{1}{|J|}\iint_{Q(\alpha J)}|\nabla U|^2\,y\,dx\,dy
\]
is finite. Let $C_H(\psi),\,C_\psi^{(H^1)},\,c_0(\psi),\,C_P$ be the window/geometry constants (Poisson plateau $c_0>0$, Hilbert envelope $C_H$, $H^1$ window constant, band–limit bound). Define
\[
\Upsilon(F):=\frac{C_H(\psi)\,M_\psi(F)+C_P}{c_0(\psi)},\qquad
M_\psi(F):=\sup_{L,t_0}\frac{1}{L}\Big|\int u(t)\,\varphi_{L,t_0}(t)\,dt\Big|.
\]
Then
\begin{equation}\label{eq:cert-imp}
\Upsilon(F)\le \tfrac{1}{2}\quad\Longrightarrow\quad
\int_{J}(-w'_F)\,dt\ \le\ \frac{\pi}{2}\quad\text{for every subinterval }J\subset I.
\end{equation}
This follows by testing \eqref{eq:PB-interval} against $\varphi_{L,t_0}$, bounding the outer term via the fixed–aperture $H^1$–BMO/Carleson estimate
$M_\psi\le \tfrac{4}{\pi}C_\psi^{(H^1)}\sqrt{C_{\mathrm{box}}}$, and using the plateau lower bound $c_0$ and $\mathrm{Bal}\le 2\pi$ (constants fixed once $\psi,\alpha$ are fixed).

\section*{Certificate invariance and rise control}

\begin{lemma}[Invariance under inversion]\label{lem:inv-wedge}
Let $F^\sharp:=1/F$. Then
\[
u^\sharp=-u,\qquad w^\sharp=-w\ (\text{a.e.}),\qquad \mu^\sharp=-\mu,\qquad C_{\mathrm{box}}(F^\sharp)=C_{\mathrm{box}}(F),
\]
\[M_\psi(F^\sharp)=M_\psi(F),\qquad \Upsilon(F^\sharp)=\Upsilon(F).\]
\end{lemma}

\begin{lemma}[Rise bound]\label{lem:rise-wedge}
If $\Upsilon(F)\le \tfrac12$, then for every subinterval $J\subset I$,
\[\int_{J} w'_F(t)\,dt\ \le\ \frac{\pi}{2}.\]
\end{lemma}

\begin{proof}
Apply \eqref{eq:cert-imp} to $F^\sharp$: $\int_J (-w'_{F^\sharp})\le \pi/2$. Since $w^\sharp=-w$, $-\int_J (-w'_F)\le \pi/2$, i.e. $\int_J w'_F\le \pi/2$.
\end{proof}

Combining \eqref{eq:cert-imp} and Lemma~\ref{lem:rise-wedge} yields
\begin{equation}\label{eq:two-sided}
\left|\int_{J} (-w'_F(t))\,dt\right|\ \le\ \frac{\pi}{2}\qquad\text{for every subinterval }J\subset I.
\end{equation}

\section*{Regularity and the canonical phase representative}

\begin{lemma}[Countable additivity of the phase–drop functional]\label{lem:add-wedge}
Define $T$ on finite disjoint unions of bounded open intervals by
\[
T\Big(\bigsqcup_{m=1}^M (a_m,b_m)\Big)\ :=\ \sum_{m=1}^M\left(\int_{\mathbb{H}}\mathrm{Bal}(z;(a_m,b_m))\,d\mu(z)
-\big(\mathcal{H}[u](b_m)-\mathcal{H}[u](a_m)\big)\right).
\]
Then $T$ is well–defined and countably additive on the algebra generated by bounded open intervals.
\end{lemma}

\begin{lemma}[Canonical phase via indefinite integral]\label{lem:W-wedge}
Fix an anchor $t_\star\in I$ at which $u,w$ have non–tangential boundary values and set $w(t_\star)=0$ (branch choice). Define
\[w^\sharp(t)\ :=\ -\,T\big((t_\star,t]\cap I\big),\qquad t\in I.\]
Then $w^\sharp\in BV(I)$ and for all $a<b$ in $I$,
\[w^\sharp(b)-w^\sharp(a)\ =\ -\,T\big((a,b)\big)\ =\ \int_{(a,b)} (-w'(t))\,dt.\]
\end{lemma}

\begin{lemma}[Identification $w^\sharp=w$ a.e.]\label{lem:identify-wedge}
The distributional derivatives of $w$ and $w^\sharp$ coincide on $I$, hence $w^\sharp-w$ is constant a.e.; with the anchor $w(t_\star)=w^\sharp(t_\star)=0$ one has $w^\sharp=w$ a.e. on $I$.
\end{lemma}

\section*{Wedge from two–sided oscillation (no extra smoothness needed)}

\begin{theorem}[Wedge Lemma: phase rise vs. drop]\label{thm:wedge}
Assume $\Upsilon(F)\le \tfrac12$ on $I$. Fix the anchored representative $w^\sharp$ of Lemma~\ref{lem:W-wedge}. Then for every $t\in I$,
\[|w^\sharp(t)|\ \le\ \frac{\pi}{2}.\]
Consequently $w(t)\in[-\tfrac{\pi}{2},\tfrac{\pi}{2}]$ for a.e. $t\in I$.
\end{theorem}

\begin{remark}[Anchoring]
The wedge is a statement modulo additive constants; fixing $w(t_\star)=0$ pins the branch. Any other choice shifts the wedge interval by the same constant.
\end{remark}


% --- Certificate–to–Wedge Theorem ---
\clearpage
% Certificate–to–Wedge Theorem (Half–Plane Schur Route)
% Ready to \input into the manuscript; assumes theorem environments are defined.

\section*{Certificate–to–Wedge Theorem}

\paragraph{Setting.}
Let $\Omega:=\{ s\in\mathbf C : \Re s>1/2\}$ and $\partial\Omega=\{1/2+it:t\in\mathbf R\}$. Fix $\alpha\ge1$. For a bounded interval $I\subset\mathbf R$ write the Carleson box
\[
Q(\alpha I):=\{\tfrac12+x+iy: y\in I,\ 0<x<\alpha|I|\},\qquad \Omega(I):=\Omega\cap\{\Im s\in I\}.
\]
Let $\mathcal J$ be analytic on $\Omega$ and set $F:=2\mathcal J$. Assume non–tangential boundary values $F^*(t)=R(t)e^{iw(t)}$ exist a.e. on $I$, with $R>0$ and phase $w\in\mathbf R$ (branch fixed up to an additive constant).

Fix a window $\psi$ supported in $I$. Denote fixed constants
\[c_0(\psi)>0,\quad C_H(\psi)>0,\quad C_P(\kappa)>0,\quad C_\psi^{(H^1)}>0,\quad C_{\mathrm{box}}>0.\]
Define the control functional $M_\psi\ge0$ and the certificate
\[\Upsilon\ :=\ \frac{C_H(\psi)\,M_\psi + C_P(\kappa)}{\,c_0(\psi)\,}.\]

\paragraph{Negative variation on an interval.}
For real $u\in L^1_{\rm loc}(I)$, define
\[\mathrm{Var}^{-}_{I}(u)\ :=\ \sup\ \sum_{j=0}^{m-1} \max\{0,\ u(t_j)-u(t_{j+1})\},\]
with the supremum over finite partitions $t_0<\cdots<t_m$ of $I$.

\subsection*{Hypotheses (to be verified for $F=2\mathcal J$)}

- (H1) Boundary trace and phase regularity: $F^*$ exists a.e. on $I$, $R\in L^1_{\rm loc}(I)$, and $w$ has locally finite variation on $I$.

- (H2) Phase–energy domination: there is a finite Borel measure $\mu$ on $Q(\alpha I)$ such that
\[\mathrm{Var}^{-}_{I}(w)\ \le\ \pi\,\mu\bigl(Q(\alpha I)\bigr).\]

- (H3) Fixed–aperture embedding for the control functional:
\[M_\psi\ \le\ \frac{4}{\pi}\,C_\psi^{(H^1)}\,\sqrt{C_{\mathrm{box}}},\qquad \mu\bigl(Q(\alpha I)\bigr)\ \le\ C_{\mathrm{box}}.\]

- (H4) Phase budget inequality for the chosen window $\psi$:
\[\mathrm{Var}^{-}_{I}(w)\ \le\ C_H(\psi)\,M_\psi\ +\ C_P(\kappa).\]

- (H5) Barrier cost for wedge exit: if $|w(t_*)|>\tfrac{\pi}{2}$ for some $t_*\in I$, then
\[\mathrm{Var}^{-}_{I}(w)\ \ge\ c_0(\psi).\]

\begin{theorem}[Certificate–to–Wedge]\label{thm:phase-certificate}
Let $F=2\mathcal J$ be analytic on $\Omega$, and let $I\subset\mathbf R$ be bounded. If \textup{(H1)}–\textup{(H5)} hold on $I$ and
\[\Upsilon\ :=\ \frac{C_H(\psi)\,M_\psi + C_P(\kappa)}{\,c_0(\psi)\,}\ \le\ \tfrac12,\]
then the boundary phase is confined to the wedge a.e. on $I$:
\[w(t)\ \in\ [-\tfrac{\pi}{2},\tfrac{\pi}{2}]\quad\text{for a.e. }t\in I.\]
\end{theorem}

\begin{proof}
If a wedge exit occurs, (H5) gives $\mathrm{Var}^{-}_{I}(w)\ge c_0(\psi)$. By (H4) and the definition of $\Upsilon$, $\mathrm{Var}^{-}_{I}(w)\le \Upsilon\,c_0(\psi)\le \tfrac12 c_0(\psi)$—a contradiction. Hence no exit occurs and the wedge holds a.e. on $I$.
\end{proof}

\begin{remark}[Local–to–global]
The theorem is local in $I$. If (H1)–(H5) and $\Upsilon\le \tfrac12$ hold uniformly on a covering of $\partial\Omega$ by overlapping intervals, the wedge holds a.e. on the whole boundary.
\end{remark}


% --- Upper bound on the tested phase drop ---
\clearpage
% Upper bound on the phase drop (tested form)
% Ready to \input into the manuscript; assumes theorem environments are defined.

\section*{Upper bound on the phase drop (tested form)}

Let $F=2\mathcal J$ be meromorphic of bounded type on $\mathbb{H}$ with boundary data $u=\log|F^*|$, $w=\arg(F^*)$. Fix a $C^1$ window $\psi$ of mean zero supported in $(-1,1)$ and define $\varphi_{L,t_0}(t)=L^{-1}\psi((t-t_0)/L)$ and $\Phi_{L,t_0}=\mathcal P[\varphi_{L,t_0}]$.

\begin{lemma}[Upper bound for the tested phase drop]\label{lem:upper-bound-tested}
For every $L>0$ and $t_0\in\mathbb{R}$,
\[\int_{\mathbb{R}} (-w'(t))\,\varphi_{L,t_0}(t)\,dt\ \le\ C_H(\psi)\,L\,M_\psi(u)\ +\ C_P(\kappa).\]
\end{lemma}

\begin{proof}
By the distributional phase–velocity identity, $\int(-w')\,\varphi_{L,t_0}=2\pi\int \Phi_{L,t_0}\,d\mu-\int \mathcal H[u']\,\varphi_{L,t_0}$. The $\mu$–term is nonnegative on boxes (or absorbed into $C_P$ if a finite prime main term is present), while
\[\left|\int \mathcal H[u']\,\varphi_{L,t_0}\right|=\left|\int u\,(\mathcal H[\varphi_{L,t_0}])'\right|\ \le\ \| (\mathcal H[\varphi_{L,t_0}])'\|_\infty\ \left|\int_I u\right|\ \le\ C_H(\psi)\,L\,M_\psi(u),\]
using $(\mathcal H[\varphi_{L,t_0}])'(t)=L^{-1}(\mathcal H[\varphi_{1,0}])'((t-t_0)/L)$ and the support of $\varphi_{L,t_0}$ in $I$.
\end{proof}


% --- Fixed–aperture normalizations ---
\clearpage
% Fixed–aperture normalizations and constants
% Ready to \input into the manuscript; assumes theorem environments are defined.

\section*{Fixed–aperture normalizations and constants}

\paragraph{Poisson and Hilbert transforms.}
We normalize the Poisson kernel and Hilbert transform on $\mathbb{R}$ by
\[P_y(x):=\frac{1}{\pi}\,\frac{y}{x^2+y^2},\qquad \mathcal H[f](t):=\frac{1}{\pi}\,\mathrm{p.v.}\int_{\mathbb{R}} \frac{f(s)}{t-s}\,ds.\]
For $\varphi\in L^1_{\rm loc}$ we write its Poisson extension $\Phi=\mathcal P[\varphi]$.

\paragraph{Cones and Carleson boxes (aperture $\alpha\ge1$).}
For an interval $I=(c-\tfrac{L}{2},c+\tfrac{L}{2})$ define
\[Q_\alpha(I):=\{x+iy:\ x\in \alpha I,\ 0<y\le \alpha L\},\qquad
\Gamma_\alpha(t):=\{(x,y): |x-t|<\alpha y,\ y>0\}.\]

\paragraph{Windows and scaling.}
Fix $\psi\in C^1_c(-1,1)$ with $\int\psi=0$. For scale $L>0$ and center $t_0$ set
\[\varphi_{L,t_0}(t):=\frac{1}{L}\,\psi\!\left(\frac{t-t_0}{L}\right),\qquad \Phi_{L,t_0}:=\mathcal P[\varphi_{L,t_0}].\]
Let $\Psi:=\mathcal P[\psi]$. Then the fixed–aperture scaling identities hold:
\[\iint_{\mathbb{H}} |\nabla \Phi_{L,t_0}|^2\,y\,dx\,dy\ =\ \frac{1}{L}\,\iint_{\mathbb{H}} |\nabla \Psi|^2\,\eta\,d\xi\,d\eta,\]
\[\int_{\mathbb{R}}\!\Big(\iint_{\Gamma_\alpha(t)} |\nabla \Phi_{L,t_0}|^2\,y\,dx\,dy\Big)^{1/2}\,dt\ =\ L\cdot \int_{\mathbb{R}}\!\Big(\iint_{\Gamma_\alpha(\tau)} |\nabla \Psi|^2\,\eta\,d\xi\,d\eta\Big)^{1/2}\,d\tau.\]

\paragraph{Window $H^1$ constant.}
Define
\[C_\psi^{(H^1)}\ :=\ \left(\frac{2}{\pi}\,\iint_{\mathbb{H}} |\nabla \Psi|^2\,\eta\,d\xi\,d\eta\right)^{1/2},\]
so that $\big(\iint |\nabla \Phi_{L,t_0}|^2\,y\big)^{1/2}\le \sqrt{\tfrac{\pi}{2}}\,L^{-1/2} C_\psi^{(H^1)}$.

\paragraph{Hilbert envelope.}
Set
\[C_H(\psi)\ :=\ \sup_{s\in\mathbb{R}}\,\left|(\mathcal H[\varphi_{1,0}])'(s)\right|,\]
so that $\|(\mathcal H[\varphi_{L,t_0}])'\|_\infty= C_H(\psi)/L$ by scaling.

\paragraph{Poisson plateau.}
For $I=(t_0-\tfrac{L}{2},t_0+\tfrac{L}{2})$ define
\[c_0(\psi):=\inf_{z\in Q_\alpha(I)} \Phi_{L,t_0}(z).\]
By scale invariance and continuity of $\Phi_{L,t_0}$, $c_0(\psi)>0$ depends only on $\psi$ and $\alpha$, and is independent of $L$ and $t_0$.

\paragraph{Carleson energy.}
For harmonic $U=\mathcal P[u]$, define
\[C_{\rm box}(U;\alpha):=\sup_I \frac{1}{|I|}\,\iint_{Q_\alpha(I)} |\nabla U|^2\,y\,dx\,dy.\]
Under Whitney scaling $|I|\le 1/(\Lambda\log\langle T\rangle)$, the constants above are independent of height $T$.


% --- Removability across zeros of xi via bounded Theta ---
\clearpage
% Removability across zeros of \xi via bounded \Theta and inverse Cayley
% Ready to \input into the manuscript; assumes theorem environments are defined.

\section*{Removability across $Z(\xi)$}

Let $\Omega=\{\Re s>1/2\}$ and set $F:=2\mathcal J$ and $\Theta:=\dfrac{F-1}{F+1}$. Suppose $\mathcal J$ is meromorphic on $\Omega$ and $\Theta$ is holomorphic on $\Omega\setminus Z$, where $Z\subset \Omega$ is a discrete set (typically $Z=Z(\xi)\cap\Omega$).

\begin{lemma}[Removable singularity for bounded $\Theta$]\label{lem:removable-Theta}
Let $z_0\in Z$ and assume $\Theta$ is holomorphic on $(\Omega\cap B(z_0,r))\setminus\{z_0\}$ and bounded there: $|\Theta|\le 1$. Then $\Theta$ extends holomorphically to $z_0$ with $|\Theta(z_0)|\le 1$.
\end{lemma}

\begin{proof}
By the Riemann removable singularity theorem, any bounded holomorphic function on a punctured neighborhood extends holomorphically at the puncture. The bound persists by continuity.
\end{proof}

\begin{lemma}[Excluding $\Theta(z_0)=1$ unless constant]\label{lem:no-one-Theta}
If $\Theta$ is holomorphic on a domain $D\subset\Omega$ and $|\Theta|\le1$ on $D$, then either $\Theta\equiv e^{i\theta}$ is unimodular constant on the component containing $z_0$, or else $|\Theta(z_0)|<1$. In particular, $\Theta(z_0)\ne1$ unless $\Theta\equiv1$ locally.
\end{lemma}

\begin{proof}
If $|\Theta|$ attains its supremum $1$ at an interior point, the maximum modulus principle forces $\Theta$ to be locally constant with $|\Theta|\equiv1$. Otherwise $|\Theta(z_0)|<1$.
\end{proof}

\begin{theorem}[Removability for $\mathcal J$ at $Z(\xi)$]\label{thm:rem-J-across}
Assume $\Theta$ is holomorphic on $\Omega\setminus Z$ with $|\Theta|\le 1$ there, and let $z_0\in Z$. Then $\Theta$ extends holomorphically to $z_0$ with $\Theta(z_0)\ne 1$. Consequently, $\mathcal J$ extends holomorphically to $z_0$ via the inverse Cayley transform
\[\mathcal J\ =\ \frac{1+\Theta}{2(1-\Theta)}.\]
In particular, $\mathcal J$ has no pole at $z_0$.
\end{theorem}

\begin{proof}
By Lemma~\ref{lem:removable-Theta}, $\Theta$ extends holomorphically to $z_0$ with $|\Theta(z_0)|\le1$. Lemma~\ref{lem:no-one-Theta} excludes $\Theta(z_0)=1$ unless $\Theta\equiv 1$ locally, which is impossible if $\mathcal J$ is nonconstant. Hence $1-\Theta(z_0)\ne0$, so the inverse Cayley formula defines a holomorphic extension of $\mathcal J$ at $z_0$.
\end{proof}


% --- Overlapping boxes: coverage and no–loss propagation ---
\clearpage
% Overlapping boxes: coverage and no-loss propagation
% Ready to \input into the manuscript; assumes theorem environments are defined.

\section*{Overlapping boxes: coverage and no–loss propagation}

\paragraph{Whitney cover of the boundary.}
There exists a countable family of intervals $\{I_k\}_{k\ge1}$ with
\[|I_k|\ \le\ \frac{1}{\Lambda\log\langle T_k\rangle}\qquad(\Lambda\ge1\ \text{fixed})\]
that covers $\mathbb{R}$ up to a null set and has uniformly bounded overlap. The associated Carleson boxes $Q_\alpha(I_k)$ cover a full–measure subset of the boundary line $\{\Re s=1/2\}$ and admit cones $\Gamma_\alpha(t)$ with $\Gamma_\alpha^{\le |I_k|}(t)\subset Q_\alpha(I_k)$.

\begin{lemma}[Boundary wedge on overlaps]\label{lem:wedge-overlap}
If $w\in[-\tfrac{\pi}{2},\tfrac{\pi}{2}]$ a.e. on each $I_k$, then $w\in[-\tfrac{\pi}{2},\tfrac{\pi}{2}]$ a.e. on $\bigcup_k I_k$.
\end{lemma}

\begin{proof}
Immediate by a.e. inclusion and countable union.
\end{proof}

\paragraph{Rectangles and no–loss union.}
Let $R_k:=\{\sigma+it:1/2<\sigma<1/2+L_k,\ |t-T_k|<H_k\}$ be rectangles with left side on $\{\Re s=1/2\}$ and chosen so that $Q_\alpha(I_k)\subset R_k$ and the ratio $H_k/L_k$ is large enough that the harmonic measure of the left side at interior points is arbitrarily close to $1$ (cf. Lemma~\ref{lem:hm-rect-unif}).

\begin{lemma}[No–loss on overlaps of rectangles]\label{lem:union-rectangles}
If $\Theta$ is holomorphic on $R_i\cup R_j$ and $|\Theta|\le 1$ on each $R_i$ and $R_j$, then $|\Theta|\le1$ on $R_i\cup R_j$.
\end{lemma}

\begin{proof}
Since $\Theta$ is the same holomorphic function on the overlap and bounded by $1$ on each set, the bound holds pointwise on the union.
\end{proof}

\begin{proposition}[Propagation across an overlapping cover]\label{prop:overlap-propagation}
Assume the height–uniform constants of Proposition~\ref{prop:uniform-unif}. Then there exists a family $\{R_k\}$ covering every compact $K\subset\Omega$ such that $|\Theta|\le1$ on each $R_k$ and therefore on $\bigcup_k R_k$. In particular, $|\Theta|\le1$ on $K$.
\end{proposition}

\begin{proof}
Choose $R_k$ with aspect ratio making the two–constants (pinch) Lemma~\ref{lem:two-const-unif} arbitrarily tight; boundary wedge on $I_k$ yields $|\Theta|\le 1$ on each $R_k$. Bounded overlap and Lemma~\ref{lem:union-rectangles} imply the bound on the union; exhaustion of $K$ by finitely many $R_k$ yields the claim.
\end{proof}


% --- Box energy: decomposition and aggregation ---
\clearpage
% Box energy: definition, unconditional decomposition, and aggregation
% Ready to \input into the manuscript; assumes theorem environments are defined.

\section*{Box energy: decomposition and aggregation}

\paragraph{Definition (fixed aperture).}
For a harmonic potential $U=\Re\log F$ with $F$ meromorphic of bounded type in $\Omega$, define the fixed–aperture box Carleson energy
\[C_{\rm box}(U;\alpha)\ :=\ \sup_{I}\ \frac{1}{|I|}\iint_{Q_\alpha(I)} |\nabla U(x,y)|^2\,y\,dx\,dy.\]
All statements below are for a fixed aperture $\alpha\ge 1$.

\paragraph{Unconditional decomposition.}
Write $U=U_{\det_2}-U_\xi+U_\Gamma$, corresponding to the log–potentials of $\det_2(I-A)$, $\xi$ and the archimedean/Gamma factor (with signs consistent with $\mathcal J$). Then for every interval $I$,
\[\frac{1}{|I|}\iint_{Q_\alpha(I)} |\nabla U|^2\,y\,\le\ \underbrace{\frac{1}{|I|}\iint_{Q_\alpha(I)} |\nabla U_{\det_2}|^2\,y}_{\le\ K_0}\ +\ \underbrace{\frac{1}{|I|}\iint_{Q_\alpha(I)} |\nabla U_\xi|^2\,y}_{\le\ K_\xi}\ +\ \underbrace{\frac{1}{|I|}\iint_{Q_\alpha(I)} |\nabla U_\Gamma|^2\,y}_{\le\ K_\Gamma}.\]
Consequently,
\begin{equation}\label{eq:cbox-split}
C_{\rm box}(U;\alpha)\ \le\ K_0\ +\ K_\xi\ +\ K_\Gamma.
\end{equation}

\paragraph{Arithmetic tail $K_0$.}
The $\det_2$–tail begins at prime powers $p^k$ with $k\ge2$; its box energy is finite and independent of $T$ under the fixed aperture, yielding a constant $K_0<\infty$ (with explicit enclosure given in the numeric appendix).

\paragraph{Archimedean term $K_\Gamma$.}
By Stirling/digamma bounds on fixed–aperture boxes, the $\Gamma$–term has bounded gradient and yields a finite, height–independent constant $K_\Gamma<\infty$.

\paragraph{Zeros ($\xi$–block) term $K_\xi$.}
With local Blaschke neutralization of zeros inside $Q_\alpha(I)$ and a cubic far–field kernel bound summed over dyadic horizontal annuli using only unconditional zero–counting, one obtains a uniform, height–independent constant $K_\xi<\infty$; see Theorem~\ref{thm:Kxi}.

\begin{theorem}[Aggregation of the box energy]\label{thm:aggregation-box}
For fixed $\alpha\ge1$, there exist absolute constants $K_0,K_\xi,K_\Gamma<\infty$ such that for every interval $I$,
\[\frac{1}{|I|}\iint_{Q_\alpha(I)} |\nabla U|^2\,y\,\le\ K_0+K_\xi+K_\Gamma.\]
Equivalently, $C_{\rm box}(U;\alpha)\le K_0+K_\xi+K_\Gamma$.
\end{theorem}

\begin{remark}[Uniformity under Whitney scaling]
Under the scale law $|I|\le 1/(\Lambda\log\langle T\rangle)$, each of $K_0,K_\xi,K_\Gamma$ is independent of height $T$; hence the bound \eqref{eq:cbox-split} is height–uniform.
\end{remark}


% --- Unconditional ξ–block bound on Carleson boxes ---
\clearpage
% Unconditional bound for the ξ–block on Carleson boxes
% Ready to \input into the manuscript; assumes theorem environments are defined.

\section*{Unconditional ξ–block bound on Carleson boxes}

\paragraph{Domain, boxes, and the ξ–block.}
Set the upper half–plane $\mathbb{H}:=\{z=x+iy:y>0\}$ with boundary $\mathbb{R}$. Fix a bounded interval $I=(c-\tfrac{L}{2},c+\tfrac{L}{2})\subset\mathbb{R}$, $L>0$, and its Carleson box
\[
Q(I):=\{x+iy: x\in I,\ 0<y\le L\},\qquad Q(\alpha I):=\{x+iy: x\in \alpha I,\ 0<y\le \alpha L\},\ \alpha>1.
\]
We study the contribution of the Riemann $\xi$–function to the weighted box energy
\[
\mathcal{E}_{\xi}(I):=\iint_{Q(I)} |\nabla U_{\xi}(z)|^2\,y\,dx\,dy,
\]
where $U_{\xi}=\Re\log \Xi$ is the (fixed–branch) potential of the $\xi$–block after transfer to $\mathbb{H}$; write $s=\tfrac{1}{2}-i z$ and set $\Xi(z):=\xi(s)$ so the critical line maps to $\partial\mathbb{H}$. Let $\mathcal{Z}_+:=\{a_j=x_j+i y_j\in\mathbb{H}\}$ be the zeros of $\Xi$ with multiplicities $m_j\in\mathbb{N}$. In $\mathbb{H}$,
\[
\log |\Xi(z)| = H(z) + \sum_{j} m_j\,\Phi_{a_j}(z),\qquad
\Phi_{a}(z):=\log\left|\frac{z-a}{z-\overline{a}}\right|,
\]
where $H$ is harmonic (archimedean and outer part). Accordingly,
\[
U_{\xi}(z)=H(z)+\sum_j m_j \,\Phi_{a_j}(z),\qquad 
\nabla\Phi_{a}(z)=\frac{z-a}{|z-a|^2}-\frac{z-\overline{a}}{|z-\overline{a}|^2}.
\]
We bound the discrete “$\xi$–block” $\sum_j m_j\Phi_{a_j}$; the smooth part $H$ is handled elsewhere and does not enter $K_{\xi}$.

\paragraph{Neutralization.}
Given $\alpha>1$, define the local Blaschke neutralizer
\[
B_I(z):=\prod_{a_j\in Q(\alpha I)}\left(\frac{z-a_j}{z-\overline{a_j}}\right)^{m_j},\qquad |B_I(t)|=1\ \text{ for a.e. } t\in\mathbb{R}.
\]
Set $\widetilde{\Xi}:=\Xi/B_I$ and $\widetilde{U}_{\xi}:=\Re\log \widetilde{\Xi} = H+\sum_{a_j\notin Q(\alpha I)} m_j \Phi_{a_j}$. Then $\widetilde{U}_\xi$ is harmonic on $Q(\alpha I)$, and
\[
\mathcal{E}_{\xi}(I)=\iint_{Q(I)} |\nabla \widetilde{U}_{\xi}|^2\,y\,dx\,dy\;+\;\underbrace{\iint_{Q(I)} |\nabla \Re\log B_I|^2\,y\,dx\,dy}_{=:\ \mathcal{E}_{B}(I)}.
\]
We bound the \emph{far–field} term $\mathcal{E}^{\mathrm{far}}_{\xi}(I):=\iint_{Q(I)} |\nabla \widetilde{U}_{\xi}|^2 y\,dx\,dy$ by a universal constant after normalization by $|I|$, and show the \emph{near–field} neutralizer term $\mathcal{E}_B(I)$ is a local sum that shifts harmlessly when boxes move.

\section*{Pointwise kernel, cubic decay, and a per–zero bound}
Let $a=x_0+i y_0\in\mathbb{H}$, $y_0>0$. Introduce $A:= (x-x_0)^2+(y-y_0)^2$ and $B:=(x-x_0)^2+(y+y_0)^2$. Differentiating $\Phi_a$ gives
\begin{align*}
\partial_x \Phi_a(z)&=\frac{4 y y_0 (x-x_0)}{A\,B},\\
\partial_y \Phi_a(z)&=\frac{2 y_0\big(y^2-(x-x_0)^2-y_0^2\big)}{A\,B}.
\end{align*}
Hence
\[
|\nabla \Phi_a(z)|^2
:=\frac{4y_0^2\big(4y^2(x-x_0)^2+\big(y^2-(x-x_0)^2-y_0^2\big)^2\big)}{A^2 B^2}.
\]

\begin{lemma}[Cubic far–field bound]\label{lem:cubic}
Fix $\alpha>1$. Let $I=(c-L/2,c+L/2)$ and $a\notin Q(\alpha I)$. Define $D:=\mathrm{dist}(x_0,I)$ and $R:=\sqrt{D^2+y_0^2}$. Then
\[
\mathcal{E}(a;I):=\iint_{Q(I)} |\nabla \Phi_a(z)|^2\,y\,dx\,dy\ \le\ \frac{64}{\alpha^2}\,\frac{y_0\,L^2}{(D+y_0)^3}\ \le\ 64\,\frac{y_0\,L^2}{R^3}.
\]
\end{lemma}

\begin{proof}
Since $a\notin Q(\alpha I)$, for $x\in I$ and $0<y\le L$ one has $|x-x_0|\ge D\ge \tfrac{\alpha-1}{2}L$ and $y\le \tfrac{2}{\alpha-1}D$. Estimating $A\,B\ge (x-x_0)^2 (y_0+y)^2$ and bounding the numerator by $(u^2+v^2)^2\le 2(u^4+v^4)$ yields $|\nabla\Phi_a|^2\le \dfrac{64y_0^2}{(x-x_0)^2(y_0+y)^4}$. Integrate in $x$ and $y$ explicitly to obtain the stated bound.
\end{proof}

\paragraph{Annular summation and zero–counting only.}
For $k\ge 0$ define dyadic horizontal annuli around $I$ by
\[
\mathcal{A}_k(I):=\Big\{x+iy\in\mathbb{H}:\ 2^k\tfrac{L}{2}\le |x-c|<2^{k+1}\tfrac{L}{2},\ 0<y\le 1\Big\}.
\]
Partition $\mathcal{Z}_+\setminus Q(\alpha I)$ into $\mathcal{Z}_k:=\mathcal{Z}_+\cap \mathcal{A}_k(I)$; for $a_j\in\mathcal{Z}_k$ one has $D\asymp 2^k L$. Summing Lemma~\ref{lem:cubic} and using $y_0\le \tfrac{1}{2}$,
\[
\frac{1}{|I|}\,\mathcal{E}^{\mathrm{far}}_{\xi}(I)\ \le\ \frac{64}{\alpha^2}\,\frac{1}{L^2}\sum_{k\ge0}\frac{\#\mathcal{Z}_k}{2^{3k}}.
\]

\begin{lemma}[Zero–counting in horizontal windows]\label{lem:zc}
There exist $A_0,A_1\ge 1$ such that for all $T\ge 3$ and $W\in(0,T]$,
\[
\#\{\rho=\beta+i\gamma: 0<\gamma\le T+W\}-\#\{\rho: 0<\gamma\le T-W\}\ \le\ A_0\,W\log T\ +\ A_1\log T.
\]
Equivalently, for $\Xi$ in $\mathbb{H}$,
\[
\#\big(\mathcal{Z}_+\cap\{x+iy: |x-T|<W\}\big)\ \le\ A_0\,W\log T + A_1\log T.
\]
\end{lemma}

Applying Lemma~\ref{lem:zc} on the horizontal window of width $2^{k}L$ centered at height $T=|c|+1$ gives
\[
\#\mathcal{Z}_k\ \le\ C_0\,(2^k L)\,\log\langle T\rangle\ +\ C_1\log\langle T\rangle,\qquad \langle T\rangle:=2+|T|.
\]
Hence
\[
\frac{1}{|I|}\,\mathcal{E}^{\mathrm{far}}_{\xi}(I)
\ \le\ \frac{64}{\alpha^2}\,\frac{\log\langle T\rangle}{L^2}\left(C_0 \sum_{k\ge0}\frac{1}{2^{2k}} + C_1\sum_{k\ge0}\frac{1}{2^{3k}}\right)
\ =:\ C^\star\,\frac{\log\langle T\rangle}{L^2}.
\]

\paragraph{Scale law and unconditional uniformity.}
With Whitney scaling $L\le 1/(\Lambda\log\langle T\rangle)$ (fixed $\Lambda\ge 1$),
\[
\frac{1}{|I|}\,\mathcal{E}^{\mathrm{far}}_{\xi}(I)\ \le\ C^\star\,\Lambda^2\ =:\ K_\xi^{\mathrm{far}}.
\]

\section*{The neutralizer term $\mathcal{E}_{B}(I)$}
By construction, $B_I$ includes exactly the zeros of $\Xi$ inside $Q(\alpha I)$. Set $U_B:=\Re\log B_I=\sum_{a_j\in Q(\alpha I)} m_j\,\Phi_{a_j}$. Its energy on $Q(I)$ is
\[
\mathcal{E}_B(I)\ =\ \sum_{a_j\in Q(\alpha I)} m_j\,\iint_{Q(I)} |\nabla\Phi_{a_j}(z)|^2 y\,dx\,dy.
\]
Each summand is nonnegative and local. As boxes move, contributions migrate between $\mathcal{E}_B$ and $\mathcal{E}^{\mathrm{far}}_{\xi}$ and remain controlled by Lemma~\ref{lem:cubic}. Taking a supremum over all boxes does not increase beyond the far–field bound. Thus we may absorb $\mathcal{E}_B$ and put $K_\xi:=K_\xi^{\mathrm{far}}$.

\section*{Conclusion}
\begin{theorem}[Unconditional $\xi$–block bound]\label{thm:Kxi}
Fix $\alpha\ge 2$ and a scale parameter $\Lambda\ge 1$. For every Carleson box $Q(I)$ at height $T$ with length $L\le 1/(\Lambda\log\langle T\rangle)$,
\[
\frac{1}{|I|}\,\iint_{Q(\alpha I)} |\nabla U_{\xi}(z)|^2\,y\,dx\,dy\ \le\ K_\xi,
\qquad K_\xi:=\frac{64}{\alpha^2}\left( \frac{C_0}{1-2^{-2}}+\frac{C_1}{1-2^{-3}}\right)\Lambda^2,
\]
where $C_0,C_1$ are the absolute constants from Lemma~\ref{lem:zc}. The proof is unconditional, using only the local neutralizer $B_I$, the explicit kernel, cubic far–field decay, and classical zero–counting.
\end{theorem}

\begin{remark}[Sharper variants]
(i) A slightly more delicate estimate yields quartic decay $R^{-4}$ pointwise; cubic suffices and is simpler. (ii) Optimizing $\alpha$ and the $y$–integral can lower constants without affecting unconditionality.
\end{remark}


% --- All heights: height–uniform constants and globalization ---
\clearpage
% Height–uniform results (all heights): constants and globalization
% Ready to \input into the manuscript; assumes theorem environments exist.

\section*{Uniformity in height and globalization}

\begin{proposition}[Height–uniform constants]\label{prop:height-uniform}
Fix $\alpha\ge 1$ and $\Lambda\ge 1$.
Let $U=\Re\log F$ for the field $F$ used in the PSC analysis (with local neutralization in each Whitney box).
Then there exist absolute constants $K_0,K_\xi,K_\Gamma<\infty$ such that for every box $Q_\alpha(I)$ with $|I|\le 1/(\Lambda\log\langle T\rangle)$ located at height $T$,
\[
  \frac{1}{|I|}\iint_{Q_\alpha(I)} |\nabla U|^2\,y\,dx\,dy\ \le\ K_0+K_\xi+K_\Gamma\ =:\ K_{\rm box},
\]
with all constants independent of $T$.
Moreover, the window constants from the fixed–aperture setup satisfy
\[
  c_0(\psi)>0,\qquad C_H(\psi)<\infty,\qquad C_\psi^{(H^1)}<\infty,
\]
depending only on $\psi$ and $\alpha$ (and not on $T$).
\end{proposition}

\begin{theorem}[Certificate on every Whitney box]\label{thm:cw-allheights}
Let $I$ be any Whitney interval at height $T$ with $|I|\le 1/(\Lambda\log\langle T\rangle)$.
Assume the PSC budgets are taken from the height–uniform bounds in Proposition~\ref{prop:height-uniform} and the fixed–aperture normalizations.
Then the boundary phase of $F$ satisfies the a.e. wedge on $I$:
\[
  w_F(t)\in\bigl[-\tfrac{\pi}{2},\tfrac{\pi}{2}\bigr]\quad\text{for a.e. }t\in I,
\]
equivalently $\Re F(\tfrac12+it)\ge 0$ a.e. on $I$.
\end{theorem}

\begin{theorem}[Global Schur bound and removability]\label{thm:schur-global}
Let $\Theta=\mathrm{Cayley}(2\mathcal J-1)$ be the Cayley transform of the Herglotz field $2\mathcal J$.
If Theorem~\ref{thm:cw-allheights} holds on every Whitney interval, then $|\Theta(s)|\le 1$ for all $s\in\Omega$.
Consequently, $\Theta$ is analytic and bounded on $\Omega$, and every would–be singularity of $\mathcal J$ in $\Omega$ is removable.
\end{theorem}




% --- Fixed–aperture H^1–BMO / Carleson embedding and M_psi bound ---
\clearpage
% Fixed–aperture H^1–BMO / Carleson embedding and the bound for M_\psi
% Ready to \input into the manuscript; assumes theorem environments are defined.

\section*{Fixed–aperture $H^1$–BMO / Carleson embedding and $M_\psi$}

\paragraph{Geometry and notation.}
Work in the upper half–plane $\mathbb{H}=\{x+iy:y>0\}$ with boundary $\mathbb{R}$. For any interval $I=(c-\tfrac{L}{2},c+\tfrac{L}{2})$ of length $L>0$ and aperture $\alpha\ge 1$, set
\[
Q_\alpha(I):=\{x+iy:\ x\in \alpha I,\ 0<y\le \alpha L\},\qquad
\Gamma_\alpha(t):=\{(x,y): |x-t|<\alpha y,\ y>0\}.
\]
Given $f\in L^1_{\rm loc}(\mathbb{R})$, its Poisson extension is $\mathcal{P}[f](x,y)=\tfrac{1}{\pi}\int \dfrac{y f(t)}{(x-t)^2+y^2}\,dt$. Write $U=\mathcal{P}[u]$ and $\Phi=\mathcal{P}[\varphi]$ for the harmonic extensions of $u$ and $\varphi$.

\paragraph{Box energy and window family.}
For harmonic $U$ define the fixed–aperture box Carleson energy
\[
C_{\rm box}(U;\alpha)\ :=\ \sup_{I}\ \frac{1}{|I|}\iint_{Q_\alpha(I)} |\nabla U(x,y)|^2\,y\,dx\,dy.
\]
Fix a $C^1$ compactly supported window $\psi$ on $(-1,1)$ with zero mean $\int\psi=0$. For scale $L>0$ and center $t_0\in\mathbb{R}$ set
\[
\varphi_{L,t_0}(t):=\frac{1}{L}\,\psi\!\left(\frac{t-t_0}{L}\right),\qquad
\Phi_{L,t_0}:=\mathcal{P}[\varphi_{L,t_0}].
\]
Define the scale–free window constant
\[
C_\psi^{(H^1)}\ :=\ \left(\frac{2}{\pi}\iint_{\mathbb{H}} |\nabla \mathcal{P}[\psi](\xi,\eta)|^2\,\eta\,d\xi\,d\eta\right)^{1/2}.
\]
Finally, define the pairing functional
\[
M_\psi(u)\ :=\ \sup_{L>0,\ t_0\in\mathbb{R}}\ \frac{1}{L}\,\left|\int_{\mathbb{R}} u(t)\,\varphi_{L,t_0}(t)\,dt\right|.
\]

\subsection*{Analytic identities and normalizations}

\begin{lemma}[Green–Poisson pairing]\label{lem:green-embedding}
If $U=\mathcal{P}[u]$ and $\Phi=\mathcal{P}[\varphi]$ with $\varphi\in C_c^\infty(\mathbb{R})$, then
\[
\int_{\mathbb{R}} u(t)\,\varphi(t)\,dt\ =\ \frac{2}{\pi}\iint_{\mathbb{H}} \nabla U(x,y)\cdot \nabla \Phi(x,y)\,y\,dx\,dy.
\]
The right–hand side is absolutely convergent if $\int \varphi=0$.
\end{lemma}

\begin{lemma}[Scaling for the window family]\label{lem:scaling-embedding}
Let $\Psi=\mathcal{P}[\psi]$ and $\Phi_{L,t_0}=\mathcal{P}[\varphi_{L,t_0}]$. Then
\[
\iint_{\mathbb{H}} |\nabla \Phi_{L,t_0}|^2\,y\,dx\,dy\ =\ \frac{1}{L}\,\iint_{\mathbb{H}} |\nabla \Psi|^2\,\eta\,d\xi\,d\eta,
\]
\[
\int_{\mathbb{R}}\!\Big(\iint_{\Gamma_\alpha(t)} |\nabla \Phi_{L,t_0}|^2\,y\,dx\,dy\Big)^{1/2}\,dt\ =\ L\cdot \int_{\mathbb{R}}\!\Big(\iint_{\Gamma_\alpha(\tau)} |\nabla \Psi|^2\,\eta\,d\xi\,d\eta\Big)^{1/2}\,d\tau.
\]
\end{lemma}

\subsection*{Embedding with constants}

\begin{lemma}[Local Carleson domination of cones]\label{lem:cone-box-embedding}
For any $t\in\mathbb{R}$ and $L>0$, the truncated cone $\Gamma_\alpha^{\le L}(t):=\{(x,y)\in\Gamma_\alpha(t): 0<y\le L\}$ satisfies
\[
\iint_{\Gamma_\alpha^{\le L}(t)}|\nabla U|^2\,y\,dx\,dy\ \le\ \iint_{Q_\alpha(I_t)}|\nabla U|^2\,y\,dx\,dy\ \le\ C_{\rm box}\,L,
\]
with $I_t=(t-\tfrac{L}{2},t+\tfrac{L}{2})$.
\end{lemma}

\begin{lemma}[Dyadic annular energy of the window]\label{lem:dyadic-tail-embedding}
There exists $C_0>0$ such that for every $L>0$ and $I=(t_0-\tfrac{L}{2},t_0+\tfrac{L}{2})$,
\[
\iint_{Q_\alpha(2^k I)\setminus Q_\alpha(2^{k-1} I)} |\nabla \Phi_{L,t_0}|^2\,y\,dx\,dy\ \le\ C_0\,2^{-2k}\cdot \frac{1}{L}\,\iint_{\mathbb{H}}|\nabla \Psi|^2\,\eta\,d\xi\,d\eta\quad (k\ge 1).
\]
\end{lemma}

\begin{theorem}[Fixed–aperture embedding]\label{thm:embedding-embedding}
Let $U=\mathcal{P}[u]$ with $C_{\rm box}(U;\alpha)<\infty$. Then for every $L>0$ and $t_0\in\mathbb{R}$,
\[
\left|\int_{\mathbb{R}} u(t)\,\varphi_{L,t_0}(t)\,dt\right|
\ \le\ \frac{2}{\pi}\,\sqrt{C_{\rm box}}\,\left(
\iint_{Q_\alpha(2I)} |\nabla \Phi_{L,t_0}|^2\,y
+\sum_{k=1}^{\infty}\iint_{Q_\alpha(2^{k+1}I)\setminus Q_\alpha(2^{k}I)} |\nabla \Phi_{L,t_0}|^2\,y
\right)^{\!1/2}
\cdot\left( \sqrt{2\alpha L}\ +\ \sum_{k=1}^{\infty}\sqrt{2^{k+1}L}\,2^{-k}\right),
\]
where $I=(t_0-\tfrac{L}{2},t_0+\tfrac{L}{2})$. In particular,
\[
\left|\int u\,\varphi_{L,t_0}\right|
\ \le\ \frac{2}{\pi}\,\sqrt{C_{\rm box}}\,
\left(\iint_{\mathbb{H}} |\nabla \Phi_{L,t_0}|^2\,y\right)^{1/2}\cdot C_1(\alpha).
\]
\end{theorem}

\begin{lemma}[Window $H^1$ normalization]\label{lem:windowH1-embedding}
With $C_\psi^{(H^1)}$ as defined and Lemma \ref{lem:scaling-embedding},
\[
\left(\iint_{\mathbb{H}} |\nabla \Phi_{L,t_0}|^2\,y\right)^{1/2}\ \le\ \sqrt{\tfrac{\pi}{2}}\,\frac{1}{\sqrt{L}}\;C_\psi^{(H^1)}.
\]
\end{lemma}

\subsection*{Main estimate for $M_\psi$}

\begin{theorem}[Bound for $M_\psi$]\label{thm:Mpsi-embedding}
Let $u\in L^1_{\rm loc}(\mathbb{R})$ with Poisson extension $U$ and $C_{\rm box}(U;\alpha)<\infty$ for fixed $\alpha\ge1$. Then
\[
M_\psi(u)\ \le\ \frac{4}{\pi}\,C_\psi^{(H^1)}\,\sqrt{C_{\rm box}(U;\alpha)}.
\]
\end{theorem}

\begin{remark}
1) $C_\psi^{(H^1)}$ depends only on the fixed window $\psi$ and aperture $\alpha$; equivalent fixed–aperture Lusin–area definitions agree by scaling.  
2) The proof is unconditional: it uses only Green–Poisson pairing, explicit Poisson kernel bounds, a dyadic annular decomposition, and the box Carleson energy $C_{\rm box}$—no assumptions on zeros or primes.  
3) The factor $\tfrac{4}{\pi}$ arises from the pairing constant $\tfrac{2}{\pi}$ and the $\,\le 2\,$ geometric factor in the embedding step.
\end{remark}


% --- Uniformity in height T and pinch/exhaustion ---
\clearpage
% Uniformity in height and pinch/exhaustion to \Omega
% Ready to \input into the manuscript; assumes theorem environments are defined.

\section*{Uniformity in height and pinch/exhaustion to $\Omega$}

\paragraph{Domain and objects.}
Let $\Omega=\{s=\sigma+it:\sigma>1/2\}$. Set $F:=2\mathcal J$ and $\Theta:=\dfrac{F-1}{F+1}$ so that $\Re F\ge 0$ is equivalent to $|\Theta|\le 1$. Work via the upper half–plane model $z=x+iy$ after $z=-i(s-\tfrac12)$. For an interval $I$ and aperture $\alpha\ge1$ denote $Q_\alpha(I)=\{x+iy:\ x\in \alpha I,\ 0<y\le \alpha|I|\}$. The box–energy is
\[
C_{\rm box}(U;\alpha):=\sup_I \frac{1}{|I|}\iint_{Q_\alpha(I)} |\nabla U|^2\,y\,dx\,dy,\qquad U=\Re\log F.
\]

\subsection*{Height–uniform constants under Whitney scaling}

For $T\in\mathbb{R}$ write $\langle T\rangle:=2+|T|$. Boxes centered at height $T$ are taken at Whitney scale
\[|I|=L(T)\ \le\ \frac{1}{\Lambda\,\log\langle T\rangle}\qquad(\Lambda\ge 1\ \text{fixed}).\]

\begin{proposition}[Height–uniform constants]\label{prop:uniform-unif}
Fix $\alpha\ge 1$ and $\Lambda\ge 1$. Then the following hold with constants independent of $T$:
\begin{itemize}
\item[(i)] (Window/geometry) $c_0(\psi)>0$, $C_H(\psi)<\infty$, $C_\psi^{(H^1)}<\infty$ depend only on $\psi,\alpha$.
\item[(ii)] (Box energy) With the neutralization described in the $\xi$–block analysis, there exist absolute $K_\xi, K_0, K_\Gamma<\infty$ such that
\[C_{\rm box}(U;\alpha)\ \le\ K_\xi+K_0+K_\Gamma\ =:\ K_{\rm box},\]
uniformly for all boxes $Q_\alpha(I)$ with $|I|\le 1/(\Lambda\log\langle T\rangle)$.
\item[(iii)] (Certificate) For every such box one has the a.e. boundary wedge on its base interval $I$:
\[w_F(t)\in[-\tfrac{\pi}{2},\tfrac{\pi}{2}]\quad\text{for a.e. }t\in I,\]
hence $\Re F(1/2+it)\ge 0$ for a.e. $t\in I$, with constants (and the inequality $\Upsilon\le \tfrac12$) independent of $T$.
\end{itemize}
\end{proposition}

\begin{proof}
(i) is purely geometric. (ii) follows from the unconditional far–field $\xi$–block bound (cubic decay + horizontal annuli) combined with the prime–power tail and the archimedean part; each piece is independent of $T$ under the scale law. (iii) follows from the certificate $\Rightarrow$ wedge via the fixed–aperture embedding $M_\psi\le \tfrac{4}{\pi}C_\psi^{(H^1)}\sqrt{C_{\rm box}}$ and the plateau $c_0(\psi)$.
\end{proof}

\begin{lemma}[Global boundary wedge (a.e.)]\label{lem:globalwedge-unif}
There exists a countable cover of $\mathbb{R}$ by intervals $\{I_k\}_{k\ge 1}$ with $|I_k|\le 1/(\Lambda\log\langle T_k\rangle)$ such that
\[w_F(t)\in[-\tfrac{\pi}{2},\tfrac{\pi}{2}]\quad\text{for a.e. }t\in \bigcup_k I_k.\]
Consequently $w_F(t)\in[-\tfrac{\pi}{2},\tfrac{\pi}{2}]$ for a.e. $t\in\mathbb{R}$.
\end{lemma}

\begin{proof}
Tile $\mathbb{R}$ by disjoint base intervals at Whitney scale in $|t|$; the union has full measure, and each $I_k$ satisfies the certificate hypotheses by Proposition~\ref{prop:uniform-unif}(ii), giving the wedge on $I_k$. Countable union preserves “a.e.”
\end{proof}

\subsection*{Pinch in rectangles: harmonic measure and two–constants}

Let $R=R(L,H,T)$ denote the axis–parallel rectangle $R:=\{\sigma+it:\ 1/2<\sigma<1/2+L,\ |t-T|<H\}$, $L,H>0$. Write $E_L$ for its left side $\{\sigma=1/2,\ |t-T|<H\}$. Let $u:=\log|\Theta|$ (subharmonic on $\Omega$ since $\Theta$ is holomorphic in $\Omega$). For $z\in R$ let $\omega_R(z,\cdot)$ denote harmonic measure on $\partial R$.

\begin{lemma}[Harmonic measure of the left side]\label{lem:hm-rect-unif}
For $R=R(L,H,T)$ and $z=\sigma+it\in R$,
\[\omega_R\big(z,E_L\big)\ \ge\ 1-\exp\!\Big(-\frac{\pi(\sigma-1/2)}{2H}\Big).\]
In particular, on the vertical midline $\sigma=1/2+\theta L$ with $0<\theta<1$,
\[\omega_R\big(z,E_L\big)\ \ge\ 1-\exp\!\Big(-\frac{\pi\theta L}{2H}\Big).\]
\end{lemma}

\begin{lemma}[Two–constants inequality (pinch form)]\label{lem:two-const-unif}
Suppose $u=\log|\Theta|$ is subharmonic on a neighborhood of $\overline R$ and $u^*\le 0$ a.e. on $E_L$. Let $M_R:=\sup_{\partial R\setminus E_L} u^+<\infty$. Then for every $z\in R$,
\[u(z)\ \le\ (1-\omega_R(z,E_L))\,M_R,\]
hence $|\Theta(z)|\le \exp\!\big((1-\omega_R(z,E_L))\,M_R\big)$.
\end{lemma}

\begin{lemma}[Analyticity of $\Theta$ on $\Omega$]\label{lem:Theta-hol-unif}
$\Theta$ extends holomorphically to all of $\Omega$; at a pole of $\mathcal J$ (zero of $\xi$) one has $\Theta\to 1$, hence the singularity is removable.
\end{lemma}

\begin{theorem}[Pinch/exhaustion to compacts]\label{thm:pinch-unif}
For every compact $K\subset\Omega$, $|\Theta(s)|\le 1$ for all $s\in K$.
\end{theorem}

\begin{proof}
Fix $K$ and rectangles $R_n:=R(L_n,H_n,0)$ with $L_n:=d_K/n$ and $H_n:=n H_K$ so that $K\subset R_n$ for $n\gg1$. Lemmas~\ref{lem:globalwedge-unif}, \ref{lem:two-const-unif}, \ref{lem:Theta-hol-unif} give
\[\log|\Theta(z)|\ \le\ (1-\omega_{R_n}(z,E_L^{(n)}))\,M_{R_n}.\]
The harmonic–measure deficit tends to 0 uniformly on $K$ (Lemma~\ref{lem:hm-rect-unif}); with $M_{R_n}<\infty$ fixed for each $n$, the bound tends to 0, proving $|\Theta|\le 1$ on $K$.
\end{proof}

\subsection*{Normal families and global consequence}

\begin{proposition}[Compactness formulation]\label{prop:compact-unif}
Let $\{R_n\}$ be as above and, for $\delta>0$, set $\Theta_{n,\delta}(s):=\Theta(s)\,\chi_{n,\delta}(s)$ with a holomorphic cutoff $\chi_{n,\delta}$ on a neighborhood of $\overline{R_n}$ such that $|\chi_{n,\delta}|\le 1$, $\chi_{n,\delta}\equiv 1$ on $R_{n-1}$, and $\sup_{\partial R_n\setminus E_L^{(n)}}\log|\Theta_{n,\delta}|\le \delta$. Then $\{\Theta_{n,\delta}\}_n$ is normal on $K$ and
\[\limsup_{n\to\infty}\ \sup_{z\in K}|\Theta_{n,\delta}(z)|\ \le\ \exp\!\big(\sup_{z\in K}(1-\omega_{R_n}(z,E_L^{(n)}))\,\delta\big)\ \to\ 1.\]
Letting $\delta\downarrow 0$ and using local uniform convergence gives $\sup_K|\Theta|\le 1$.
\end{proposition}

\begin{theorem}[Global Schur bound and removability]\label{thm:global-unif}
For all $s\in\Omega$, $|\Theta(s)|\le 1$ and $\Re F(s)\ge 0$. Consequently every would–be singularity of $\mathcal J$ in $\Omega$ is removable, so $\mathcal J$ has no pole in $\Omega$.
\end{theorem}


% --- Proof Map (concise) ---
% Included for reader orientation; does not affect proofs.
\clearpage
% Proof Map (Bounded-Real Route via Product Certificate)
% Ready to \input into the manuscript; no preamble/macros required.

\section*{Proof Map (Bounded-Real Route via Product Certificate)}

\paragraph{Ambient domain.} Let $\Omega:=\{s\in\mathbb C:\Re s>\tfrac12\}$. Write $s=\tfrac12+\sigma+it$, $\sigma>0$. Let $\xi(s)=\tfrac12 s(1-s)\,\pi^{-s/2}\,\Gamma(\tfrac{s}{2})\,\zeta(s)$.

\paragraph{Objects.} On $\ell^2(\mathcal P)$ let $A(s)e_p:=p^{-s}e_p$. Define the outer-normalized ratio
\[
  \mathcal J(s)\ :=\ \frac{\det\nolimits_2(I-A(s))}{\mathcal O(s)\,\xi(s)},
  \qquad
  \Theta(s)\ :=\ \frac{2\mathcal J(s)-1}{2\mathcal J(s)+1}.
\]
Target: $2\mathcal J$ is Herglotz on $\Omega$ (i.e. $\Re(2\mathcal J)\ge 0$), equivalently $\Theta$ is Schur on $\Omega$.

\paragraph{Window and geometry.} Fix an even $\psi\in C^\infty$ with $\psi\equiv1$ on $[-1,1]$, $0\le\psi\le1$, compact support. For an interval $I=[T-L,T+L]$ with $L\le 1/\log\langle T\rangle$ and aperture $\alpha\in[1,2]$, the Carleson box is
\[
  Q(\alpha I):=\{(\sigma,t): |t-T|\le \alpha L,\ 0<\sigma\le \alpha|I|\},\qquad
  \Omega(I):=\{\tfrac12+\sigma+it: (\sigma,t)\in Q(\alpha I)\}\subset\Omega.
\]

\paragraph{Boundary phase density.} With boundary phase $w(t)$ one has on $I$
\[
  \int_I (-w'(t))\,dt\ \le\ \pi\,\mu\big(Q(I)\big),
\]
where $\mu$ is the Carleson/energy measure induced by the neutralized potential.

\paragraph{Locked certificate to (P+).} Independent inputs:
\[
  c_0(\psi),\quad C_H(\psi),\quad C_P(\kappa),\quad C^{(H^1)}_\psi,\quad C_{\mathrm{box}}.
\]
The fixed-aperture embedding gives
\[
  M_\psi\ \le\ \frac{4}{\pi}\,C^{(H^1)}_\psi\sqrt{C_{\mathrm{box}}}.
\]
Hence the boundary certificate
\[
  \Upsilon\ :=\ \frac{C_H(\psi)\,M_\psi + C_P(\kappa)}{c_0(\psi)}\ <\ \frac{\pi}{2},
\]
which forces $w\in[-\tfrac{\pi}{2},\tfrac{\pi}{2}]$ a.e. on the base (condition (P+)).

\paragraph{Poisson and Cayley.} From (P+) and the Poisson integral, $\Re(2\mathcal J)\ge 0$ on $\Omega(I)$, so $2\mathcal J$ is Herglotz. Therefore the Cayley transform is Schur: $|\Theta|\le1$ on $\Omega(I)$.

\paragraph{Removable singularities at $Z(\xi)$.} If $\rho\in Z(\xi)$, boundedness of $\Theta$ on a punctured disc implies $\Theta$ (and hence $\mathcal J$) extends holomorphically across $\rho$.

\paragraph{Globalization/pinch.} Exhaust $\Omega$ by overlapping boxes and use the maximum-modulus pinch to propagate $|\Theta|\le1$ to the interior and exclude zeros of $\xi$ inside $\Omega$. By the functional equation, all nontrivial zeros lie on $\Re s=\tfrac12$; thus RH.

\paragraph{Independence (no circularity).}
\begin{itemize}
  \item $C_{\mathrm{box}}=K_0+K_\xi+\|U_\Gamma\|_{\mathrm{area}}$ with (i) prime tails $K_0$, (ii) $\xi$-block $K_\xi$ (neutralized far-field + near-field), (iii) Archimedean $\Gamma$-term; each bounded independently.
  \item $M_\psi\le \tfrac{4}{\pi}C_\psi^{(H^1)}\sqrt{C_{\mathrm{box}}}$ via fixed-aperture $H^1$–BMO/Carleson embedding (one-time numeric evaluation).
\end{itemize}

\paragraph{PSC status.} The Prime–Tail Schur–Covering inequality can be presented separately; the main proof chain to (P+) and RH runs solely through the \emph{product} certificate $\Upsilon$.


\end{document}
% --- Proof Map (concise) ---
% Included for reader orientation; does not affect proofs.
\clearpage
% Proof Map (Bounded-Real Route via Product Certificate)
% Ready to \input into the manuscript; no preamble/macros required.

\section*{Proof Map (Bounded-Real Route via Product Certificate)}

\paragraph{Ambient domain.} Let $\Omega:=\{s\in\mathbb C:\Re s>\tfrac12\}$. Write $s=\tfrac12+\sigma+it$, $\sigma>0$. Let $\xi(s)=\tfrac12 s(1-s)\,\pi^{-s/2}\,\Gamma(\tfrac{s}{2})\,\zeta(s)$.

\paragraph{Objects.} On $\ell^2(\mathcal P)$ let $A(s)e_p:=p^{-s}e_p$. Define the outer-normalized ratio
\[
  \mathcal J(s)\ :=\ \frac{\det\nolimits_2(I-A(s))}{\mathcal O(s)\,\xi(s)},
  \qquad
  \Theta(s)\ :=\ \frac{2\mathcal J(s)-1}{2\mathcal J(s)+1}.
\]
Target: $2\mathcal J$ is Herglotz on $\Omega$ (i.e. $\Re(2\mathcal J)\ge 0$), equivalently $\Theta$ is Schur on $\Omega$.

\paragraph{Window and geometry.} Fix an even $\psi\in C^\infty$ with $\psi\equiv1$ on $[-1,1]$, $0\le\psi\le1$, compact support. For an interval $I=[T-L,T+L]$ with $L\le 1/\log\langle T\rangle$ and aperture $\alpha\in[1,2]$, the Carleson box is
\[
  Q(\alpha I):=\{(\sigma,t): |t-T|\le \alpha L,\ 0<\sigma\le \alpha|I|\},\qquad
  \Omega(I):=\{\tfrac12+\sigma+it: (\sigma,t)\in Q(\alpha I)\}\subset\Omega.
\]

\paragraph{Boundary phase density.} With boundary phase $w(t)$ one has on $I$
\[
  \int_I (-w'(t))\,dt\ \le\ \pi\,\mu\big(Q(I)\big),
\]
where $\mu$ is the Carleson/energy measure induced by the neutralized potential.

\paragraph{Locked certificate to (P+).} Independent inputs:
\[
  c_0(\psi),\quad C_H(\psi),\quad C_P(\kappa),\quad C^{(H^1)}_\psi,\quad C_{\mathrm{box}}.
\]
The fixed-aperture embedding gives
\[
  M_\psi\ \le\ \frac{4}{\pi}\,C^{(H^1)}_\psi\sqrt{C_{\mathrm{box}}}.
\]
Hence the boundary certificate
\[
  \Upsilon\ :=\ \frac{C_H(\psi)\,M_\psi + C_P(\kappa)}{c_0(\psi)}\ <\ \frac{\pi}{2},
\]
which forces $w\in[-\tfrac{\pi}{2},\tfrac{\pi}{2}]$ a.e. on the base (condition (P+)).

\paragraph{Poisson and Cayley.} From (P+) and the Poisson integral, $\Re(2\mathcal J)\ge 0$ on $\Omega(I)$, so $2\mathcal J$ is Herglotz. Therefore the Cayley transform is Schur: $|\Theta|\le1$ on $\Omega(I)$.

\paragraph{Removable singularities at $Z(\xi)$.} If $\rho\in Z(\xi)$, boundedness of $\Theta$ on a punctured disc implies $\Theta$ (and hence $\mathcal J$) extends holomorphically across $\rho$.

\paragraph{Globalization/pinch.} Exhaust $\Omega$ by overlapping boxes and use the maximum-modulus pinch to propagate $|\Theta|\le1$ to the interior and exclude zeros of $\xi$ inside $\Omega$. By the functional equation, all nontrivial zeros lie on $\Re s=\tfrac12$; thus RH.

\paragraph{Independence (no circularity).}
\begin{itemize}
  \item $C_{\mathrm{box}}=K_0+K_\xi+\|U_\Gamma\|_{\mathrm{area}}$ with (i) prime tails $K_0$, (ii) $\xi$-block $K_\xi$ (neutralized far-field + near-field), (iii) Archimedean $\Gamma$-term; each bounded independently.
  \item $M_\psi\le \tfrac{4}{\pi}C_\psi^{(H^1)}\sqrt{C_{\mathrm{box}}}$ via fixed-aperture $H^1$–BMO/Carleson embedding (one-time numeric evaluation).
\end{itemize}

\paragraph{PSC status.} The Prime–Tail Schur–Covering inequality can be presented separately; the main proof chain to (P+) and RH runs solely through the \emph{product} certificate $\Upsilon$.

\end{document}