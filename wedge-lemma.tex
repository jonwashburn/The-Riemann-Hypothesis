% The Wedge Lemma (Phase Rise vs. Drop): Two–sided control and boundary regularity
% Ready to \input into the manuscript; assumes theorem environments are defined.

\section*{The Wedge Lemma (phase rise vs. drop)}

\paragraph{Setting.}
Work in the upper half–plane $\mathbb{H}=\{x+iy:y>0\}$ with boundary $\mathbb{R}$. Let $F$ be meromorphic in $\mathbb{H}$, of bounded type, with a.e. non–tangential boundary values. Write
\[
\log F=U+iV, \qquad u:=\log|F|\in L^1_{\mathrm{loc}}(\mathbb{R}),\qquad
w:=\text{a.e. boundary trace of }V\ (\text{the boundary phase}).
\]
Fix an interval $I=(t_0-\tfrac{L}{2},t_0+\tfrac{L}{2})$ and an aperture $\alpha>1$;
set $Q(\alpha I):=\{x+iy:\ x\in \alpha I,\ 0<y\le \alpha L\}$.

\paragraph{Phase–balayage (interval form).}
Let $\mu$ be the signed zero–pole measure of $F$ in $\mathbb{H}$ (poles positive, zeros negative, with multiplicity). For any bounded open interval $J=(a,b)\subset \mathbb{R}$ whose endpoints are points of approximate continuity for $u,w$ one has
\begin{equation}\label{eq:PB-interval}
\int_{J} (-w'(t))\,dt
\;=\;\int_{\mathbb{H}}\mathrm{Bal}(z;J)\,d\mu(z)\;-
\;\big(\mathcal{H}[u](b)-\mathcal{H}[u](a)\big),
\end{equation}
where $\mathrm{Bal}(x+iy;J)=\int_{a}^{b}\tfrac{2y}{(t-x)^2+y^2}\,dt\in[0,2\pi]$ is the balayage kernel and $\mathcal{H}$ the Hilbert transform.

\paragraph{The certificate format.}
Let $\psi$ be a fixed $C^1$ window of mean $0$ supported in $(-1,1)$ and set $\varphi_{L,t_0}(t)=L^{-1}\psi\big(\frac{t-t_0}{L}\big)$. Assume the box energy
\[
C_{\mathrm{box}}=\sup_{J}\frac{1}{|J|}\iint_{Q(\alpha J)}|\nabla U|^2\,y\,dx\,dy
\]
is finite. Let $C_H(\psi),\,C_\psi^{(H^1)},\,c_0(\psi),\,C_P$ be the window/geometry constants (Poisson plateau $c_0>0$, Hilbert envelope $C_H$, $H^1$ window constant, band–limit bound). Define
\[
\Upsilon(F):=\frac{C_H(\psi)\,M_\psi(F)+C_P}{c_0(\psi)},\qquad
M_\psi(F):=\sup_{L,t_0}\frac{1}{L}\Big|\int u(t)\,\varphi_{L,t_0}(t)\,dt\Big|.
\]
Then
\begin{equation}\label{eq:cert-imp}
\Upsilon(F)\le \tfrac{1}{2}\quad\Longrightarrow\quad
\int_{J}(-w'_F)\,dt\ \le\ \frac{\pi}{2}\quad\text{for every subinterval }J\subset I.
\end{equation}
This follows by testing \eqref{eq:PB-interval} against $\varphi_{L,t_0}$, bounding the outer term via the fixed–aperture $H^1$–BMO/Carleson estimate
$M_\psi\le \tfrac{4}{\pi}C_\psi^{(H^1)}\sqrt{C_{\mathrm{box}}}$, and using the plateau lower bound $c_0$ and $\mathrm{Bal}\le 2\pi$ (constants fixed once $\psi,\alpha$ are fixed).

\section*{Certificate invariance and rise control}

\begin{lemma}[Invariance under inversion]\label{lem:inv-wedge}
Let $F^\sharp:=1/F$. Then
\[
u^\sharp=-u,\qquad w^\sharp=-w\ (\text{a.e.}),\qquad \mu^\sharp=-\mu,\qquad C_{\mathrm{box}}(F^\sharp)=C_{\mathrm{box}}(F),
\]
\[M_\psi(F^\sharp)=M_\psi(F),\qquad \Upsilon(F^\sharp)=\Upsilon(F).\]
\end{lemma}

\begin{lemma}[Rise bound]\label{lem:rise-wedge}
If $\Upsilon(F)\le \tfrac12$, then for every subinterval $J\subset I$,
\[\int_{J} w'_F(t)\,dt\ \le\ \frac{\pi}{2}.\]
\end{lemma}

\begin{proof}
Apply \eqref{eq:cert-imp} to $F^\sharp$: $\int_J (-w'_{F^\sharp})\le \pi/2$. Since $w^\sharp=-w$, $-\int_J (-w'_F)\le \pi/2$, i.e. $\int_J w'_F\le \pi/2$.
\end{proof}

Combining \eqref{eq:cert-imp} and Lemma~\ref{lem:rise-wedge} yields
\begin{equation}\label{eq:two-sided}
\left|\int_{J} (-w'_F(t))\,dt\right|\ \le\ \frac{\pi}{2}\qquad\text{for every subinterval }J\subset I.
\end{equation}

\section*{Regularity and the canonical phase representative}

\begin{lemma}[Countable additivity of the phase–drop functional]\label{lem:add-wedge}
Define $T$ on finite disjoint unions of bounded open intervals by
\[
T\Big(\bigsqcup_{m=1}^M (a_m,b_m)\Big)\ :=\ \sum_{m=1}^M\left(\int_{\mathbb{H}}\mathrm{Bal}(z;(a_m,b_m))\,d\mu(z)
-\big(\mathcal{H}[u](b_m)-\mathcal{H}[u](a_m)\big)\right).
\]
Then $T$ is well–defined and countably additive on the algebra generated by bounded open intervals.
\end{lemma}

\begin{lemma}[Canonical phase via indefinite integral]\label{lem:W-wedge}
Fix an anchor $t_\star\in I$ at which $u,w$ have non–tangential boundary values and set $w(t_\star)=0$ (branch choice). Define
\[w^\sharp(t)\ :=\ -\,T\big((t_\star,t]\cap I\big),\qquad t\in I.\]
Then $w^\sharp\in BV(I)$ and for all $a<b$ in $I$,
\[w^\sharp(b)-w^\sharp(a)\ =\ -\,T\big((a,b)\big)\ =\ \int_{(a,b)} (-w'(t))\,dt.\]
\end{lemma}

\begin{lemma}[Identification $w^\sharp=w$ a.e.]\label{lem:identify-wedge}
The distributional derivatives of $w$ and $w^\sharp$ coincide on $I$, hence $w^\sharp-w$ is constant a.e.; with the anchor $w(t_\star)=w^\sharp(t_\star)=0$ one has $w^\sharp=w$ a.e. on $I$.
\end{lemma}

\section*{Wedge from two–sided oscillation (no extra smoothness needed)}

\begin{theorem}[Wedge Lemma: phase rise vs. drop]\label{thm:wedge}
Assume $\Upsilon(F)\le \tfrac12$ on $I$. Fix the anchored representative $w^\sharp$ of Lemma~\ref{lem:W-wedge}. Then for every $t\in I$,
\[|w^\sharp(t)|\ \le\ \frac{\pi}{2}.\]
Consequently $w(t)\in[-\tfrac{\pi}{2},\tfrac{\pi}{2}]$ for a.e. $t\in I$.
\end{theorem}

\begin{remark}[Anchoring]
The wedge is a statement modulo additive constants; fixing $w(t_\star)=0$ pins the branch. Any other choice shifts the wedge interval by the same constant.
\end{remark}
